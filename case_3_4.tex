\section{$4$-valent {\sc Eberhard}-like theorems with triangles}\label{sec:3:4}

In this section we want to prove $4$-valent {\sc Eberhard}-like theorems with triangles, i.e. for $q = [q_3 \times 3, q_l \times l]$, $w = [w_4 \times 4]$, $l > 4$, $\gcd(q_3, q_l) = 1$. As it turns out in \autoref{sec:negative:results}, this is only possible for all admissible pairs of sequences on all closed orientable $2$-manifolds, if $3 \nmid l$. For all the proofs, we want to use \autoref{thm:main:const}, therefore we need to have a construction scheme for patches with arbitrary large $l$-gons. These we get by the next two constructions:

\begin{construction}\label{const:edge:replacement:3:4:1}
  \begin{cinput}
  \item A $4$-patch with $p$-vector $p$ and a specified edge with exactly one vertex incident to some $k_1$-gon and the other vertex incident to some $k_2$-gon.
  \end{cinput}
  \begin{coutput}
  \item A new $4$-patch with $p$-vector $p - [k_1, k_2] + [2k \times 3] + [k_1 + k, k_2 + k]$ for all $k \in \nats$.
  \item If every two faces of the $3$-patch meet proper, then this is carried over to the new patch.
  \end{coutput}
  \begin{cdescription}
    Using the replacement of the single edge as seen in \autoref{fig:const:edge:replacement:3:4:1} results in a new map with $p$-vector $p - [k_1, k_2] + [2 \times 3] + [k_1 + 1, k_2 + 1]$. The thick line on the left is the specified edge and the thick line on the right is a new edge which we can use to repeat the construction. Every time we use this construction we add two new triangles while increasing the number of vertices of the left and right polygon by one; doing this $k$ times has the desired result, since we only inserted $4$-valent vertices.
    \begin{tikzfigure}{\label{fig:const:edge:replacement:3:4:1}}{}
      \matrix (m) [column sep=1cm] {
        \begin{scope}
          \draw[lsquare] (-1, 0) -- (1, 0);
          \draw (-1.2, 0.5) -- (-1, 0) -- (-1.2, -0.5);
          \draw (1.2, 0.5) -- (1, 0) -- (1.2, -0.5);
          \draw (-1, 0) -- (-0.8, -0.5);
          \draw (1, 0) -- (0.8, 0.5);
          \node (k1) at (-2, 0) {$k_1$};
          \node (k2) at (2, 0) {$k_2$};
          \draw[lface] (-1, 0) -- (k1);
          \draw[lface] ( 1, 0) -- (k2);
        \end{scope}
        &
        \begin{scope}
          \draw[lsquare] (-1, 0.5) -- (1, 0.5);
          \draw (-1, -0.5) -- (1, -0.5);
          \draw (-1.2, 1) -- (-1, 0.5) -- (-1, -0.5) -- (-1.2, -1);
          \draw (1.2, 1) -- (1, 0.5) -- (1, -0.5) -- (1.2, -1);
          \draw (-1, 0.5) -- (1, -0.5);
          \draw (-1, -0.5) -- (-0.8, -1);
          \draw (1, 0.5) -- (0.8, 1);
          \node (k1) at (-2, 0) {$k_1$};
          \node (k2) at (2, 0) {$k_2$};
          \draw[lface] (-1, 0.5) -- (k1);
          \draw[lface] ( 1, 0.5) -- (k2);
        \end{scope}
        \\
      };
    \end{tikzfigure}
  \end{cdescription}
\end{construction}

\begin{construction}\label{const:edge:replacement:3:4:2}
  \begin{cinput}
  \item A $4$-patch with $p$-vector $p$ and a specified edge which is the common edge of a $k_1$-gon and a $k_2$-gon.
  \end{cinput}
  \begin{coutput}
  \item A new $4$-patch with $p$-vector $p - [k_1, k_2] + [(6k) \times 3] + [k_1 + 3k, k_2 + 3k]$ for all $k \in \nats$.
  \end{coutput}
  \begin{cdescription}
    Using the replacement of the path as seen in \autoref{fig:const:edge:replacement:3:4:2} results in a new map with $p$-vector $p - [k_1, k_2] + [(6k) \times 3] + [k_1 + 3, k_2 + 3]$. The thick drawn path on the left is the specified path and the thick drawn path on the right is a new path which we can use to repeat the construction. Every time we use this construction we add six new triangles while increasing the number of vertices of the left and right polygon by three; doing this $k$ times has the desired result, since we only inserted $4$-valent vertices.
    \begin{tikzfigure}{\label{fig:const:edge:replacement:3:4:2}}{}
      \matrix (m) [column sep=1cm] {
        \begin{scope}
          \draw[ldiamond] (0, -1) -- (0, 1);
          \draw (0.5, -1.2) -- (0, -1) -- (-0.5, -1.2);
          \draw (0.5, 1.2) -- (0, 1) -- (-0.5, 1.2);
          \node (k1) at (-1, 0) {$k_1$};
          \node (k2) at (1, 0) {$k_2$};
          \draw[lface] (0, 0) -- (k1);
          \draw[lface] (0, 0) -- (k2);
        \end{scope}
        &
        \begin{scope}
          \draw[ldiamond] (0, 0.6) -- (0, 1.2);
          \draw (0, 0.6) -- (0, -1);
          \draw (0.5, -1.2) -- (0, -1) -- (-0.5, -1.2);
          \draw (0.5, 1.4) -- (0, 1.2) -- (-0.5, 1.4);
          
          \draw (-0.5, 0) -- (0, -0.6) -- (0.5, 0);
          \draw (-0.5, 0) -- (0, -0.2) -- (0.5, 0);
          \draw (-0.5, 0) -- (0, 0.6) -- (0.5, 0);
          \draw (-0.5, 0) -- (0, 0.2) -- (0.5, 0);
          

          \node (k1) at (-2, 0) {$k_1 + 3$};
          \node (k2) at (2, 0) {$k_2 + 3$};
          \draw[lface] (0, 0.9) -- (k1);
          \draw[lface] (0, 0.9) -- (k2);

        \end{scope}
        \\
      };
    \end{tikzfigure}  
  \end{cdescription}
\end{construction}

\begin{construction}\label{const:edge:replacement:3:4:3}
  \begin{cinput}
  \item A $4$-patch with $p$-vector $p$ and a specified vertex which is adjacent to both $k_1$-gon and a $k_2$-gon which do not share an edge containing this vertex.
  \end{cinput}
  \begin{coutput}
  \item A new $4$-patch with $p$-vector $p - [k_1, k_2] + [(6k) \times 3] + [k_1 + 3k, k_2 + 3k]$ for all $k \in \nats$.
  \end{coutput}
  \begin{cdescription}
    Using the replacement of the path as seen in \autoref{fig:const:edge:replacement:3:4:2} results in a new map with $p$-vector $p - [k_1, k_2] + [(6k) \times 3] + [k_1 + 3, k_2 + 3]$. The thick drawn path on the left is the specified path and the thick drawn path on the right is a new path which we can use to repeat the construction. Every time we use this construction we add six new triangles while increasing the number of vertices of the left and right polygon by three; doing this $k$ times has the desired result, since we only inserted $4$-valent vertices.
    \begin{tikzfigure}{\label{fig:const:edge:replacement:3:4:2}}{}
      \matrix (m) [column sep=1cm] {
        \begin{scope}
          \draw (0.2, -0.5) -- (0, 0) -- (-0.2, -0.5);
          \draw (0.2, 0.5) -- (0, 0) -- (-0.2, 0.5);
          \node (k1) at (-1, 0) {$k_1$};
          \node (k2) at (1, 0) {$k_2$};
          \node[lvertex] at (0, 0) {};
          \draw[lface] (0, 0) -- (k1);
          \draw[lface] (0, 0) -- (k2);
        \end{scope}
        &
        \begin{scope}
          \draw (0.2, -1.5) -- (0, -1) -- (-0.2, -1.5);
          \draw (0.2, 1.5) -- (0, 1) -- (-0.2, 1.5);
          \draw (0, -1) -- (-0.25, -0.5) -- (-0.25, 0.5) -- (0, 1);
          \draw (0, -1) -- ( 0.25, -0.5) -- ( 0.25, 0.5) -- (0, 1);
          \draw (-0.25, -0.5) -- ( 0.25,  0.5) -- (-0.25,  0.5);
          \draw (-0.25,  0.5) -- ( 0.25, -0.5) -- (-0.25, -0.5);
          \node (k1) at (-2, 0) {$k_1 + 3$};
          \node (k2) at (2, 0) {$k_2 + 3$};
          \node[lvertex] at (0, 1) {};
          \draw[lface] (0, 1) -- (k1);
          \draw[lface] (0, 1) -- (k2);
        \end{scope}
        \\
      };
    \end{tikzfigure}  
  \end{cdescription}
\end{construction}

\clearpage
\begin{theorem}
  Let $p$ and $v$ be a pair of admissible sequences for a orientable closed $2$-manifold $S$. Then $(p, v)$ is $[(3k + 1) \times 3, (3k+5)]$-$[4]$-realizable for all $k \in \nats$.
  \begin{proof}
    An expansion $4$-patch with outer tuple $o = (1, 2, 1, 3, 2, 3)$, a corresponding $o$-$4$-gonal $4$-patch and a expansion $4$-patch with the polyhedral property, all consisting only of triangles and pentagons, are shown in \autoref{fig:expansion:patch:3:5:4}. Using \autoref{const:edge:replacement:3:4:1} and \autoref{const:edge:replacement:3:4:2} on the edges as indicated in all three we get new $4$-patches consisting only of triangles and $3k + 5$-gons. Therefore we can apply \autoref{thm:main:const}.
  \end{proof}
\end{theorem}
\begin{tikzfigure2}{}
  \begin{tikzsubfigure}{}{}{0.5}
    \begin{scope}[yscale=0.866]
      \draw (0.5,1)--(-0.5,1)--(-1.5,3)--(-4,6)--(-3.5,7)--(-2.5,7)--(-1,6.5)--(-0.5,7)--(0.5,7)--(1,7.5)--(2.5,7)--(1.5,3)--(0.5,1);
      \draw (-1.5,3)--(-0.5,4.5)--(0.5,3)--(0.5,1); %1
      \draw (-4,6)--(-1,6)--(-1,6.5); %8
      \draw[lsquare] (-1.5,3)--(-1,6); %4
      \draw (-0.5,4.5) --(-1,6); %3
      \draw (-0.5,4.5) --(0.5,7); %7
      \draw[lsquare] (0.5,3) --(0.5,7); %5
      \draw (0.5,3) --(1,7.5); %2

      \fill[black] (0.5,1)    circle (2pt);
      \fill[black] (-0.5,1)   circle (2pt);
      \fill[black] (-1.5,3)   circle (2pt);
      \fill[black] (-4,6)     circle (2pt);
      \fill[black] (-3.5,7)   circle (2pt);
      \fill[black] (-2.5,7)   circle (2pt);
      \fill[black] (-1,6.5)   circle (2pt);
      \fill[black] (-0.5,7)   circle (2pt);
      \fill[black] (0.5,7)    circle (2pt);
      \fill[black] (1,7.5)    circle (2pt);
      \fill[black] (2.5,7)    circle (2pt);
      \fill[black] (1.5,3)    circle (2pt);
      \fill[black] (-0.5,4.5) circle (2pt);
      \fill[black] (0.5,3)    circle (2pt);
      \fill[black] (-1,6)     circle (2pt);
      
      \node (n1) at (-2  ,6.5) {$5$};
      \node (n2) at (-0.5,2.5) {$5$};
      \node (n3) at ( 1,4) {$5$};
      \node (n4) at (-0.25,6.25) {$5$};
      \node at (0.75,6.75) {$3$};
      \node at (-1.8,4.5) {$3$};
      \node at (-0.75,4.5) {$3$};
      \node at (0,4.5) {$3$};

      \draw[lface] (-1,6)--(n1);
      \draw[lface] (-1.5,3)--(n2);
      \draw[lface] (0.5,3)--(n3);
      \draw[lface] (0.5,7)--(n4);
      
      \node[anchor= 90] at (0.5,1)  {$i_{0}'$};
      \node[anchor= 90] at (-0.5,1) {$i_0$};
      \node[anchor=  0] at (-1.5,3) {$i_1$};
      \node[anchor= 30] at (-4,6)   {$i_2=o_0$};
      \node[anchor=300] at (-3.5,7) {$\mathbf{o_1}$};
      \node[anchor=270] at (-2.5,7) {$o_2$};
      \node[anchor=315] at (-1,6.5) {$o_3$};
      \node[anchor=270] at (-0.5,7) {$o_4$};
      \node[anchor=270] at (0.5,7)  {$o_5$};
      \node[anchor=270] at (1,7.5)  {$o_6$};
      \node[anchor=240] at (2.5,7)  {$i_2'=o_7$};
      \node[anchor=180] at (1.5,3)  {$i_1'$};
    \end{scope}
  \end{tikzsubfigure}~
  \begin{tikzsubfigure}{}{}{0.5}
    \begin{scope}[scale=0.55]
      \begin{scope}[yscale=0.866]
        \draw[very thick] (0.5,1)--(-0.5,1)--(-1.5,3)--(-4,6)--(-3.5,7)--(-2.5,7)--(-1,6.5)--(-0.5,7)--(0.5,7)--(1,7.5)--(2.5,7)--(1.5,3)--(0.5,1);
        \draw (-1.5,3)--(-0.5,4.5)--(0.5,3)--(0.5,1); %1
        \draw (-4,6)--(-1,6)--(-1,6.5); %8
        \draw (-1.5,3)--(-1,6); %4
        \draw (-0.5,4.5) --(-1,6); %3
        \draw (-0.5,4.5) --(0.5,7); %7
        \draw (0.5,3) --(0.5,7); %5
        \draw (0.5,3) --(1,7.5); %2

        \fill[black] (0.5,1)    circle (3pt);
        \fill[black] (-0.5,1)   circle (3pt);
        \fill[black] (-1.5,3)   circle (3pt);
        \fill[black] (-4,6)     circle (3pt);
        \fill[black] (-3.5,7)   circle (3pt);
        \fill[black] (-2.5,7)   circle (3pt);
        \fill[black] (-1,6.5)   circle (3pt);
        \fill[black] (-0.5,7)   circle (3pt);
        \fill[black] (0.5,7)    circle (3pt);
        \fill[black] (1,7.5)    circle (3pt);
        \fill[black] (2.5,7)    circle (3pt);
        \fill[black] (1.5,3)    circle (3pt);
        \fill[black] (-0.5,4.5) circle (3pt);
        \fill[black] (0.5,3)    circle (3pt);
        \fill[black] (-1,6)     circle (3pt);
      \end{scope}
      \begin{scope}[rotate=-60, yscale=0.866]
        \draw[very thick] (0.5,1)--(-0.5,1)--(-1.5,3)--(-4,6)--(-3.5,7)--(-2.5,7)--(-1,6.5)--(-0.5,7)--(0.5,7)--(1,7.5)--(2.5,7)--(1.5,3)--(0.5,1);
        \draw (-1.5,3)--(-0.5,4.5)--(0.5,3)--(0.5,1); %1
        \draw (-4,6)--(-1,6)--(-1,6.5); %8
        \draw (-1.5,3)--(-1,6); %4
        \draw (-0.5,4.5) --(-1,6); %3
        \draw (-0.5,4.5) --(0.5,7); %7
        \draw (0.5,3) --(0.5,7); %5
        \draw (0.5,3) --(1,7.5); %2


        \fill[black] (0.5,1)    circle (3pt);
        \fill[black] (-0.5,1)   circle (3pt);
        \fill[black] (-1.5,3)   circle (3pt);
        \fill[black] (-4,6)     circle (3pt);
        \fill[black] (-3.5,7)   circle (3pt);
        \fill[black] (-2.5,7)   circle (3pt);
        \fill[black] (-1,6.5)   circle (3pt);
        \fill[black] (-0.5,7)   circle (3pt);
        \fill[black] (0.5,7)    circle (3pt);
        \fill[black] (1,7.5)    circle (3pt);
        \fill[black] (2.5,7)    circle (3pt);
        \fill[black] (1.5,3)    circle (3pt);
        \fill[black] (-0.5,4.5) circle (3pt);
        \fill[black] (0.5,3)    circle (3pt);
        \fill[black] (-1,6)     circle (3pt);
      \end{scope}
      \begin{scope}[yscale=0.866,shift={(0 cm,14 cm)},rotate=180]
        \draw[very thick] (0.5,1)--(-0.5,1)--(-1.5,3)--(-4,6)--(-3.5,7)--(-2.5,7)--(-1,6.5)--(-0.5,7)--(0.5,7)--(1,7.5)--(2.5,7)--(1.5,3)--(0.5,1);
        \draw (-1.5,3)--(-0.5,4.5)--(0.5,3)--(0.5,1); %1
        \draw (-4,6)--(-1,6)--(-1,6.5); %8
        \draw (-1.5,3)--(-1,6); %4
        \draw (-0.5,4.5) --(-1,6); %3
        \draw (-0.5,4.5) --(0.5,7); %7
        \draw (0.5,3) --(0.5,7); %5
        \draw (0.5,3) --(1,7.5); %2


        \fill[black] (0.5,1)    circle (3pt);
        \fill[black] (-0.5,1)   circle (3pt);
        \fill[black] (-1.5,3)   circle (3pt);
        \fill[black] (-4,6)     circle (3pt);
        \fill[black] (-3.5,7)   circle (3pt);
        \fill[black] (-2.5,7)   circle (3pt);
        \fill[black] (-1,6.5)   circle (3pt);
        \fill[black] (-0.5,7)   circle (3pt);
        \fill[black] (0.5,7)    circle (3pt);
        \fill[black] (1,7.5)    circle (3pt);
        \fill[black] (2.5,7)    circle (3pt);
        \fill[black] (1.5,3)    circle (3pt);
        \fill[black] (-0.5,4.5) circle (3pt);
        \fill[black] (0.5,3)    circle (3pt);
        \fill[black] (-1,6)     circle (3pt);
      \end{scope}
      \begin{scope}[shift={(0 cm,12.124 cm)},rotate=120,yscale=0.866]
        \draw[very thick] (0.5,1)--(-0.5,1)--(-1.5,3)--(-4,6)--(-3.5,7)--(-2.5,7)--(-1,6.5)--(-0.5,7)--(0.5,7)--(1,7.5)--(2.5,7)--(1.5,3)--(0.5,1);
        \draw (-1.5,3)--(-0.5,4.5)--(0.5,3)--(0.5,1); %1
        \draw (-4,6)--(-1,6)--(-1,6.5); %8
        \draw (-1.5,3)--(-1,6); %4
        \draw (-0.5,4.5) --(-1,6); %3
        \draw (-0.5,4.5) --(0.5,7); %7
        \draw (0.5,3) --(0.5,7); %5
        \draw (0.5,3) --(1,7.5); %2
        \draw (0.5,1)--(-0.5,1)--(-1.5,3)--(-4,6)--(-3.5,7)--(-2.5,7)--(-1,6.5)--(-0.5,7)--(0.5,7)--(1,7.5)--(2.5,7)--(1.5,3)--(0.5,1);


        \fill[black] (0.5,1)    circle (3pt);
        \fill[black] (-0.5,1)   circle (3pt);
        \fill[black] (-1.5,3)   circle (3pt);
        \fill[black] (-4,6)     circle (3pt);
        \fill[black] (-3.5,7)   circle (3pt);
        \fill[black] (-2.5,7)   circle (3pt);
        \fill[black] (-1,6.5)   circle (3pt);
        \fill[black] (-0.5,7)   circle (3pt);
        \fill[black] (0.5,7)    circle (3pt);
        \fill[black] (1,7.5)    circle (3pt);
        \fill[black] (2.5,7)    circle (3pt);
        \fill[black] (1.5,3)    circle (3pt);
        \fill[black] (-0.5,4.5) circle (3pt);
        \fill[black] (0.5,3)    circle (3pt);
        \fill[black] (-1,6)     circle (3pt);
      \end{scope}
    \end{scope}
  \end{tikzsubfigure}
  \begin{tikzsubfigure}{}{}{1.0}
    \begin{scope}[scale=5]
      \coordinate (x0) at (-0.145531, 0.231610);
\coordinate (x1) at (-0.063274, -0.419927);
\coordinate (x2) at (0.405804, -0.125166);
\coordinate (x3) at (-0.993712, 0.111964);
\coordinate (x4) at (-0.993712, -0.111964);
\coordinate (x5) at (-0.330279, -0.943883);
\coordinate (x6) at (-0.111964, -0.993712);
\coordinate (x7) at (0.111964, -0.993712);
\coordinate (x8) at (0.943883, -0.330279);
\coordinate (x9) at (0.993712, -0.111964);
\coordinate (x10) at (0.993712, 0.111964);
\coordinate (x11) at (0.330279, 0.943883);
\coordinate (x12) at (0.111964, 0.993712);
\coordinate (x13) at (-0.943883, 0.330279);
\coordinate (x14) at (-0.111964, 0.993712);
\coordinate (x15) at (-0.532032, 0.846724);
\coordinate (x16) at (-0.707107, 0.707107);
\coordinate (x17) at (-0.846724, 0.532032);
\coordinate (x18) at (-0.330279, 0.943883);
\coordinate (x19) at (-0.846724, -0.532032);
\coordinate (x20) at (-0.707107, -0.707107);
\coordinate (x21) at (-0.532032, -0.846724);
\coordinate (x22) at (-0.943883, -0.330279);
\coordinate (x23) at (0.532032, -0.846724);
\coordinate (x24) at (0.707107, -0.707107);
\coordinate (x25) at (0.846724, -0.532032);
\coordinate (x26) at (0.330279, -0.943883);
\coordinate (x27) at (0.846724, 0.532032);
\coordinate (x28) at (0.707107, 0.707107);
\coordinate (x29) at (0.532032, 0.846724);
\coordinate (x30) at (0.943883, 0.330279);
\draw (x0) -- (x1);
\draw (x1) -- (x2);
\draw (x2) -- (x0);
\draw (x0) -- (x3);
\draw (x3) -- (x4);
\draw[very thick] (x4) -- (x5);
\draw (x5) -- (x1);
\draw (x5) -- (x6);
\draw (x6) -- (x1);
\draw (x6) -- (x7);
\draw[very thick] (x7) -- (x8);
\draw (x8) -- (x2);
\draw (x8) -- (x9);
\draw (x9) -- (x2);
\draw (x9) -- (x10);
\draw[very thick] (x10) -- (x11);
\draw (x11) -- (x0);
\draw (x11) -- (x12);
\draw[very thick] (x12) -- (x13);
\draw[very thick] (x13) -- (x3);
\draw (x12) -- (x14);
\draw[very thick] (x14) -- (x13);
\draw (x14) -- (x15);
\draw (x15) -- (x16);
\draw (x16) -- (x17);
\draw[very thick] (x17) -- (x13);
\draw (x14) -- (x18);
\draw (x18) -- (x15);
\draw (x4) -- (x19);
\draw (x19) -- (x20);
\draw (x20) -- (x21);
\draw (x21) -- (x5);
\draw (x4) -- (x22);
\draw (x22) -- (x19);
\draw (x7) -- (x23);
\draw (x23) -- (x24);
\draw (x24) -- (x25);
\draw (x25) -- (x8);
\draw (x7) -- (x26);
\draw (x26) -- (x23);
\draw (x10) -- (x27);
\draw (x27) -- (x28);
\draw (x28) -- (x29);
\draw (x29) -- (x11);
\draw (x10) -- (x30);
\draw (x30) -- (x27);
\node at (-0.298232, 0.087497) {0};
\node at (0.065666, -0.104494) {1};
\node at (-0.505302, -0.226440) {2};
\node at (-0.168506, -0.785841) {3};
\node at (0.257283, -0.572559) {4};
\node at (0.781133, -0.189136) {5};
\node at (0.515595, 0.210065) {6};
\node at (-0.328177, 0.522290) {7};
\node at (-0.314628, 0.772568) {8};
\node at (-0.628342, 0.681971) {9};
\node at (-0.324759, 0.928107) {10};
\node at (-0.681971, -0.628342) {11};
\node at (-0.928107, -0.324759) {12};
\node at (0.628342, -0.681971) {13};
\node at (0.324759, -0.928107) {14};
\node at (0.681971, 0.628342) {15};
\node at (0.928107, 0.324759) {16};

    \end{scope}
  \end{tikzsubfigure}
\end{tikzfigure2}

\begin{theorem}
  Let $p$ and $v$ be a pair of admissible sequences for a orientable closed $2$-manifold $S$. Then $(p, v)$ is $[(3k + 3) \times 5, (3k+7)]$-$[4]$-realizable for all $k \in \nats$.
  \begin{proof}
    An expansion $4$-patch with outer tuple $o = (1, 0, 1, 1, 2, 1)$, a corresponding $o$-$4$-gonal $4$-patch and a expansion $4$-patch with the polyhedral property, all consisting only of triangles and heptagons, are shown in \autoref{fig:expansion:patch:3:4:7}. Using \autoref{const:edge:replacement:3:4:1} and \autoref{const:edge:replacement:3:4:2} on the edges as indicated in all three we get new $4$-patches consisting only of triangles and $3k + 7$-gons. Therefore we can apply \autoref{thm:main:const}.
  \end{proof}
\end{theorem}
\begin{tikzfigure2}{}
  \begin{tikzsubfigure}{}{}{0.5}
    \begin{scope}[yscale=0.866]
      \draw(0,1.5)--(-1,2)--(-0.5,3)--(0.5,3.5)--(1.5,3)--(2.5,3)--(4.5,2.5)--(5.5,3)--(6.5,2.5)--(5.5,2)--(5,1)--(4,0.5)--(3,0.5)--(2,0)--(2,1)--(1,1.5)--(0,1.5);
      \draw (3,0.5)--(2,1)--(2,2)--(1,1.5);
      \draw (1.5,3)--(2,2)--(2.5,3);
      \draw (5,1)--(4.5,2.5)--(5.5,2)--(5.5,3);
      
      \node (n1) at (0.5,2.5) {$7$};
      \node (n2) at (3.5,1.5) {$7$};
      \node at (2.25,0.5) {$3$};
      \node at (1.75,1.5) {$3$};
      \node at (2,2.666) {$3$};
      \node at (5,1.8) {$3$};
      \node at (5.2,2.5) {$3$};
      \node at (5.8,2.5) {$3$};

      \draw[lface](2,2)--(n1);
      \draw[lface](2,2)--(n2);
      
      \fill[black] (0,1.5)   circle (2pt);
      \fill[black] (-1,2)    circle (2pt);
      \fill[black] (-0.5,3)  circle (2pt);  
      \fill[black] (0.5,3.5) circle (2pt); 
      \fill[black] (1.5,3)   circle (2pt); 
      \fill[black] (2.5,3)   circle (2pt); 
      \fill[black] (4.5,2.5) circle (2pt); 
      \fill[black] (5.5,3)   circle (2pt); 
      \fill[black] (6.5,2.5) circle (2pt); 
      \fill[black] (5.5,2)   circle (2pt);
      \fill[black] (5,1)     circle (2pt);
      \fill[black] (4,0.5)   circle (2pt);
      \fill[black] (3,0.5)   circle (2pt);
      \fill[black] (2,0)     circle (2pt);
      \fill[black] (2,1)     circle (2pt);
      \fill[black] (1,1.5)   circle (2pt); 
      \fill[black] (2,2)     circle (2pt);
      
      \node[lvertex] at (2,2) {};
      \node[anchor= 90] at (0,1.5)   {$i_{0}$};
      \node[anchor=  0] at (-1,2)    {$i_{1}$};
      \node[anchor=330] at (-0.5,3)  {$i_{2}=o_{0}$};
      \node[anchor=270] at (0.5,3.5) {$\mathbf{o_{1}}$};
      \node[anchor=240] at (1.5,3)   {$o_{2}$};
      \node[anchor=270] at (2.5,3)   {$o_{3}$};
      \node[anchor=270] at (4.5,2.5) {$o_{4}$};
      \node[anchor=270] at (5.5,3)   {$o_{5}$};
      \node[anchor=180] at (6.5,2.5) {$o_{6}$};
      \node[anchor=150] at (5.5,2)   {$o_{7}$};
      \node[anchor=140] at (5,1)     {$o_{8}$};
      \node[anchor=150] at (4,0.5)   {$o_{9}$};
      \node[anchor= 90] at (3,0.5)   {$o_{10}$};
      \node[anchor= 90] at (2,0)     {$i_{2}'=o_{11}$};
      \node[anchor= 45] at (2,1)     {$i_{1}'$};
      \node[anchor= 90] at (1,1.5)   {$i_{0}'$};

    \end{scope}

  %   \begin{scope}[yscale=0.866]
  %     \draw (0.5,1)--(-0.5,1)--(-2,4)--(-4.25,4.8333)--(-4,6)--(-3.5,7)--(-2.5,7)--(-1.5,7.333)--(-0.5,7)--(0.5,7)--(1.5,6.666)--(2,4)--(0.5,1);
  %     \draw (-0.5,1)--(-0.5,4.5)--(-1.5,5)--(0,6)--(0.5,7);
  %     \draw(-0.5,4.5)--(-2,4)--(-1.5,5)--(-4.25,4.8333);
  %     \draw (-0.5,4.5)--(0,6)--(-0.5,7);
  %     \draw (-4,6)--(-2.5,7);

  %     \fill[black] (0.5,1)       circle (2pt);
  %     \fill[black] (-0.5,1)      circle (2pt);
  %     \fill[black] (-2,4)        circle (2pt);
  %     \fill[black] (-4.25,4.8333)circle (2pt);
  %     \fill[black] (-4,6)        circle (2pt);
  %     \fill[black] (-3.5,7)      circle (2pt);
  %     \fill[black] (-2.5,7)      circle (2pt);
  %     \fill[black] (-1.5,7.333)  circle (2pt);
  %     \fill[black] (-0.5,7)      circle (2pt);
  %     \fill[black] (0.5,7)       circle (2pt);
  %     \fill[black] (1.5,6.666)   circle (2pt);
  %     \fill[black] (2,4)         circle (2pt);
  %     \fill[black] (0,6)         circle (2pt);
  %     \fill[black] (-1.5,5)      circle (2pt);
  %     \fill[black] (-0.5,4.5)    circle (2pt);

  %     \node[anchor= 90] at (0.5,1)        {$i_{0}'$};
  %     \node[anchor= 90] at (-0.5,1)       {$i_{0}$};
  %     \node[anchor= 60] at (-2,4)         {$i_{1}$};
  %     \node[anchor= 90] at (-4.25,4.8333) {$i_2=o_{0}$};
  %     \node[anchor=  0] at (-4,6)         {$o_{1}$};
  %     \node[anchor=335] at (-3.5,7)       {$o_{2}$};
  %     \node[anchor=270] at (-2.5,7)       {$o_{3}$};
  %     \node[anchor=270] at (-1.5,7.333)   {$o_{4}$};
  %     \node[anchor=270] at (-0.5,7)       {$o_{5}$};
  %     \node[anchor=270] at (0.5,7)        {$o_{6}$};
  %     \node[anchor=210] at (1.5,6.666)    {$i_{2}'=o_7$};
  %     \node[anchor=180] at (2,4)          {$i_{1}'$};
      
  %     \node (n1) at (0.5,4.5)  {$7$};
  %     \node at (-1,3.5)   {$3$};
  %     \node at (-1.5,4.5) {$3$};
  %     \node at (-2.4,4.5) {$3$};
  %     \node (n2) at (-2,6) {$7$};
  %     \node at (-3.4,6.7){$3$};
  %     \node at (0,6.6)    {$3$};
  %     \node at (-0.7,5)   {$3$};

  %     \node[lvertex] at (0,6) {};
      
  %     \draw[lface] (0,6)--(n1);
  %     \draw[lface] (0,6)--(n2);
  %   \end{scope}
  % \end{tikzsubfigure}~
  % \begin{tikzsubfigure}{}{}{0.5}
  %   \begin{scope}[scale=0.55]
  %     \begin{scope}[yscale=0.866]
  %       \draw[very thick] (0.5,1)--(-0.5,1)--(-2,4)--(-4.25,4.8333)--(-4,6)--(-3.5,7)--(-2.5,7)--(-1.5,7.333)--(-0.5,7)--(0.5,7)--(1.5,6.666)--(2,4)--(0.5,1);
  %       \draw (-0.5,1)--(-0.5,4.5)--(-1.5,5)--(0,6)--(0.5,7);
  %       \draw(-0.5,4.5)--(-2,4)--(-1.5,5)--(-4.25,4.8333);
  %       \draw (-0.5,4.5)--(0,6)--(-0.5,7);
  %       \draw (-4,6)--(-2.5,7);
  %       \fill[black] (0.5,1)       circle (2pt);
  %       \fill[black] (-0.5,1)      circle (2pt);
  %       \fill[black] (-2,4)        circle (2pt);
  %       \fill[black] (-4.25,4.8333)circle (2pt);
  %       \fill[black] (-4,6)        circle (2pt);
  %       \fill[black] (-3.5,7)      circle (2pt);
  %       \fill[black] (-2.5,7)      circle (2pt);
  %       \fill[black] (-1.5,7.333)  circle (2pt);
  %       \fill[black] (-0.5,7)      circle (2pt);
  %       \fill[black] (0.5,7)       circle (2pt);
  %       \fill[black] (1.5,6.666)   circle (2pt);
  %       \fill[black] (2,4)         circle (2pt);
  %       \fill[black] (0,6)         circle (2pt);
  %       \fill[black] (-1.5,5)      circle (2pt);
  %       \fill[black] (-0.5,4.5)    circle (2pt);
  %     \end{scope}
  %     \begin{scope}[rotate=-60, yscale=0.866]
  %       \draw[very thick] (0.5,1)--(-0.5,1)--(-2,4)--(-4.25,4.8333)--(-4,6)--(-3.5,7)--(-2.5,7)--(-1.5,7.333)--(-0.5,7)--(0.5,7)--(1.5,6.666)--(2,4)--(0.5,1);
  %       \draw (-0.5,1)--(-0.5,4.5)--(-1.5,5)--(0,6)--(0.5,7);
  %       \draw(-0.5,4.5)--(-2,4)--(-1.5,5)--(-4.25,4.8333);
  %       \draw (-0.5,4.5)--(0,6)--(-0.5,7);
  %       \draw (-4,6)--(-2.5,7);
  %       \fill[black] (0.5,1)       circle (2pt);
  %       \fill[black] (-0.5,1)      circle (2pt);
  %       \fill[black] (-2,4)        circle (2pt);
  %       \fill[black] (-4.25,4.8333)circle (2pt);
  %       \fill[black] (-4,6)        circle (2pt);
  %       \fill[black] (-3.5,7)      circle (2pt);
  %       \fill[black] (-2.5,7)      circle (2pt);
  %       \fill[black] (-1.5,7.333)  circle (2pt);
  %       \fill[black] (-0.5,7)      circle (2pt);
  %       \fill[black] (0.5,7)       circle (2pt);
  %       \fill[black] (1.5,6.666)   circle (2pt);
  %       \fill[black] (2,4)         circle (2pt);
  %       \fill[black] (0,6)         circle (2pt);
  %       \fill[black] (-1.5,5)      circle (2pt);
  %       \fill[black] (-0.5,4.5)    circle (2pt);
  %     \end{scope}
  %     \begin{scope}[yscale=0.866,shift={(0 cm,14 cm)},rotate=180]
  %       \draw[very thick] (0.5,1)--(-0.5,1)--(-2,4)--(-4.25,4.8333)--(-4,6)--(-3.5,7)--(-2.5,7)--(-1.5,7.333)--(-0.5,7)--(0.5,7)--(1.5,6.666)--(2,4)--(0.5,1);
  %       \draw (-0.5,1)--(-0.5,4.5)--(-1.5,5)--(0,6)--(0.5,7);
  %       \draw(-0.5,4.5)--(-2,4)--(-1.5,5)--(-4.25,4.8333);
  %       \draw (-0.5,4.5)--(0,6)--(-0.5,7);
  %       \draw (-4,6)--(-2.5,7);
  %       \fill[black] (0.5,1)       circle (2pt);
  %       \fill[black] (-0.5,1)      circle (2pt);
  %       \fill[black] (-2,4)        circle (2pt);
  %       \fill[black] (-4.25,4.8333)circle (2pt);
  %       \fill[black] (-4,6)        circle (2pt);
  %       \fill[black] (-3.5,7)      circle (2pt);
  %       \fill[black] (-2.5,7)      circle (2pt);
  %       \fill[black] (-1.5,7.333)  circle (2pt);
  %       \fill[black] (-0.5,7)      circle (2pt);
  %       \fill[black] (0.5,7)       circle (2pt);
  %       \fill[black] (1.5,6.666)   circle (2pt);
  %       \fill[black] (2,4)         circle (2pt);
  %       \fill[black] (0,6)         circle (2pt);
  %       \fill[black] (-1.5,5)      circle (2pt);
  %       \fill[black] (-0.5,4.5)    circle (2pt);
  %     \end{scope}
  %     \begin{scope}[shift={(0cm, 12.124cm)},rotate=120,yscale=0.866]
  %       \draw[very thick] (0.5,1)--(-0.5,1)--(-2,4)--(-4.25,4.8333)--(-4,6)--(-3.5,7)--(-2.5,7)--(-1.5,7.333)--(-0.5,7)--(0.5,7)--(1.5,6.666)--(2,4)--(0.5,1);
  %       \draw (-0.5,1)--(-0.5,4.5)--(-1.5,5)--(0,6)--(0.5,7);
  %       \draw(-0.5,4.5)--(-2,4)--(-1.5,5)--(-4.25,4.8333);
  %       \draw (-0.5,4.5)--(0,6)--(-0.5,7);
  %       \draw (-4,6)--(-2.5,7);
  %       \fill[black] (0.5,1)       circle (2pt);
  %       \fill[black] (-0.5,1)      circle (2pt);
  %       \fill[black] (-2,4)        circle (2pt);
  %       \fill[black] (-4.25,4.8333)circle (2pt);
  %       \fill[black] (-4,6)        circle (2pt);
  %       \fill[black] (-3.5,7)      circle (2pt);
  %       \fill[black] (-2.5,7)      circle (2pt);
  %       \fill[black] (-1.5,7.333)  circle (2pt);
  %       \fill[black] (-0.5,7)      circle (2pt);
  %       \fill[black] (0.5,7)       circle (2pt);
  %       \fill[black] (1.5,6.666)   circle (2pt);
  %       \fill[black] (2,4)         circle (2pt);
  %       \fill[black] (0,6)         circle (2pt);
  %       \fill[black] (-1.5,5)      circle (2pt);
  %       \fill[black] (-0.5,4.5)    circle (2pt);
  %     \end{scope}
  %   \end{scope}
  \end{tikzsubfigure}
  \begin{tikzsubfigure}{}{}{1.0}
    \begin{scope}[scale=5]
      % \node (n0) at (0.198518, -0.127580) {0};
% \node (n1) at (0.800523, 0.062019) {1};
% \node (n2) at (0.931003, 0.140216) {2};
% \node (n3) at (0.910836, 0.232924) {3};
% \node (n4) at (0.640446, 0.281189) {4};
% \node (n5) at (0.138539, 0.124657) {5};
% \node (n6) at (0.152558, -0.085147) {6};
% \node (n7) at (0.363865, -0.075274) {7};
% \node (n8) at (0.627263, -0.283950) {8};
% \node (n9) at (0.231615, -0.789247) {9};
% \node (n10) at (0.053369, -0.891434) {10};
% \node (n11) at (-0.109343, -0.608463) {11};
% \node (n12) at (-0.511642, -0.595120) {12};
% \node (n13) at (-0.727638, -0.306583) {13};
% \node (n14) at (-0.513635, 0.127724) {14};
% \node[anchor=0] (n15) at (-0.970516, 0.211123) {15};
% \node (n16) at (-0.933787, -0.213847) {16};
% \node (n17) at (0.215101, -0.928496) {17};
% \node[anchor=180] (n18) at (0.970516, -0.211123) {18};
% \node (n19) at (0.306193, 0.726498) {19};
% \node[anchor=240] (n20) at (0.595211, 0.795109) {20};
% \node[anchor=260] (n21) at (0.116488, 0.977285) {21};
% \node[anchor=260] (n22) at (0.211123, 0.970516) {22};
% \node (n23) at (-0.204433, 0.852693) {23};
% \node (n24) at (-0.573205, 0.690563) {24};
% \node[anchor=280] (n25) at (-0.475997, 0.871723) {25};
% \node[anchor=300] (n26) at (-0.795109, 0.595211) {26};
% \node (n27) at (-0.706031, -0.594383) {27};
% \node[anchor= 30] (n28) at (-0.871723, -0.475997) {28};
% \node[anchor= 60] (n29) at (-0.595211, -0.795109) {29};
% \node (n30) at (0.592349, -0.705907) {30};
% \node[anchor=120] (n31) at (0.475997, -0.871723) {31};
% \node[anchor=150] (n32) at (0.795109, -0.595211) {32};

\node (n1) at (0.800523, 0.062019) {3};
\node (n2) at (0.931003, 0.140216) {3};
\node (n3) at (0.910836, 0.232924) {3};
\node (n4) at (0.640446, 0.281189) {7};
\node (n5) at (0.138539, 0.124657) {3};
\node (n6) at (0.152558, -0.085147) {3};
\node (n7) at (0.363865, -0.075274) {3};
\node (n8) at (0.627263, -0.283950) {7};
\node (n9) at (0.231615, -0.789247) {3};
\node (n10) at (0.053369, -0.891434) {3};
\node (n11) at (-0.109343, -0.608463) {7};
\node (n12) at (-0.511642, -0.595120) {3};
\node (n13) at (-0.727638, -0.306583) {3};
\node (n14) at (-0.513635, 0.127724) {7};
\node[anchor=0] (n15) at (-0.970516, 0.211123) {3};
\node (n16) at (-0.933787, -0.213847) {3};
\node (n17) at (0.215101, -0.928496) {3};
\node[anchor=180] (n18) at (0.970516, -0.211123) {3};
\node (n19) at (0.306193, 0.726498) {7};
\node[anchor=240] (n20) at (0.595211, 0.795109) {3};
\node[anchor=260] (n21) at (0.116488, 0.977285) {3};
\node[anchor=260] (n22) at (0.211123, 0.970516) {3};
\node (n23) at (-0.204433, 0.852693) {3};
\node (n24) at (-0.573205, 0.690563) {7};
\node[anchor=280] (n25) at (-0.475997, 0.871723) {3};
\node[anchor=300] (n26) at (-0.795109, 0.595211) {3};
\node (n27) at (-0.706031, -0.594383) {7};
\node[anchor= 30] (n28) at (-0.871723, -0.475997) {3};
\node[anchor= 60] (n29) at (-0.595211, -0.795109) {3};
\node (n30) at (0.592349, -0.705907) {7};
\node[anchor=120] (n31) at (0.475997, -0.871723) {3};
\node[anchor=150] (n32) at (0.795109, -0.595211) {3};


\fill[black] (0.997452, 0.071339) circle (0.3pt);
\fill[black] (0.818411, 0.136744) circle (0.3pt);
\fill[black] (0.585705, -0.022025) circle (0.3pt);
\fill[black] (0.977147, 0.212565) circle (0.3pt);
\fill[black] (0.936950, 0.349464) circle (0.3pt);
\fill[black] (0.877679, 0.479249) circle (0.3pt);
\fill[black] (0.800541, 0.599278) circle (0.3pt);
\fill[black] (0.207721, 0.383537) circle (0.3pt);
\fill[black] (0.256114, 0.042078) circle (0.3pt);
\fill[black] (-0.048218, -0.051643) circle (0.3pt);
\fill[black] (0.249777, -0.245876) circle (0.3pt);
\fill[black] (0.160107, -0.679399) circle (0.3pt);
\fill[black] (0.463398, -0.690889) circle (0.3pt);
\fill[black] (0.936950, -0.349464) circle (0.3pt);
\fill[black] (0.997452, -0.071339) circle (0.3pt);
\fill[black] (0.071339, -0.997452) circle (0.3pt);
\fill[black] (-0.071339, -0.997452) circle (0.3pt);
\fill[black] (-0.493700, -0.370772) circle (0.3pt);
\fill[black] (-0.349464, -0.936950) circle (0.3pt);
\fill[black] (-0.212565, -0.977147) circle (0.3pt);
\fill[black] (-0.691761, -0.477638) circle (0.3pt);
\fill[black] (-0.997452, -0.071339) circle (0.3pt);
\fill[black] (-0.329395, 0.583481) circle (0.3pt);
\fill[black] (-0.936950, 0.349464) circle (0.3pt);
\fill[black] (-0.997452, 0.071339) circle (0.3pt);
\fill[black] (-0.977147, 0.212565) circle (0.3pt);
\fill[black] (-0.977147, -0.212565) circle (0.3pt);
\fill[black] (0.212565, -0.977147) circle (0.3pt);
\fill[black] (0.977147, -0.212565) circle (0.3pt);
\fill[black] (0.707107, 0.707107) circle (0.3pt);
\fill[black] (0.479249, 0.877679) circle (0.3pt);
\fill[black] (0.349464, 0.936950) circle (0.3pt);
\fill[black] (-0.071339, 0.997452) circle (0.3pt);
\fill[black] (0.599278, 0.800541) circle (0.3pt);
\fill[black] (0.071339, 0.997452) circle (0.3pt);
\fill[black] (0.212565, 0.977147) circle (0.3pt);
\fill[black] (-0.212565, 0.977147) circle (0.3pt);
\fill[black] (-0.349464, 0.936950) circle (0.3pt);
\fill[black] (-0.599278, 0.800541) circle (0.3pt);
\fill[black] (-0.707107, 0.707107) circle (0.3pt);
\fill[black] (-0.877679, 0.479249) circle (0.3pt);
\fill[black] (-0.479249, 0.877679) circle (0.3pt);
\fill[black] (-0.800541, 0.599278) circle (0.3pt);
\fill[black] (-0.936950, -0.349464) circle (0.3pt);
\fill[black] (-0.800541, -0.599278) circle (0.3pt);
\fill[black] (-0.707107, -0.707107) circle (0.3pt);
\fill[black] (-0.479249, -0.877679) circle (0.3pt);
\fill[black] (-0.877679, -0.479249) circle (0.3pt);
\fill[black] (-0.599278, -0.800541) circle (0.3pt);
\fill[black] (0.349464, -0.936950) circle (0.3pt);
\fill[black] (0.599278, -0.800541) circle (0.3pt);
\fill[black] (0.707107, -0.707107) circle (0.3pt);
\fill[black] (0.877679, -0.479249) circle (0.3pt);
\fill[black] (0.479249, -0.877679) circle (0.3pt);
\fill[black] (0.800541, -0.599278) circle (0.3pt);

\coordinate (x0) at (0.997452, 0.071339);
\coordinate (x1) at (0.818411, 0.136744);
\coordinate (x2) at (0.585705, -0.022025);
\coordinate (x3) at (0.977147, 0.212565);
\coordinate (x4) at (0.936950, 0.349464);
\coordinate (x5) at (0.877679, 0.479249);
\coordinate (x6) at (0.800541, 0.599278);
\coordinate (x7) at (0.207721, 0.383537);
\coordinate (x8) at (0.256114, 0.042078);
\coordinate (x9) at (-0.048218, -0.051643);
\coordinate (x10) at (0.249777, -0.245876);
\coordinate (x11) at (0.160107, -0.679399);
\coordinate (x12) at (0.463398, -0.690889);
\coordinate (x13) at (0.936950, -0.349464);
\coordinate (x14) at (0.997452, -0.071339);
\coordinate (x15) at (0.071339, -0.997452);
\coordinate (x16) at (-0.071339, -0.997452);
\coordinate (x17) at (-0.493700, -0.370772);
\coordinate (x18) at (-0.349464, -0.936950);
\coordinate (x19) at (-0.212565, -0.977147);
\coordinate (x20) at (-0.691761, -0.477638);
\coordinate (x21) at (-0.997452, -0.071339);
\coordinate (x22) at (-0.329395, 0.583481);
\coordinate (x23) at (-0.936950, 0.349464);
\coordinate (x24) at (-0.997452, 0.071339);
\coordinate (x25) at (-0.977147, 0.212565);
\coordinate (x26) at (-0.977147, -0.212565);
\coordinate (x27) at (0.212565, -0.977147);
\coordinate (x28) at (0.977147, -0.212565);
\coordinate (x29) at (0.707107, 0.707107);
\coordinate (x30) at (0.479249, 0.877679);
\coordinate (x31) at (0.349464, 0.936950);
\coordinate (x32) at (-0.071339, 0.997452);
\coordinate (x33) at (0.599278, 0.800541);
\coordinate (x34) at (0.071339, 0.997452);
\coordinate (x35) at (0.212565, 0.977147);
\coordinate (x36) at (-0.212565, 0.977147);
\coordinate (x37) at (-0.349464, 0.936950);
\coordinate (x38) at (-0.599278, 0.800541);
\coordinate (x39) at (-0.707107, 0.707107);
\coordinate (x40) at (-0.877679, 0.479249);
\coordinate (x41) at (-0.479249, 0.877679);
\coordinate (x42) at (-0.800541, 0.599278);
\coordinate (x43) at (-0.936950, -0.349464);
\coordinate (x44) at (-0.800541, -0.599278);
\coordinate (x45) at (-0.707107, -0.707107);
\coordinate (x46) at (-0.479249, -0.877679);
\coordinate (x47) at (-0.877679, -0.479249);
\coordinate (x48) at (-0.599278, -0.800541);
\coordinate (x49) at (0.349464, -0.936950);
\coordinate (x50) at (0.599278, -0.800541);
\coordinate (x51) at (0.707107, -0.707107);
\coordinate (x52) at (0.877679, -0.479249);
\coordinate (x53) at (0.479249, -0.877679);
\coordinate (x54) at (0.800541, -0.599278);


\draw (x0) -- (x1);
\draw (x1) -- (x2);
\draw (x2) -- (x0);
\draw (x0) -- (x3);
\draw (x3) -- (x1);
\draw (x3) -- (x4);
\draw (x4) -- (x1);
\draw (x4) -- (x5);
\draw (x5) -- (x6);
\draw (x6) -- (x7);
\draw (x7) -- (x8);
\draw (x8) -- (x2);
\draw (x7) -- (x9);
\draw[lsquare] (x9) -- (x8);
\draw[lface] (x8) -- (n4);
\draw[lface] (x9) -- (n11);
\draw (x9) -- (x10);
\draw (x10) -- (x8);
\draw (x10) -- (x2);
\draw (x10) -- (x11);
\draw (x11) -- (x12);
\draw[ldiamond] (x12) -- (x13) node[midway] (m1) {};
\draw[lface] (m1) -- (n8);
\draw[lface] (m1) -- (n30);
\draw (x13) -- (x14);
\draw (x14) -- (x0);
\draw (x11) -- (x15);
\draw (x15) -- (x12);
\draw (x11) -- (x16);
\draw (x16) -- (x15);
\draw (x9) -- (x17);
\draw (x17) -- (x18);
\draw (x18) -- (x19);
\draw (x19) -- (x16);
\draw (x17) -- (x20);
\draw (x20) -- (x18);
\draw (x17) -- (x21);
\draw[lsquare] (x21) -- (x20);
\draw[lface] (x21) -- (n14);
\draw[lface] (x20) -- (n27);
\draw (x7) -- (x22);
\draw (x22) -- (x23);
\draw (x23) -- (x24);
\draw (x24) -- (x21);
\draw (x23) -- (x25);
\draw (x25) -- (x24);
\draw (x21) -- (x26);
\draw (x26) -- (x20);
\draw (x15) -- (x27);
\draw (x27) -- (x12);
\draw (x13) -- (x28);
\draw (x28) -- (x14);
\draw (x6) -- (x29);
\draw (x29) -- (x30);
\draw (x30) -- (x31);
\draw (x31) -- (x32);
\draw (x32) -- (x22);
\draw (x29) -- (x33);
\draw (x33) -- (x30);
\draw (x31) -- (x34);
\draw (x34) -- (x32);
\draw (x31) -- (x35);
\draw (x35) -- (x34);
\draw (x32) -- (x36);
\draw (x36) -- (x22);
\draw (x36) -- (x37);
\draw (x37) -- (x38);
\draw (x38) -- (x39);
\draw (x39) -- (x40);
\draw (x40) -- (x23);
\draw (x37) -- (x41);
\draw (x41) -- (x38);
\draw (x39) -- (x42);
\draw (x42) -- (x40);
\draw (x26) -- (x43);
\draw (x43) -- (x44);
\draw (x44) -- (x45);
\draw (x45) -- (x46);
\draw (x46) -- (x18);
\draw (x43) -- (x47);
\draw (x47) -- (x44);
\draw (x45) -- (x48);
\draw (x48) -- (x46);
\draw (x27) -- (x49);
\draw (x49) -- (x50);
\draw (x50) -- (x51);
\draw (x51) -- (x52);
\draw (x52) -- (x13);
\draw (x49) -- (x53);
\draw (x53) -- (x50);
\draw (x51) -- (x54);
\draw (x54) -- (x52);

\node[lvertex] at (x22) {};
\draw[lface] (x22) -- (n19);
\draw[lface] (x22) -- (n24);

    \end{scope}
  \end{tikzsubfigure}
\end{tikzfigure2}
