\section{$4$-valent {\sc Eberhard}-like theorems with triangles}\label{sec:3:4}

In this section we want to prove $4$-valent {\sc Eberhard}-like theorems with triangles, i.e. for $q = [q_3 \times 3, q_l \times l]$, $w = [w_4 \times 4]$, $l > 4$, $\gcd(q_3, q_l) = 1$. As it turns out in \autoref{sec:negative:results}, this is only possible for all admissible pairs of sequences on all closed orientable $2$-manifolds, if $3 \nmid l$. For all the proofs, we want to use \autoref{thm:main:const}, therefore we need to have a construction scheme for patches with arbitrary large $l$-gons. These we get by the next two constructions:

\begin{construction}\label{const:edge:replacement:3:4:1}
  \begin{cinput}
  \item A $4$-patch with $p$-vector $p$ and a specified edge with exactly one vertex incident to some $k_1$-gon and the other vertex  incident to some $k_2$-gon.
  \end{cinput}
  \begin{coutput}
  \item A new $4$-patch with $p$-vector $p - [k_1, k_2] + [2k \times 3] + [k_1 + k, k_2 + k]$ for all $k \in \nats$.
  \item If every two faces of the $3$-patch meet proper, then this is carried over to the new patch.
  \end{coutput}
  \begin{cdescription}
    Using the replacement of the single edge as seen in \autoref{fig:const:edge:replacement:3:4:1} results in a new map with $p$-vector $p - [k_1, k_2] + [2 \times 3] + [k_1 + 1, k_2 + 1]$. The thick line on the left is the specified edge and the thick line on the right is a new edge which we can use to repeat the construction. Every time we use this construction we add two new triangles while increasing the number of vertices of the left and right polygon by one; doing this $k$ times has the desired result, since we only inserted $4$-valent vertices.
    \begin{tikzfigure}{\label{fig:const:edge:replacement:3:4:1}}{}
      \matrix (m) [column sep=1cm] {
        \begin{scope}
          \draw[very thick] (-1, 0) -- (1, 0);
          \draw (-1.2, 0.5) -- (-1, 0) -- (-1.2, -0.5);
          \draw (1.2, 0.5) -- (1, 0) -- (1.2, -0.5);
          \draw (-1, 0) -- (-0.8, -0.5);
          \draw (1, 0) -- (0.8, 0.5);
          \node at (-1.5, 0) {$k_1$};
          \node at (1.5, 0) {$k_2$};
        \end{scope}
        &
        \begin{scope}
          \draw[very thick] (-1, 0.5) -- (1, 0.5);
          \draw (-1, -0.5) -- (1, -0.5);
          \draw (-1.2, 1) -- (-1, 0.5) -- (-1, -0.5) -- (-1.2, -1);
          \draw (1.2, 1) -- (1, 0.5) -- (1, -0.5) -- (1.2, -1);
          \draw (-1, 0.5) -- (1, -0.5);
          \draw (-1, -0.5) -- (-0.8, -1);
          \draw (1, 0.5) -- (0.8, 1);
          \node at (-1.5, 0) {$k_1$};
          \node at (1.5, 0) {$k_2$};
        \end{scope}
        \\
      };
    \end{tikzfigure}
  \end{cdescription}
\end{construction}

\begin{construction}\label{const:edge:replacement:3:4:2}
  \begin{cinput}
  \item A $4$-patch with $p$-vector $p$ and a specified path of length $2$ with the start vertex incident to some $k_1$-gon and the end vertex incident to some $k_2$-gon, while the vertex in the middle is incident to neither polygon.
  \end{cinput}
  \begin{coutput}
  \item A new $4$-patch with $p$-vector $p - [k_1, k_2] + [(6k) \times 3] + [k_1 + 3k, k_2 + 3k]$ for all $k \in \nats$.
  \end{coutput}
  \begin{cdescription}
    Using the replacement of the path as seen in \autoref{fig:const:edge:replacement:3:4:2} results in a new map with $p$-vector $p - [k_1, k_2] + [(6k) \times 3] + [k_1 + 3, k_2 + 3]$. The thick drawn path on the left is the specified path and the thick drawn path on the right is a new path which we can use to repeat the construction. Every time we use this construction we add six new triangles while increasing the number of vertices of the left and right polygon by three; doing this $k$ times has the desired result, since we only inserted $4$-valent vertices.
    \begin{tikzfigure}{\label{fig:const:edge:replacement:3:4:2}}{}
      \matrix (m) [column sep=1cm] {
        \begin{scope}
          \draw[very thick] (0, -1) -- (0, 1);
          \draw (0.5, -1.2) -- (0, -1) -- (-0.5, -1.2);
          \draw (0.5, 1.2) -- (0, 1) -- (-0.5, 1.2);
          \node at (-1, 0) {$k_1$};
          \node at (1, 0) {$k_2$};
        \end{scope}
        &
        \begin{scope}
          \draw[very thick] (0, -1) -- (0, 1);
          \draw (0.5, -1.2) -- (0, -1) -- (-0.5, -1.2);
          \draw (0.5, 1.2) -- (0, 1) -- (-0.5, 1.2);
          
          \draw (-0.5, 0) -- (0, -0.6) -- (0.5, 0);
          \draw (-0.5, 0) -- (0, -0.2) -- (0.5, 0);
          \draw (-0.5, 0) -- (0, 0.6) -- (0.5, 0);
          \draw (-0.5, 0) -- (0, 0.2) -- (0.5, 0);
          

          \node[anchor=east] at (-2, 0) {$k_1 + 3$};
          \node[anchor=west] at (2, 0) {$k_2 + 3$};
        \end{scope}
        \\
      };
    \end{tikzfigure}  
  \end{cdescription}
\end{construction}

We now can split the infinitely many cases in two series, each with a different residual class of $l \mod 3$:

\begin{theorem}
  Let $p$ and $v$ be a pair of admissible sequences for a orientable closed $2$-manifold $S$. Then $(p, v)$ is $[(3k + 1) \times 3, (3k+5)]$-$[4]$-realizable for all $k \in \nats$.
  \begin{proof}
    An expansion $4$-patch with outer tuple $o = (1, 2, 1, 3, 2, 3)$, a corresponding $o$-$4$-gonal $4$-patch and a expansion $4$-patch with the polyhedral property, all consisting only of triangles and pentagons, are shown in \autoref{fig:expansion:patch:3:5:4}. Using \autoref{const:edge:replacement:3:4:1} and \autoref{const:edge:replacement:3:4:2} on the edges as indicated in all three we get new $4$-patches consisting only of triangles and $3k + 5$-gons. Therefore we can apply \autoref{thm:main:const}.
    \begin{tikzfigure}{\label{fig:expansion:patch:3:5:4}}{}
      \matrix (m) [column sep=1cm, row sep=1cm] {
        \begin{scope}[yscale=0.866]
          \draw (0.5,1)--(-0.5,1)--(-1.5,3)--(-4,6)--(-3.5,7)--(-2.5,7)--(-1,6.5)--(-0.5,7)--(0.5,7)--(1,7.5)--(2.5,7)--(1.5,3)--(0.5,1);
          \draw (-1.5,3)--(-0.5,4.5)--(0.5,3)--(0.5,1); %1
          \draw (-4,6)--(-1,6)--(-1,6.5); %8
          \draw[lsquare] (-1.5,3)--(-1,6); %4
          \draw (-0.5,4.5) --(-1,6); %3
          \draw (-0.5,4.5) --(0.5,7); %7
          \draw[lsquare] (0.5,3) --(0.5,7); %5
          \draw (0.5,3) --(1,7.5); %2

          \draw[lface] (-1,6)--(-2,6.5);
          \draw[lface] (-1.5,3)--(-0.5,2.5);
          \draw[lface] (0.5,3)--(1.5,3.5);
          \draw[lface] (0.5,7)--(-0.25,6.25);
          
          \node[anchor= 90] at (0.5,1)  {$i_{0}'$};
          \node[anchor= 90] at (-0.5,1) {$i_0$};
          \node[anchor=  0] at (-1.5,3) {$i_1$};
          \node[anchor= 30] at (-4,6)   {$i_2=o_0$};
          \node[anchor=300] at (-3.5,7) {$o_1$};
          \node[anchor=270] at (-2.5,7) {$o_2$};
          \node[anchor=315] at (-1,6.5) {$o_3$};
          \node[anchor=270] at (-0.5,7) {$o_4$};
          \node[anchor=270] at (0.5,7)  {$o_5$};
          \node[anchor=270] at (1,7.5)  {$o_6$};
          \node[anchor=240] at (2.5,7)  {$i_2'=o_7$};
          \node[anchor=180] at (1.5,3)  {$i_1'$};
       
          
        \end{scope}
        &
        \begin{scope}[scale=0.6]
        \begin{scope}[yscale=0.866]
          \draw[very thick] (0.5,1)--(-0.5,1)--(-1.5,3)--(-4,6)--(-3.5,7)--(-2.5,7)--(-1,6.5)--(-0.5,7)--(0.5,7)--(1,7.5)--(2.5,7)--(1.5,3)--(0.5,1);
          \draw (-1.5,3)--(-0.5,4.5)--(0.5,3)--(0.5,1); %1
          \draw (-4,6)--(-1,6)--(-1,6.5); %8
          \draw (-1.5,3)--(-1,6); %4
          \draw (-0.5,4.5) --(-1,6); %3
          \draw (-0.5,4.5) --(0.5,7); %7
          \draw (0.5,3) --(0.5,7); %5
          \draw (0.5,3) --(1,7.5); %2
        \end{scope}
        \begin{scope}[rotate=-60, yscale=0.866]
          \draw[very thick] (0.5,1)--(-0.5,1)--(-1.5,3)--(-4,6)--(-3.5,7)--(-2.5,7)--(-1,6.5)--(-0.5,7)--(0.5,7)--(1,7.5)--(2.5,7)--(1.5,3)--(0.5,1);
          \draw (-1.5,3)--(-0.5,4.5)--(0.5,3)--(0.5,1); %1
          \draw (-4,6)--(-1,6)--(-1,6.5); %8
          \draw (-1.5,3)--(-1,6); %4
          \draw (-0.5,4.5) --(-1,6); %3
          \draw (-0.5,4.5) --(0.5,7); %7
          \draw (0.5,3) --(0.5,7); %5
          \draw (0.5,3) --(1,7.5); %2
          \end{scope}
          \begin{scope}[yscale=0.866,shift={(0 cm,14 cm)},rotate=180]
            \draw[very thick] (0.5,1)--(-0.5,1)--(-1.5,3)--(-4,6)--(-3.5,7)--(-2.5,7)--(-1,6.5)--(-0.5,7)--(0.5,7)--(1,7.5)--(2.5,7)--(1.5,3)--(0.5,1);
            \draw (-1.5,3)--(-0.5,4.5)--(0.5,3)--(0.5,1); %1
          \draw (-4,6)--(-1,6)--(-1,6.5); %8
          \draw (-1.5,3)--(-1,6); %4
          \draw (-0.5,4.5) --(-1,6); %3
          \draw (-0.5,4.5) --(0.5,7); %7
          \draw (0.5,3) --(0.5,7); %5
          \draw (0.5,3) --(1,7.5); %2
          \end{scope}
          \begin{scope}[shift={(0 cm,12.124 cm)},rotate=120,yscale=0.866]
   \draw[very thick] (0.5,1)--(-0.5,1)--(-1.5,3)--(-4,6)--(-3.5,7)--(-2.5,7)--(-1,6.5)--(-0.5,7)--(0.5,7)--(1,7.5)--(2.5,7)--(1.5,3)--(0.5,1);
            \draw (-1.5,3)--(-0.5,4.5)--(0.5,3)--(0.5,1); %1
          \draw (-4,6)--(-1,6)--(-1,6.5); %8
          \draw (-1.5,3)--(-1,6); %4
          \draw (-0.5,4.5) --(-1,6); %3
          \draw (-0.5,4.5) --(0.5,7); %7
          \draw (0.5,3) --(0.5,7); %5
          \draw (0.5,3) --(1,7.5); %2
    \draw (0.5,1)--(-0.5,1)--(-1.5,3)--(-4,6)--(-3.5,7)--(-2.5,7)--(-1,6.5)--(-0.5,7)--(0.5,7)--(1,7.5)--(2.5,7)--(1.5,3)--(0.5,1);
          \end{scope}
          \end{scope}
        %\begin{scope}[scale=3, yshift=25]
        %  \coordinate (x0) at (0.232399, -0.396938);
\coordinate (x1) at (0.493399, -0.204630);
\coordinate (x2) at (0.754551, -0.012194);
\coordinate (x3) at (0.433937, 0.043731);
\coordinate (x4) at (0.222521, -0.974928);
\coordinate (x5) at (0.900969, -0.433884);
\coordinate (x6) at (-0.099991, -0.073589);
\coordinate (x7) at (0.900969, 0.433884);
\coordinate (x8) at (0.222521, 0.974928);
\coordinate (x9) at (-0.623490, 0.781831);
\coordinate (x10) at (-1.000000, 0.000000);
\coordinate (x11) at (-0.623490, -0.781831);
\draw (x0) -- (x1);
\draw (x1) -- (x2);
\draw (x2) -- (x3);
\draw (x3) -- (x0);
\draw (x0) -- (x4);
\draw (x4) -- (x5);
\draw (x5) -- (x2);
\draw (x3) -- (x6);
\draw[very thick] (x6) -- (x0);
\draw (x6) -- (x4);
\draw (x5) -- (x7);
\draw[very thick] (x7) -- (x2);
\draw (x7) -- (x3);
\draw (x7) -- (x8);
\draw (x8) -- (x9);
\draw (x9) -- (x6);
\draw (x9) -- (x10);
\draw (x10) -- (x11);
\draw (x11) -- (x4);
\node at (-0.000000, -0.000000) {0};
\node at (0.478571, -0.142508) {1};
\node at (0.520768, -0.404515) {2};
\node at (0.188782, -0.142266) {3};
\node at (0.118310, -0.481818) {4};
\node at (0.852163, -0.004065) {5};
\node at (0.696485, 0.155140) {6};
\node at (0.166789, 0.432157) {7};
\node at (-0.424890, -0.209703) {8};

        %\end{scope}
        
        %\begin{scope}[scale=3, yshift=25]
        %  \coordinate (x0) at (-0.145531, 0.231610);
\coordinate (x1) at (-0.063274, -0.419927);
\coordinate (x2) at (0.405804, -0.125166);
\coordinate (x3) at (-0.993712, 0.111964);
\coordinate (x4) at (-0.993712, -0.111964);
\coordinate (x5) at (-0.330279, -0.943883);
\coordinate (x6) at (-0.111964, -0.993712);
\coordinate (x7) at (0.111964, -0.993712);
\coordinate (x8) at (0.943883, -0.330279);
\coordinate (x9) at (0.993712, -0.111964);
\coordinate (x10) at (0.993712, 0.111964);
\coordinate (x11) at (0.330279, 0.943883);
\coordinate (x12) at (0.111964, 0.993712);
\coordinate (x13) at (-0.943883, 0.330279);
\coordinate (x14) at (-0.111964, 0.993712);
\coordinate (x15) at (-0.532032, 0.846724);
\coordinate (x16) at (-0.707107, 0.707107);
\coordinate (x17) at (-0.846724, 0.532032);
\coordinate (x18) at (-0.330279, 0.943883);
\coordinate (x19) at (-0.846724, -0.532032);
\coordinate (x20) at (-0.707107, -0.707107);
\coordinate (x21) at (-0.532032, -0.846724);
\coordinate (x22) at (-0.943883, -0.330279);
\coordinate (x23) at (0.532032, -0.846724);
\coordinate (x24) at (0.707107, -0.707107);
\coordinate (x25) at (0.846724, -0.532032);
\coordinate (x26) at (0.330279, -0.943883);
\coordinate (x27) at (0.846724, 0.532032);
\coordinate (x28) at (0.707107, 0.707107);
\coordinate (x29) at (0.532032, 0.846724);
\coordinate (x30) at (0.943883, 0.330279);
\draw (x0) -- (x1);
\draw (x1) -- (x2);
\draw (x2) -- (x0);
\draw (x0) -- (x3);
\draw (x3) -- (x4);
\draw[very thick] (x4) -- (x5);
\draw (x5) -- (x1);
\draw (x5) -- (x6);
\draw (x6) -- (x1);
\draw (x6) -- (x7);
\draw[very thick] (x7) -- (x8);
\draw (x8) -- (x2);
\draw (x8) -- (x9);
\draw (x9) -- (x2);
\draw (x9) -- (x10);
\draw[very thick] (x10) -- (x11);
\draw (x11) -- (x0);
\draw (x11) -- (x12);
\draw[very thick] (x12) -- (x13);
\draw[very thick] (x13) -- (x3);
\draw (x12) -- (x14);
\draw[very thick] (x14) -- (x13);
\draw (x14) -- (x15);
\draw (x15) -- (x16);
\draw (x16) -- (x17);
\draw[very thick] (x17) -- (x13);
\draw (x14) -- (x18);
\draw (x18) -- (x15);
\draw (x4) -- (x19);
\draw (x19) -- (x20);
\draw (x20) -- (x21);
\draw (x21) -- (x5);
\draw (x4) -- (x22);
\draw (x22) -- (x19);
\draw (x7) -- (x23);
\draw (x23) -- (x24);
\draw (x24) -- (x25);
\draw (x25) -- (x8);
\draw (x7) -- (x26);
\draw (x26) -- (x23);
\draw (x10) -- (x27);
\draw (x27) -- (x28);
\draw (x28) -- (x29);
\draw (x29) -- (x11);
\draw (x10) -- (x30);
\draw (x30) -- (x27);
\node at (-0.298232, 0.087497) {0};
\node at (0.065666, -0.104494) {1};
\node at (-0.505302, -0.226440) {2};
\node at (-0.168506, -0.785841) {3};
\node at (0.257283, -0.572559) {4};
\node at (0.781133, -0.189136) {5};
\node at (0.515595, 0.210065) {6};
\node at (-0.328177, 0.522290) {7};
\node at (-0.314628, 0.772568) {8};
\node at (-0.628342, 0.681971) {9};
\node at (-0.324759, 0.928107) {10};
\node at (-0.681971, -0.628342) {11};
\node at (-0.928107, -0.324759) {12};
\node at (0.628342, -0.681971) {13};
\node at (0.324759, -0.928107) {14};
\node at (0.681971, 0.628342) {15};
\node at (0.928107, 0.324759) {16};

        %\end{scope}
        \\
      };
    \end{tikzfigure}
  \end{proof}
\end{theorem}

\begin{theorem}
  Let $p$ and $v$ be a pair of admissible sequences for a orientable closed $2$-manifold $S$. Then $(p, v)$ is $[(3k + 3) \times 5, (3k+7)]$-$[4]$-realizable for all $k \in \nats$.
  \begin{proof}
    An expansion $4$-patch with outer tuple $o = (1, 0, 1, 1, 2, 1)$, a corresponding $o$-$4$-gonal $4$-patch and a expansion $4$-patch with the polyhedral property, all consisting only of triangles and heptagons, are shown in \autoref{fig:expansion:patch:3:4:7}. Using \autoref{const:edge:replacement:3:4:1} and \autoref{const:edge:replacement:3:4:2} on the edges as indicated in all three we get new $4$-patches consisting only of triangles and $3k + 7$-gons. Therefore we can apply \autoref{thm:main:const}.
    \begin{tikzfigure}{\label{fig:expansion:patch:3:4:7}}{}
      \matrix (m) [column sep=1cm] {
        \begin{scope}[scale=3, yshift=25]
          \documentclass[a4paper]{article}
\usepackage[left=0.5cm, right=0.5cm, top=1cm, bottom=1cm]{geometry}
\usepackage{tikz}
\begin{document}
\begin{figure}[h!]
\centering
\begin{tikzpicture}[scale = 10, auto]
\coordinate (x0) at (0.275697, 0.629962);
\coordinate (x1) at (0.426590, 0.581452);
\coordinate (x2) at (0.576706, 0.535472);
\coordinate (x3) at (0.900969, 0.433884);
\coordinate (x4) at (0.222521, 0.974928);
\coordinate (x5) at (-0.048782, 0.732065);
\coordinate (x6) at (0.426122, 0.582941);
\coordinate (x7) at (-0.623490, 0.781831);
\coordinate (x8) at (-1.000000, 0.000000);
\coordinate (x9) at (-0.623490, -0.781831);
\coordinate (x10) at (0.900969, -0.433884);
\coordinate (x11) at (0.222521, -0.974928);
\draw (0.275697, 0.629962) -- (0.426590, 0.581452);
\draw (0.426590, 0.581452) -- (0.576706, 0.535472);
\draw (0.576706, 0.535472) -- (0.900969, 0.433884);
\draw (0.900969, 0.433884) -- (0.222521, 0.974928);
\draw (0.222521, 0.974928) -- (-0.048782, 0.732065);
\draw (-0.048782, 0.732065) -- (0.275697, 0.629962);
\draw (0.275697, 0.629962) -- (0.576706, 0.535472);
\draw (-0.048782, 0.732065) -- (0.426122, 0.582941);
\draw (0.426122, 0.582941) -- (0.275697, 0.629962);
\draw (0.426122, 0.582941) -- (0.576706, 0.535472);
\draw (0.426122, 0.582941) -- (0.900969, 0.433884);
\draw (0.222521, 0.974928) -- (-0.623490, 0.781831);
\draw (-0.623490, 0.781831) -- (-0.048782, 0.732065);
\draw (-0.623490, 0.781831) -- (-1.000000, 0.000000);
\draw (-1.000000, 0.000000) -- (-0.623490, -0.781831);
\draw (-0.623490, -0.781831) -- (0.900969, -0.433884);
\draw (0.900969, -0.433884) -- (0.900969, 0.433884);
\draw (-0.623490, -0.781831) -- (0.222521, -0.974928);
\draw (0.222521, -0.974928) -- (0.900969, -0.433884);
\node at (-0.000000, -0.000000) {0};
\node at (0.392283, 0.647960) {1};
\node at (0.426331, 0.582295) {2};
\node at (0.217679, 0.648323) {3};
\node at (0.426175, 0.582792) {4};
\node at (0.634599, 0.517432) {5};
\node at (-0.149917, 0.829608) {6};
\node at (-0.009672, 0.187858) {7};
\node at (0.166667, -0.730214) {8};
\end{tikzpicture}
\caption{Graph}

\end{figure}
\end{document}

        \end{scope}
        &
        \begin{scope}[scale=3, yshift=25]
          \coordinate (x0) at (-0.040973, 0.504900);
\coordinate (x1) at (-0.306524, 0.637361);
\coordinate (x2) at (-0.566100, 0.266443);
\coordinate (x3) at (-0.111964, 0.993712);
\coordinate (x4) at (-0.330279, 0.943883);
\coordinate (x5) at (-0.532032, 0.846724);
\coordinate (x6) at (-0.846724, 0.532032);
\coordinate (x7) at (-0.943883, 0.330279);
\coordinate (x8) at (-0.993712, 0.111964);
\coordinate (x9) at (-0.707107, 0.707107);
\coordinate (x10) at (-0.993712, -0.111964);
\coordinate (x11) at (-0.441672, -0.313643);
\coordinate (x12) at (0.296604, -0.348829);
\coordinate (x13) at (0.993712, 0.111964);
\coordinate (x14) at (0.547325, 0.363829);
\coordinate (x15) at (0.943883, 0.330279);
\coordinate (x16) at (0.846724, 0.532032);
\coordinate (x17) at (0.707107, 0.707107);
\coordinate (x18) at (0.532032, 0.846724);
\coordinate (x19) at (0.111964, 0.993712);
\coordinate (x20) at (0.330279, 0.943883);
\coordinate (x21) at (-0.943883, -0.330279);
\coordinate (x22) at (-0.846724, -0.532032);
\coordinate (x23) at (-0.707107, -0.707107);
\coordinate (x24) at (-0.532032, -0.846724);
\coordinate (x25) at (-0.111964, -0.993712);
\coordinate (x26) at (0.111964, -0.993712);
\coordinate (x27) at (-0.330279, -0.943883);
\coordinate (x28) at (0.293726, -0.719159);
\coordinate (x29) at (0.330279, -0.943883);
\coordinate (x30) at (0.532032, -0.846724);
\coordinate (x31) at (0.707107, -0.707107);
\coordinate (x32) at (0.846724, -0.532032);
\coordinate (x33) at (0.993712, -0.111964);
\coordinate (x34) at (0.943883, -0.330279);
\draw (-0.040973, 0.504900) -- (-0.306524, 0.637361);
\draw (-0.306524, 0.637361) -- (-0.566100, 0.266443);
\draw (-0.566100, 0.266443) -- (-0.040973, 0.504900);
\draw (-0.040973, 0.504900) -- (-0.111964, 0.993712);
\draw (-0.111964, 0.993712) -- (-0.306524, 0.637361);
\draw (-0.111964, 0.993712) -- (-0.330279, 0.943883);
\draw (-0.330279, 0.943883) -- (-0.306524, 0.637361);
\draw (-0.330279, 0.943883) -- (-0.532032, 0.846724);
\draw (-0.532032, 0.846724) -- (-0.846724, 0.532032);
\draw (-0.846724, 0.532032) -- (-0.943883, 0.330279);
\draw (-0.943883, 0.330279) -- (-0.993712, 0.111964);
\draw (-0.993712, 0.111964) -- (-0.566100, 0.266443);
\draw (-0.532032, 0.846724) -- (-0.707107, 0.707107);
\draw (-0.707107, 0.707107) -- (-0.846724, 0.532032);
\draw (-0.993712, 0.111964) -- (-0.993712, -0.111964);
\draw (-0.993712, -0.111964) -- (-0.566100, 0.266443);
\draw (-0.993712, -0.111964) -- (-0.441672, -0.313643);
\draw (-0.441672, -0.313643) -- (0.296604, -0.348829);
\draw (0.296604, -0.348829) -- (0.993712, 0.111964);
\draw (0.993712, 0.111964) -- (0.547325, 0.363829);
\draw (0.547325, 0.363829) -- (-0.040973, 0.504900);
\draw (0.993712, 0.111964) -- (0.943883, 0.330279);
\draw (0.943883, 0.330279) -- (0.547325, 0.363829);
\draw (0.943883, 0.330279) -- (0.846724, 0.532032);
\draw (0.846724, 0.532032) -- (0.547325, 0.363829);
\draw (0.846724, 0.532032) -- (0.707107, 0.707107);
\draw (0.707107, 0.707107) -- (0.532032, 0.846724);
\draw (0.532032, 0.846724) -- (0.111964, 0.993712);
\draw (0.111964, 0.993712) -- (-0.111964, 0.993712);
\draw (0.532032, 0.846724) -- (0.330279, 0.943883);
\draw (0.330279, 0.943883) -- (0.111964, 0.993712);
\draw (-0.993712, -0.111964) -- (-0.943883, -0.330279);
\draw (-0.943883, -0.330279) -- (-0.441672, -0.313643);
\draw (-0.943883, -0.330279) -- (-0.846724, -0.532032);
\draw (-0.846724, -0.532032) -- (-0.441672, -0.313643);
\draw (-0.846724, -0.532032) -- (-0.707107, -0.707107);
\draw (-0.707107, -0.707107) -- (-0.532032, -0.846724);
\draw (-0.532032, -0.846724) -- (-0.111964, -0.993712);
\draw (-0.111964, -0.993712) -- (0.111964, -0.993712);
\draw (0.111964, -0.993712) -- (0.296604, -0.348829);
\draw (-0.532032, -0.846724) -- (-0.330279, -0.943883);
\draw (-0.330279, -0.943883) -- (-0.111964, -0.993712);
\draw (0.111964, -0.993712) -- (0.293726, -0.719159);
\draw (0.293726, -0.719159) -- (0.296604, -0.348829);
\draw (0.111964, -0.993712) -- (0.330279, -0.943883);
\draw (0.330279, -0.943883) -- (0.293726, -0.719159);
\draw (0.330279, -0.943883) -- (0.532032, -0.846724);
\draw (0.532032, -0.846724) -- (0.293726, -0.719159);
\draw (0.532032, -0.846724) -- (0.707107, -0.707107);
\draw (0.707107, -0.707107) -- (0.846724, -0.532032);
\draw (0.846724, -0.532032) -- (0.993712, -0.111964);
\draw (0.993712, -0.111964) -- (0.993712, 0.111964);
\draw (0.846724, -0.532032) -- (0.943883, -0.330279);
\draw (0.943883, -0.330279) -- (0.993712, -0.111964);
\node at (-0.178613, 0.510448) {0};
\node at (-0.304533, 0.469568) {1};
\node at (-0.153154, 0.711991) {2};
\node at (-0.249589, 0.858319) {3};
\node at (-0.645608, 0.524098) {4};
\node at (-0.695288, 0.695288) {5};
\node at (-0.851175, 0.088814) {6};
\node at (-0.029260, 0.067528) {7};
\node at (0.828307, 0.268691) {8};
\node at (0.779311, 0.408713) {9};
\node at (0.370316, 0.706002) {10};
\node at (0.324759, 0.928107) {11};
\node at (-0.793089, -0.251962) {12};
\node at (-0.744093, -0.391985) {13};
\node at (-0.318704, -0.676537) {14};
\node at (-0.324759, -0.928107) {15};
\node at (0.234098, -0.687234) {16};
\node at (0.245323, -0.885585) {17};
\node at (0.385346, -0.836589) {18};
\node at (0.666231, -0.450550) {19};
\node at (0.928107, -0.324759) {20};

        \end{scope}
        \\
      };
    \end{tikzfigure}
  \end{proof}
\end{theorem}