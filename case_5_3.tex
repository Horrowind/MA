\section{$3$-valent {\sc Eberhard}-like theorems with pentagons}\label{sec:5:3}

\begin{construction}\label{const:edge:replacement:5:1}
  \begin{cinput}
  \item A $3$-patch with $p$-vector $p$ and a specified edge with exactly one vertex incident to some $k_1$-gon and the other vertex  incident to some $k_2$-gon.
  \end{cinput}
  \begin{coutput}
  \item A new $3$-patch with $p$-vector $p - [k_1, k_2] + [10k \times 5] + [k_1 + 5k, k_2 + 5k]$ for all $k \in \nats$.
  \end{coutput}
  \begin{cdescription}
    Using the replacement of the single edge as seen in \autoref{fig:const:edge:replacement:5:1} results in a new map with $p$-vector $p - [k_1, k_2] + [10 \times 5] + [k_1 + 5, k_2 + 5]$. The thick line on the left is the specified line and the thick line on the right is a new line which we can use to repeat the construction. Every time we use this construction we add ten new pentagons while increasing the number of vertices of the left and right polygon by 5; doing this $k$ times has the desired result, since we only inserted $3$-valent vertices.
    \begin{tikzfigure}{\label{fig:const:edge:replacement:5:1}}{}
      \matrix (m) [column sep=1cm] {
        \begin{scope}
          \draw[very thick] (-1, 0) -- (1, 0);
          \draw (-1.2, 0.5) -- (-1, 0) -- (-1.2, -0.5);
          \draw (1.2, 0.5) -- (1, 0) -- (1.2, -0.5);
          \node at (-1.5, 0) {$k_1$};
          \node at (1.5, 0) {$k_2$};
        \end{scope}
        &
        \begin{scope}
          \draw[very thick] (-1, 2.5) -- (1, 2.5);
          \draw (-1, -2.5) -- (1, -2.5);
          \draw (-1.2, 3) -- (-1, 2.5);
          \draw (-1.2, -3) -- (-1, -2.5);
          \draw (1.2, 3) -- (1, 2.5);
          \draw (1.2, -3) -- (1, -2.5);
          \draw (1, 2.5) -- (0.9, 1.5) -- (0.85, 0.5) -- (0.85, -0.5) -- (0.9, -1.5) -- (1, -2.5);
          \draw (-1, 2.5) -- (-0.9, 1.5) -- (-0.85, 0.5) -- (-0.85, -0.5) -- (-0.9, -1.5) -- (-1, -2.5);
          \draw (0.9, 1.5) -- (0, 1.4) -- (-0.9, 1.5);
          \draw (0.9, -1.5) -- (0, -1.4) -- (-0.9, -1.5);
          \draw (0.85, 0.5) -- (0.4, 0.5) -- (0, 0.6) -- (-0.4, 0.5) -- (-0.85, 0.5);
          \draw (0.85, -0.5) -- (0.4, -0.5) -- (0, -0.6) -- (-0.4, -0.5) -- (-0.85, -0.5);
          \draw (0, 1.4) -- (0, 0.6);
          \draw (0, -1.4) -- (0, -0.6);
          \draw (0.4, 0.5) -- (0.3, 0) -- (0.4, -0.5);
          \draw (-0.4, 0.5) -- (-0.3, 0) -- (-0.4, -0.5);
          \draw (0.3, 0) -- (-0.3, 0);
          \node at (-1.8, 0) {$k_1 + 5$};
          \node at (1.8, 0) {$k_2 + 5$};
          \node at (0, 2) {$5$};
          \node at (0, -2) {$5$};
          \node at (0.6, 1) {$5$};
          \node at (-0.6, 1) {$5$};
          \node at (0.6, -1) {$5$};
          \node at (-0.6, -1) {$5$};
          \node at (-0.6, 0) {$5$};
          \node at (0.6, 0) {$5$};
          \node at (0, -0.3) {$5$};
          \node at (0, 0.3) {$5$};
        \end{scope}
        \\
      };
    \end{tikzfigure}
  \end{cdescription}
\end{construction}

\begin{construction}\label{const:edge:replacement:5:2}
  \begin{cinput}
  \item A $3$-patch with $p$-vector $p$ and a specified path of length $2$ with the start vertex incident to some $k_1$-gon and the end vertex incident to some $k_2$-gon, while the vertex in the middle is incident to neither polygon.
  \end{cinput}
  \begin{coutput}
  \item A new $3$-patch with $p$-vector $p - [k_1, k_2] + [10k \times 5] + [k_1 + 5k, k_2 + 5k]$ for all $k \in \nats$.
  \end{coutput}
  \begin{cdescription}
    Using the replacement of the path as seen in \autoref{fig:const:edge:replacement:5:2} results in a new map with $p$-vector $p - [k_1, k_2] + [10 \times 5] + [k_1 + 5, k_2 + 5]$. The thick drawn path on the left is the specified path and the thick drawn path on the right is a new path which we can use to repeat the construction. Every time we use this construction we add ten new pentagons while increasing the number of vertices of the left and right polygon by 5; doing this $k$ times has the desired result, since we only inserted $3$-valent vertices.
    \begin{tikzfigure}{\label{fig:const:edge:replacement:5:2}}{}
      \matrix (m) [column sep=1cm] {
        \begin{scope}
          \draw[very thick] (-1, 0) -- (0, 0.2) -- (1, 0);
          \draw (-1.2, 0.5) -- (-1, 0) -- (-1.2, -0.5);
          \draw (1.2, 0.5) -- (1, 0) -- (1.2, -0.5);
          \draw (0, 0.2) -- (0, 0.8);
          \node at (-1.5, 0) {$k_1$};
          \node at (1.5, 0) {$k_2$};
        \end{scope}
        &
        \begin{scope}
          \draw (0, 2.7) -- (0, 3.3);
          \draw[very thick] (1, 2.5) -- (0, 2.7) -- (-1, 2.5);
          \draw (-0.9, 1.5) -- (0.9, 1.5);
          \draw (-1.2, 3) -- (-1, 2.5);
          \draw (-1.2, -3) -- (-1, -2.5);
          \draw (1.2, 3) -- (1, 2.5);
          \draw (1.2, -3) -- (1, -2.5);
          \draw (1, 2.5) -- (0.9, 1.5) -- (0.85, 0.5) -- (0.85, -0.5) -- (0.9, -1.5) -- (1, -2.5);
          \draw (-1, 2.5) -- (-0.9, 1.5) -- (-0.85, 0.5) -- (-0.85, -0.5) -- (-0.9, -1.5) -- (-1, -2.5);
          \draw (0.85, 0.5) -- (0, 0.4) -- (-0.85, 0.5);
          \draw (-1, -2.5) -- (0, -2.3) -- (1, -2.5);
          \draw (0.85, -0.5) -- (0.4, -0.5) -- (0, -0.4) -- (-0.4, -0.5) -- (-0.85, -0.5);
          \draw (0.9, -1.5) -- (0.4, -1.5) -- (0, -1.6) -- (-0.4, -1.5) -- (-0.9, -1.5);
          \draw (0, 0.4) -- (0, -0.4);
          \draw (0, -2.3) -- (0, -1.6);
          \draw (0.4, -0.5) -- (0.3, -1) -- (0.4, -1.5);
          \draw (-0.4, -0.5) -- (-0.3, -1) -- (-0.4, -1.5);
          \draw (0.3, -1) -- (-0.3, -1);
          \node at (-1.8, 0) {$k_1 + 5$};
          \node at (1.8, 0) {$k_2 + 5$};
          \node at (0, 1) {$5$};
          \node at (0, 2) {$5$};
          \node at (0.6, 0) {$5$};
          \node at (-0.6, 0) {$5$};
          \node at (0.6, -2) {$5$};
          \node at (-0.6, -2) {$5$};
          \node at (-0.6, -1) {$5$};
          \node at (0.6, -1) {$5$};
          \node at (0, -1.3) {$5$};
          \node at (0, -0.7) {$5$};
        \end{scope}
        \\
      };
    \end{tikzfigure}
  \end{cdescription}
\end{construction}

\begin{construction}\label{const:edge:replacement:5:3}
  \begin{cinput}
  \item A $3$-patch with $p$-vector $p$ and a specified edge incident to both some $k_1$-gon and some $k_2$-gon.
  \end{cinput}
  \begin{coutput}
  \item A new $3$-patch with $p$-vector $p - [k_1, k_2] + [(10k) \times 5] + [5k + k_1 , 5k + k_2]$ for all $k \in \nats$.
  \end{coutput}
  \begin{cdescription}
    Using the replacement of the single edge as seen in \autoref{fig:const:edge:replacement:5:3} results in a new map with $p$-vector $p - [k_1, k_2] + [10 \times 5] + [k_1 + 5, k_2 + 5]$. The thick line on the left is the specified line and the thick line on the right is a new line which we can use to repeat the construction. Every time we use this construction we add four new quadrangles while increasing the number of vertices of the left and right polygon; doing this $k$ times has the desired result, since we only insert $3$-valent vertices.
    \begin{tikzfigure}{\label{fig:const:edge:replacement:5:3}}{}
      \matrix (m) [column sep=1cm] {
        \begin{scope}
          \draw[very thick] (0, -1) -- (0, 1);
          \draw (0.5, -1.2) -- (0, -1) -- (-0.5, -1.2);
          \draw (0.5, 1.2) -- (0, 1) -- (-0.5, 1.2);
          \node at (-1, 0) {$k_1$};
          \node at (1, 0) {$k_2$};
        \end{scope}
        &
        \begin{scope}
          \draw (0.5, -2.7) -- (0, -2.5) -- (-0.5, -2.7);
          \draw (0.5, 2.7) -- (0, 2.5) -- (-0.5, 2.7);
          \draw[very thick] (0, 2.5) -- (0, 2);
          \draw (0, -2.5) -- (0, -2);
          \draw (0, -2) -- (-1.5 , -1.5) -- (-2, 0) -- (-1.5, 1.5) -- (0, 2) -- (1.5, 1.5) -- (2, 0) -- (1.5, -1.5) -- (0, -2);
          \draw (-0.5, -1) -- (-0.5, -0.5) -- (-1, 0) -- (-0.5, 0.5) -- (-0.5, 1) -- (0.5, 1) -- (0.5, 0.5) -- (1, 0) -- (0.5, -0.5) -- (0.5, -1) -- (-0.5, -1);
          \draw (-0.5, -0.5) -- (0, -0.25) -- (0.5, -0.5);
          \draw (-0.5, 0.5) -- (0, 0.25) -- (0.5, 0.5);
          \draw (-1.5, -1.5) -- (-0.5, -1);
          \draw (1.5, -1.5) -- (0.5, -1);
          \draw (-1.5, 1.5) -- (-0.5, 1);
          \draw (1.5, 1.5) -- (0.5, 1);
          \draw (0, -0.25) -- (0, 0.25);
          \draw (-2, 0) -- (-1, 0);
          \draw (2, 0) -- (1, 0);

          \node at (-1.25, -0.5) {$5$};
          \node at (1.25, -0.5) {$5$};
          \node at (-1.25, 0.5) {$5$};
          \node at (1.25, 0.5) {$5$};
          \node at (-0.5, 0) {$5$};
          \node at (0.5, 0) {$5$};
          \node at (0, -1.5) {$5$};
          \node at (0, 1.5) {$5$};
          \node at (0, -0.75) {$5$};
          \node at (0, 0.75) {$5$};

        \end{scope}
        \\
      };
    \end{tikzfigure}
  \end{cdescription}
\end{construction}

\begin{theorem}
  Let $p$ and $v$ be a pair of admissible sequences for some orientable closed $2$-manifold $S$ with {\sc Euler}-characteristic $\chi$, i.e. \eqref{eq:vp:3} and \eqref{eq:handshake} holds for some $e \in \nats$. Then $(p, v)$ is $[(5k + 1) \times 5, (5k+7)]$-$[3]$-realizable for all $k \in \nats$.
  \begin{proof}
    To apply \autoref{thm:main:const}, we need an expansion $3$-patch with outer tuple $o$ and a corresponding $o$-$k$-gonal $3$-patch, both of which only contain pentagons and $5k+7$-gons. These are depicted in \autoref{fig:expansion:patch:5:7}.

\begin{tikzfigure}{\label{fig:expansion:patch:5:7}}{}
      \matrix (m) [column sep=1cm] {
        \begin{scope}[scale=3, yshift=25]
          \documentclass[a4paper]{article}
\usepackage[left=0.5cm, right=0.5cm, top=1cm, bottom=1cm]{geometry}
\usepackage{tikz}
\begin{document}
\begin{figure}[h!]
\centering
\begin{tikzpicture}[scale = 10, auto]
\coordinate (x0) at (0.939693, -0.342020);
\coordinate (x1) at (0.634063, -0.299186);
\coordinate (x2) at (0.328124, -0.256744);
\coordinate (x3) at (0.500000, -0.866025);
\coordinate (x4) at (0.939693, 0.342020);
\coordinate (x5) at (0.091511, 0.342104);
\coordinate (x6) at (-0.766044, 0.642788);
\coordinate (x7) at (-1.000000, 0.000000);
\coordinate (x8) at (-0.766044, -0.642788);
\coordinate (x9) at (-0.173648, -0.984808);
\coordinate (x10) at (0.500000, 0.866025);
\coordinate (x11) at (-0.173648, 0.984808);
\draw (0.939693, -0.342020) -- (0.634063, -0.299186);
\draw (0.634063, -0.299186) -- (0.328124, -0.256744);
\draw[thick] (0.328124, -0.256744) -- (0.500000, -0.866025);
\draw (0.500000, -0.866025) -- (0.939693, -0.342020);
\draw (0.939693, -0.342020) -- (0.939693, 0.342020);
\draw (0.939693, 0.342020) -- (0.091511, 0.342104);
\draw (0.091511, 0.342104) -- (0.328124, -0.256744);
\draw (0.091511, 0.342104) -- (-0.766044, 0.642788);
\draw (-0.766044, 0.642788) -- (-1.000000, 0.000000);
\draw (-1.000000, 0.000000) -- (-0.766044, -0.642788);
\draw (-0.766044, -0.642788) -- (-0.173648, -0.984808);
\draw (-0.173648, -0.984808) -- (0.500000, -0.866025);
\draw (0.939693, 0.342020) -- (0.500000, 0.866025);
\draw (0.500000, 0.866025) -- (-0.173648, 0.984808);
\draw (-0.173648, 0.984808) -- (-0.766044, 0.642788);
\node at (-0.000000, 0.000000) {0};
\node at (0.600470, -0.440994) {1};
\node at (0.586617, -0.042765) {2};
\node at (-0.255157, -0.252210) {3};
\node at (0.118302, 0.635549) {4};
\end{tikzpicture}
\caption{Graph}

\end{figure}
\end{document}

        \end{scope}
        &
        \begin{scope}[scale=3, yshift=25]
          \coordinate (x0) at (-0.882596, 0.144489);
\coordinate (x1) at (-0.817805, 0.052066);
\coordinate (x2) at (-0.717351, 0.117014);
\coordinate (x3) at (-0.680700, 0.275346);
\coordinate (x4) at (-0.812941, 0.277494);
\coordinate (x5) at (-0.998308, 0.058145);
\coordinate (x6) at (-0.998308, -0.058145);
\coordinate (x7) at (-0.854325, -0.072140);
\coordinate (x8) at (-0.746725, -0.176568);
\coordinate (x9) at (-0.695612, -0.029457);
\coordinate (x10) at (-0.577701, -0.003240);
\coordinate (x11) at (-0.572673, 0.412353);
\coordinate (x12) at (-0.847214, 0.420134);
\coordinate (x13) at (-0.984808, 0.173648);
\coordinate (x14) at (-0.711199, -0.354740);
\coordinate (x15) at (-0.572753, -0.418177);
\coordinate (x16) at (-0.549509, 0.835488);
\coordinate (x17) at (-0.642788, 0.766044);
\coordinate (x18) at (-0.727374, 0.686242);
\coordinate (x19) at (-0.984808, -0.173648);
\coordinate (x20) at (-0.957990, -0.286803);
\coordinate (x21) at (-0.782466, -0.508334);
\coordinate (x22) at (-0.642788, -0.766044);
\coordinate (x23) at (-0.549509, -0.835488);
\coordinate (x24) at (-0.448799, -0.893633);
\coordinate (x25) at (-0.448799, 0.893633);
\coordinate (x26) at (-0.342020, -0.939693);
\coordinate (x27) at (-0.230616, -0.973045);
\coordinate (x28) at (0.022529, -0.386804);
\coordinate (x29) at (-0.022528, 0.386799);
\coordinate (x30) at (-0.342020, 0.939693);
\coordinate (x31) at (0.342020, -0.939693);
\coordinate (x32) at (0.448799, -0.893633);
\coordinate (x33) at (0.448799, 0.893633);
\coordinate (x34) at (0.342020, 0.939693);
\coordinate (x35) at (0.230616, 0.973045);
\coordinate (x36) at (0.549509, -0.835488);
\coordinate (x37) at (0.648552, -0.425935);
\coordinate (x38) at (0.716422, -0.012589);
\coordinate (x39) at (0.655278, 0.408542);
\coordinate (x40) at (0.549509, 0.835488);
\coordinate (x41) at (0.826848, -0.421223);
\coordinate (x42) at (0.984808, -0.173648);
\coordinate (x43) at (0.998308, -0.058145);
\coordinate (x44) at (0.998308, 0.058145);
\coordinate (x45) at (0.841254, 0.282416);
\coordinate (x46) at (0.792667, 0.497122);
\coordinate (x47) at (0.642788, 0.766044);
\coordinate (x48) at (0.642788, -0.766044);
\coordinate (x49) at (0.727374, -0.686242);
\coordinate (x50) at (0.984808, 0.173648);
\coordinate (x51) at (0.957990, 0.286803);
\coordinate (x52) at (0.918216, 0.396080);
\coordinate (x53) at (0.866025, 0.500000);
\coordinate (x54) at (0.802123, 0.597159);
\coordinate (x55) at (0.727374, 0.686242);
\coordinate (x56) at (0.116093, 0.993238);
\coordinate (x57) at (0.000000, 1.000000);
\coordinate (x58) at (-0.116093, 0.993238);
\coordinate (x59) at (-0.230616, 0.973045);
\coordinate (x60) at (-0.802123, 0.597159);
\coordinate (x61) at (-0.866025, 0.500000);
\coordinate (x62) at (-0.918216, 0.396080);
\coordinate (x63) at (-0.957990, 0.286803);
\coordinate (x64) at (-0.918216, -0.396080);
\coordinate (x65) at (-0.866025, -0.500000);
\coordinate (x66) at (-0.802123, -0.597159);
\coordinate (x67) at (-0.727374, -0.686242);
\coordinate (x68) at (-0.116093, -0.993238);
\coordinate (x69) at (-0.000000, -1.000000);
\coordinate (x70) at (0.116093, -0.993238);
\coordinate (x71) at (0.230616, -0.973045);
\coordinate (x72) at (0.802123, -0.597159);
\coordinate (x73) at (0.866025, -0.500000);
\coordinate (x74) at (0.918216, -0.396080);
\coordinate (x75) at (0.957990, -0.286803);
\draw (-0.882596, 0.144489) -- (-0.817805, 0.052066);
\draw (-0.817805, 0.052066) -- (-0.717351, 0.117014);
\draw (-0.717351, 0.117014) -- (-0.680700, 0.275346);
\draw (-0.680700, 0.275346) -- (-0.812941, 0.277494);
\draw (-0.812941, 0.277494) -- (-0.882596, 0.144489);
\draw (-0.882596, 0.144489) -- (-0.998308, 0.058145);
\draw (-0.998308, 0.058145) -- (-0.998308, -0.058145);
\draw (-0.998308, -0.058145) -- (-0.854325, -0.072140);
\draw (-0.854325, -0.072140) -- (-0.817805, 0.052066);
\draw (-0.854325, -0.072140) -- (-0.746725, -0.176568);
\draw (-0.746725, -0.176568) -- (-0.695612, -0.029457);
\draw (-0.695612, -0.029457) -- (-0.717351, 0.117014);
\draw (-0.695612, -0.029457) -- (-0.577701, -0.003240);
\draw (-0.577701, -0.003240) -- (-0.572673, 0.412353);
\draw (-0.572673, 0.412353) -- (-0.680700, 0.275346);
\draw (-0.812941, 0.277494) -- (-0.847214, 0.420134);
\draw (-0.847214, 0.420134) -- (-0.984808, 0.173648);
\draw (-0.984808, 0.173648) -- (-0.998308, 0.058145);
\draw (-0.746725, -0.176568) -- (-0.711199, -0.354740);
\draw (-0.711199, -0.354740) -- (-0.572753, -0.418177);
\draw (-0.572753, -0.418177) -- (-0.577701, -0.003240);
\draw (-0.572673, 0.412353) -- (-0.549509, 0.835488);
\draw (-0.549509, 0.835488) -- (-0.642788, 0.766044);
\draw (-0.642788, 0.766044) -- (-0.727374, 0.686242);
\draw (-0.727374, 0.686242) -- (-0.847214, 0.420134);
\draw (-0.998308, -0.058145) -- (-0.984808, -0.173648);
\draw (-0.984808, -0.173648) -- (-0.957990, -0.286803);
\draw (-0.957990, -0.286803) -- (-0.782466, -0.508334);
\draw (-0.782466, -0.508334) -- (-0.711199, -0.354740);
\draw (-0.782466, -0.508334) -- (-0.642788, -0.766044);
\draw (-0.642788, -0.766044) -- (-0.549509, -0.835488);
\draw (-0.549509, -0.835488) -- (-0.572753, -0.418177);
\draw (-0.549509, -0.835488) -- (-0.448799, -0.893633);
\draw (-0.448799, -0.893633) -- (-0.448799, 0.893633);
\draw (-0.448799, 0.893633) -- (-0.549509, 0.835488);
\draw (-0.448799, -0.893633) -- (-0.342020, -0.939693);
\draw (-0.342020, -0.939693) -- (-0.230616, -0.973045);
\draw (-0.230616, -0.973045) -- (0.022529, -0.386804);
\draw (0.022529, -0.386804) -- (-0.022528, 0.386799);
\draw (-0.022528, 0.386799) -- (-0.342020, 0.939693);
\draw (-0.342020, 0.939693) -- (-0.448799, 0.893633);
\draw (0.022529, -0.386804) -- (0.342020, -0.939693);
\draw (0.342020, -0.939693) -- (0.448799, -0.893633);
\draw (0.448799, -0.893633) -- (0.448799, 0.893633);
\draw (0.448799, 0.893633) -- (0.342020, 0.939693);
\draw (0.342020, 0.939693) -- (0.230616, 0.973045);
\draw (0.230616, 0.973045) -- (-0.022528, 0.386799);
\draw (0.448799, -0.893633) -- (0.549509, -0.835488);
\draw (0.549509, -0.835488) -- (0.648552, -0.425935);
\draw (0.648552, -0.425935) -- (0.716422, -0.012589);
\draw (0.716422, -0.012589) -- (0.655278, 0.408542);
\draw (0.655278, 0.408542) -- (0.549509, 0.835488);
\draw (0.549509, 0.835488) -- (0.448799, 0.893633);
\draw (0.648552, -0.425935) -- (0.826848, -0.421223);
\draw (0.826848, -0.421223) -- (0.984808, -0.173648);
\draw (0.984808, -0.173648) -- (0.998308, -0.058145);
\draw (0.998308, -0.058145) -- (0.716422, -0.012589);
\draw (0.998308, -0.058145) -- (0.998308, 0.058145);
\draw (0.998308, 0.058145) -- (0.841254, 0.282416);
\draw (0.841254, 0.282416) -- (0.655278, 0.408542);
\draw (0.841254, 0.282416) -- (0.792667, 0.497122);
\draw (0.792667, 0.497122) -- (0.642788, 0.766044);
\draw (0.642788, 0.766044) -- (0.549509, 0.835488);
\draw (0.549509, -0.835488) -- (0.642788, -0.766044);
\draw (0.642788, -0.766044) -- (0.727374, -0.686242);
\draw (0.727374, -0.686242) -- (0.826848, -0.421223);
\draw (0.998308, 0.058145) -- (0.984808, 0.173648);
\draw (0.984808, 0.173648) -- (0.957990, 0.286803);
\draw (0.957990, 0.286803) -- (0.792667, 0.497122);
\draw (0.957990, 0.286803) -- (0.918216, 0.396080);
\draw (0.918216, 0.396080) -- (0.866025, 0.500000);
\draw (0.866025, 0.500000) -- (0.802123, 0.597159);
\draw (0.802123, 0.597159) -- (0.727374, 0.686242);
\draw (0.727374, 0.686242) -- (0.642788, 0.766044);
\draw (0.230616, 0.973045) -- (0.116093, 0.993238);
\draw (0.116093, 0.993238) -- (0.000000, 1.000000);
\draw (0.000000, 1.000000) -- (-0.116093, 0.993238);
\draw (-0.116093, 0.993238) -- (-0.230616, 0.973045);
\draw (-0.230616, 0.973045) -- (-0.342020, 0.939693);
\draw (-0.727374, 0.686242) -- (-0.802123, 0.597159);
\draw (-0.802123, 0.597159) -- (-0.866025, 0.500000);
\draw (-0.866025, 0.500000) -- (-0.918216, 0.396080);
\draw (-0.918216, 0.396080) -- (-0.957990, 0.286803);
\draw (-0.957990, 0.286803) -- (-0.984808, 0.173648);
\draw (-0.957990, -0.286803) -- (-0.918216, -0.396080);
\draw (-0.918216, -0.396080) -- (-0.866025, -0.500000);
\draw (-0.866025, -0.500000) -- (-0.802123, -0.597159);
\draw (-0.802123, -0.597159) -- (-0.727374, -0.686242);
\draw (-0.727374, -0.686242) -- (-0.642788, -0.766044);
\draw (-0.230616, -0.973045) -- (-0.116093, -0.993238);
\draw (-0.116093, -0.993238) -- (-0.000000, -1.000000);
\draw (-0.000000, -1.000000) -- (0.116093, -0.993238);
\draw (0.116093, -0.993238) -- (0.230616, -0.973045);
\draw (0.230616, -0.973045) -- (0.342020, -0.939693);
\draw (0.727374, -0.686242) -- (0.802123, -0.597159);
\draw (0.802123, -0.597159) -- (0.866025, -0.500000);
\draw (0.866025, -0.500000) -- (0.918216, -0.396080);
\draw (0.918216, -0.396080) -- (0.957990, -0.286803);
\draw (0.957990, -0.286803) -- (0.984808, -0.173648);

        \end{scope}
        &
        \begin{scope}[scale=3, yshift=25]
          \coordinate (x0) at (-0.025719, -0.140221);
\coordinate (x1) at (-0.048581, -0.117640);
\coordinate (x2) at (-0.058419, -0.158944);
\coordinate (x3) at (-0.003761, -0.214534);
\coordinate (x4) at (-0.056720, -0.074753);
\coordinate (x5) at (-0.127330, -0.127974);
\coordinate (x6) at (-0.223332, -0.159863);
\coordinate (x7) at (0.037355, -0.303210);
\coordinate (x8) at (0.297466, -0.453253);
\coordinate (x9) at (0.462717, 0.037231);
\coordinate (x10) at (-0.013981, 0.005909);
\coordinate (x11) at (-0.490391, -0.022612);
\coordinate (x12) at (-1.000000, 0.000000);
\coordinate (x13) at (-0.623490, -0.781831);
\coordinate (x14) at (0.222521, -0.974928);
\coordinate (x15) at (0.900969, -0.433884);
\coordinate (x16) at (0.900969, 0.433884);
\coordinate (x17) at (0.222521, 0.974928);
\coordinate (x18) at (-0.623490, 0.781831);
\draw (-0.025719, -0.140221) -- (-0.048581, -0.117640);
\draw (-0.048581, -0.117640) -- (-0.058419, -0.158944);
\draw (-0.058419, -0.158944) -- (-0.003761, -0.214534);
\draw (-0.003761, -0.214534) -- (-0.025719, -0.140221);
\draw (-0.025719, -0.140221) -- (-0.056720, -0.074753);
\draw[very thick] (-0.056720, -0.074753) -- (-0.127330, -0.127974);
\draw[very thick] (-0.127330, -0.127974) -- (-0.058419, -0.158944);
\draw (-0.127330, -0.127974) -- (-0.223332, -0.159863);
\draw (-0.223332, -0.159863) -- (0.037355, -0.303210);
\draw (0.037355, -0.303210) -- (-0.003761, -0.214534);
\draw (0.037355, -0.303210) -- (0.297466, -0.453253);
\draw[very thick] (0.297466, -0.453253) -- (0.462717, 0.037231);
\draw (0.462717, 0.037231) -- (-0.013981, 0.005909);
\draw (-0.013981, 0.005909) -- (-0.056720, -0.074753);
\draw (-0.013981, 0.005909) -- (-0.490391, -0.022612);
\draw (-0.490391, -0.022612) -- (-0.223332, -0.159863);
\draw (-0.490391, -0.022612) -- (-1.000000, 0.000000);
\draw (-1.000000, 0.000000) -- (-0.623490, -0.781831);
\draw (-0.623490, -0.781831) -- (0.222521, -0.974928);
\draw (0.222521, -0.974928) -- (0.297466, -0.453253);
\draw (0.222521, -0.974928) -- (0.900969, -0.433884);
\draw (0.900969, -0.433884) -- (0.900969, 0.433884);
\draw (0.900969, 0.433884) -- (0.462717, 0.037231);
\draw (0.900969, 0.433884) -- (0.222521, 0.974928);
\draw (0.222521, 0.974928) -- (-0.623490, 0.781831);
\draw (-0.623490, 0.781831) -- (-1.000000, 0.000000);
\node at (-0.000000, 0.000000) {0};
\node at (-0.034120, -0.157835) {1};
\node at (-0.063354, -0.123906) {2};
\node at (-0.075097, -0.192905) {3};
\node at (0.099622, -0.163262) {4};
\node at (-0.182351, -0.075859) {5};
\node at (-0.254267, -0.385100) {6};
\node at (0.556928, -0.278190) {7};
\node at (-0.077379, 0.315882) {8};

        \end{scope}
        \\
      };
    \end{tikzfigure}
  \end{proof}
\end{theorem}

\begin{theorem}
  Let $p$ and $v$ be a pair of admissible sequences for some orientable closed $2$-manifold $S$ with {\sc Euler}-characteristic $\chi$, i.e. \eqref{eq:vp:3} and \eqref{eq:handshake} holds for some $e \in \nats$. Then $(p, v)$ is $[(5k + 2) \times 5, (5k+8)]$-$[3]$-realizable for all $k \in \nats$.
  \begin{proof}
    To apply \autoref{thm:main:const}, we need an expansion $3$-patch with outer tuple $o$ and a corresponding $o$-$k$-gonal $3$-patch, both of which only contain pentagons and $5k+8$-gons. These are depicted in \autoref{fig:expansion:patch:5:8}.

\begin{tikzfigure}{\label{fig:expansion:patch:5:8}}{}
      \matrix (m) [column sep=1cm] {
        \begin{scope}[scale=3, yshift=25]
          \coordinate (x0) at (0.568199, -0.201929);
\coordinate (x1) at (0.448960, -0.399099);
\coordinate (x2) at (0.328743, -0.595831);
\coordinate (x3) at (0.654861, -0.755750);
\coordinate (x4) at (0.959493, -0.281733);
\coordinate (x5) at (0.204800, -0.052065);
\coordinate (x6) at (-0.032816, -0.518344);
\coordinate (x7) at (-0.415415, -0.909632);
\coordinate (x8) at (0.142315, -0.989821);
\coordinate (x9) at (0.959493, 0.281733);
\coordinate (x10) at (0.236438, 0.479720);
\coordinate (x11) at (-0.415415, 0.909632);
\coordinate (x12) at (-0.841254, 0.540641);
\coordinate (x13) at (-1.000000, 0.000000);
\coordinate (x14) at (-0.841254, -0.540641);
\coordinate (x15) at (0.654861, 0.755750);
\coordinate (x16) at (0.142315, 0.989821);
\draw (0.568199, -0.201929) -- (0.448960, -0.399099);
\draw (0.448960, -0.399099) -- (0.328743, -0.595831);
\draw (0.328743, -0.595831) -- (0.654861, -0.755750);
\draw (0.654861, -0.755750) -- (0.959493, -0.281733);
\draw (0.959493, -0.281733) -- (0.568199, -0.201929);
\draw (0.568199, -0.201929) -- (0.204800, -0.052065);
\draw (0.204800, -0.052065) -- (-0.032816, -0.518344);
\draw (-0.032816, -0.518344) -- (0.328743, -0.595831);
\draw (-0.032816, -0.518344) -- (-0.415415, -0.909632);
\draw (-0.415415, -0.909632) -- (0.142315, -0.989821);
\draw (0.142315, -0.989821) -- (0.654861, -0.755750);
\draw (0.959493, -0.281733) -- (0.959493, 0.281733);
\draw (0.959493, 0.281733) -- (0.236438, 0.479720);
\draw (0.236438, 0.479720) -- (0.204800, -0.052065);
\draw (0.236438, 0.479720) -- (-0.415415, 0.909632);
\draw (-0.415415, 0.909632) -- (-0.841254, 0.540641);
\draw (-0.841254, 0.540641) -- (-1.000000, 0.000000);
\draw (-1.000000, 0.000000) -- (-0.841254, -0.540641);
\draw (-0.841254, -0.540641) -- (-0.415415, -0.909632);
\draw (0.959493, 0.281733) -- (0.654861, 0.755750);
\draw (0.654861, 0.755750) -- (0.142315, 0.989821);
\draw (0.142315, 0.989821) -- (-0.415415, 0.909632);
\node at (-0.000000, -0.000000) {0};
\node at (0.592051, -0.446868) {1};
\node at (0.303577, -0.353454) {2};
\node at (0.135537, -0.753876) {3};
\node at (0.585685, 0.045145) {4};
\node at (-0.388114, -0.011336) {5};
\node at (0.315538, 0.683331) {6};

        \end{scope}
        &
        \begin{scope}[scale=3, yshift=25]
          \coordinate (x0) at (-0.297003, 0.445473);
\coordinate (x1) at (-0.470182, 0.606872);
\coordinate (x2) at (-0.630409, 0.404927);
\coordinate (x3) at (-0.747357, 0.175386);
\coordinate (x4) at (-0.528641, 0.085018);
\coordinate (x5) at (-0.009460, 0.760053);
\coordinate (x6) at (-0.371662, 0.928368);
\coordinate (x7) at (-0.458227, 0.888835);
\coordinate (x8) at (-0.540641, 0.841254);
\coordinate (x9) at (-0.756908, 0.486478);
\coordinate (x10) at (-0.989821, 0.142315);
\coordinate (x11) at (-0.998867, 0.047582);
\coordinate (x12) at (-0.998867, -0.047582);
\coordinate (x13) at (-0.695335, -0.307174);
\coordinate (x14) at (-0.618159, -0.786053);
\coordinate (x15) at (-0.540641, -0.841254);
\coordinate (x16) at (0.540641, 0.841254);
\coordinate (x17) at (0.458227, 0.888835);
\coordinate (x18) at (-0.458227, -0.888835);
\coordinate (x19) at (0.009400, -0.760084);
\coordinate (x20) at (0.296922, -0.445520);
\coordinate (x21) at (0.528605, -0.085087);
\coordinate (x22) at (0.695323, 0.307178);
\coordinate (x23) at (0.618159, 0.786053);
\coordinate (x24) at (0.469997, -0.606880);
\coordinate (x25) at (0.630228, -0.405177);
\coordinate (x26) at (0.747371, -0.175522);
\coordinate (x27) at (0.371662, -0.928368);
\coordinate (x28) at (0.458227, -0.888835);
\coordinate (x29) at (0.540641, -0.841254);
\coordinate (x30) at (0.756899, -0.486486);
\coordinate (x31) at (0.989821, -0.142315);
\coordinate (x32) at (0.998867, -0.047582);
\coordinate (x33) at (0.998867, 0.047582);
\coordinate (x34) at (0.989821, 0.142315);
\coordinate (x35) at (0.971812, 0.235759);
\coordinate (x36) at (0.945001, 0.327068);
\coordinate (x37) at (0.755750, 0.654861);
\coordinate (x38) at (0.690079, 0.723734);
\coordinate (x39) at (0.909632, 0.415415);
\coordinate (x40) at (0.866025, 0.500000);
\coordinate (x41) at (0.814576, 0.580057);
\coordinate (x42) at (0.371662, 0.928368);
\coordinate (x43) at (0.281733, 0.959493);
\coordinate (x44) at (0.189251, 0.981929);
\coordinate (x45) at (-0.189251, 0.981929);
\coordinate (x46) at (-0.281733, 0.959493);
\coordinate (x47) at (0.095056, 0.995472);
\coordinate (x48) at (0.000000, 1.000000);
\coordinate (x49) at (-0.095056, 0.995472);
\coordinate (x50) at (-0.618159, 0.786053);
\coordinate (x51) at (-0.690079, 0.723734);
\coordinate (x52) at (-0.755750, 0.654861);
\coordinate (x53) at (-0.945001, 0.327068);
\coordinate (x54) at (-0.971812, 0.235759);
\coordinate (x55) at (-0.814576, 0.580057);
\coordinate (x56) at (-0.866025, 0.500000);
\coordinate (x57) at (-0.909632, 0.415415);
\coordinate (x58) at (-0.989821, -0.142315);
\coordinate (x59) at (-0.971812, -0.235759);
\coordinate (x60) at (-0.945001, -0.327068);
\coordinate (x61) at (-0.755750, -0.654861);
\coordinate (x62) at (-0.690079, -0.723734);
\coordinate (x63) at (-0.909632, -0.415415);
\coordinate (x64) at (-0.866025, -0.500000);
\coordinate (x65) at (-0.814576, -0.580057);
\coordinate (x66) at (-0.371662, -0.928368);
\coordinate (x67) at (-0.281733, -0.959493);
\coordinate (x68) at (-0.189251, -0.981929);
\coordinate (x69) at (0.189251, -0.981929);
\coordinate (x70) at (0.281733, -0.959493);
\coordinate (x71) at (-0.095056, -0.995472);
\coordinate (x72) at (-0.000000, -1.000000);
\coordinate (x73) at (0.095056, -0.995472);
\coordinate (x74) at (0.618159, -0.786053);
\coordinate (x75) at (0.690079, -0.723734);
\coordinate (x76) at (0.755750, -0.654861);
\coordinate (x77) at (0.945001, -0.327068);
\coordinate (x78) at (0.971812, -0.235759);
\coordinate (x79) at (0.814576, -0.580057);
\coordinate (x80) at (0.866025, -0.500000);
\coordinate (x81) at (0.909632, -0.415415);

\draw (-0.297003, 0.445473) -- (-0.470182, 0.606872);
\draw (-0.470182, 0.606872) -- (-0.630409, 0.404927);
\draw (-0.630409, 0.404927) -- (-0.747357, 0.175386);
\draw (-0.747357, 0.175386) -- (-0.528641, 0.085018);
\draw (-0.528641, 0.085018) -- (-0.297003, 0.445473);
\draw (-0.297003, 0.445473) -- (-0.009460, 0.760053);
\draw (-0.009460, 0.760053) -- (-0.371662, 0.928368);
\draw[very thick] (-0.371662, 0.928368) -- (-0.458227, 0.888835);
\draw (-0.458227, 0.888835) -- (-0.470182, 0.606872);
\draw[very thick] (-0.458227, 0.888835) -- (-0.540641, 0.841254);
\draw (-0.540641, 0.841254) -- (-0.756908, 0.486478);
\draw (-0.756908, 0.486478) -- (-0.630409, 0.404927);
\draw (-0.756908, 0.486478) -- (-0.989821, 0.142315);
\draw (-0.989821, 0.142315) -- (-0.998867, 0.047582);
\draw (-0.998867, 0.047582) -- (-0.747357, 0.175386);
\draw (-0.998867, 0.047582) -- (-0.998867, -0.047582);
\draw (-0.998867, -0.047582) -- (-0.695335, -0.307174);
\draw (-0.695335, -0.307174) -- (-0.528641, 0.085018);
\draw[ldiamond] (-0.695335, -0.307174) -- (-0.618159, -0.786053);
\draw (-0.618159, -0.786053) -- (-0.540641, -0.841254);
\draw (-0.540641, -0.841254) -- (0.540641, 0.841254);
\draw (0.540641, 0.841254) -- (0.458227, 0.888835);
\draw (0.458227, 0.888835) -- (-0.009460, 0.760053);
\draw (-0.540641, -0.841254) -- (-0.458227, -0.888835);
\draw (-0.458227, -0.888835) -- (0.009400, -0.760084);
\draw (0.009400, -0.760084) -- (0.296922, -0.445520);
\draw (0.296922, -0.445520) -- (0.528605, -0.085087);
\draw (0.528605, -0.085087) -- (0.695323, 0.307178);
\draw[ldiamond] (0.695323, 0.307178) -- (0.618159, 0.786053);
\draw (0.618159, 0.786053) -- (0.540641, 0.841254);
\draw (0.296922, -0.445520) -- (0.469997, -0.606880);
\draw (0.469997, -0.606880) -- (0.630228, -0.405177);
\draw (0.630228, -0.405177) -- (0.747371, -0.175522);
\draw (0.747371, -0.175522) -- (0.528605, -0.085087);
\draw (0.009400, -0.760084) -- (0.371662, -0.928368);
\draw[very thick] (0.371662, -0.928368) -- (0.458227, -0.888835);
\draw (0.458227, -0.888835) -- (0.469997, -0.606880);
\draw[very thick] (0.458227, -0.888835) -- (0.540641, -0.841254);
\draw (0.540641, -0.841254) -- (0.756899, -0.486486);
\draw (0.756899, -0.486486) -- (0.630228, -0.405177);
\draw (0.756899, -0.486486) -- (0.989821, -0.142315);
\draw (0.989821, -0.142315) -- (0.998867, -0.047582);
\draw (0.998867, -0.047582) -- (0.747371, -0.175522);
\draw (0.998867, -0.047582) -- (0.998867, 0.047582);
\draw (0.998867, 0.047582) -- (0.695323, 0.307178);
\draw (0.998867, 0.047582) -- (0.989821, 0.142315);
\draw (0.989821, 0.142315) -- (0.971812, 0.235759);
\draw (0.971812, 0.235759) -- (0.945001, 0.327068);
\draw (0.945001, 0.327068) -- (0.755750, 0.654861);
\draw (0.755750, 0.654861) -- (0.690079, 0.723734);
\draw (0.690079, 0.723734) -- (0.618159, 0.786053);
\draw (0.945001, 0.327068) -- (0.909632, 0.415415);
\draw (0.909632, 0.415415) -- (0.866025, 0.500000);
\draw (0.866025, 0.500000) -- (0.814576, 0.580057);
\draw (0.814576, 0.580057) -- (0.755750, 0.654861);
\draw (0.458227, 0.888835) -- (0.371662, 0.928368);
\draw (0.371662, 0.928368) -- (0.281733, 0.959493);
\draw (0.281733, 0.959493) -- (0.189251, 0.981929);
\draw (0.189251, 0.981929) -- (-0.189251, 0.981929);
\draw (-0.189251, 0.981929) -- (-0.281733, 0.959493);
\draw (-0.281733, 0.959493) -- (-0.371662, 0.928368);
\draw (0.189251, 0.981929) -- (0.095056, 0.995472);
\draw (0.095056, 0.995472) -- (0.000000, 1.000000);
\draw (0.000000, 1.000000) -- (-0.095056, 0.995472);
\draw (-0.095056, 0.995472) -- (-0.189251, 0.981929);
\draw (-0.540641, 0.841254) -- (-0.618159, 0.786053);
\draw (-0.618159, 0.786053) -- (-0.690079, 0.723734);
\draw (-0.690079, 0.723734) -- (-0.755750, 0.654861);
\draw (-0.755750, 0.654861) -- (-0.945001, 0.327068);
\draw (-0.945001, 0.327068) -- (-0.971812, 0.235759);
\draw (-0.971812, 0.235759) -- (-0.989821, 0.142315);
\draw (-0.755750, 0.654861) -- (-0.814576, 0.580057);
\draw (-0.814576, 0.580057) -- (-0.866025, 0.500000);
\draw (-0.866025, 0.500000) -- (-0.909632, 0.415415);
\draw (-0.909632, 0.415415) -- (-0.945001, 0.327068);
\draw (-0.998867, -0.047582) -- (-0.989821, -0.142315);
\draw (-0.989821, -0.142315) -- (-0.971812, -0.235759);
\draw (-0.971812, -0.235759) -- (-0.945001, -0.327068);
\draw (-0.945001, -0.327068) -- (-0.755750, -0.654861);
\draw (-0.755750, -0.654861) -- (-0.690079, -0.723734);
\draw (-0.690079, -0.723734) -- (-0.618159, -0.786053);
\draw (-0.945001, -0.327068) -- (-0.909632, -0.415415);
\draw (-0.909632, -0.415415) -- (-0.866025, -0.500000);
\draw (-0.866025, -0.500000) -- (-0.814576, -0.580057);
\draw (-0.814576, -0.580057) -- (-0.755750, -0.654861);
\draw (-0.458227, -0.888835) -- (-0.371662, -0.928368);
\draw (-0.371662, -0.928368) -- (-0.281733, -0.959493);
\draw (-0.281733, -0.959493) -- (-0.189251, -0.981929);
\draw (-0.189251, -0.981929) -- (0.189251, -0.981929);
\draw (0.189251, -0.981929) -- (0.281733, -0.959493);
\draw (0.281733, -0.959493) -- (0.371662, -0.928368);
\draw (-0.189251, -0.981929) -- (-0.095056, -0.995472);
\draw (-0.095056, -0.995472) -- (-0.000000, -1.000000);
\draw (-0.000000, -1.000000) -- (0.095056, -0.995472);
\draw (0.095056, -0.995472) -- (0.189251, -0.981929);
\draw (0.540641, -0.841254) -- (0.618159, -0.786053);
\draw (0.618159, -0.786053) -- (0.690079, -0.723734);
\draw (0.690079, -0.723734) -- (0.755750, -0.654861);
\draw (0.755750, -0.654861) -- (0.945001, -0.327068);
\draw (0.945001, -0.327068) -- (0.971812, -0.235759);
\draw (0.971812, -0.235759) -- (0.989821, -0.142315);
\draw (0.755750, -0.654861) -- (0.814576, -0.580057);
\draw (0.814576, -0.580057) -- (0.866025, -0.500000);
\draw (0.866025, -0.500000) -- (0.909632, -0.415415);
\draw (0.909632, -0.415415) -- (0.945001, -0.327068);

% \node at (0.228530, 0.078419) {0};
% \node at (-0.534718, 0.343535) {1};
% \node at (-0.321307, 0.725920) {2};
% \node at (-0.571273, 0.645673) {3};
% \node at (-0.824672, 0.251338) {4};
% \node at (-0.793813, -0.009354) {5};
% \node at (-0.211296, 0.135769) {6};
% \node at (0.211273, -0.135787) {7};
% \node at (0.534625, -0.343637) {8};
% \node at (0.321242, -0.725937) {9};
% \node at (0.571198, -0.645726) {10};
% \node at (0.824637, -0.251416) {11};
% \node at (0.793807, 0.009314) {12};
% \node at (0.833101, 0.403069) {13};
% \node at (0.858197, 0.495480) {14};
% \node at (0.056096, 0.923559) {15};
% \node at (0.000000, 0.990960) {16};
% \node at (-0.783521, 0.524690) {17};
% \node at (-0.858197, 0.495480) {18};
% \node at (-0.833103, -0.403068) {19};
% \node at (-0.858197, -0.495480) {20};
% \node at (-0.056103, -0.923562) {21};
% \node at (-0.000000, -0.990960) {22};
% \node at (0.783520, -0.524691) {23};
% \node at (0.858197, -0.495480) {24};

\fill[black] (-0.297003, 0.445473) circle (0.2pt);
\fill[black] (-0.470182, 0.606872) circle (0.2pt);
\fill[black] (-0.630409, 0.404927) circle (0.2pt);
\fill[black] (-0.747357, 0.175386) circle (0.2pt);
\fill[black] (-0.528641, 0.085018) circle (0.2pt);
\fill[black] (-0.009460, 0.760053) circle (0.2pt);
\fill[black] (-0.371662, 0.928368) circle (0.2pt);
\fill[black] (-0.458227, 0.888835) circle (0.2pt);
\fill[black] (-0.540641, 0.841254) circle (0.2pt);
\fill[black] (-0.756908, 0.486478) circle (0.2pt);
\fill[black] (-0.989821, 0.142315) circle (0.2pt);
\fill[black] (-0.998867, 0.047582) circle (0.2pt);
\fill[black] (-0.998867, -0.047582) circle (0.2pt);
\fill[black] (-0.695335, -0.307174) circle (0.2pt);
\fill[black] (-0.618159, -0.786053) circle (0.2pt);
\fill[black] (-0.540641, -0.841254) circle (0.2pt);
\fill[black] (0.540641, 0.841254) circle (0.2pt);
\fill[black] (0.458227, 0.888835) circle (0.2pt);
\fill[black] (-0.458227, -0.888835) circle (0.2pt);
\fill[black] (0.009400, -0.760084) circle (0.2pt);
\fill[black] (0.296922, -0.445520) circle (0.2pt);
\fill[black] (0.528605, -0.085087) circle (0.2pt);
\fill[black] (0.695323, 0.307178) circle (0.2pt);
\fill[black] (0.618159, 0.786053) circle (0.2pt);
\fill[black] (0.469997, -0.606880) circle (0.2pt);
\fill[black] (0.630228, -0.405177) circle (0.2pt);
\fill[black] (0.747371, -0.175522) circle (0.2pt);
\fill[black] (0.371662, -0.928368) circle (0.2pt);
\fill[black] (0.458227, -0.888835) circle (0.2pt);
\fill[black] (0.540641, -0.841254) circle (0.2pt);
\fill[black] (0.756899, -0.486486) circle (0.2pt);
\fill[black] (0.989821, -0.142315) circle (0.2pt);
\fill[black] (0.998867, -0.047582) circle (0.2pt);
\fill[black] (0.998867, 0.047582) circle (0.2pt);
\fill[black] (0.989821, 0.142315) circle (0.2pt);
\fill[black] (0.971812, 0.235759) circle (0.2pt);
\fill[black] (0.945001, 0.327068) circle (0.2pt);
\fill[black] (0.755750, 0.654861) circle (0.2pt);
\fill[black] (0.690079, 0.723734) circle (0.2pt);
\fill[black] (0.909632, 0.415415) circle (0.2pt);
\fill[black] (0.866025, 0.500000) circle (0.2pt);
\fill[black] (0.814576, 0.580057) circle (0.2pt);
\fill[black] (0.371662, 0.928368) circle (0.2pt);
\fill[black] (0.281733, 0.959493) circle (0.2pt);
\fill[black] (0.189251, 0.981929) circle (0.2pt);
\fill[black] (-0.189251, 0.981929) circle (0.2pt);
\fill[black] (-0.281733, 0.959493) circle (0.2pt);
\fill[black] (0.095056, 0.995472) circle (0.2pt);
\fill[black] (0.000000, 1.000000) circle (0.2pt);
\fill[black] (-0.095056, 0.995472) circle (0.2pt);
\fill[black] (-0.618159, 0.786053) circle (0.2pt);
\fill[black] (-0.690079, 0.723734) circle (0.2pt);
\fill[black] (-0.755750, 0.654861) circle (0.2pt);
\fill[black] (-0.945001, 0.327068) circle (0.2pt);
\fill[black] (-0.971812, 0.235759) circle (0.2pt);
\fill[black] (-0.814576, 0.580057) circle (0.2pt);
\fill[black] (-0.866025, 0.500000) circle (0.2pt);
\fill[black] (-0.909632, 0.415415) circle (0.2pt);
\fill[black] (-0.989821, -0.142315) circle (0.2pt);
\fill[black] (-0.971812, -0.235759) circle (0.2pt);
\fill[black] (-0.945001, -0.327068) circle (0.2pt);
\fill[black] (-0.755750, -0.654861) circle (0.2pt);
\fill[black] (-0.690079, -0.723734) circle (0.2pt);
\fill[black] (-0.909632, -0.415415) circle (0.2pt);
\fill[black] (-0.866025, -0.500000) circle (0.2pt);
\fill[black] (-0.814576, -0.580057) circle (0.2pt);
\fill[black] (-0.371662, -0.928368) circle (0.2pt);
\fill[black] (-0.281733, -0.959493) circle (0.2pt);
\fill[black] (-0.189251, -0.981929) circle (0.2pt);
\fill[black] (0.189251, -0.981929) circle (0.2pt);
\fill[black] (0.281733, -0.959493) circle (0.2pt);
\fill[black] (-0.095056, -0.995472) circle (0.2pt);
\fill[black] (-0.000000, -1.000000) circle (0.2pt);
\fill[black] (0.095056, -0.995472) circle (0.2pt);
\fill[black] (0.618159, -0.786053) circle (0.2pt);
\fill[black] (0.690079, -0.723734) circle (0.2pt);
\fill[black] (0.755750, -0.654861) circle (0.2pt);
\fill[black] (0.945001, -0.327068) circle (0.2pt);
\fill[black] (0.971812, -0.235759) circle (0.2pt);
\fill[black] (0.814576, -0.580057) circle (0.2pt);
\fill[black] (0.866025, -0.500000) circle (0.2pt);
\fill[black] (0.909632, -0.415415) circle (0.2pt);

\node at (-0.534718, 0.343535) {5};
\node at (-0.321307, 0.725920) {5};
\node at (-0.571273, 0.645673) {5};
\node at (-0.824672, 0.251338) {5};
\node at (-0.793813, -0.009354) {5};
\node at (-0.211296, 0.135769) {8};
\node at (0.211273, -0.135787) {8};
\node at (0.534625, -0.343637) {5};
\node at (0.321242, -0.725937) {5};
\node at (0.571198, -0.645726) {5};
\node at (0.824637, -0.251416) {5};
\node at (0.793807, 0.009314) {5};
\node at (0.833101, 0.403069) {8};
\node[anchor=210] at (0.858197, 0.495480) {5};
\node at (0.056096, 0.923559) {8};
\node[anchor=270] at (0.000000, 0.990960) {5};
\node at (-0.783521, 0.524690) {8};
\node[anchor=330] at (-0.858197, 0.495480) {5};
\node at (-0.833103, -0.403068) {8};
\node[anchor= 30] at (-0.858197, -0.495480) {5};
\node at (-0.056103, -0.923562) {8};
\node[anchor= 90] at (-0.000000, -0.990960) {5};
\node at (0.783520, -0.524691) {8};
\node[anchor=150] at (0.858197, -0.495480) {5};

        \end{scope}
        &
        \begin{scope}[scale=3, yshift=25]
          \documentclass[a4paper]{article}
\usepackage[left=0.5cm, right=0.5cm, top=1cm, bottom=1cm]{geometry}
\usepackage{tikz}
\begin{document}
\begin{figure}[h!]
\centering
\begin{tikzpicture}[scale = 10, auto]
\coordinate (x0) at (0.378255, -0.201787);
\coordinate (x1) at (0.296490, -0.216703);
\coordinate (x2) at (0.219159, -0.241103);
\coordinate (x3) at (0.236254, -0.266664);
\coordinate (x4) at (0.307342, -0.236173);
\coordinate (x5) at (0.478413, -0.176913);
\coordinate (x6) at (0.587066, -0.113444);
\coordinate (x7) at (0.541404, 0.101799);
\coordinate (x8) at (0.161601, -0.195378);
\coordinate (x9) at (0.145731, -0.271603);
\coordinate (x10) at (0.330568, -0.275513);
\coordinate (x11) at (0.419119, -0.256129);
\coordinate (x12) at (0.175670, -0.312229);
\coordinate (x13) at (0.250926, -0.312899);
\coordinate (x14) at (0.091374, -0.348932);
\coordinate (x15) at (0.006254, -0.427117);
\coordinate (x16) at (0.226680, -0.379633);
\coordinate (x17) at (0.447573, -0.339481);
\coordinate (x18) at (0.671582, -0.313389);
\coordinate (x19) at (-0.218643, -0.491234);
\coordinate (x20) at (-0.623490, -0.781831);
\coordinate (x21) at (0.222521, -0.974928);
\coordinate (x22) at (0.900969, -0.433884);
\coordinate (x23) at (0.900969, 0.433884);
\coordinate (x24) at (0.222521, 0.974928);
\coordinate (x25) at (-0.623490, 0.781831);
\coordinate (x26) at (-1.000000, 0.000000);
\draw (0.378255, -0.201787) -- (0.296490, -0.216703);
\draw (0.296490, -0.216703) -- (0.219159, -0.241103);
\draw (0.219159, -0.241103) -- (0.236254, -0.266664);
\draw (0.236254, -0.266664) -- (0.307342, -0.236173);
\draw (0.307342, -0.236173) -- (0.378255, -0.201787);
\draw (0.378255, -0.201787) -- (0.478413, -0.176913);
\draw (0.478413, -0.176913) -- (0.587066, -0.113444);
\draw (0.587066, -0.113444) -- (0.541404, 0.101799);
\draw (0.541404, 0.101799) -- (0.161601, -0.195378);
\draw[very thick] (0.161601, -0.195378) -- (0.145731, -0.271603);
\draw[very thick] (0.145731, -0.271603) -- (0.219159, -0.241103);
\draw (0.307342, -0.236173) -- (0.330568, -0.275513);
\draw (0.330568, -0.275513) -- (0.419119, -0.256129);
\draw (0.419119, -0.256129) -- (0.478413, -0.176913);
\draw (0.236254, -0.266664) -- (0.175670, -0.312229);
\draw (0.175670, -0.312229) -- (0.250926, -0.312899);
\draw (0.250926, -0.312899) -- (0.330568, -0.275513);
\draw (0.145731, -0.271603) -- (0.091374, -0.348932);
\draw (0.091374, -0.348932) -- (0.175670, -0.312229);
\draw (0.091374, -0.348932) -- (0.006254, -0.427117);
\draw (0.006254, -0.427117) -- (0.226680, -0.379633);
\draw (0.226680, -0.379633) -- (0.250926, -0.312899);
\draw (0.226680, -0.379633) -- (0.447573, -0.339481);
\draw (0.447573, -0.339481) -- (0.419119, -0.256129);
\draw (0.447573, -0.339481) -- (0.671582, -0.313389);
\draw[very thick] (0.671582, -0.313389) -- (0.587066, -0.113444);
\draw (0.161601, -0.195378) -- (-0.218643, -0.491234);
\draw (-0.218643, -0.491234) -- (0.006254, -0.427117);
\draw (-0.218643, -0.491234) -- (-0.623490, -0.781831);
\draw (-0.623490, -0.781831) -- (0.222521, -0.974928);
\draw (0.222521, -0.974928) -- (0.900969, -0.433884);
\draw (0.900969, -0.433884) -- (0.671582, -0.313389);
\draw (0.900969, -0.433884) -- (0.900969, 0.433884);
\draw (0.900969, 0.433884) -- (0.541404, 0.101799);
\draw (0.900969, 0.433884) -- (0.222521, 0.974928);
\draw (0.222521, 0.974928) -- (-0.623490, 0.781831);
\draw (-0.623490, 0.781831) -- (-1.000000, 0.000000);
\draw (-1.000000, 0.000000) -- (-0.623490, -0.781831);
\node at (-0.000000, 0.000000) {0};
\node at (0.287500, -0.232486) {1};
\node at (0.351015, -0.164392) {2};
\node at (0.382740, -0.229303) {3};
\node at (0.260152, -0.280696) {4};
\node at (0.173638, -0.288106) {5};
\node at (0.150181, -0.356162) {6};
\node at (0.334974, -0.312731) {7};
\node at (0.520751, -0.239871) {8};
\node at (0.037264, -0.346853) {9};
\node at (0.204181, -0.517687) {10};
\node at (0.720398, -0.065007) {11};
\node at (-0.079891, 0.103000) {12};
\end{tikzpicture}
\caption{Graph}

\end{figure}
\end{document}

        \end{scope}
        \\
      };
    \end{tikzfigure}
  \end{proof}
\end{theorem}


% \begin{lemma}\label{thm:expansion:patch:4:7}
%   There are an expansion $3$-patch with outer tuple $o = (1, 1, 1, 2, 2, 2)$ and a corresponding $o$-$6$-gonal $3$-patch consisting only of $4$-gons and $4k + 7$-gons for all $k \in \nats$.
%   \begin{proof}
%     Using \autoref{const:edge:replacement:4:2} on the edges drawn thick in both $3$-patches in \autoref{fig:expansion:patch:4:7} results in $3$-patches consisting only $n$-gons with the specified amount of vertices. The left patch is an expansion patch with outer tuple $o$, while the right patch is $o$-$6$-gonal, these properties do not change when using \autoref{const:edge:replacement:4:2}.
%     \begin{tikzfigure}{\label{fig:expansion:patch:4:7}}{}
%       \matrix (m) [column sep=1cm] {
%         \begin{scope}[scale=1.2]
%           \draw (0.5, 2.5) -- (-0.5, 2) -- (-1, 1) -- (-0.1, 0) -- (1.1, 0) -- (2, 1) -- (1.5, 2);
%           \draw[very thick] (0.5, 2.5) -- (1.5, 2);
%           \draw (0.5, 2.5) -- (0, 3.5) -- (0.9, 4.5) -- (2.1, 4.5) -- (3, 3.5) -- (2.5, 2.5) -- (1.5, 2);
%           \draw (2, 1) -- (3, 1.5) -- (2.5, 2.5);
%           \node[anchor=90] at (-0.1, 0) {$i_0$};
%           \node[anchor=90] at (1.2, 0) {$i'_0$};
%           \node[anchor=120] at (2, 1) {$i'_1$};
%           \node[anchor=180] at (3, 1.5) {$i'_2 = o_7$};
%           \node[anchor=180] at (2.5, 2.5) {$o_6$};
%           \node[anchor=180] at (3, 3.5) {$o_5$};
%           \node[anchor=-120] at (2.1, 4.5) {$o_4$};
%           \node[anchor=-60] at (0.9, 4.5) {$o_3$};
%           \node[anchor=0] at (0, 3.5) {$\mathbf{o_2}$};
%           \node[anchor=-20] at (0.5, 2.5) {$o_{1}$};
%           \node[anchor=0] at (-0.5, 2) {$i_2 = o_{0}$};
%           \node[anchor=45] at (-1, 1) {$i_1$};
%         \end{scope}
%         &
%         \begin{scope}[scale=3, yshift=25]
%           \coordinate (x0) at (-0.636833, -0.430484);
\coordinate (x1) at (-0.648324, -0.371449);
\coordinate (x2) at (-0.685325, -0.322172);
\coordinate (x3) at (-0.727924, -0.392224);
\coordinate (x4) at (-0.552087, -0.487859);
\coordinate (x5) at (-0.612740, -0.361879);
\coordinate (x6) at (-0.673200, -0.235415);
\coordinate (x7) at (-0.724186, -0.094568);
\coordinate (x8) at (-0.997204, 0.074730);
\coordinate (x9) at (-0.997204, -0.074730);
\coordinate (x10) at (-0.834593, -0.402435);
\coordinate (x11) at (-0.680173, -0.733052);
\coordinate (x12) at (-0.563320, -0.826239);
\coordinate (x13) at (-0.466831, -0.616557);
\coordinate (x14) at (-0.320629, -0.450392);
\coordinate (x15) at (-0.447803, -0.240925);
\coordinate (x16) at (-0.082005, -0.522040);
\coordinate (x17) at (-0.179768, -0.265015);
\coordinate (x18) at (-0.023974, -0.035278);
\coordinate (x19) at (-0.478520, 0.314020);
\coordinate (x20) at (-0.974928, 0.222521);
\coordinate (x21) at (-0.433884, -0.900969);
\coordinate (x22) at (-0.054646, -0.897083);
\coordinate (x23) at (0.135465, -0.756001);
\coordinate (x24) at (0.433884, -0.900969);
\coordinate (x25) at (0.563320, -0.826239);
\coordinate (x26) at (0.467152, -0.330729);
\coordinate (x27) at (0.294755, -0.955573);
\coordinate (x28) at (0.974928, -0.222521);
\coordinate (x29) at (0.997204, -0.074730);
\coordinate (x30) at (-0.433884, 0.900969);
\coordinate (x31) at (-0.563320, 0.826239);
\coordinate (x32) at (0.997204, 0.074730);
\coordinate (x33) at (0.741454, 0.422017);
\coordinate (x34) at (0.448614, 0.657866);
\coordinate (x35) at (0.122004, 0.844377);
\coordinate (x36) at (-0.294755, 0.955573);
\coordinate (x37) at (0.680173, 0.733052);
\coordinate (x38) at (0.563320, 0.826239);
\coordinate (x39) at (0.433884, 0.900969);
\coordinate (x40) at (0.294755, 0.955573);
\coordinate (x41) at (0.149042, 0.988831);
\coordinate (x42) at (0.000000, 1.000000);
\coordinate (x43) at (-0.149042, 0.988831);
\coordinate (x44) at (-0.680173, 0.733052);
\coordinate (x45) at (-0.781831, 0.623490);
\coordinate (x46) at (-0.866025, 0.500000);
\coordinate (x47) at (-0.930874, 0.365341);
\coordinate (x48) at (-0.974928, -0.222521);
\coordinate (x49) at (-0.930874, -0.365341);
\coordinate (x50) at (-0.866025, -0.500000);
\coordinate (x51) at (-0.781831, -0.623490);
\coordinate (x52) at (-0.294755, -0.955573);
\coordinate (x53) at (-0.149042, -0.988831);
\coordinate (x54) at (-0.000000, -1.000000);
\coordinate (x55) at (0.149042, -0.988831);
\coordinate (x56) at (0.680173, -0.733052);
\coordinate (x57) at (0.781831, -0.623490);
\coordinate (x58) at (0.866025, -0.500000);
\coordinate (x59) at (0.930874, -0.365341);
\coordinate (x60) at (0.974928, 0.222521);
\coordinate (x61) at (0.930874, 0.365341);
\coordinate (x62) at (0.866025, 0.500000);
\coordinate (x63) at (0.781831, 0.623490);
\draw (-0.636833, -0.430484) -- (-0.648324, -0.371449);
\draw (-0.648324, -0.371449) -- (-0.685325, -0.322172);
\draw (-0.685325, -0.322172) -- (-0.727924, -0.392224);
\draw (-0.727924, -0.392224) -- (-0.636833, -0.430484);
\draw (-0.636833, -0.430484) -- (-0.552087, -0.487859);
\draw (-0.552087, -0.487859) -- (-0.612740, -0.361879);
\draw (-0.612740, -0.361879) -- (-0.648324, -0.371449);
\draw (-0.612740, -0.361879) -- (-0.673200, -0.235415);
\draw (-0.673200, -0.235415) -- (-0.685325, -0.322172);
\draw (-0.673200, -0.235415) -- (-0.724186, -0.094568);
\draw (-0.724186, -0.094568) -- (-0.997204, 0.074730);
\draw (-0.997204, 0.074730) -- (-0.997204, -0.074730);
\draw[very thick] (-0.997204, -0.074730) -- (-0.834593, -0.402435);
\draw (-0.834593, -0.402435) -- (-0.727924, -0.392224);
\draw (-0.834593, -0.402435) -- (-0.680173, -0.733052);
\draw (-0.680173, -0.733052) -- (-0.563320, -0.826239);
\draw (-0.563320, -0.826239) -- (-0.466831, -0.616557);
\draw[very thick] (-0.466831, -0.616557) -- (-0.552087, -0.487859);
\draw (-0.466831, -0.616557) -- (-0.320629, -0.450392);
\draw (-0.320629, -0.450392) -- (-0.447803, -0.240925);
\draw (-0.447803, -0.240925) -- (-0.724186, -0.094568);
\draw (-0.320629, -0.450392) -- (-0.082005, -0.522040);
\draw (-0.082005, -0.522040) -- (-0.179768, -0.265015);
\draw (-0.179768, -0.265015) -- (-0.447803, -0.240925);
\draw (-0.179768, -0.265015) -- (-0.023974, -0.035278);
\draw (-0.023974, -0.035278) -- (-0.478520, 0.314020);
\draw[very thick] (-0.478520, 0.314020) -- (-0.974928, 0.222521);
\draw (-0.974928, 0.222521) -- (-0.997204, 0.074730);
\draw (-0.563320, -0.826239) -- (-0.433884, -0.900969);
\draw[very thick] (-0.433884, -0.900969) -- (-0.054646, -0.897083);
\draw (-0.054646, -0.897083) -- (0.135465, -0.756001);
\draw (0.135465, -0.756001) -- (-0.082005, -0.522040);
\draw (0.135465, -0.756001) -- (0.433884, -0.900969);
\draw (0.433884, -0.900969) -- (0.563320, -0.826239);
\draw[very thick] (0.563320, -0.826239) -- (0.467152, -0.330729);
\draw (0.467152, -0.330729) -- (-0.023974, -0.035278);
\draw (-0.054646, -0.897083) -- (0.294755, -0.955573);
\draw (0.294755, -0.955573) -- (0.433884, -0.900969);
\draw (0.467152, -0.330729) -- (0.974928, -0.222521);
\draw (0.974928, -0.222521) -- (0.997204, -0.074730);
\draw (0.997204, -0.074730) -- (-0.433884, 0.900969);
\draw (-0.433884, 0.900969) -- (-0.563320, 0.826239);
\draw (-0.563320, 0.826239) -- (-0.478520, 0.314020);
\draw (0.997204, -0.074730) -- (0.997204, 0.074730);
\draw[very thick] (0.997204, 0.074730) -- (0.741454, 0.422017);
\draw (0.741454, 0.422017) -- (0.448614, 0.657866);
\draw (0.448614, 0.657866) -- (0.122004, 0.844377);
\draw (0.122004, 0.844377) -- (-0.294755, 0.955573);
\draw[very thick] (-0.294755, 0.955573) -- (-0.433884, 0.900969);
\draw (0.741454, 0.422017) -- (0.680173, 0.733052);
\draw (0.680173, 0.733052) -- (0.563320, 0.826239);
\draw (0.563320, 0.826239) -- (0.448614, 0.657866);
\draw (0.563320, 0.826239) -- (0.433884, 0.900969);
\draw (0.433884, 0.900969) -- (0.122004, 0.844377);
\draw (0.433884, 0.900969) -- (0.294755, 0.955573);
\draw (0.294755, 0.955573) -- (0.149042, 0.988831);
\draw (0.149042, 0.988831) -- (0.000000, 1.000000);
\draw (0.000000, 1.000000) -- (-0.149042, 0.988831);
\draw (-0.149042, 0.988831) -- (-0.294755, 0.955573);
\draw (-0.563320, 0.826239) -- (-0.680173, 0.733052);
\draw (-0.680173, 0.733052) -- (-0.781831, 0.623490);
\draw (-0.781831, 0.623490) -- (-0.866025, 0.500000);
\draw (-0.866025, 0.500000) -- (-0.930874, 0.365341);
\draw (-0.930874, 0.365341) -- (-0.974928, 0.222521);
\draw (-0.997204, -0.074730) -- (-0.974928, -0.222521);
\draw (-0.974928, -0.222521) -- (-0.930874, -0.365341);
\draw (-0.930874, -0.365341) -- (-0.866025, -0.500000);
\draw (-0.866025, -0.500000) -- (-0.781831, -0.623490);
\draw (-0.781831, -0.623490) -- (-0.680173, -0.733052);
\draw (-0.433884, -0.900969) -- (-0.294755, -0.955573);
\draw (-0.294755, -0.955573) -- (-0.149042, -0.988831);
\draw (-0.149042, -0.988831) -- (-0.000000, -1.000000);
\draw (-0.000000, -1.000000) -- (0.149042, -0.988831);
\draw (0.149042, -0.988831) -- (0.294755, -0.955573);
\draw (0.563320, -0.826239) -- (0.680173, -0.733052);
\draw (0.680173, -0.733052) -- (0.781831, -0.623490);
\draw (0.781831, -0.623490) -- (0.866025, -0.500000);
\draw (0.866025, -0.500000) -- (0.930874, -0.365341);
\draw (0.930874, -0.365341) -- (0.974928, -0.222521);
\draw (0.997204, 0.074730) -- (0.974928, 0.222521);
\draw (0.974928, 0.222521) -- (0.930874, 0.365341);
\draw (0.930874, 0.365341) -- (0.866025, 0.500000);
\draw (0.866025, 0.500000) -- (0.781831, 0.623490);
\draw (0.781831, 0.623490) -- (0.680173, 0.733052);


%         \end{scope}
%         \\
%       };
%     \end{tikzfigure}
%   \end{proof}
% \end{lemma}

% \begin{lemma}\label{thm:expansion:patch:4:9}
%   There are an expansion $3$-patch with outer tuple $o = (1, 1, 1, 1, 1, 2, 2, 2, 2, 2)$ and a corresponding $o$-$6$-gonal $3$-patch consisting only of $4$-gons and $4k + 9$-gons for all $k \in \nats$.
%   \begin{proof}
%     Using \autoref{const:edge:replacement:4:2} on the edges drawn thick in both $3$-patches in \autoref{fig:expansion:patch:4:9} results in $3$-patches consisting only $n$-gons with the specified amount of vertices. The left patch is an expansion patch with outer tuple $o$, while the right patch is $o$-$6$-gonal, these properties do not change when using \autoref{const:edge:replacement:4:2}.
%     \begin{tikzfigure}{\label{fig:expansion:patch:4:9}}{}
%       \matrix (m) [column sep=1cm] {
%         \begin{scope}[scale=0.8]
%           \draw (-0.5, 3) -- (-1, 2) -- (-1, 1) -- (0, 0) -- (1, 0) -- (2, 1) -- (2, 2) -- (1.5, 3) -- (0.5, 3.5);
%           \draw[very thick] (0.5, 3.5) -- (-0.5, 3);
%           \draw (2, 1) -- (3, 1) -- (3, 2) -- (2, 2);
%           \draw (3, 2) -- (2.5, 3) -- (1.5, 3);
%           \draw (2.5, 3) -- (1, 4.5);
%           \draw (0.5, 3.5) -- (1, 4.5) -- (1, 5.5) -- (0, 6.5) -- (-1, 6.5) -- (-2, 5.5) -- (-2, 4.5) -- (-1.5, 3.5) -- (-0.5, 3);
%           \node[anchor=90] at (0, 0) {$i_0$};
%           \node[anchor=90] at (1, 0) {$i'_0$};
%           \node[anchor=120] at (2, 1) {$i'_1$};
%           \node[anchor=180] at (3, 1) {$i'_2 = o_{11}$};
%           \node[anchor=180] at (3, 2) {$o_{10}$};
%           \node[anchor=-150] at (2.5, 3) {$o_9$};
%           \node[anchor=-150] at (1, 4.5) {$o_8$};
%           \node[anchor=-150] at (1, 5.5) {$o_7$};
%           \node[anchor=-110] at (0, 6.5) {$o_6$};
%           \node[anchor=-70] at (-1, 6.5) {$o_5$};
%           \node[anchor=-20] at (-2, 5.5) {$o_4$};
%           \node[anchor=0] at (-2, 4.5) {$o_3$};
%           \node[anchor=60] at (-1.5, 3.5) {$\mathbf{o_2}$};
%           \node[anchor=20] at (-0.5, 3) {$o_{1}$};
%           \node[anchor=0] at (-1, 2) {$i_2 = o_{0}$};
%           \node[anchor=45] at (-1, 1) {$i_1$};
%         \end{scope}
%         &
%         \begin{scope}[scale=3, yshift=25]
%           \coordinate (x0) at (0.575695, -0.197192);
\coordinate (x1) at (0.556573, -0.078916);
\coordinate (x2) at (0.456424, -0.065169);
\coordinate (x3) at (0.411774, -0.321955);
\coordinate (x4) at (0.755417, -0.108068);
\coordinate (x5) at (0.608514, 0.028443);
\coordinate (x6) at (0.486005, 0.193713);
\coordinate (x7) at (0.448511, 0.482721);
\coordinate (x8) at (0.182722, 0.346782);
\coordinate (x9) at (0.092916, 0.112802);
\coordinate (x10) at (0.057548, -0.133347);
\coordinate (x11) at (0.077844, -0.383380);
\coordinate (x12) at (0.294498, -0.589016);
\coordinate (x13) at (-0.057365, 0.134413);
\coordinate (x14) at (-0.092752, -0.111739);
\coordinate (x15) at (-0.182716, -0.345924);
\coordinate (x16) at (-0.077701, 0.384190);
\coordinate (x17) at (-0.294379, 0.589529);
\coordinate (x18) at (-0.411671, 0.322705);
\coordinate (x19) at (-0.456219, 0.065992);
\coordinate (x20) at (-0.485771, -0.192942);
\coordinate (x21) at (-0.448469, -0.482209);
\coordinate (x22) at (-0.556027, 0.081036);
\coordinate (x23) at (-0.607771, -0.026861);
\coordinate (x24) at (-0.575798, 0.198657);
\coordinate (x25) at (-0.755220, 0.109252);
\coordinate (x26) at (-0.989821, 0.142315);
\coordinate (x27) at (-0.998867, 0.047582);
\coordinate (x28) at (-0.844574, -0.338160);
\coordinate (x29) at (-0.690079, -0.723734);
\coordinate (x30) at (-0.618159, -0.786053);
\coordinate (x31) at (-0.540641, -0.841254);
\coordinate (x32) at (-0.129512, -0.900491);
\coordinate (x33) at (0.281733, -0.959493);
\coordinate (x34) at (0.371662, -0.928368);
\coordinate (x35) at (0.458227, -0.888835);
\coordinate (x36) at (0.715092, -0.562406);
\coordinate (x37) at (0.971812, -0.235759);
\coordinate (x38) at (0.989821, -0.142315);
\coordinate (x39) at (0.998867, -0.047582);
\coordinate (x40) at (0.844604, 0.338085);
\coordinate (x41) at (0.690079, 0.723734);
\coordinate (x42) at (0.618159, 0.786053);
\coordinate (x43) at (0.540641, 0.841254);
\coordinate (x44) at (0.129432, 0.900502);
\coordinate (x45) at (-0.281733, 0.959493);
\coordinate (x46) at (-0.371662, 0.928368);
\coordinate (x47) at (-0.458227, 0.888835);
\coordinate (x48) at (-0.715142, 0.562342);
\coordinate (x49) at (-0.971812, 0.235759);
\coordinate (x50) at (-0.540641, 0.841254);
\coordinate (x51) at (-0.742622, 0.584088);
\coordinate (x52) at (-0.945001, 0.327068);
\coordinate (x53) at (-0.998867, -0.047582);
\coordinate (x54) at (-0.877098, -0.351207);
\coordinate (x55) at (-0.755750, -0.654861);
\coordinate (x56) at (-0.458227, -0.888835);
\coordinate (x57) at (-0.134523, -0.935174);
\coordinate (x58) at (0.189251, -0.981929);
\coordinate (x59) at (0.540641, -0.841254);
\coordinate (x60) at (0.742703, -0.583985);
\coordinate (x61) at (0.945001, -0.327068);
\coordinate (x62) at (0.998867, 0.047582);
\coordinate (x63) at (0.877146, 0.351086);
\coordinate (x64) at (0.755750, 0.654861);
\coordinate (x65) at (0.134394, 0.935192);
\coordinate (x66) at (-0.189251, 0.981929);
\coordinate (x67) at (0.458227, 0.888835);
\coordinate (x68) at (0.371662, 0.928368);
\coordinate (x69) at (0.281733, 0.959493);
\coordinate (x70) at (0.189251, 0.981929);
\coordinate (x71) at (0.095056, 0.995472);
\coordinate (x72) at (0.000000, 1.000000);
\coordinate (x73) at (-0.095056, 0.995472);
\coordinate (x74) at (-0.618159, 0.786053);
\coordinate (x75) at (-0.690079, 0.723734);
\coordinate (x76) at (-0.755750, 0.654861);
\coordinate (x77) at (-0.814576, 0.580057);
\coordinate (x78) at (-0.866025, 0.500000);
\coordinate (x79) at (-0.909632, 0.415415);
\coordinate (x80) at (-0.989821, -0.142315);
\coordinate (x81) at (-0.971812, -0.235759);
\coordinate (x82) at (-0.945001, -0.327068);
\coordinate (x83) at (-0.909632, -0.415415);
\coordinate (x84) at (-0.866025, -0.500000);
\coordinate (x85) at (-0.814576, -0.580057);
\coordinate (x86) at (-0.371662, -0.928368);
\coordinate (x87) at (-0.281733, -0.959493);
\coordinate (x88) at (-0.189251, -0.981929);
\coordinate (x89) at (-0.095056, -0.995472);
\coordinate (x90) at (-0.000000, -1.000000);
\coordinate (x91) at (0.095056, -0.995472);
\coordinate (x92) at (0.618159, -0.786053);
\coordinate (x93) at (0.690079, -0.723734);
\coordinate (x94) at (0.755750, -0.654861);
\coordinate (x95) at (0.814576, -0.580057);
\coordinate (x96) at (0.866025, -0.500000);
\coordinate (x97) at (0.909632, -0.415415);
\coordinate (x98) at (0.989821, 0.142315);
\coordinate (x99) at (0.971812, 0.235759);
\coordinate (x100) at (0.945001, 0.327068);
\coordinate (x101) at (0.909632, 0.415415);
\coordinate (x102) at (0.866025, 0.500000);
\coordinate (x103) at (0.814576, 0.580057);
\draw (0.575695, -0.197192) -- (0.556573, -0.078916);
\draw (0.556573, -0.078916) -- (0.456424, -0.065169);
\draw (0.456424, -0.065169) -- (0.411774, -0.321955);
\draw (0.411774, -0.321955) -- (0.575695, -0.197192);
\draw (0.575695, -0.197192) -- (0.755417, -0.108068);
\draw (0.755417, -0.108068) -- (0.608514, 0.028443);
\draw (0.608514, 0.028443) -- (0.556573, -0.078916);
\draw (0.608514, 0.028443) -- (0.486005, 0.193713);
\draw (0.486005, 0.193713) -- (0.456424, -0.065169);
\draw (0.486005, 0.193713) -- (0.448511, 0.482721);
\draw (0.448511, 0.482721) -- (0.182722, 0.346782);
\draw (0.182722, 0.346782) -- (0.092916, 0.112802);
\draw (0.092916, 0.112802) -- (0.057548, -0.133347);
\draw (0.057548, -0.133347) -- (0.077844, -0.383380);
\draw (0.077844, -0.383380) -- (0.294498, -0.589016);
\draw (0.294498, -0.589016) -- (0.411774, -0.321955);
\draw (0.092916, 0.112802) -- (-0.057365, 0.134413);
\draw (-0.057365, 0.134413) -- (-0.092752, -0.111739);
\draw (-0.092752, -0.111739) -- (0.057548, -0.133347);
\draw (-0.092752, -0.111739) -- (-0.182716, -0.345924);
\draw (-0.182716, -0.345924) -- (0.077844, -0.383380);
\draw (0.182722, 0.346782) -- (-0.077701, 0.384190);
\draw (-0.077701, 0.384190) -- (-0.057365, 0.134413);
\draw (-0.077701, 0.384190) -- (-0.294379, 0.589529);
\draw (-0.294379, 0.589529) -- (-0.411671, 0.322705);
\draw (-0.411671, 0.322705) -- (-0.456219, 0.065992);
\draw (-0.456219, 0.065992) -- (-0.485771, -0.192942);
\draw (-0.485771, -0.192942) -- (-0.448469, -0.482209);
\draw (-0.448469, -0.482209) -- (-0.182716, -0.345924);
\draw (-0.456219, 0.065992) -- (-0.556027, 0.081036);
\draw (-0.556027, 0.081036) -- (-0.607771, -0.026861);
\draw (-0.607771, -0.026861) -- (-0.485771, -0.192942);
\draw (-0.411671, 0.322705) -- (-0.575798, 0.198657);
\draw (-0.575798, 0.198657) -- (-0.556027, 0.081036);
\draw (-0.575798, 0.198657) -- (-0.755220, 0.109252);
\draw (-0.755220, 0.109252) -- (-0.607771, -0.026861);
\draw (-0.755220, 0.109252) -- (-0.989821, 0.142315);
\draw (-0.989821, 0.142315) -- (-0.998867, 0.047582);
\draw (-0.998867, 0.047582) -- (-0.844574, -0.338160);
\draw (-0.844574, -0.338160) -- (-0.690079, -0.723734);
\draw (-0.690079, -0.723734) -- (-0.618159, -0.786053);
\draw (-0.618159, -0.786053) -- (-0.448469, -0.482209);
\draw (-0.618159, -0.786053) -- (-0.540641, -0.841254);
\draw (-0.540641, -0.841254) -- (-0.129512, -0.900491);
\draw (-0.129512, -0.900491) -- (0.281733, -0.959493);
\draw (0.281733, -0.959493) -- (0.371662, -0.928368);
\draw (0.371662, -0.928368) -- (0.294498, -0.589016);
\draw (0.371662, -0.928368) -- (0.458227, -0.888835);
\draw (0.458227, -0.888835) -- (0.715092, -0.562406);
\draw (0.715092, -0.562406) -- (0.971812, -0.235759);
\draw (0.971812, -0.235759) -- (0.989821, -0.142315);
\draw (0.989821, -0.142315) -- (0.755417, -0.108068);
\draw (0.989821, -0.142315) -- (0.998867, -0.047582);
\draw (0.998867, -0.047582) -- (0.844604, 0.338085);
\draw (0.844604, 0.338085) -- (0.690079, 0.723734);
\draw (0.690079, 0.723734) -- (0.618159, 0.786053);
\draw (0.618159, 0.786053) -- (0.448511, 0.482721);
\draw (0.618159, 0.786053) -- (0.540641, 0.841254);
\draw (0.540641, 0.841254) -- (0.129432, 0.900502);
\draw (0.129432, 0.900502) -- (-0.281733, 0.959493);
\draw (-0.281733, 0.959493) -- (-0.371662, 0.928368);
\draw (-0.371662, 0.928368) -- (-0.294379, 0.589529);
\draw (-0.371662, 0.928368) -- (-0.458227, 0.888835);
\draw (-0.458227, 0.888835) -- (-0.715142, 0.562342);
\draw (-0.715142, 0.562342) -- (-0.971812, 0.235759);
\draw (-0.971812, 0.235759) -- (-0.989821, 0.142315);
\draw (-0.458227, 0.888835) -- (-0.540641, 0.841254);
\draw (-0.540641, 0.841254) -- (-0.742622, 0.584088);
\draw (-0.742622, 0.584088) -- (-0.715142, 0.562342);
\draw (-0.742622, 0.584088) -- (-0.945001, 0.327068);
\draw (-0.945001, 0.327068) -- (-0.971812, 0.235759);
\draw (-0.998867, 0.047582) -- (-0.998867, -0.047582);
\draw (-0.998867, -0.047582) -- (-0.877098, -0.351207);
\draw (-0.877098, -0.351207) -- (-0.844574, -0.338160);
\draw (-0.877098, -0.351207) -- (-0.755750, -0.654861);
\draw (-0.755750, -0.654861) -- (-0.690079, -0.723734);
\draw (-0.540641, -0.841254) -- (-0.458227, -0.888835);
\draw (-0.458227, -0.888835) -- (-0.134523, -0.935174);
\draw (-0.134523, -0.935174) -- (-0.129512, -0.900491);
\draw (-0.134523, -0.935174) -- (0.189251, -0.981929);
\draw (0.189251, -0.981929) -- (0.281733, -0.959493);
\draw (0.458227, -0.888835) -- (0.540641, -0.841254);
\draw (0.540641, -0.841254) -- (0.742703, -0.583985);
\draw (0.742703, -0.583985) -- (0.715092, -0.562406);
\draw (0.742703, -0.583985) -- (0.945001, -0.327068);
\draw (0.945001, -0.327068) -- (0.971812, -0.235759);
\draw (0.998867, -0.047582) -- (0.998867, 0.047582);
\draw (0.998867, 0.047582) -- (0.877146, 0.351086);
\draw (0.877146, 0.351086) -- (0.844604, 0.338085);
\draw (0.877146, 0.351086) -- (0.755750, 0.654861);
\draw (0.755750, 0.654861) -- (0.690079, 0.723734);
\draw (0.129432, 0.900502) -- (0.134394, 0.935192);
\draw (0.134394, 0.935192) -- (-0.189251, 0.981929);
\draw (-0.189251, 0.981929) -- (-0.281733, 0.959493);
\draw (0.540641, 0.841254) -- (0.458227, 0.888835);
\draw (0.458227, 0.888835) -- (0.134394, 0.935192);
\draw (0.458227, 0.888835) -- (0.371662, 0.928368);
\draw (0.371662, 0.928368) -- (0.281733, 0.959493);
\draw (0.281733, 0.959493) -- (0.189251, 0.981929);
\draw (0.189251, 0.981929) -- (0.095056, 0.995472);
\draw (0.095056, 0.995472) -- (0.000000, 1.000000);
\draw (0.000000, 1.000000) -- (-0.095056, 0.995472);
\draw (-0.095056, 0.995472) -- (-0.189251, 0.981929);
\draw (-0.540641, 0.841254) -- (-0.618159, 0.786053);
\draw (-0.618159, 0.786053) -- (-0.690079, 0.723734);
\draw (-0.690079, 0.723734) -- (-0.755750, 0.654861);
\draw (-0.755750, 0.654861) -- (-0.814576, 0.580057);
\draw (-0.814576, 0.580057) -- (-0.866025, 0.500000);
\draw (-0.866025, 0.500000) -- (-0.909632, 0.415415);
\draw (-0.909632, 0.415415) -- (-0.945001, 0.327068);
\draw (-0.998867, -0.047582) -- (-0.989821, -0.142315);
\draw (-0.989821, -0.142315) -- (-0.971812, -0.235759);
\draw (-0.971812, -0.235759) -- (-0.945001, -0.327068);
\draw (-0.945001, -0.327068) -- (-0.909632, -0.415415);
\draw (-0.909632, -0.415415) -- (-0.866025, -0.500000);
\draw (-0.866025, -0.500000) -- (-0.814576, -0.580057);
\draw (-0.814576, -0.580057) -- (-0.755750, -0.654861);
\draw (-0.458227, -0.888835) -- (-0.371662, -0.928368);
\draw (-0.371662, -0.928368) -- (-0.281733, -0.959493);
\draw (-0.281733, -0.959493) -- (-0.189251, -0.981929);
\draw (-0.189251, -0.981929) -- (-0.095056, -0.995472);
\draw (-0.095056, -0.995472) -- (-0.000000, -1.000000);
\draw (-0.000000, -1.000000) -- (0.095056, -0.995472);
\draw (0.095056, -0.995472) -- (0.189251, -0.981929);
\draw (0.540641, -0.841254) -- (0.618159, -0.786053);
\draw (0.618159, -0.786053) -- (0.690079, -0.723734);
\draw (0.690079, -0.723734) -- (0.755750, -0.654861);
\draw (0.755750, -0.654861) -- (0.814576, -0.580057);
\draw (0.814576, -0.580057) -- (0.866025, -0.500000);
\draw (0.866025, -0.500000) -- (0.909632, -0.415415);
\draw (0.909632, -0.415415) -- (0.945001, -0.327068);
\draw (0.998867, 0.047582) -- (0.989821, 0.142315);
\draw (0.989821, 0.142315) -- (0.971812, 0.235759);
\draw (0.971812, 0.235759) -- (0.945001, 0.327068);
\draw (0.945001, 0.327068) -- (0.909632, 0.415415);
\draw (0.909632, 0.415415) -- (0.866025, 0.500000);
\draw (0.866025, 0.500000) -- (0.814576, 0.580057);
\draw (0.814576, 0.580057) -- (0.755750, 0.654861);

\fill[black] (0.575695, -0.197192) circle (0.4pt);
\fill[black] (0.556573, -0.078916) circle (0.4pt);
\fill[black] (0.456424, -0.065169) circle (0.4pt);
\fill[black] (0.411774, -0.321955) circle (0.4pt);
\fill[black] (0.755417, -0.108068) circle (0.4pt);
\fill[black] (0.608514, 0.028443) circle (0.4pt);
\fill[black] (0.486005, 0.193713) circle (0.4pt);
\fill[black] (0.448511, 0.482721) circle (0.4pt);
\fill[black] (0.182722, 0.346782) circle (0.4pt);
\fill[black] (0.092916, 0.112802) circle (0.4pt);
\fill[black] (0.057548, -0.133347) circle (0.4pt);
\fill[black] (0.077844, -0.383380) circle (0.4pt);
\fill[black] (0.294498, -0.589016) circle (0.4pt);
\fill[black] (-0.057365, 0.134413) circle (0.4pt);
\fill[black] (-0.092752, -0.111739) circle (0.4pt);
\fill[black] (-0.182716, -0.345924) circle (0.4pt);
\fill[black] (-0.077701, 0.384190) circle (0.4pt);
\fill[black] (-0.294379, 0.589529) circle (0.4pt);
\fill[black] (-0.411671, 0.322705) circle (0.4pt);
\fill[black] (-0.456219, 0.065992) circle (0.4pt);
\fill[black] (-0.485771, -0.192942) circle (0.4pt);
\fill[black] (-0.448469, -0.482209) circle (0.4pt);
\fill[black] (-0.556027, 0.081036) circle (0.4pt);
\fill[black] (-0.607771, -0.026861) circle (0.4pt);
\fill[black] (-0.575798, 0.198657) circle (0.4pt);
\fill[black] (-0.755220, 0.109252) circle (0.4pt);
\fill[black] (-0.989821, 0.142315) circle (0.4pt);
\fill[black] (-0.998867, 0.047582) circle (0.4pt);
\fill[black] (-0.844574, -0.338160) circle (0.4pt);
\fill[black] (-0.690079, -0.723734) circle (0.4pt);
\fill[black] (-0.618159, -0.786053) circle (0.4pt);
\fill[black] (-0.540641, -0.841254) circle (0.4pt);
\fill[black] (-0.129512, -0.900491) circle (0.4pt);
\fill[black] (0.281733, -0.959493) circle (0.4pt);
\fill[black] (0.371662, -0.928368) circle (0.4pt);
\fill[black] (0.458227, -0.888835) circle (0.4pt);
\fill[black] (0.715092, -0.562406) circle (0.4pt);
\fill[black] (0.971812, -0.235759) circle (0.4pt);
\fill[black] (0.989821, -0.142315) circle (0.4pt);
\fill[black] (0.998867, -0.047582) circle (0.4pt);
\fill[black] (0.844604, 0.338085) circle (0.4pt);
\fill[black] (0.690079, 0.723734) circle (0.4pt);
\fill[black] (0.618159, 0.786053) circle (0.4pt);
\fill[black] (0.540641, 0.841254) circle (0.4pt);
\fill[black] (0.129432, 0.900502) circle (0.4pt);
\fill[black] (-0.281733, 0.959493) circle (0.4pt);
\fill[black] (-0.371662, 0.928368) circle (0.4pt);
\fill[black] (-0.458227, 0.888835) circle (0.4pt);
\fill[black] (-0.715142, 0.562342) circle (0.4pt);
\fill[black] (-0.971812, 0.235759) circle (0.4pt);
\fill[black] (-0.540641, 0.841254) circle (0.4pt);
\fill[black] (-0.742622, 0.584088) circle (0.4pt);
\fill[black] (-0.945001, 0.327068) circle (0.4pt);
\fill[black] (-0.998867, -0.047582) circle (0.4pt);
\fill[black] (-0.877098, -0.351207) circle (0.4pt);
\fill[black] (-0.755750, -0.654861) circle (0.4pt);
\fill[black] (-0.458227, -0.888835) circle (0.4pt);
\fill[black] (-0.134523, -0.935174) circle (0.4pt);
\fill[black] (0.189251, -0.981929) circle (0.4pt);
\fill[black] (0.540641, -0.841254) circle (0.4pt);
\fill[black] (0.742703, -0.583985) circle (0.4pt);
\fill[black] (0.945001, -0.327068) circle (0.4pt);
\fill[black] (0.998867, 0.047582) circle (0.4pt);
\fill[black] (0.877146, 0.351086) circle (0.4pt);
\fill[black] (0.755750, 0.654861) circle (0.4pt);
\fill[black] (0.134394, 0.935192) circle (0.4pt);
\fill[black] (-0.189251, 0.981929) circle (0.4pt);
\fill[black] (0.458227, 0.888835) circle (0.4pt);
\fill[black] (0.371662, 0.928368) circle (0.4pt);
\fill[black] (0.281733, 0.959493) circle (0.4pt);
\fill[black] (0.189251, 0.981929) circle (0.4pt);
\fill[black] (0.095056, 0.995472) circle (0.4pt);
\fill[black] (0.000000, 1.000000) circle (0.4pt);
\fill[black] (-0.095056, 0.995472) circle (0.4pt);
\fill[black] (-0.618159, 0.786053) circle (0.4pt);
\fill[black] (-0.690079, 0.723734) circle (0.4pt);
\fill[black] (-0.755750, 0.654861) circle (0.4pt);
\fill[black] (-0.814576, 0.580057) circle (0.4pt);
\fill[black] (-0.866025, 0.500000) circle (0.4pt);
\fill[black] (-0.909632, 0.415415) circle (0.4pt);
\fill[black] (-0.989821, -0.142315) circle (0.4pt);
\fill[black] (-0.971812, -0.235759) circle (0.4pt);
\fill[black] (-0.945001, -0.327068) circle (0.4pt);
\fill[black] (-0.909632, -0.415415) circle (0.4pt);
\fill[black] (-0.866025, -0.500000) circle (0.4pt);
\fill[black] (-0.814576, -0.580057) circle (0.4pt);
\fill[black] (-0.371662, -0.928368) circle (0.4pt);
\fill[black] (-0.281733, -0.959493) circle (0.4pt);
\fill[black] (-0.189251, -0.981929) circle (0.4pt);
\fill[black] (-0.095056, -0.995472) circle (0.4pt);
\fill[black] (-0.000000, -1.000000) circle (0.4pt);
\fill[black] (0.095056, -0.995472) circle (0.4pt);
\fill[black] (0.618159, -0.786053) circle (0.4pt);
\fill[black] (0.690079, -0.723734) circle (0.4pt);
\fill[black] (0.755750, -0.654861) circle (0.4pt);
\fill[black] (0.814576, -0.580057) circle (0.4pt);
\fill[black] (0.866025, -0.500000) circle (0.4pt);
\fill[black] (0.909632, -0.415415) circle (0.4pt);
\fill[black] (0.989821, 0.142315) circle (0.4pt);
\fill[black] (0.971812, 0.235759) circle (0.4pt);
\fill[black] (0.945001, 0.327068) circle (0.4pt);
\fill[black] (0.909632, 0.415415) circle (0.4pt);
\fill[black] (0.866025, 0.500000) circle (0.4pt);
\fill[black] (0.814576, 0.580057) circle (0.4pt);

%         \end{scope}
%         \\
%       };
%     \end{tikzfigure}
%   \end{proof}
% \end{lemma}

% \begin{lemma}\label{thm:expansion:patch:4:10}
%   There are an expansion $3$-patch with outer tuple $o = (1, 1, 2, 1, 2, 2)$ and a corresponding $o$-$6$-gonal $3$-patch consisting only of $4$-gons and $4k + 10$-gons for all $k \in \nats$.
%   \begin{proof}
%     Using \autoref{const:edge:replacement:4:2} on the edges drawn thick in both $3$-patches in \autoref{fig:expansion:patch:4:10} results in $3$-patches consisting only $n$-gons with the specified amount of vertices. The left patch is an expansion patch with outer tuple $o$, while the right patch is $o$-$6$-gonal, these properties do not change when using \autoref{const:edge:replacement:4:2}.
%     \begin{tikzfigure}{\label{fig:expansion:patch:4:10}}{\todo{better picture}}
%       \matrix (m) [column sep=1cm] {
%         \begin{scope}[scale=3]
%           \coordinate (x0) at (0.900969, -0.433884);
\coordinate (x1) at (0.529927, 0.030827);
\coordinate (x2) at (0.158789, 0.495608);
\coordinate (x3) at (-0.623490, 0.781831);
\coordinate (x4) at (-1.000000, 0.000000);
\coordinate (x5) at (-0.623490, -0.781831);
\coordinate (x6) at (0.222521, -0.974928);
\coordinate (x7) at (0.900969, 0.433884);
\coordinate (x8) at (0.222521, 0.974928);
\draw (0.900969, -0.433884) -- (0.529927, 0.030827);
\draw (0.529927, 0.030827) -- (0.158789, 0.495608);
\draw (0.158789, 0.495608) -- (-0.623490, 0.781831);
\draw (-0.623490, 0.781831) -- (-1.000000, 0.000000);
\draw (-1.000000, 0.000000) -- (-0.623490, -0.781831);
\draw (-0.623490, -0.781831) -- (0.222521, -0.974928);
\draw (0.222521, -0.974928) -- (0.900969, -0.433884);
\draw (0.900969, -0.433884) -- (0.900969, 0.433884);
\draw (0.900969, 0.433884) -- (0.158789, 0.495608);
\draw (0.900969, 0.433884) -- (0.222521, 0.974928);
\draw (0.222521, 0.974928) -- (-0.623490, 0.781831);

%         \end{scope}
%         &
%         \begin{scope}[scale=3]
%           \coordinate (x0) at (-0.489667, 0.574133);
\coordinate (x1) at (-0.582787, 0.425731);
\coordinate (x2) at (-0.460723, 0.497361);
\coordinate (x3) at (-0.342392, 0.574507);
\coordinate (x4) at (-0.503762, 0.739269);
\coordinate (x5) at (-0.639636, 0.713636);
\coordinate (x6) at (-0.781831, 0.623490);
\coordinate (x7) at (-0.866025, 0.500000);
\coordinate (x8) at (-0.930874, 0.365341);
\coordinate (x9) at (-0.974928, 0.222521);
\coordinate (x10) at (-0.997204, 0.074730);
\coordinate (x11) at (-0.715461, 0.275987);
\coordinate (x12) at (-0.442911, 0.449633);
\coordinate (x13) at (-0.169681, 0.622248);
\coordinate (x14) at (-0.433884, 0.900969);
\coordinate (x15) at (-0.563320, 0.826239);
\coordinate (x16) at (-0.680173, 0.733052);
\coordinate (x17) at (0.125157, 0.791453);
\coordinate (x18) at (0.149042, 0.988831);
\coordinate (x19) at (0.000000, 1.000000);
\coordinate (x20) at (-0.149042, 0.988831);
\coordinate (x21) at (-0.294755, 0.955573);
\coordinate (x22) at (0.433884, 0.900969);
\coordinate (x23) at (0.294755, 0.955573);
\coordinate (x24) at (-0.997204, -0.074730);
\coordinate (x25) at (-0.518526, 0.030249);
\coordinate (x26) at (0.046678, 0.379833);
\coordinate (x27) at (0.563320, 0.826239);
\coordinate (x28) at (-0.974928, -0.222521);
\coordinate (x29) at (-0.930874, -0.365341);
\coordinate (x30) at (-0.866025, -0.500000);
\coordinate (x31) at (-0.781831, -0.623490);
\coordinate (x32) at (-0.680173, -0.733052);
\coordinate (x33) at (-0.563320, -0.826239);
\coordinate (x34) at (-0.278180, -0.603149);
\coordinate (x35) at (-0.018597, -0.327896);
\coordinate (x36) at (0.107571, 0.005772);
\coordinate (x37) at (0.255728, -0.367789);
\coordinate (x38) at (0.381386, -0.119020);
\coordinate (x39) at (-0.433884, -0.900969);
\coordinate (x40) at (-0.273134, -0.819539);
\coordinate (x41) at (-0.294755, -0.955573);
\coordinate (x42) at (-0.149042, -0.988831);
\coordinate (x43) at (-0.000000, -1.000000);
\coordinate (x44) at (0.149042, -0.988831);
\coordinate (x45) at (0.294755, -0.955573);
\coordinate (x46) at (0.433884, -0.900969);
\coordinate (x47) at (0.387886, -0.629052);
\coordinate (x48) at (0.563320, -0.826239);
\coordinate (x49) at (0.583270, -0.671976);
\coordinate (x50) at (0.680173, -0.733052);
\coordinate (x51) at (0.781831, -0.623490);
\coordinate (x52) at (0.866025, -0.500000);
\coordinate (x53) at (0.930874, -0.365341);
\coordinate (x54) at (0.974928, -0.222521);
\coordinate (x55) at (0.997204, -0.074730);
\coordinate (x56) at (0.701510, -0.044622);
\coordinate (x57) at (0.849717, 0.142774);
\coordinate (x58) at (0.930874, 0.365341);
\coordinate (x59) at (0.866025, 0.500000);
\coordinate (x60) at (0.781831, 0.623490);
\coordinate (x61) at (0.680173, 0.733052);
\coordinate (x62) at (0.997204, 0.074730);
\coordinate (x63) at (0.974928, 0.222521);
\draw (-0.489667, 0.574133) -- (-0.582787, 0.425731);
\draw (-0.582787, 0.425731) -- (-0.460723, 0.497361);
\draw (-0.460723, 0.497361) -- (-0.342392, 0.574507);
\draw (-0.342392, 0.574507) -- (-0.489667, 0.574133);
\draw (-0.489667, 0.574133) -- (-0.503762, 0.739269);
\draw (-0.503762, 0.739269) -- (-0.639636, 0.713636);
\draw (-0.639636, 0.713636) -- (-0.781831, 0.623490);
\draw (-0.781831, 0.623490) -- (-0.866025, 0.500000);
\draw (-0.866025, 0.500000) -- (-0.930874, 0.365341);
\draw (-0.930874, 0.365341) -- (-0.974928, 0.222521);
\draw (-0.974928, 0.222521) -- (-0.997204, 0.074730);
\draw (-0.997204, 0.074730) -- (-0.715461, 0.275987);
\draw (-0.715461, 0.275987) -- (-0.582787, 0.425731);
\draw (-0.715461, 0.275987) -- (-0.442911, 0.449633);
\draw (-0.442911, 0.449633) -- (-0.460723, 0.497361);
\draw (-0.442911, 0.449633) -- (-0.169681, 0.622248);
\draw (-0.169681, 0.622248) -- (-0.342392, 0.574507);
\draw (-0.503762, 0.739269) -- (-0.433884, 0.900969);
\draw (-0.433884, 0.900969) -- (-0.563320, 0.826239);
\draw (-0.563320, 0.826239) -- (-0.639636, 0.713636);
\draw (-0.563320, 0.826239) -- (-0.680173, 0.733052);
\draw (-0.680173, 0.733052) -- (-0.781831, 0.623490);
\draw (-0.169681, 0.622248) -- (0.125157, 0.791453);
\draw (0.125157, 0.791453) -- (0.149042, 0.988831);
\draw (0.149042, 0.988831) -- (0.000000, 1.000000);
\draw (0.000000, 1.000000) -- (-0.149042, 0.988831);
\draw (-0.149042, 0.988831) -- (-0.294755, 0.955573);
\draw (-0.294755, 0.955573) -- (-0.433884, 0.900969);
\draw (0.125157, 0.791453) -- (0.433884, 0.900969);
\draw (0.433884, 0.900969) -- (0.294755, 0.955573);
\draw (0.294755, 0.955573) -- (0.149042, 0.988831);
\draw (-0.997204, 0.074730) -- (-0.997204, -0.074730);
\draw (-0.997204, -0.074730) -- (-0.518526, 0.030249);
\draw (-0.518526, 0.030249) -- (0.046678, 0.379833);
\draw (0.046678, 0.379833) -- (0.563320, 0.826239);
\draw (0.563320, 0.826239) -- (0.433884, 0.900969);
\draw (-0.997204, -0.074730) -- (-0.974928, -0.222521);
\draw (-0.974928, -0.222521) -- (-0.930874, -0.365341);
\draw (-0.930874, -0.365341) -- (-0.518526, 0.030249);
\draw (-0.930874, -0.365341) -- (-0.866025, -0.500000);
\draw (-0.866025, -0.500000) -- (-0.781831, -0.623490);
\draw (-0.781831, -0.623490) -- (-0.680173, -0.733052);
\draw (-0.680173, -0.733052) -- (-0.563320, -0.826239);
\draw (-0.563320, -0.826239) -- (-0.278180, -0.603149);
\draw (-0.278180, -0.603149) -- (-0.018597, -0.327896);
\draw (-0.018597, -0.327896) -- (0.107571, 0.005772);
\draw (0.107571, 0.005772) -- (0.046678, 0.379833);
\draw (-0.018597, -0.327896) -- (0.255728, -0.367789);
\draw (0.255728, -0.367789) -- (0.381386, -0.119020);
\draw (0.381386, -0.119020) -- (0.107571, 0.005772);
\draw (-0.563320, -0.826239) -- (-0.433884, -0.900969);
\draw (-0.433884, -0.900969) -- (-0.273134, -0.819539);
\draw (-0.273134, -0.819539) -- (-0.278180, -0.603149);
\draw (-0.433884, -0.900969) -- (-0.294755, -0.955573);
\draw (-0.294755, -0.955573) -- (-0.149042, -0.988831);
\draw (-0.149042, -0.988831) -- (-0.273134, -0.819539);
\draw (-0.149042, -0.988831) -- (-0.000000, -1.000000);
\draw (-0.000000, -1.000000) -- (0.149042, -0.988831);
\draw (0.149042, -0.988831) -- (0.294755, -0.955573);
\draw (0.294755, -0.955573) -- (0.433884, -0.900969);
\draw (0.433884, -0.900969) -- (0.387886, -0.629052);
\draw (0.387886, -0.629052) -- (0.255728, -0.367789);
\draw (0.433884, -0.900969) -- (0.563320, -0.826239);
\draw (0.563320, -0.826239) -- (0.583270, -0.671976);
\draw (0.583270, -0.671976) -- (0.387886, -0.629052);
\draw (0.563320, -0.826239) -- (0.680173, -0.733052);
\draw (0.680173, -0.733052) -- (0.781831, -0.623490);
\draw (0.781831, -0.623490) -- (0.583270, -0.671976);
\draw (0.781831, -0.623490) -- (0.866025, -0.500000);
\draw (0.866025, -0.500000) -- (0.930874, -0.365341);
\draw (0.930874, -0.365341) -- (0.974928, -0.222521);
\draw (0.974928, -0.222521) -- (0.997204, -0.074730);
\draw (0.997204, -0.074730) -- (0.701510, -0.044622);
\draw (0.701510, -0.044622) -- (0.381386, -0.119020);
\draw (0.701510, -0.044622) -- (0.849717, 0.142774);
\draw (0.849717, 0.142774) -- (0.930874, 0.365341);
\draw (0.930874, 0.365341) -- (0.866025, 0.500000);
\draw (0.866025, 0.500000) -- (0.781831, 0.623490);
\draw (0.781831, 0.623490) -- (0.680173, 0.733052);
\draw (0.680173, 0.733052) -- (0.563320, 0.826239);
\draw (0.997204, -0.074730) -- (0.997204, 0.074730);
\draw (0.997204, 0.074730) -- (0.849717, 0.142774);
\draw (0.997204, 0.074730) -- (0.974928, 0.222521);
\draw (0.974928, 0.222521) -- (0.930874, 0.365341);

%         \end{scope}
%         \\
%       };
%     \end{tikzfigure}
%   \end{proof}
% \end{lemma}


% \begin{lemma}\label{thm:expansion:patch:poly:4:k}
%   There is an expansion $3$-patch with the polyhedral property consisting of $4$-gons and $k$-gons, $k \geq 7$.
%   \begin{proof}
%     Use \autoref{const:edge:replacement:4:1} on the edges drawn thick in \autoref{fig:expansion:patch:poly:4}. \todo{argue that this has the polyhedral property}.
%     \begin{tikzfigure}{\label{fig:expansion:patch:poly:4}}{\todo{better picture}}
%         \begin{scope}[y=0.80pt, x=0.80pt, yscale=-1.000000, xscale=1.000000, inner sep=0pt, outer sep=0pt, shift={(49.84488,-74.40945)}, scale=0.6]
    \path[draw=black,line join=miter,line cap=butt,even odd rule,line width=0.800pt]
      (306.8594,591.7323) -- (322.2024,582.8740) -- (337.5453,591.7323) --
      (337.5453,609.4488) -- (322.2024,618.3071) -- (306.8594,609.4488) -- cycle;
    \path[draw=black,line join=miter,line cap=butt,even odd rule,line width=0.800pt]
      (322.2024,582.8740) -- (322.2024,547.4409);
    \path[draw=black,line join=miter,line cap=butt,even odd rule,line width=0.800pt]
      (337.5453,591.7323) -- (368.2313,574.0157);
    \path[draw=black,line join=miter,line cap=butt,even odd rule,line width=0.800pt]
      (337.5453,609.4488) -- (368.2313,627.1653);
    \path[draw=black,line join=miter,line cap=butt,even odd rule,line width=0.800pt]
      (322.2024,618.3071) -- (322.2024,653.7402);
    \path[draw=black,line join=miter,line cap=butt,even odd rule,line width=0.800pt]
      (306.8594,609.4488) -- (276.1735,627.1653);
    \path[draw=black,line join=miter,line cap=butt,even odd rule,line width=0.800pt]
      (306.8594,591.7323) -- (276.1735,574.0158);
    \path[draw=black,line join=miter,line cap=butt,even odd rule,line width=0.800pt]
      (276.1735,574.0158) -- (260.8305,547.4409) -- (276.1735,520.8661) --
      (306.8594,520.8661) -- (322.2024,547.4409);
    \path[draw=black,line join=miter,line cap=butt,miter limit=4.00,even odd
      rule,line width=0.800pt] (322.2024,547.4409) -- (337.5453,520.8661) --
      (368.2313,520.8661) -- (383.5742,547.4409) -- (368.2313,574.0157) --
      (398.9172,574.0157) -- (414.2602,600.5905) -- (398.9172,627.1653) --
      (368.2313,627.1653) -- (383.5742,653.7402) -- (368.2313,680.3150) --
      (337.5453,680.3150) -- (322.2024,653.7402) -- (306.8594,680.3150) --
      (276.1735,680.3150) -- (260.8305,653.7402) -- (276.1735,627.1653) --
      (245.4875,627.1653) -- (230.1445,600.5905) -- (245.4875,574.0157) --
      (276.1735,574.0157);
    \path[draw=black,line join=miter,line cap=butt,even odd rule,line width=0.800pt]
      (306.8594,520.8661) -- (322.2024,494.2913) -- (337.5453,520.8661);
    \path[draw=black,line join=miter,line cap=butt,even odd rule,line width=0.800pt]
      (383.5742,547.4409) -- (414.2602,547.4409) -- (398.9172,574.0157);
    \path[draw=black,line join=miter,line cap=butt,even odd rule,line width=0.800pt]
      (398.9172,627.1653) -- (414.2602,653.7402) -- (383.5742,653.7402);
    \path[draw=black,line join=miter,line cap=butt,even odd rule,line width=0.800pt]
      (337.5453,680.3150) -- (322.2024,706.8898) -- (306.8594,680.3150);
    \path[draw=black,line join=miter,line cap=butt,even odd rule,line width=0.800pt]
      (260.8305,653.7402) -- (230.1446,653.7402) -- (245.4875,627.1653);
    \path[draw=black,line join=miter,line cap=butt,even odd rule,line width=0.800pt]
      (245.4875,574.0158) -- (230.1445,547.4409) -- (260.8305,547.4409);
    \path[draw=black,line join=miter,line cap=butt,even odd rule,line width=0.800pt]
      (260.8305,494.2913) -- (276.1735,520.8661);
    \path[draw=black,line join=miter,line cap=butt,even odd rule,line width=0.800pt]
      (322.2024,476.5748) -- (322.2024,494.2913);
    \path[draw=black,line join=miter,line cap=butt,even odd rule,line width=0.800pt]
      (368.2313,520.8661) -- (383.5742,494.2913);
    \path[draw=black,line join=miter,line cap=butt,even odd rule,line width=0.800pt]
      (414.2602,547.4409) -- (429.6031,538.5827);
    \path[draw=black,line join=miter,line cap=butt,even odd rule,line width=0.800pt]
      (414.2602,600.5906) -- (444.9461,600.5906);
    \path[draw=black,line join=miter,line cap=butt,even odd rule,line width=0.800pt]
      (414.2602,653.7402) -- (429.6031,662.5984);
    \path[draw=black,line join=miter,line cap=butt,even odd rule,line width=0.800pt]
      (368.2313,680.3150) -- (383.5742,706.8898);
    \path[draw=black,line join=miter,line cap=butt,even odd rule,line width=0.800pt]
      (322.2024,706.8898) -- (322.2024,724.6063);
    \path[draw=black,line join=miter,line cap=butt,even odd rule,line width=0.800pt]
      (276.1735,680.3150) -- (260.8305,706.8898);
    \path[draw=black,line join=miter,line cap=butt,even odd rule,line width=0.800pt]
      (230.1446,653.7402) -- (214.8016,662.5984);
    \path[draw=black,line join=miter,line cap=butt,even odd rule,line width=0.800pt]
      (230.1446,600.5906) -- (199.4586,600.5906);
    \path[draw=black,line join=miter,line cap=butt,even odd rule,line width=0.800pt]
      (230.1446,547.4409) -- (214.8016,538.5827);
    \path[draw=black,line join=miter,line cap=butt,even odd rule,line width=0.800pt]
      (214.8016,538.5827) -- (199.4586,512.0079) -- (168.7727,512.0079) --
      (184.1156,538.5827) -- cycle;
    \path[draw=black,line join=miter,line cap=butt,even odd rule,line width=0.800pt]
      (322.2024,476.5748) -- (306.8594,450.0000) -- (322.2024,423.4252) --
      (337.5453,450.0000) -- (322.2024,476.5748);
    \path[draw=black,line join=miter,line cap=butt,even odd rule,line width=0.800pt]
      (429.6032,538.5827) -- (444.9461,512.0079) -- (475.6321,512.0079) --
      (460.2891,538.5827) -- (429.6032,538.5827);
    \path[draw=black,line join=miter,line cap=butt,even odd rule,line width=0.800pt]
      (429.6032,662.5984) -- (460.2891,662.5984) -- (475.6321,689.1732) --
      (444.9461,689.1732) -- (429.6032,662.5984);
    \path[draw=black,line join=miter,line cap=butt,even odd rule,line width=0.800pt]
      (322.2024,724.6063) -- (337.5453,751.1811) -- (322.2024,777.7559) --
      (306.8594,751.1811) -- (322.2024,724.6063);
    \path[draw=black,line join=miter,line cap=butt,even odd rule,line width=0.800pt]
      (214.8016,662.5984) -- (199.4586,689.1732) -- (168.7727,689.1732) --
      (184.1156,662.5984) -- cycle;
    \path[draw=black,line join=miter,line cap=butt,even odd rule,line width=0.800pt]
      (230.1446,441.1417) -- (230.1446,476.5748) -- (260.8305,494.2913) --
      (260.8305,458.8583) -- (230.1446,441.1417);
    \path[draw=black,line join=miter,line cap=butt,even odd rule,line width=0.800pt]
      (383.5742,494.2913) -- (383.5742,458.8583) -- (414.2602,441.1417) --
      (414.2602,476.5748) -- (383.5742,494.2913);
    \path[draw=black,line join=miter,line cap=butt,even odd rule,line width=0.800pt]
      (444.9461,600.5906) -- (475.6321,582.8740) -- (506.3180,600.5906) --
      (475.6321,618.3071) -- (444.9461,600.5906);
    \path[draw=black,line join=miter,line cap=butt,even odd rule,line width=0.800pt]
      (383.5742,706.8898) -- (383.5742,742.3228) -- (414.2602,760.0394) --
      (414.2602,724.6063) -- (383.5742,706.8898);
    \path[draw=black,line join=miter,line cap=butt,even odd rule,line width=0.800pt]
      (260.8305,706.8898) -- (260.8305,742.3228) -- (230.1446,760.0394) --
      (230.1446,724.6063) -- (260.8305,706.8898);
    \path[draw=black,line join=miter,line cap=butt,even odd rule,line width=0.800pt]
      (199.4586,600.5906) .. controls (199.4586,600.5906) and (168.7727,618.3071) ..
      (168.7727,618.3071) -- (138.0867,600.5906) -- (168.7727,582.8740) --
      (199.4586,600.5906);
    \path[draw=black,line join=miter,line cap=butt,even odd rule,line width=0.800pt]
      (184.1156,538.5827) -- (168.7727,582.8740);
    \path[draw=black,line join=miter,line cap=butt,even odd rule,line width=0.800pt]
      (199.4586,512.0079) -- (230.1446,476.5748);
    \path[draw=black,line join=miter,line cap=butt,even odd rule,line width=0.800pt]
      (260.8305,458.8583) -- (306.8594,450.0000);
    \path[draw=black,line join=miter,line cap=butt,even odd rule,line width=0.800pt]
      (337.5453,450.0000) -- (383.5742,458.8583);
    \path[draw=black,line join=miter,line cap=butt,even odd rule,line width=0.800pt]
      (414.2602,476.5748) -- (444.9461,512.0079);
    \path[draw=black,line join=miter,line cap=butt,even odd rule,line width=0.800pt]
      (460.2891,538.5827) -- (475.6321,582.8740);
    \path[draw=black,line join=miter,line cap=butt,even odd rule,line width=0.800pt]
      (475.6321,618.3071) -- (460.2891,662.5984);
    \path[draw=black,line join=miter,line cap=butt,even odd rule,line width=0.800pt]
      (444.9461,689.1732) -- (414.2602,724.6063);
    \path[draw=black,line join=miter,line cap=butt,even odd rule,line width=0.800pt]
      (383.5742,742.3228) -- (337.5453,751.1811);
    \path[draw=black,line join=miter,line cap=butt,even odd rule,line width=0.800pt]
      (306.8594,751.1811) -- (260.8305,742.3228);
    \path[draw=black,line join=miter,line cap=butt,even odd rule,line width=0.800pt]
      (230.1446,724.6063) -- (199.4586,689.1732);
    \path[draw=black,line join=miter,line cap=butt,even odd rule,line width=0.800pt]
      (184.1156,662.5984) -- (168.7727,618.3071);
    \path[draw=black,line join=miter,line cap=butt,even odd rule,line width=0.800pt]
      (168.7727,689.1732) -- (153.4297,698.0315);
    \path[draw=black,line join=miter,line cap=butt,even odd rule,line width=0.800pt]
      (168.7727,512.0079) -- (153.4297,503.1496);
    \path[draw=black,line join=miter,line cap=butt,even odd rule,line width=0.800pt]
      (322.2024,423.4252) -- (322.2024,405.7087);
    \path[draw=black,line join=miter,line cap=butt,even odd rule,line width=0.800pt]
      (475.6321,512.0079) -- (490.9750,503.1496);
    \path[draw=black,line join=miter,line cap=butt,even odd rule,line width=0.800pt]
      (414.2602,441.1417) -- (429.6031,414.5669);
    \path[draw=black,line join=miter,line cap=butt,even odd rule,line width=0.800pt]
      (506.3180,600.5906) -- (537.0039,600.5906);
    \path[draw=black,line join=miter,line cap=butt,even odd rule,line width=0.800pt]
      (475.6321,689.1732) -- (490.9750,698.0315);
    \path[draw=black,line join=miter,line cap=butt,even odd rule,line width=0.800pt]
      (414.2602,760.0394) -- (429.6031,786.6142);
    \path[draw=black,line join=miter,line cap=butt,even odd rule,line width=0.800pt]
      (322.2024,777.7559) -- (322.2024,795.4724);
    \path[draw=black,line join=miter,line cap=butt,even odd rule,line width=0.800pt]
      (230.1446,760.0394) -- (214.8016,786.6142);
    \path[draw=black,line join=miter,line cap=butt,even odd rule,line width=0.800pt]
      (138.0867,600.5906) -- (107.4008,600.5906);
    \path[draw=black,line join=miter,line cap=butt,even odd rule,line width=0.800pt]
      (230.1446,441.1417) -- (214.8016,414.5669);
    \path[draw=black,line join=miter,line cap=butt,even odd rule,line width=0.800pt]
      (107.4008,600.5906) -- (92.0578,520.8661) -- (153.4297,503.1496) --
      (138.0867,441.1417) -- (214.8016,414.5669) -- (276.1735,361.4173) --
      (322.2024,405.7087) -- (368.2313,361.4173) -- (429.6031,414.5669) --
      (506.3180,441.1417) -- (490.9750,503.1496) -- (552.3469,520.8661) --
      (537.0039,600.5906) -- (552.3469,680.3150) -- (490.9750,698.0315) --
      (506.3180,760.0394) -- (429.6031,786.6142) -- (368.2313,839.7638) --
      (322.2024,795.4724) -- (276.1735,839.7638) -- (214.8016,786.6142) --
      (138.0867,760.0394) -- (153.4297,698.0315) -- (92.0578,680.3150) -- cycle;
    \path[draw=black,line join=miter,line cap=butt,even odd rule,line width=0.800pt]
      (92.0578,520.8661) -- (46.0289,494.2913) -- (46.0289,458.8583) --
      (61.3719,432.2835) -- (92.0578,414.5669) -- (138.0867,441.1417);
    \path[draw=black,line join=miter,line cap=butt,even odd rule,line width=0.800pt]
      (276.1735,361.4173) -- (276.1735,308.2677) -- (306.8594,290.5512) --
      (337.5453,290.5512) -- (368.2313,308.2677) -- (368.2313,361.4173);
    \path[draw=black,line join=miter,line cap=butt,even odd rule,line width=0.800pt]
      (506.3180,441.1417) -- (552.3469,414.5669) -- (583.0328,432.2835) --
      (598.3758,458.8583) -- (598.3758,494.2913) -- (552.3469,520.8661);
    \path[draw=black,line join=miter,line cap=butt,even odd rule,line width=0.800pt]
      (552.3469,680.3150) -- (598.3758,706.8898) -- (598.3758,742.3228) --
      (583.0328,768.8976) -- (552.3469,786.6142) -- (506.3180,760.0394);
    \path[draw=black,line join=miter,line cap=butt,even odd rule,line width=0.800pt]
      (368.2313,839.7638) -- (368.2313,892.9134) -- (337.5453,910.6299) --
      (306.8594,910.6299) -- (276.1735,892.9134) -- (276.1735,839.7638);
    \path[draw=black,line join=miter,line cap=butt,even odd rule,line width=0.800pt]
      (138.0867,760.0394) -- (92.0578,786.6142) -- (61.3719,768.8976) --
      (46.0289,742.3228) -- (46.0289,706.8898) -- (92.0578,680.3150);
    \path[draw=black,line join=miter,line cap=butt,miter limit=4.00,even odd
      rule,line width=2.400pt] (368.2313,627.1653) -- (383.5742,653.7402);
    \path[draw=black,line join=miter,line cap=butt,miter limit=4.00,even odd
      rule,line width=2.400pt] (322.2024,653.7402) -- (306.8594,680.3150);
    \path[draw=black,miter limit=4.00,line width=2.400pt] (276.1735,627.1653) --
      (245.4875,627.1653);
    \path[draw=black,miter limit=4.00,line width=2.400pt] (276.1735,574.0158) --
      (260.8305,547.4409);
    \path[draw=black,miter limit=4.00,line width=2.400pt] (322.2024,547.4409) --
      (337.5453,520.8661);
    \path[draw=black,miter limit=4.00,line width=2.400pt] (368.2313,574.0158) --
      (398.9172,574.0158);
    \path[draw=black,miter limit=4.00,line width=2.400pt] (429.6032,662.5984) --
      (444.9461,689.1732);
    \path[draw=black,miter limit=4.00,line width=2.400pt] (322.2024,724.6063) --
      (306.8594,751.1811);
    \path[draw=black,miter limit=4.00,line width=2.400pt] (214.8016,662.5984) --
      (184.1156,662.5984);
    \path[draw=black,miter limit=4.00,line width=2.400pt] (214.8016,538.5827) --
      (199.4586,512.0079);
    \path[draw=black,miter limit=4.00,line width=2.400pt] (322.2024,476.5748) --
      (337.5453,450.0000);
    \path[draw=black,miter limit=4.00,line width=2.400pt] (429.6032,538.5827) --
      (460.2891,538.5827);
    \path[draw=black,miter limit=4.00,line width=2.400pt] (537.0039,600.5906) --
      (552.3469,520.8661);
    \path[draw=black,miter limit=4.00,line width=2.400pt] (429.6032,414.5669) --
      (368.2313,361.4173);
    \path[draw=black,miter limit=4.00,line width=2.400pt] (214.8016,414.5669) --
      (138.0867,441.1417);
    \path[draw=black,miter limit=4.00,line width=2.400pt] (107.4008,600.5906) --
      (92.0578,680.3150);
    \path[draw=black,miter limit=4.00,line width=2.400pt] (214.8016,786.6142) --
      (276.1735,839.7638);
    \path[draw=black,miter limit=4.00,line width=2.400pt] (429.6032,786.6142) --
      (506.3180,760.0394);
  \end{scope}


%     \end{tikzfigure}
%   \end{proof}
% \end{lemma}


% \begin{theorem}
%   Let $p$ and $v$ be sequences for which \eqref{eq:vp:3} holds for some orientable closed $2$-manifold $S$ with {\sc Euler}-characteristic $\chi$. Then $(p, v)$ is $[(4k + 1) \times 4, 2 \times (4k+7)]$-$[3]$-realizable and $[(4k + 3) \times 4, 2 \times (4k+9)]$-$[3]$-realizable for all $k \in \nats$. When not being in the case $\chi = 2$, $\sum_{k=3,\, 2 \nmid k}^{m} p_k = 0$ and $\sum_{k=4, \,3 \nmid k}^m v_k = 1$, it is also $[(4k + 4) \times 4, 2 \times (4k+10)]$-$[3]$-realizable.
%   \begin{proof}
%     \autoref{thm:expansion:patch:4:7}, \autoref{thm:expansion:patch:4:9}, \autoref{thm:expansion:patch:4:10} and \autoref{thm:expansion:patch:poly:4:k} deliver the necessary patches to apply \autoref{thm:main:const}, which shows the claim.
%   \end{proof}
% \end{theorem}