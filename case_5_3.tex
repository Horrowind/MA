\mysection{$3$-valent {\sc Eberhard}-like theorems with pentagons}\label{sec:5:3}

In this section we want to prove $3$-valent {\sc Eberhard}-like theorems with pentagons, i.e. for $q = [q_5 \times 4, q_l \times l]$, $w = [w_3 \times 3]$, $l > 6$, $\gcd(q_5, q_l) = 1$. As it turns out in \autoref{sec:negative:results}, this is only possible for all admissible pairs of sequences on all closed orientable $2$-manifolds, if $5 \nmid l$. We split the remaining cases in four series, each with a residual class of $l \mod 5$. 
\clearpage
For all the proofs, we want to use \autoref{thm:main:const}, therefore we need to have a construction scheme for patches with arbitrary large $l$-gons. These we get by the next three constructions:
\begin{construction}\label{const:edge:replacement:5:1} When we want to use this construction in this section we label an edge (the specified edge) with a square and point with arrows to a $k_1$-gon and a $k_2$-gon.
  \begin{cinput}
  \item A $3$-patch with $p$-vector $p$ and a specified edge with exactly one vertex incident to some $k_1$-gon and the other vertex incident to some $k_2$-gon.
  \end{cinput}
  \begin{coutput}
  \item A $3$-patch with $p$-vector $p - [k_1, k_2] + [10k \times 5] + [k_1 + 5k, k_2 + 5k]$ for all $k \in \nats$.%
  \item If every two faces of the $3$-patch meet properly, then this is carried over to the new patch.
  \end{coutput}
  \begin{cdescription}
    Using the replacement of the single edge as seen in \autoref{fig:const:edge:replacement:5:1} results in a new $3$-patch with $p$-vector $p - [k_1, k_2] + [10 \times 5] + [k_1 + 5, k_2 + 5]$. The line on the left labeled with a square is the specified edge and the line on the right labeled with a diamond is a new edge which we can use to repeat the construction. Every time we use this construction we add ten pentagons while increasing the number of vertices of the left and right polygon by five; doing this $k$ times gives the desired $3$-patch. That all faces meet properly follows by induction as this property is preserved in each step.
    \begin{tikzfigure}{\label{fig:const:edge:replacement:5:1}}{}
      \matrix (m) [column sep=1cm] {
        \begin{scope}
          \draw[lsquare] (-1, 0) -- (1, 0);
          \draw (-1.2, 0.5) -- (-1, 0) -- (-1.2, -0.5);
          \draw (1.2, 0.5) -- (1, 0) -- (1.2, -0.5);
          \node (k1) at (-2.5, 0) {$k_1$};
          \node (k2) at (2.5, 0) {$k_2$};
          \draw[lface] (-1, 0) -- (k1);
          \draw[lface] ( 1, 0) -- (k2);
          \fill[black] (-1,0) circle(2pt);
          \fill[black] (1,0) circle(2pt);

        \end{scope}
        &
        \begin{scope}
          \draw[lsquare] (-1, 2.5) -- (1, 2.5);
          \draw (-1, -2.5) -- (1, -2.5);
          \draw (-1.2, 3) -- (-1, 2.5);
          \draw (-1.2, -3) -- (-1, -2.5);
          \draw (1.2, 3) -- (1, 2.5);
          \draw (1.2, -3) -- (1, -2.5);
          \draw (1, 2.5) -- (0.9, 1.5) -- (0.85, 0.5) -- (0.85, -0.5) -- (0.9, -1.5) -- (1, -2.5);
          \draw (-1, 2.5) -- (-0.9, 1.5) -- (-0.85, 0.5) -- (-0.85, -0.5) -- (-0.9, -1.5) -- (-1, -2.5);
          \draw (0.9, 1.5) -- (0, 1.4) -- (-0.9, 1.5);
          \draw (0.9, -1.5) -- (0, -1.4) -- (-0.9, -1.5);
          \draw (0.85, 0.5) -- (0.4, 0.5) -- (0, 0.6) -- (-0.4, 0.5) -- (-0.85, 0.5);
          \draw (0.85, -0.5) -- (0.4, -0.5) -- (0, -0.6) -- (-0.4, -0.5) -- (-0.85, -0.5);
          \draw (0, 1.4) -- (0, 0.6);
          \draw (0, -1.4) -- (0, -0.6);
          \draw (0.4, 0.5) -- (0.3, 0) -- (0.4, -0.5);
          \draw (-0.4, 0.5) -- (-0.3, 0) -- (-0.4, -0.5);
          \draw (0.3, 0) -- (-0.3, 0);
          \node (k1) at (-1.8, 0) {$k_1 + 5$};
          \node (k2) at (1.8, 0) {$k_2 + 5$};
          \node at (0, 2) {$5$};
          \node at (0, -2) {$5$};
          \node at (0.6, 1) {$5$};
          \node at (-0.6, 1) {$5$};
          \node at (0.6, -1) {$5$};
          \node at (-0.6, -1) {$5$};
          \node at (-0.6, 0) {$5$};
          \node at (0.6, 0) {$5$};
          \node at (0, -0.3) {$5$};
          \node at (0, 0.3) {$5$};

          \draw[lface] (-1, 2.5) -- (k1);
          \draw[lface] ( 1, 2.5) -- (k2);

          \fill[black] (-1,2.5) circle(2pt);
          \fill[black] ( 1,2.5) circle(2pt);
          \fill[black] (-1,-2.5) circle(2pt);
          \fill[black] ( 1,-2.5) circle(2pt);

          \fill[black] (0.9,1.5) circle(2pt);
          \fill[black] (0.9,-1.5) circle(2pt);
          \fill[black] (-0.9,1.5) circle(2pt);
          \fill[black] (-0.9,-1.5) circle(2pt);

          \fill[black] (0.85,0.5) circle(2pt);
          \fill[black] (0.85,-0.5) circle(2pt);
          \fill[black] (-0.85,0.5) circle(2pt);
          \fill[black] (-0.85,-0.5) circle(2pt);

          \fill[black] (0.4,-0.5) circle(2pt);
          \fill[black] (0.4,0.5) circle(2pt);
          \fill[black] (-0.4,-0.5) circle(2pt);
          \fill[black] (-0.4,0.5) circle(2pt);

          \fill[black] (0,1.4) circle(2pt);
          \fill[black] (0,-1.4) circle(2pt);
          \fill[black] (0,0.6) circle(2pt);
          \fill[black] (0,-0.6) circle(2pt);
          \fill[black] (0.3,0) circle(2pt);
          \fill[black] (-0.3,0) circle(2pt);
        \end{scope}
        \\
      };
    \end{tikzfigure}
  \end{cdescription}
\end{construction}

\begin{construction}\label{const:edge:replacement:5:2} When we want to use this construction in this section we draw a path of length two (the specified path) thick and point with arrows to a $k_1$-gon and a $k_2$-gon.
  \begin{cinput}
  \item A $3$-patch with $p$-vector $p$ and a specified path of length two beginning at a $k_1$-gon and some leading to a $k_2$-gon; no edge of the path is adjacent to any of the polygons.
  \item If every two faces of the $3$-patch meet properly, then this is carried over to the new patch.
  \end{cinput}
  \begin{coutput}
  \item A $3$-patch with $p$-vector $p - [k_1, k_2] + [(10k) \times 5] + [5k + k_1 , 5k + k_2]$ for all $k \in \nats$.%
  \end{coutput}
  \begin{cdescription}
    Using the replacement of the path as seen in \autoref{fig:const:edge:replacement:5:2} results in a new $3$-patch with $p$-vector $p - [k_1, k_2] + [10 \times 5] + [k_1 + 5, k_2 + 5]$. The path drawn thick on the left is the specified path and the path on the right is a new path which we can use to repeat the construction. Every time we use this construction we add ten pentagons while increasing the number of vertices of the left and right polygon by five; doing this $k$ times gives the desired $3$-patch. That all faces meet properly follows by induction as this property is preserved in each step.
    \begin{tikzfigure}{\label{fig:const:edge:replacement:5:2}}{}
      \matrix (m) [column sep=1cm] {
        \begin{scope}
          \draw[very thick] (-1, 0) -- (0, 0.2) -- (1, 0);
          \draw (-1.2, 0.5) -- (-1, 0) -- (-1.2, -0.5);
          \draw (1.2, 0.5) -- (1, 0) -- (1.2, -0.5);
          \draw (0, 0.2) -- (0, 0.8);
          \node (k1) at (-2.5, 0) {$k_1$};
          \node (k2) at (2.5, 0) {$k_2$};
          \draw[lface] (-1, 0) -- (k1);
          \draw[lface] ( 1, 0) -- (k2);
          \fill[black] (-1,0) circle(2pt);
          \fill[black] (1,0) circle(2pt);
          \fill[black] (0,0.2) circle(2pt);

        \end{scope}
        &
        \begin{scope}
          \draw (0, 2.7) -- (0, 3.3);
          \draw[very thick] (1, 2.5) -- (0, 2.7) -- (-1, 2.5);
          \draw (-0.9, 1.5) -- (0.9, 1.5);
          \draw (-1.2, 3) -- (-1, 2.5);
          \draw (-1.2, -3) -- (-1, -2.5);
          \draw (1.2, 3) -- (1, 2.5);
          \draw (1.2, -3) -- (1, -2.5);
          \draw (1, 2.5) -- (0.9, 1.5) -- (0.85, 0.5) -- (0.85, -0.5) -- (0.9, -1.5) -- (1, -2.5);
          \draw (-1, 2.5) -- (-0.9, 1.5) -- (-0.85, 0.5) -- (-0.85, -0.5) -- (-0.9, -1.5) -- (-1, -2.5);
          \draw (0.85, 0.5) -- (0, 0.4) -- (-0.85, 0.5);
          \draw (-1, -2.5) -- (0, -2.3) -- (1, -2.5);
          \draw (0.85, -0.5) -- (0.4, -0.5) -- (0, -0.4) -- (-0.4, -0.5) -- (-0.85, -0.5);
          \draw (0.9, -1.5) -- (0.4, -1.5) -- (0, -1.6) -- (-0.4, -1.5) -- (-0.9, -1.5);
          \draw (0, 0.4) -- (0, -0.4);
          \draw (0, -2.3) -- (0, -1.6);
          \draw (0.4, -0.5) -- (0.3, -1) -- (0.4, -1.5);
          \draw (-0.4, -0.5) -- (-0.3, -1) -- (-0.4, -1.5);
          \draw (0.3, -1) -- (-0.3, -1);
          \node (k1) at (-1.8, 0) {$k_1 + 5$};
          \node (k2) at (1.8, 0) {$k_2 + 5$};
          \node at (0, 1) {$5$};
          \node at (0, 2) {$5$};
          \node at (0.6, 0) {$5$};
          \node at (-0.6, 0) {$5$};
          \node at (0.6, -2) {$5$};
          \node at (-0.6, -2) {$5$};
          \node at (-0.6, -1) {$5$};
          \node at (0.6, -1) {$5$};
          \node at (0, -1.3) {$5$};
          \node at (0, -0.7) {$5$};
          \draw[lface] (-1, 2.5) -- (k1);
          \draw[lface] ( 1, 2.5) -- (k2);

          \fill[black] (-1,2.5) circle(2pt);
          \fill[black] ( 1,2.5) circle(2pt);
          \fill[black] ( 0,2.7) circle(2pt);
          \fill[black] (-0.9,1.5) circle(2pt);
          \fill[black] (0.9,1.5) circle(2pt);

          % \fill[black] (-1.2,-3) circle(2pt);
          % \fill[black] (-1.2,3) circle(2pt);
          % \fill[black] (1.2,-3) circle(2pt);
          % \fill[black] (1.2,3) circle(2pt);

          \fill[black] (-0.85,0.5) circle(2pt);
          \fill[black] (0.85,0.5) circle(2pt);
          \fill[black] (-0.85,-0.5) circle(2pt);
          \fill[black] (0.85,-0.5) circle(2pt);

          \fill[black] (0,0.4) circle(2pt);
          \fill[black] (0,-0.4) circle(2pt);

          \fill[black] (0.4,-0.5) circle(2pt);
          \fill[black] (-0.4,-0.5) circle(2pt);
          % \fill[black] (0.4,0.5) circle(2pt);
          % \fill[black] (-0.4,0.5) circle(2pt);
          \fill[black] (0,-2.3) circle(2pt);
          \fill[black] (0,-1.6) circle(2pt);

          \fill[black] (0.9,-1.5) circle(2pt);
          \fill[black] (-0.9,-1.5) circle(2pt);
          \fill[black] (0.4,-1.5) circle(2pt);
          \fill[black] (-0.4,-1.5) circle(2pt);

          \fill[black] (0.3,-1) circle(2pt);
          \fill[black] (-0.3,-1) circle(2pt);

          \fill[black] (1,-2.5) circle(2pt);
          \fill[black] (-1,-2.5) circle(2pt);


        \end{scope}
        \\
      };
    \end{tikzfigure}%
  \end{cdescription}%
\end{construction}%
\begin{construction}\label{const:edge:replacement:5:3} When we want to use this construction in this section we label an edge (the specified edge) with a diamond and point with arrows to a $k_1$-gon and a $k_2$-gon.
  \begin{cinput}
  \item A $3$-patch with $p$-vector $p$ and a specified edge incident to a $k_1$-gon and a $k_2$-gon.
  \end{cinput}
  \begin{coutput}
  \item A $3$-patch with $p$-vector $p - [k_1, k_2] + [(10k) \times 5] + [5k + k_1 , 5k + k_2]$ for all $k \in \nats$.%
  \end{coutput}
  \begin{cdescription}
    Using the replacement of the single edge as seen in \autoref{fig:const:edge:replacement:5:3} results in a new $3$-patch with $p$-vector $p - [k_1, k_2] + [10 \times 5] + [k_1 + 5, k_2 + 5]$. The line on the left labeled with a diamond is the specified edge and the line on the right labeled with a diamond is a new edge which we can use to repeat the construction. Every time we use this construction we add ten pentagons while increasing the number of vertices of the left and right polygon by five; doing this $k$ times gives the desired $3$-patch.
    \begin{tikzfigure}{\label{fig:const:edge:replacement:5:3}}{}
      \matrix (m) [column sep=1cm] {
        \begin{scope}
          \draw[ldiamond] (0, -1) -- (0, 1);
          \draw (0.5, -1.2) -- (0, -1) -- (-0.5, -1.2);
          \draw (0.5, 1.2) -- (0, 1) -- (-0.5, 1.2);
          \node (k1) at (-1, 0) {$k_1$};
          \node (k2) at (1, 0) {$k_2$};
          \draw[lface] (0, 0) -- (k1);
          \draw[lface] (0, 0) -- (k2);

          \fill[black] (0,-1) circle(2pt);
          \fill[black] (0,1) circle(2pt);

        \end{scope}
        &
        \begin{scope}
          \draw (0.5, -2.7) -- (0, -2.5) -- (-0.5, -2.7);
          \draw (0.5, 2.7) -- (0, 2.5) -- (-0.5, 2.7);
          \draw[ldiamond] (0, 2.5) -- (0, 2);
          \draw (0, -2.5) -- (0, -2);
          \draw (0, -2) -- (-1.5 , -1.5) -- (-2, 0) -- (-1.5, 1.5) -- (0, 2) -- (1.5, 1.5) -- (2, 0) -- (1.5, -1.5) -- (0, -2);
          \draw (-0.5, -1) -- (-0.5, -0.5) -- (-1, 0) -- (-0.5, 0.5) -- (-0.5, 1) -- (0.5, 1) -- (0.5, 0.5) -- (1, 0) -- (0.5, -0.5) -- (0.5, -1) -- (-0.5, -1);
          \draw (-0.5, -0.5) -- (0, -0.25) -- (0.5, -0.5);
          \draw (-0.5, 0.5) -- (0, 0.25) -- (0.5, 0.5);
          \draw (-1.5, -1.5) -- (-0.5, -1);
          \draw (1.5, -1.5) -- (0.5, -1);
          \draw (-1.5, 1.5) -- (-0.5, 1);
          \draw (1.5, 1.5) -- (0.5, 1);
          \draw (0, -0.25) -- (0, 0.25);
          \draw (-2, 0) -- (-1, 0);
          \draw (2, 0) -- (1, 0);

          \node at (-1.25, -0.5) {$5$};
          \node at (1.25, -0.5)  {$5$};
          \node at (-1.25, 0.5)  {$5$};
          \node at (1.25, 0.5) {$5$};
          \node at (-0.5, 0) {$5$};
          \node at (0.5, 0) {$5$};
          \node at (0, -1.5) {$5$};
          \node at (0, 1.5) {$5$};
          \node at (0, -0.75) {$5$};
          \node at (0, 0.75) {$5$};
          \node[anchor=east] (k1) at (-2, 1.5) {$k_1 + 5$};
          \node[anchor=west] (k2) at (2, 1.5) {$k_2 + 5$};
          \draw[lface] (0, 2.25) -- (k1);
          \draw[lface] (0, 2.25) -- (k2);

          \fill[black] (0,-2.5) circle(2pt);
          \fill[black] (0,2.5) circle(2pt);
          \fill[black] (0,2) circle(2pt);
          \fill[black] (0,-2) circle(2pt);

          \fill[black] (-0.5,-1) circle(2pt);
          \fill[black] (-0.5,1) circle(2pt);
          \fill[black] (0.5,1) circle(2pt);
          \fill[black] (0.5,-1) circle(2pt);

          \fill[black] (0.5,0.5) circle(2pt);
          \fill[black] (-0.5,0.5) circle(2pt);
          \fill[black] (-0.5,-0.5) circle(2pt);
          \fill[black] (0.5,-0.5) circle(2pt);

          \fill[black] (2,0) circle(2pt);
          \fill[black] (-2,0) circle(2pt);

          \fill[black] (0,-0.25) circle(2pt);
          \fill[black] (0,0.25) circle(2pt);

          \fill[black] (-1.5,1.5) circle(2pt);
          \fill[black] (-1.5,-1.5) circle(2pt);
          \fill[black] (1.5,-1.5) circle(2pt);
          \fill[black] (1.5,1.5) circle(2pt);

          \fill[black] (1,0) circle(2pt);
          \fill[black] (-1,0) circle(2pt);


        \end{scope}
        \\
      };
    \end{tikzfigure}
  \end{cdescription}
\end{construction}
% We now can split the infinitely many cases in four series, each with a different residual class of $l \mod 5$:
\clearpage
\begin{theorem}
  Let $p$ and $v$ be a pair of admissible sequences for an orientable closed $2$-manifold $S$. Then $(p, v)$ is $[(5k + 1) \times 5, (5k+7)]$-$[3]$-realizable for all $k \in \nats$.
  \begin{proof}
    An expansion $3$-patch $\mathcal{P}_N$ with outer tuple $o = (1, 1, 1, 2, 1, 2, 2, 2)$ is shown in \autoref{fig:expansion:patch:5:7:a} and a corresponding $o$-$6$-gonal $3$-patch $\mathcal{P}_F$ is shown in \autoref{fig:expansion:patch:5:7:c}, both consisting of pentagons and heptagons. By using \autoref{const:edge:replacement:5:1} and \autoref{const:edge:replacement:5:2} as indicated we get $3$-patches consisting of only pentagons and $(5k+7)$-gons, $k \in \nats$. We can see in \autoref{fig:expansion:patch:5:7:b} that $\mathcal{P}_N$ has the polyhedral property, thus we can apply \autoref{thm:main:const} with $\mathcal{P}_P \defeq \mathcal{P}_N$.
  \end{proof}%
\end{theorem}%
\begin{tikzfigure2}{}
  \begin{tikzsubfigure}{\label{fig:expansion:patch:5:7:a}}{$\mathcal{P}_N = \mathcal{P}_P$}{0.5}
    \begin{scope}[scale=0.7, yscale=0.866]

      \node[anchor= 90] at (-0.5, 1)      {$i_{0}$};
      \node[anchor= 90] at (-2.5, 5)      {$i_{1}$};
      \node[anchor= 45] at (-4.5, 6.333)  {$i_{2}=o_{0}$};
      \node[anchor=  0] at (-4, 8)        {$o_{1}$};
      \node[anchor=  0] at (-4.5, 9)      {$\mathbf{o_{2}}$};
      \node[anchor=  0] at (-4, 10)       {$o_{3}$};
      \node[anchor=-45] at (-2.5, 10.333) {$o_{4}$};
      \node[anchor=-45] at (-1, 10)       {$o_{5}$};
      \node[anchor=230] at (-0.5, 9)      {$o_{6}$};
      \node[anchor=230] at (0.5, 9)       {$o_{7}$};
      \node[anchor=230] at (1, 8)         {$o_{8}$};
      \node[anchor=230] at (2.5, 7.666)   {$i_{2}'=o_{9}$};
      \node[anchor=180] at (2.5, 5)       {$i_{1}'$};
      \node[anchor=180] at (0.5, 1)       {$i_{0}'$};
      \node (n1) at (-1, 5) {$7$};
      \node (n2) at (-3, 9) {$7$};
      \node at (-3, 6.5)    {$5$};
      \node at (-0.5,8)     {$5$};

      \fill[black]  (-0.5, 1)      circle (3pt);
      \fill[black]  (-2.5, 5)      circle (3pt);
      \fill[black]  (-4.5, 6.333)  circle (3pt);
      \fill[black]  (-4, 8)        circle (3pt);
      \fill[black]  (-4.5, 9)      circle (3pt);
      \fill[black]  (-4, 10)       circle (3pt);
      \fill[black]  (-2.5, 10.333) circle (3pt);
      \fill[black]  (-1, 10)       circle (3pt);
      \fill[black]  (-0.5, 9)      circle (3pt);
      \fill[black]  (0.5, 9)       circle (3pt);
      \fill[black]  (1, 8)         circle (3pt);
      \fill[black]  (2.5, 7.666)   circle (3pt);
      \fill[black]  (2.5, 5)       circle (3pt);
      \fill[black]  (0.5, 1)       circle (3pt);
      \fill[black]  (-2, 6)        circle (3pt);
      \fill[black]  (-2, 8)       circle (3pt);

      \draw (-0.5, 1) -- (-2.5, 5) -- (-4.5, 6.333) -- (-4, 8) -- (-4.5, 9) -- (-4, 10) -- (-2.5, 10.333) -- (-1, 10) -- (-0.5, 9) -- (0.5, 9) -- (1, 8) -- (2.5, 7.666) -- (2.5, 5) -- (0.5, 1) -- (-0.5, 1);
      \draw (-2.5, 5) -- (-2, 6) -- (1, 8);
      \draw (-0.5, 9) -- (-2, 8) -- (-4, 8);
      \draw[lsquare] (-2, 6) -- (-2, 8);
      \draw[lface] (-2, 6) -- (n1);
      \draw[lface] (-2, 8) -- (n2);

    \end{scope}
  \end{tikzsubfigure}%
  \begin{tikzsubfigure}{\label{fig:expansion:patch:5:7:b}}{Edge patch of $\mathcal{P}_N$}{0.5}
    \begin{scope}[scale=0.35]
      \begin{scope}[yscale=0.866]
        \draw[very thick] (-0.5, 1) -- (-2.5, 5) -- (-4.5, 6.333) -- (-4, 8) -- (-4.5, 9) -- (-4, 10) -- (-2.5, 10.333) -- (-1, 10) -- (-0.5, 9) -- (0.5, 9) -- (1, 8) -- (2.5, 7.666) -- (2.5, 5) -- (0.5, 1) -- (-0.5, 1);
        \draw (-2.5, 5) -- (-2, 6) -- (1, 8);
        \draw (-2, 6) -- (-2, 8) -- (-4, 8);
        \draw (-2, 8) -- (-0.5, 9);


        \fill[black]  (-0.5, 1)      circle (4.5pt);
        \fill[black]  (-2.5, 5)      circle (4.5pt);
        \fill[black]  (-4.5, 6.333)  circle (4.5pt);
        \fill[black]  (-4, 8)        circle (4.5pt);
        \fill[black]  (-4.5, 9)      circle (4.5pt);
        \fill[black]  (-4, 10)       circle (4.5pt);
        \fill[black]  (-2.5, 10.333) circle (4.5pt);
        \fill[black]  (-1, 10)       circle (4.5pt);
        \fill[black]  (-0.5, 9)      circle (4.5pt);
        \fill[black]  (0.5, 9)       circle (4.5pt);
        \fill[black]  (1, 8)         circle (4.5pt);
        \fill[black]  (2.5, 7.666)   circle (4.5pt);
        \fill[black]  (2.5, 5)       circle (4.5pt);
        \fill[black]  (0.5, 1)       circle (4.5pt);
        \fill[black]  (-2, 6)        circle (4.5pt);
        \fill[black]  (-2, 8)        circle (4.5pt);

      \end{scope}
      \begin{scope}[rotate=-60, yscale=0.866]
        \draw[very thick] (-0.5, 1) -- (-2.5, 5) -- (-4.5, 6.333) -- (-4, 8) -- (-4.5, 9) -- (-4, 10) -- (-2.5, 10.333) -- (-1, 10) -- (-0.5, 9) -- (0.5, 9) -- (1, 8) -- (2.5, 7.666) -- (2.5, 5) -- (0.5, 1) -- (-0.5, 1);
        \draw (-2.5, 5) -- (-2, 6) -- (1, 8);
        \draw (-2, 6) -- (-2, 8) -- (-4, 8);
        \draw (-2, 8) -- (-0.5, 9);


        \fill[black]  (-0.5, 1)      circle (4.5pt);
        \fill[black]  (-2.5, 5)      circle (4.5pt);
        \fill[black]  (-4.5, 6.333)  circle (4.5pt);
        \fill[black]  (-4, 8)        circle (4.5pt);
        \fill[black]  (-4.5, 9)      circle (4.5pt);
        \fill[black]  (-4, 10)       circle (4.5pt);
        \fill[black]  (-2.5, 10.333) circle (4.5pt);
        \fill[black]  (-1, 10)       circle (4.5pt);
        \fill[black]  (-0.5, 9)      circle (4.5pt);
        \fill[black]  (0.5, 9)       circle (4.5pt);
        \fill[black]  (1, 8)         circle (4.5pt);
        \fill[black]  (2.5, 7.666)   circle (4.5pt);
        \fill[black]  (2.5, 5)       circle (4.5pt);
        \fill[black]  (0.5, 1)       circle (4.5pt);
        \fill[black]  (-2, 6)        circle (4.5pt);
        \fill[black]  (-2, 8)        circle (4.5pt);

      \end{scope}
      \begin{scope}[yscale=0.866,shift={(0 cm,18 cm)},rotate=180]
        \draw[very thick] (-0.5, 1) -- (-2.5, 5) -- (-4.5, 6.333) -- (-4, 8) -- (-4.5, 9) -- (-4, 10) -- (-2.5, 10.333) -- (-1, 10) -- (-0.5, 9) -- (0.5, 9) -- (1, 8) -- (2.5, 7.666) -- (2.5, 5) -- (0.5, 1) -- (-0.5, 1);
        \draw (-2.5, 5) -- (-2, 6) -- (1, 8);
        \draw (-2, 6) -- (-2, 8) -- (-4, 8);
        \draw (-2, 8) -- (-0.5, 9);


        \fill[black]  (-0.5, 1)      circle (4.5pt);
        \fill[black]  (-2.5, 5)      circle (4.5pt);
        \fill[black]  (-4.5, 6.333)  circle (4.5pt);
        \fill[black]  (-4, 8)        circle (4.5pt);
        \fill[black]  (-4.5, 9)      circle (4.5pt);
        \fill[black]  (-4, 10)       circle (4.5pt);
        \fill[black]  (-2.5, 10.333) circle (4.5pt);
        \fill[black]  (-1, 10)       circle (4.5pt);
        \fill[black]  (-0.5, 9)      circle (4.5pt);
        \fill[black]  (0.5, 9)       circle (4.5pt);
        \fill[black]  (1, 8)         circle (4.5pt);
        \fill[black]  (2.5, 7.666)   circle (4.5pt);
        \fill[black]  (2.5, 5)       circle (4.5pt);
        \fill[black]  (0.5, 1)       circle (4.5pt);
        \fill[black]  (-2, 6)        circle (4.5pt);
        \fill[black]  (-2, 8)        circle (4.5pt);

      \end{scope}
      \begin{scope}[shift={(0 cm,15.588 cm)},rotate=120,yscale=0.866]
        \draw[very thick] (-0.5, 1) -- (-2.5, 5) -- (-4.5, 6.333) -- (-4, 8) -- (-4.5, 9) -- (-4, 10) -- (-2.5, 10.333) -- (-1, 10) -- (-0.5, 9) -- (0.5, 9) -- (1, 8) -- (2.5, 7.666) -- (2.5, 5) -- (0.5, 1) -- (-0.5, 1);
        \draw (-2.5, 5) -- (-2, 6) -- (1, 8);
        \draw (-2, 6) -- (-2, 8) -- (-4, 8);
        \draw (-2, 8) -- (-0.5, 9);


        \fill[black]  (-0.5, 1)      circle (4.5pt);
        \fill[black]  (-2.5, 5)      circle (4.5pt);
        \fill[black]  (-4.5, 6.333)  circle (4.5pt);
        \fill[black]  (-4, 8)        circle (4.5pt);
        \fill[black]  (-4.5, 9)      circle (4.5pt);
        \fill[black]  (-4, 10)       circle (4.5pt);
        \fill[black]  (-2.5, 10.333) circle (4.5pt);
        \fill[black]  (-1, 10)       circle (4.5pt);
        \fill[black]  (-0.5, 9)      circle (4.5pt);
        \fill[black]  (0.5, 9)       circle (4.5pt);
        \fill[black]  (1, 8)         circle (4.5pt);
        \fill[black]  (2.5, 7.666)   circle (4.5pt);
        \fill[black]  (2.5, 5)       circle (4.5pt);
        \fill[black]  (0.5, 1)       circle (4.5pt);
        \fill[black]  (-2, 6)        circle (4.5pt);
        \fill[black]  (-2, 8)        circle (4.5pt);

      \end{scope}
    \end{scope}
  \end{tikzsubfigure}
  \begin{tikzsubfigure}{\label{fig:expansion:patch:5:7:c}}{$\mathcal{P}_F$}{1.0}
    \begin{scope}[scale=5]
      \node (n1) at (-0.915015, -0.259623) {5};
\node (n2) at (-0.696317, -0.557905) {5};
\node (n3) at (-0.841949, -0.135639) {5};
\node (n4) at (-0.834995, 0.218317) {5};
\node (n5) at (-0.573848, 0.004301) {7};
\node (n6) at (-0.679017, 0.626996) {5};
\node (n7) at (-0.258893, -0.139007) {7};
\node (n8) at (0.258894, 0.139006) {7};
\node (n9) at (0.573838, -0.004296) {7};
\node (n10) at (0.834967, -0.218343) {5};
\node (n11) at (0.841926, 0.135652) {5};
\node (n12) at (0.696298, 0.557908) {5};
\node (n13) at (0.678994, -0.627024) {5};
\node (n14) at (0.915000, 0.259623) {5};
\node (n15) at (0.815303, 0.532786) {7};
\node (n16) at (-0.052078, 0.894149) {7};
\node (n17) at (-0.869057, 0.437306) {7};
\node (n18) at (-0.815303, -0.532792) {7};
\node (n19) at (0.052078, -0.894149) {7};
\node (n20) at (0.869043, -0.437328) {7};

% \node (n0) at (-0.395585, -0.023040) {0};
% \node (n1) at (-0.915015, -0.259623) {1};
% \node (n2) at (-0.696317, -0.557905) {2};
% \node (n3) at (-0.841949, -0.135639) {3};
% \node (n4) at (-0.834995, 0.218317) {4};
% \node (n5) at (-0.573848, 0.004301) {5};
% \node (n6) at (-0.679017, 0.626996) {6};
% \node (n7) at (-0.258893, -0.139007) {7};
% \node (n8) at (0.258894, 0.139006) {7};
% \node (n9) at (0.573838, -0.004296) {9};
% \node (n10) at (0.834967, -0.218343) {10};
% \node (n11) at (0.841926, 0.135652) {11};
% \node (n12) at (0.696298, 0.557908) {12};
% \node (n13) at (0.678994, -0.627024) {13};
% \node (n14) at (0.915000, 0.259623) {14};
% \node (n15) at (0.815303, 0.532786) {15};
% \node (n16) at (-0.052078, 0.894149) {16};
% \node (n17) at (-0.869057, 0.437306) {17};
% \node (n18) at (-0.815303, -0.532792) {18};
% \node (n19) at (0.052078, -0.894149) {19};
% \node (n20) at (0.869043, -0.437328) {20};

\node (x0) at (-0.792605, -0.497214) {};
\node (x1) at (-0.841363, -0.282306) {};
\node (x2) at (-0.998308, -0.058145) {};
\node (x3) at (-0.984808, -0.173648) {};
\node (x4) at (-0.957990, -0.286803) {};
\node (x5) at (-0.642788, -0.766044) {};
\node (x6) at (-0.549509, -0.835488) {};
\node (x7) at (-0.655322, -0.408472) {};
\node (x8) at (-0.716442, 0.012586) {};
\node (x9) at (-0.998308, 0.058145) {};
\node (x10) at (-0.648555, 0.425995) {};
\node (x11) at (-0.826862, 0.421212) {};
\node (x12) at (-0.984808, 0.173648) {};
\node (x13) at (-0.448799, -0.893633) {};
\node (x14) at (-0.448799, 0.893633) {};
\node (x15) at (-0.549509, 0.835488) {};
\node (x16) at (-0.642788, 0.766044) {};
\node (x17) at (-0.727374, 0.686242) {};
\node (x18) at (-0.342020, -0.939693) {};
\node (x19) at (-0.230616, -0.973045) {};
\node (x20) at (0.022528, -0.386787) {};
\node (x21) at (-0.022528, 0.386783) {};
\node (x22) at (-0.342020, 0.939693) {};
\node (x23) at (0.342020, -0.939693) {};
\node (x24) at (0.448799, -0.893633) {};
\node (x25) at (0.448799, 0.893633) {};
\node (x26) at (0.342020, 0.939693) {};
\node (x27) at (0.230616, 0.973045) {};
\node (x28) at (0.549509, -0.835488) {};
\node (x29) at (0.648530, -0.425986) {};
\node (x30) at (0.716423, -0.012575) {};
\node (x31) at (0.655299, 0.408487) {};
\node (x32) at (0.549509, 0.835488) {};
\node (x33) at (0.826769, -0.421361) {};
\node (x34) at (0.984808, -0.173648) {};
\node (x35) at (0.998308, -0.058145) {};
\node (x36) at (0.998308, 0.058145) {};
\node (x37) at (0.841290, 0.282349) {};
\node (x38) at (0.792604, 0.497172) {};
\node (x39) at (0.642788, 0.766044) {};
\node (x40) at (0.642788, -0.766044) {};
\node (x41) at (0.727374, -0.686242) {};
\node (x42) at (0.984808, 0.173648) {};
\node (x43) at (0.957990, 0.286803) {};
\node (x44) at (0.918216, 0.396080) {};
\node (x45) at (0.866025, 0.500000) {};
\node (x46) at (0.802123, 0.597159) {};
\node (x47) at (0.727374, 0.686242) {};
\node (x48) at (0.116093, 0.993238) {};
\node (x49) at (0.000000, 1.000000) {};
\node (x50) at (-0.116093, 0.993238) {};
\node (x51) at (-0.230616, 0.973045) {};
\node (x52) at (-0.802123, 0.597159) {};
\node (x53) at (-0.866025, 0.500000) {};
\node (x54) at (-0.918216, 0.396080) {};
\node (x55) at (-0.957990, 0.286803) {};
\node (x56) at (-0.918216, -0.396080) {};
\node (x57) at (-0.866025, -0.500000) {};
\node (x58) at (-0.802123, -0.597159) {};
\node (x59) at (-0.727374, -0.686242) {};
\node (x60) at (-0.116093, -0.993238) {};
\node (x61) at (-0.000000, -1.000000) {};
\node (x62) at (0.116093, -0.993238) {};
\node (x63) at (0.230616, -0.973045) {};
\node (x64) at (0.802123, -0.597159) {};
\node (x65) at (0.866025, -0.500000) {};
\node (x66) at (0.918216, -0.396080) {};
\node (x67) at (0.957990, -0.286803) {};

\draw (-0.792605, -0.497214) -- (-0.841363, -0.282306);
\draw (-0.841363, -0.282306) -- (-0.998308, -0.058145);
\draw (-0.998308, -0.058145) -- (-0.984808, -0.173648);
\draw (-0.984808, -0.173648) -- (-0.957990, -0.286803);
\draw (-0.957990, -0.286803) -- (-0.792605, -0.497214);
\draw (-0.792605, -0.497214) -- (-0.642788, -0.766044);
\draw[very thick] (-0.642788, -0.766044) -- (-0.549509, -0.835488);
\draw[lface] (-0.642788, -0.766044) -- (n18);
\draw (-0.549509, -0.835488) -- (-0.655322, -0.408472);
\draw (-0.655322, -0.408472) -- (-0.841363, -0.282306);
\draw (-0.655322, -0.408472) -- (-0.716442, 0.012586);
\draw (-0.716442, 0.012586) -- (-0.998308, 0.058145);
\draw (-0.998308, 0.058145) -- (-0.998308, -0.058145);
\draw (-0.716442, 0.012586) -- (-0.648555, 0.425995);
\draw (-0.648555, 0.425995) -- (-0.826862, 0.421212);
\draw (-0.826862, 0.421212) -- (-0.984808, 0.173648);
\draw (-0.984808, 0.173648) -- (-0.998308, 0.058145);
\draw[very thick] (-0.549509, -0.835488) -- (-0.448799, -0.893633);
\draw[lface] (-0.448799, -0.893633) -- (n7);
\draw (-0.448799, -0.893633) -- (-0.448799, 0.893633);
\draw (-0.448799, 0.893633) -- (-0.549509, 0.835488);
\draw (-0.549509, 0.835488) -- (-0.648555, 0.425995);
\draw[very thick] (-0.549509, 0.835488) -- (-0.642788, 0.766044);
\draw[very thick] (-0.642788, 0.766044) -- (-0.727374, 0.686242);
\draw[lface] (-0.549509, 0.835488) -- (n5);
\draw[lface] (-0.727374, 0.686242) -- (n17);
\draw (-0.727374, 0.686242) -- (-0.826862, 0.421212);
\draw (-0.448799, -0.893633) -- (-0.342020, -0.939693);
\draw (-0.342020, -0.939693) -- (-0.230616, -0.973045);
\draw (-0.230616, -0.973045) -- (0.022528, -0.386787);
\draw[lsquare] (0.022528, -0.386787) -- (-0.022528, 0.386783);
\draw[lface] (0.022528, -0.386787) -- (n16);
\draw[lface] (-0.022528, 0.386787) -- (n19);
\draw (-0.022528, 0.386783) -- (-0.342020, 0.939693);
\draw (-0.342020, 0.939693) -- (-0.448799, 0.893633);
\draw (0.022528, -0.386787) -- (0.342020, -0.939693);
\draw (0.342020, -0.939693) -- (0.448799, -0.893633);
\draw (0.448799, -0.893633) -- (0.448799, 0.893633);
\draw (0.448799, 0.893633) -- (0.342020, 0.939693);
\draw (0.342020, 0.939693) -- (0.230616, 0.973045);
\draw (0.230616, 0.973045) -- (-0.022528, 0.386783);
\draw (0.448799, -0.893633) -- (0.549509, -0.835488);
\draw (0.549509, -0.835488) -- (0.648530, -0.425986);
\draw (0.648530, -0.425986) -- (0.716423, -0.012575);
\draw (0.716423, -0.012575) -- (0.655299, 0.408487);
\draw (0.655299, 0.408487) -- (0.549509, 0.835488);
\draw[very thick] (0.549509, 0.835488) -- (0.448799, 0.893633);
\draw[lface] (0.448799, 0.893633) -- (n8);
\draw (0.648530, -0.425986) -- (0.826769, -0.421361);
\draw (0.826769, -0.421361) -- (0.984808, -0.173648);
\draw (0.984808, -0.173648) -- (0.998308, -0.058145);
\draw (0.998308, -0.058145) -- (0.716423, -0.012575);
\draw (0.998308, -0.058145) -- (0.998308, 0.058145);
\draw (0.998308, 0.058145) -- (0.841290, 0.282349);
\draw (0.841290, 0.282349) -- (0.655299, 0.408487);
\draw (0.841290, 0.282349) -- (0.792604, 0.497172);
\draw (0.792604, 0.497172) -- (0.642788, 0.766044);
\draw[very thick] (0.642788, 0.766044) -- (0.549509, 0.835488);
\draw[lface] (0.642788, 0.766044) -- (n15);
\draw[very thick] (0.549509, -0.835488) -- (0.642788, -0.766044);
\draw[lface] (0.549509, -0.835488) -- (n9);
\draw[very thick] (0.642788, -0.766044) -- (0.727374, -0.686242);
\draw[lface] (0.727374, -0.686242) -- (n20);
\draw (0.727374, -0.686242) -- (0.826769, -0.421361);
\draw (0.998308, 0.058145) -- (0.984808, 0.173648);
\draw (0.984808, 0.173648) -- (0.957990, 0.286803);
\draw (0.957990, 0.286803) -- (0.792604, 0.497172);
\draw (0.957990, 0.286803) -- (0.918216, 0.396080);
\draw (0.918216, 0.396080) -- (0.866025, 0.500000);
\draw (0.866025, 0.500000) -- (0.802123, 0.597159);
\draw (0.802123, 0.597159) -- (0.727374, 0.686242);
\draw (0.727374, 0.686242) -- (0.642788, 0.766044);
\draw (0.230616, 0.973045) -- (0.116093, 0.993238);
\draw (0.116093, 0.993238) -- (0.000000, 1.000000);
\draw (0.000000, 1.000000) -- (-0.116093, 0.993238);
\draw (-0.116093, 0.993238) -- (-0.230616, 0.973045);
\draw (-0.230616, 0.973045) -- (-0.342020, 0.939693);
\draw (-0.727374, 0.686242) -- (-0.802123, 0.597159);
\draw (-0.802123, 0.597159) -- (-0.866025, 0.500000);
\draw (-0.866025, 0.500000) -- (-0.918216, 0.396080);
\draw (-0.918216, 0.396080) -- (-0.957990, 0.286803);
\draw (-0.957990, 0.286803) -- (-0.984808, 0.173648);
\draw (-0.957990, -0.286803) -- (-0.918216, -0.396080);
\draw (-0.918216, -0.396080) -- (-0.866025, -0.500000);
\draw (-0.866025, -0.500000) -- (-0.802123, -0.597159);
\draw (-0.802123, -0.597159) -- (-0.727374, -0.686242);
\draw (-0.727374, -0.686242) -- (-0.642788, -0.766044);
\draw (-0.230616, -0.973045) -- (-0.116093, -0.993238);
\draw (-0.116093, -0.993238) -- (-0.000000, -1.000000);
\draw (-0.000000, -1.000000) -- (0.116093, -0.993238);
\draw (0.116093, -0.993238) -- (0.230616, -0.973045);
\draw (0.230616, -0.973045) -- (0.342020, -0.939693);
\draw (0.727374, -0.686242) -- (0.802123, -0.597159);
\draw (0.802123, -0.597159) -- (0.866025, -0.500000);
\draw (0.866025, -0.500000) -- (0.918216, -0.396080);
\draw (0.918216, -0.396080) -- (0.957990, -0.286803);
\draw (0.957990, -0.286803) -- (0.984808, -0.173648);

\fill[black] (-0.792605, -0.497214) circle (0.3pt);
\fill[black] (-0.841363, -0.282306) circle (0.3pt);
\fill[black] (-0.998308, -0.058145) circle (0.3pt);
\fill[black] (-0.984808, -0.173648) circle (0.3pt);
\fill[black] (-0.957990, -0.286803) circle (0.3pt);
\fill[black] (-0.642788, -0.766044) circle (0.3pt);
\fill[black] (-0.549509, -0.835488) circle (0.3pt);
\fill[black] (-0.655322, -0.408472) circle (0.3pt);
\fill[black] (-0.716442, 0.012586) circle (0.3pt);
\fill[black] (-0.998308, 0.058145) circle (0.3pt);
\fill[black] (-0.648555, 0.425995) circle (0.3pt);
\fill[black] (-0.826862, 0.421212) circle (0.3pt);
\fill[black] (-0.984808, 0.173648) circle (0.3pt);
\fill[black] (-0.448799, -0.893633) circle (0.3pt);
\fill[black] (-0.448799, 0.893633) circle (0.3pt);
\fill[black] (-0.549509, 0.835488) circle (0.3pt);
\fill[black] (-0.642788, 0.766044) circle (0.3pt);
\fill[black] (-0.727374, 0.686242) circle (0.3pt);
\fill[black] (-0.342020, -0.939693) circle (0.3pt);
\fill[black] (-0.230616, -0.973045) circle (0.3pt);
\fill[black] (0.022528, -0.386787) circle (0.3pt);
\fill[black] (-0.022528, 0.386783) circle (0.3pt);
\fill[black] (-0.342020, 0.939693) circle (0.3pt);
\fill[black] (0.342020, -0.939693) circle (0.3pt);
\fill[black] (0.448799, -0.893633) circle (0.3pt);
\fill[black] (0.448799, 0.893633) circle (0.3pt);
\fill[black] (0.342020, 0.939693) circle (0.3pt);
\fill[black] (0.230616, 0.973045) circle (0.3pt);
\fill[black] (0.549509, -0.835488) circle (0.3pt);
\fill[black] (0.648530, -0.425986) circle (0.3pt);
\fill[black] (0.716423, -0.012575) circle (0.3pt);
\fill[black] (0.655299, 0.408487) circle (0.3pt);
\fill[black] (0.549509, 0.835488) circle (0.3pt);
\fill[black] (0.826769, -0.421361) circle (0.3pt);
\fill[black] (0.984808, -0.173648) circle (0.3pt);
\fill[black] (0.998308, -0.058145) circle (0.3pt);
\fill[black] (0.998308, 0.058145) circle (0.3pt);
\fill[black] (0.841290, 0.282349) circle (0.3pt);
\fill[black] (0.792604, 0.497172) circle (0.3pt);
\fill[black] (0.642788, 0.766044) circle (0.3pt);
\fill[black] (0.642788, -0.766044) circle (0.3pt);
\fill[black] (0.727374, -0.686242) circle (0.3pt);
\fill[black] (0.984808, 0.173648) circle (0.3pt);
\fill[black] (0.957990, 0.286803) circle (0.3pt);
\fill[black] (0.918216, 0.396080) circle (0.3pt);
\fill[black] (0.866025, 0.500000) circle (0.3pt);
\fill[black] (0.802123, 0.597159) circle (0.3pt);
\fill[black] (0.727374, 0.686242) circle (0.3pt);
\fill[black] (0.116093, 0.993238) circle (0.3pt);
\fill[black] (0.000000, 1.000000) circle (0.3pt);
\fill[black] (-0.116093, 0.993238) circle (0.3pt);
\fill[black] (-0.230616, 0.973045) circle (0.3pt);
\fill[black] (-0.802123, 0.597159) circle (0.3pt);
\fill[black] (-0.866025, 0.500000) circle (0.3pt);
\fill[black] (-0.918216, 0.396080) circle (0.3pt);
\fill[black] (-0.957990, 0.286803) circle (0.3pt);
\fill[black] (-0.918216, -0.396080) circle (0.3pt);
\fill[black] (-0.866025, -0.500000) circle (0.3pt);
\fill[black] (-0.802123, -0.597159) circle (0.3pt);
\fill[black] (-0.727374, -0.686242) circle (0.3pt);
\fill[black] (-0.116093, -0.993238) circle (0.3pt);
\fill[black] (-0.000000, -1.000000) circle (0.3pt);
\fill[black] (0.116093, -0.993238) circle (0.3pt);
\fill[black] (0.230616, -0.973045) circle (0.3pt);
\fill[black] (0.802123, -0.597159) circle (0.3pt);
\fill[black] (0.866025, -0.500000) circle (0.3pt);
\fill[black] (0.918216, -0.396080) circle (0.3pt);
\fill[black] (0.957990, -0.286803) circle (0.3pt);



    \end{scope}
  \end{tikzsubfigure}
\end{tikzfigure2}
\clearpage
\begin{theorem}
  Let $p$ and $v$ be a pair of admissible sequences for an orientable closed $2$-manifold $S$. Then $(p, v)$ is $[(5k + 2) \times 5, (5k+8)]$-$[3]$-realizable for all $k \in \nats$.
  \begin{proof}
    An expansion $3$-patch $\mathcal{P}_N$ with outer tuple $o = (1, 2, 2, 2, 1, 2, 1, 1, 1, 2)$ is shown in \autoref{fig:expansion:patch:5:8:a} and a corresponding $o$-$6$-gonal $3$-patch $\mathcal{P}_F$ is shown in \autoref{fig:expansion:patch:5:8:c}, both consisting of pentagons and octagons. By using \autoref{const:edge:replacement:5:1}, \autoref{const:edge:replacement:5:2} and \autoref{const:edge:replacement:5:3} as indicated we get $3$-patches consisting of only pentagons and $(5k+8)$-gons, $k \in \nats$. We can see in \autoref{fig:expansion:patch:5:8:b} that $\mathcal{P}_N$ has the polyhedral property, thus we can apply \autoref{thm:main:const} with $\mathcal{P}_P \defeq \mathcal{P}_N$.
  \end{proof}
\end{theorem}%
\begin{tikzfigure2}{}{}
  \begin{tikzsubfigure}{\label{fig:expansion:patch:5:8:a}}{$\mathcal{P}_N = \mathcal{P}_P$}{0.5}
    \begin{scope}[scale=0.55, yscale=0.866]
      \draw (-0.5, 1) -- (-1, 2) -- (-5.5, 5) -- (-8, 4) -- (-8.5, 5) -- (-8, 6) -- (-8.5, 7) -- (-8.5, 9) -- (-7, 10) -- (-6, 10) -- (-5.5, 11) -- (-4.5, 11) -- (-4, 10) -- (-2.5, 9) -- (-1, 10) -- (1, 8) -- (1, 2) -- (0.5, 1) -- (-0.5, 1);
      \draw (-5.5, 5) -- (-6, 6) -- (-8, 6);
      \draw (-1, 2) -- (-2.5, 5) (-4.5, 7) -- (-6, 6);
      \draw (-4.5, 7) -- (-4.5, 9) -- (-6, 10);
      \draw (-4.5, 9) -- (-4, 10);
      \draw (-2.5, 5) -- (-2.5, 9);
      \draw[lsquare] (-2.5, 5) -- (-4.5, 7);
      
      \fill[black]  (-0.5, 1)  circle(3pt);
      \fill[black]  (-1, 2)    circle(3pt);
      \fill[black]  (-8.5, 5)  circle(3pt);
      \fill[black]  (-8, 4)    circle(3pt);
      \fill[black]  (-5.5, 5)  circle(3pt);
      \fill[black]  (-8, 6)    circle(3pt);
      \fill[black]  (-8.5, 7)  circle(3pt);
      \fill[black]  (-8.5, 9)  circle(3pt);
      \fill[black]  (-7, 10)   circle(3pt);
      \fill[black]  (-6, 10)   circle(3pt);
      \fill[black]  (-5.5, 11) circle(3pt);
      \fill[black]  (-4.5, 11) circle(3pt);
      \fill[black]  (-4, 10)   circle(3pt);
      \fill[black]  (-2.5, 9)  circle(3pt);
      \fill[black]  (-1, 10)   circle(3pt);
      \fill[black]  (1, 8)     circle(3pt);
      \fill[black]  (1, 2)     circle(3pt);
      \fill[black]  (0.5, 1)   circle(3pt);
      \fill[black]  (-2.5, 5)  circle(3pt);
      \fill[black]  (-4.5, 7)  circle(3pt);
      \fill[black]  (-4.5, 9)  circle(3pt);
      \fill[black]  (-6, 6)    circle(3pt);


      \node[anchor= 90] at (-0.5, 1)  {$i_{0}$};
      \node[anchor= 45] at (-1, 2)    {$i_{1}$};
      \node[anchor=  0] at (-8.5, 5)  {$o_{1}$};
      \node[anchor= 90] at (-8, 4)    {$i_{3}=o_{0}$};
      \node[anchor= 90] at (-5.5, 5)  {$i_{2}$};
      \node[anchor=  0] at (-8, 6)    {$o_{2}$};
      \node[anchor=  0] at (-8.5, 7)  {$o_{3}$};
      \node[anchor=  0] at (-8.5, 9)  {$o_{4}$};
      \node[anchor=-30] at (-7, 10)   {$o_{5}$};
      \node[anchor=-60] at (-6, 10)   {$o_{6}$};
      \node[anchor=270] at (-5.5, 11) {$\mathbf{o_{7}}$};
      \node[anchor=270] at (-4.5, 11) {$o_{8}$};
      \node[anchor=240] at (-4, 10)   {$o_{9}$};
      \node[anchor=270] at (-2.5, 9)  {$o_{10}$};
      \node[anchor=270] at (-1, 10)   {$i_{3}'=o_{11}$};
      \node[anchor=210] at (1, 8)     {$i_{2}'$};
      \node[anchor=180] at (1, 2)     {$i_{1}'$};
      \node[anchor= 90] at (0.5, 1)   {$i_{0}'$};

      \node (k1) at (-1,6) {$8$};
      \node (k2) at (-7,8) {$8$};
      \node at (-4,5.5) {$5$};
      \node at (-3.3,7) {$5$};
      \node at (-5,10) {$5$};
      \node at (-7,5) {$5$};

      \draw[lface] (-2.5, 5) -- (k1);
      \draw[lface] (-4.5, 7) -- (k2);

    \end{scope}
  \end{tikzsubfigure}%
  \begin{tikzsubfigure}{\label{fig:expansion:patch:5:8:b}}{Edge patch of $\mathcal{P}_N$}{0.5}
    \begin{scope}[scale=0.3]
      \begin{scope}[yscale=0.866]
        \draw[very thick] (-0.5, 1) -- (-1, 2) -- (-5.5, 5) -- (-8, 4) -- (-8.5, 5) -- (-8, 6) -- (-8.5, 7) -- (-8.5, 9) -- (-7, 10) -- (-6, 10) -- (-5.5, 11) -- (-4.5, 11) -- (-4, 10) -- (-2.5, 9) -- (-1, 10) -- (1, 8) -- (1, 2) -- (0.5, 1) -- (-0.5, 1);
        \draw (-5.5, 5) -- (-6, 6) -- (-8, 6);
        \draw (-1, 2) -- (-2.5, 5) -- (-4.5, 7) -- (-6, 6);
        \draw (-4.5, 7) -- (-4.5, 9) -- (-6, 10);
        \draw (-4.5, 9) -- (-4, 10);
        \draw (-2.5, 5) -- (-2.5, 9);

        \fill[black]  (-0.5, 1)  circle(4.5pt);
        \fill[black]  (-1, 2)    circle(4.5pt);
        \fill[black]  (-8.5, 5)  circle(4.5pt);
        \fill[black]  (-8, 4)    circle(4.5pt);
        \fill[black]  (-5.5, 5)  circle(4.5pt);
        \fill[black]  (-8, 6)    circle(4.5pt);
        \fill[black]  (-8.5, 7)  circle(4.5pt);
        \fill[black]  (-8.5, 9)  circle(4.5pt);
        \fill[black]  (-7, 10)   circle(4.5pt);
        \fill[black]  (-6, 10)   circle(4.5pt);
        \fill[black]  (-5.5, 11) circle(4.5pt);
        \fill[black]  (-4.5, 11) circle(4.5pt);
        \fill[black]  (-4, 10)   circle(4.5pt);
        \fill[black]  (-2.5, 9)  circle(4.5pt);
        \fill[black]  (-1, 10)   circle(4.5pt);
        \fill[black]  (1, 8)     circle(4.5pt);
        \fill[black]  (1, 2)     circle(4.5pt);
        \fill[black]  (0.5, 1)   circle(4.5pt);
        \fill[black]  (-2.5, 5)  circle(4.5pt);
        \fill[black]  (-4.5, 7)  circle(4.5pt);
        \fill[black]  (-4.5, 9)  circle(4.5pt);
        \fill[black]  (-6, 6)    circle(4.5pt);

      \end{scope}
      \begin{scope}[rotate=-60, yscale=0.866]
        \draw[very thick] (-0.5, 1) -- (-1, 2) -- (-5.5, 5) -- (-8, 4) -- (-8.5, 5) -- (-8, 6) -- (-8.5, 7) -- (-8.5, 9) -- (-7, 10) -- (-6, 10) -- (-5.5, 11) -- (-4.5, 11) -- (-4, 10) -- (-2.5, 9) -- (-1, 10) -- (1, 8) -- (1, 2) -- (0.5, 1) -- (-0.5, 1);
        \draw (-5.5, 5) -- (-6, 6) -- (-8, 6);
        \draw (-1, 2) -- (-2.5, 5) -- (-4.5, 7) -- (-6, 6);
        \draw (-4.5, 7) -- (-4.5, 9) -- (-6, 10);
        \draw (-4.5, 9) -- (-4, 10);
        \draw (-2.5, 5) -- (-2.5, 9);


        \fill[black]  (-0.5, 1)  circle(4.5pt);
        \fill[black]  (-1, 2)    circle(4.5pt);
        \fill[black]  (-8.5, 5)  circle(4.5pt);
        \fill[black]  (-8, 4)    circle(4.5pt);
        \fill[black]  (-5.5, 5)  circle(4.5pt);
        \fill[black]  (-8, 6)    circle(4.5pt);
        \fill[black]  (-8.5, 7)  circle(4.5pt);
        \fill[black]  (-8.5, 9)  circle(4.5pt);
        \fill[black]  (-7, 10)   circle(4.5pt);
        \fill[black]  (-6, 10)   circle(4.5pt);
        \fill[black]  (-5.5, 11) circle(4.5pt);
        \fill[black]  (-4.5, 11) circle(4.5pt);
        \fill[black]  (-4, 10)   circle(4.5pt);
        \fill[black]  (-2.5, 9)  circle(4.5pt);
        \fill[black]  (-1, 10)   circle(4.5pt);
        \fill[black]  (1, 8)     circle(4.5pt);
        \fill[black]  (1, 2)     circle(4.5pt);
        \fill[black]  (0.5, 1)   circle(4.5pt);
        \fill[black]  (-2.5, 5)  circle(4.5pt);
        \fill[black]  (-4.5, 7)  circle(4.5pt);
        \fill[black]  (-4.5, 9)  circle(4.5pt);
        \fill[black]  (-6, 6)    circle(4.5pt);

      \end{scope}
      \begin{scope}[yscale=0.866,shift={(0 cm,22 cm)},rotate=180]
        \draw[very thick] (-0.5, 1) -- (-1, 2) -- (-5.5, 5) -- (-8, 4) -- (-8.5, 5) -- (-8, 6) -- (-8.5, 7) -- (-8.5, 9) -- (-7, 10) -- (-6, 10) -- (-5.5, 11) -- (-4.5, 11) -- (-4, 10) -- (-2.5, 9) -- (-1, 10) -- (1, 8) -- (1, 2) -- (0.5, 1) -- (-0.5, 1);
        \draw (-5.5, 5) -- (-6, 6) -- (-8, 6);
        \draw (-1, 2) -- (-2.5, 5) -- (-4.5, 7) -- (-6, 6);
        \draw (-4.5, 7) -- (-4.5, 9) -- (-6, 10);
        \draw (-4.5, 9) -- (-4, 10);
        \draw (-2.5, 5) -- (-2.5, 9);


        \fill[black]  (-0.5, 1)  circle(4.5pt);
        \fill[black]  (-1, 2)    circle(4.5pt);
        \fill[black]  (-8.5, 5)  circle(4.5pt);
        \fill[black]  (-8, 4)    circle(4.5pt);
        \fill[black]  (-5.5, 5)  circle(4.5pt);
        \fill[black]  (-8, 6)    circle(4.5pt);
        \fill[black]  (-8.5, 7)  circle(4.5pt);
        \fill[black]  (-8.5, 9)  circle(4.5pt);
        \fill[black]  (-7, 10)   circle(4.5pt);
        \fill[black]  (-6, 10)   circle(4.5pt);
        \fill[black]  (-5.5, 11) circle(4.5pt);
        \fill[black]  (-4.5, 11) circle(4.5pt);
        \fill[black]  (-4, 10)   circle(4.5pt);
        \fill[black]  (-2.5, 9)  circle(4.5pt);
        \fill[black]  (-1, 10)   circle(4.5pt);
        \fill[black]  (1, 8)     circle(4.5pt);
        \fill[black]  (1, 2)     circle(4.5pt);
        \fill[black]  (0.5, 1)   circle(4.5pt);
        \fill[black]  (-2.5, 5)  circle(4.5pt);
        \fill[black]  (-4.5, 7)  circle(4.5pt);
        \fill[black]  (-4.5, 9)  circle(4.5pt);
        \fill[black]  (-6, 6)    circle(4.5pt);

      \end{scope}
      \begin{scope}[shift={(0 cm,19.052 cm)},rotate=120,yscale=0.866]
        \draw[very thick] (-0.5, 1) -- (-1, 2) -- (-5.5, 5) -- (-8, 4) -- (-8.5, 5) -- (-8, 6) -- (-8.5, 7) -- (-8.5, 9) -- (-7, 10) -- (-6, 10) -- (-5.5, 11) -- (-4.5, 11) -- (-4, 10) -- (-2.5, 9) -- (-1, 10) -- (1, 8) -- (1, 2) -- (0.5, 1) -- (-0.5, 1);
        \draw (-5.5, 5) -- (-6, 6) -- (-8, 6);
        \draw (-1, 2) -- (-2.5, 5) -- (-4.5, 7) -- (-6, 6);
        \draw (-4.5, 7) -- (-4.5, 9) -- (-6, 10);
        \draw (-4.5, 9) -- (-4, 10);
        \draw (-2.5, 5) -- (-2.5, 9);


        \fill[black]  (-0.5, 1)  circle(4.5pt);
        \fill[black]  (-1, 2)    circle(4.5pt);
        \fill[black]  (-8.5, 5)  circle(4.5pt);
        \fill[black]  (-8, 4)    circle(4.5pt);
        \fill[black]  (-5.5, 5)  circle(4.5pt);
        \fill[black]  (-8, 6)    circle(4.5pt);
        \fill[black]  (-8.5, 7)  circle(4.5pt);
        \fill[black]  (-8.5, 9)  circle(4.5pt);
        \fill[black]  (-7, 10)   circle(4.5pt);
        \fill[black]  (-6, 10)   circle(4.5pt);
        \fill[black]  (-5.5, 11) circle(4.5pt);
        \fill[black]  (-4.5, 11) circle(4.5pt);
        \fill[black]  (-4, 10)   circle(4.5pt);
        \fill[black]  (-2.5, 9)  circle(4.5pt);
        \fill[black]  (-1, 10)   circle(4.5pt);
        \fill[black]  (1, 8)     circle(4.5pt);
        \fill[black]  (1, 2)     circle(4.5pt);
        \fill[black]  (0.5, 1)   circle(4.5pt);
        \fill[black]  (-2.5, 5)  circle(4.5pt);
        \fill[black]  (-4.5, 7)  circle(4.5pt);
        \fill[black]  (-4.5, 9)  circle(4.5pt);
        \fill[black]  (-6, 6)    circle(4.5pt);

      \end{scope}
    \end{scope}
  \end{tikzsubfigure}
  \begin{tikzsubfigure}{\label{fig:expansion:patch:5:8:c}}{$\mathcal{P}_F$}{1.0}
    \begin{scope}[scale=5]
      \coordinate (x0) at (-0.297003, 0.445473);
\coordinate (x1) at (-0.470182, 0.606872);
\coordinate (x2) at (-0.630409, 0.404927);
\coordinate (x3) at (-0.747357, 0.175386);
\coordinate (x4) at (-0.528641, 0.085018);
\coordinate (x5) at (-0.009460, 0.760053);
\coordinate (x6) at (-0.371662, 0.928368);
\coordinate (x7) at (-0.458227, 0.888835);
\coordinate (x8) at (-0.540641, 0.841254);
\coordinate (x9) at (-0.756908, 0.486478);
\coordinate (x10) at (-0.989821, 0.142315);
\coordinate (x11) at (-0.998867, 0.047582);
\coordinate (x12) at (-0.998867, -0.047582);
\coordinate (x13) at (-0.695335, -0.307174);
\coordinate (x14) at (-0.618159, -0.786053);
\coordinate (x15) at (-0.540641, -0.841254);
\coordinate (x16) at (0.540641, 0.841254);
\coordinate (x17) at (0.458227, 0.888835);
\coordinate (x18) at (-0.458227, -0.888835);
\coordinate (x19) at (0.009400, -0.760084);
\coordinate (x20) at (0.296922, -0.445520);
\coordinate (x21) at (0.528605, -0.085087);
\coordinate (x22) at (0.695323, 0.307178);
\coordinate (x23) at (0.618159, 0.786053);
\coordinate (x24) at (0.469997, -0.606880);
\coordinate (x25) at (0.630228, -0.405177);
\coordinate (x26) at (0.747371, -0.175522);
\coordinate (x27) at (0.371662, -0.928368);
\coordinate (x28) at (0.458227, -0.888835);
\coordinate (x29) at (0.540641, -0.841254);
\coordinate (x30) at (0.756899, -0.486486);
\coordinate (x31) at (0.989821, -0.142315);
\coordinate (x32) at (0.998867, -0.047582);
\coordinate (x33) at (0.998867, 0.047582);
\coordinate (x34) at (0.989821, 0.142315);
\coordinate (x35) at (0.971812, 0.235759);
\coordinate (x36) at (0.945001, 0.327068);
\coordinate (x37) at (0.755750, 0.654861);
\coordinate (x38) at (0.690079, 0.723734);
\coordinate (x39) at (0.909632, 0.415415);
\coordinate (x40) at (0.866025, 0.500000);
\coordinate (x41) at (0.814576, 0.580057);
\coordinate (x42) at (0.371662, 0.928368);
\coordinate (x43) at (0.281733, 0.959493);
\coordinate (x44) at (0.189251, 0.981929);
\coordinate (x45) at (-0.189251, 0.981929);
\coordinate (x46) at (-0.281733, 0.959493);
\coordinate (x47) at (0.095056, 0.995472);
\coordinate (x48) at (0.000000, 1.000000);
\coordinate (x49) at (-0.095056, 0.995472);
\coordinate (x50) at (-0.618159, 0.786053);
\coordinate (x51) at (-0.690079, 0.723734);
\coordinate (x52) at (-0.755750, 0.654861);
\coordinate (x53) at (-0.945001, 0.327068);
\coordinate (x54) at (-0.971812, 0.235759);
\coordinate (x55) at (-0.814576, 0.580057);
\coordinate (x56) at (-0.866025, 0.500000);
\coordinate (x57) at (-0.909632, 0.415415);
\coordinate (x58) at (-0.989821, -0.142315);
\coordinate (x59) at (-0.971812, -0.235759);
\coordinate (x60) at (-0.945001, -0.327068);
\coordinate (x61) at (-0.755750, -0.654861);
\coordinate (x62) at (-0.690079, -0.723734);
\coordinate (x63) at (-0.909632, -0.415415);
\coordinate (x64) at (-0.866025, -0.500000);
\coordinate (x65) at (-0.814576, -0.580057);
\coordinate (x66) at (-0.371662, -0.928368);
\coordinate (x67) at (-0.281733, -0.959493);
\coordinate (x68) at (-0.189251, -0.981929);
\coordinate (x69) at (0.189251, -0.981929);
\coordinate (x70) at (0.281733, -0.959493);
\coordinate (x71) at (-0.095056, -0.995472);
\coordinate (x72) at (-0.000000, -1.000000);
\coordinate (x73) at (0.095056, -0.995472);
\coordinate (x74) at (0.618159, -0.786053);
\coordinate (x75) at (0.690079, -0.723734);
\coordinate (x76) at (0.755750, -0.654861);
\coordinate (x77) at (0.945001, -0.327068);
\coordinate (x78) at (0.971812, -0.235759);
\coordinate (x79) at (0.814576, -0.580057);
\coordinate (x80) at (0.866025, -0.500000);
\coordinate (x81) at (0.909632, -0.415415);

\draw (-0.297003, 0.445473) -- (-0.470182, 0.606872);
\draw (-0.470182, 0.606872) -- (-0.630409, 0.404927);
\draw (-0.630409, 0.404927) -- (-0.747357, 0.175386);
\draw (-0.747357, 0.175386) -- (-0.528641, 0.085018);
\draw (-0.528641, 0.085018) -- (-0.297003, 0.445473);
\draw (-0.297003, 0.445473) -- (-0.009460, 0.760053);
\draw (-0.009460, 0.760053) -- (-0.371662, 0.928368);
\draw[very thick] (-0.371662, 0.928368) -- (-0.458227, 0.888835);
\draw (-0.458227, 0.888835) -- (-0.470182, 0.606872);
\draw[very thick] (-0.458227, 0.888835) -- (-0.540641, 0.841254);
\draw (-0.540641, 0.841254) -- (-0.756908, 0.486478);
\draw (-0.756908, 0.486478) -- (-0.630409, 0.404927);
\draw (-0.756908, 0.486478) -- (-0.989821, 0.142315);
\draw (-0.989821, 0.142315) -- (-0.998867, 0.047582);
\draw (-0.998867, 0.047582) -- (-0.747357, 0.175386);
\draw (-0.998867, 0.047582) -- (-0.998867, -0.047582);
\draw (-0.998867, -0.047582) -- (-0.695335, -0.307174);
\draw (-0.695335, -0.307174) -- (-0.528641, 0.085018);
\draw[ldiamond] (-0.695335, -0.307174) -- (-0.618159, -0.786053);
\draw (-0.618159, -0.786053) -- (-0.540641, -0.841254);
\draw (-0.540641, -0.841254) -- (0.540641, 0.841254);
\draw (0.540641, 0.841254) -- (0.458227, 0.888835);
\draw (0.458227, 0.888835) -- (-0.009460, 0.760053);
\draw (-0.540641, -0.841254) -- (-0.458227, -0.888835);
\draw (-0.458227, -0.888835) -- (0.009400, -0.760084);
\draw (0.009400, -0.760084) -- (0.296922, -0.445520);
\draw (0.296922, -0.445520) -- (0.528605, -0.085087);
\draw (0.528605, -0.085087) -- (0.695323, 0.307178);
\draw[ldiamond] (0.695323, 0.307178) -- (0.618159, 0.786053);
\draw (0.618159, 0.786053) -- (0.540641, 0.841254);
\draw (0.296922, -0.445520) -- (0.469997, -0.606880);
\draw (0.469997, -0.606880) -- (0.630228, -0.405177);
\draw (0.630228, -0.405177) -- (0.747371, -0.175522);
\draw (0.747371, -0.175522) -- (0.528605, -0.085087);
\draw (0.009400, -0.760084) -- (0.371662, -0.928368);
\draw[very thick] (0.371662, -0.928368) -- (0.458227, -0.888835);
\draw (0.458227, -0.888835) -- (0.469997, -0.606880);
\draw[very thick] (0.458227, -0.888835) -- (0.540641, -0.841254);
\draw (0.540641, -0.841254) -- (0.756899, -0.486486);
\draw (0.756899, -0.486486) -- (0.630228, -0.405177);
\draw (0.756899, -0.486486) -- (0.989821, -0.142315);
\draw (0.989821, -0.142315) -- (0.998867, -0.047582);
\draw (0.998867, -0.047582) -- (0.747371, -0.175522);
\draw (0.998867, -0.047582) -- (0.998867, 0.047582);
\draw (0.998867, 0.047582) -- (0.695323, 0.307178);
\draw (0.998867, 0.047582) -- (0.989821, 0.142315);
\draw (0.989821, 0.142315) -- (0.971812, 0.235759);
\draw (0.971812, 0.235759) -- (0.945001, 0.327068);
\draw (0.945001, 0.327068) -- (0.755750, 0.654861);
\draw (0.755750, 0.654861) -- (0.690079, 0.723734);
\draw (0.690079, 0.723734) -- (0.618159, 0.786053);
\draw (0.945001, 0.327068) -- (0.909632, 0.415415);
\draw (0.909632, 0.415415) -- (0.866025, 0.500000);
\draw (0.866025, 0.500000) -- (0.814576, 0.580057);
\draw (0.814576, 0.580057) -- (0.755750, 0.654861);
\draw (0.458227, 0.888835) -- (0.371662, 0.928368);
\draw (0.371662, 0.928368) -- (0.281733, 0.959493);
\draw (0.281733, 0.959493) -- (0.189251, 0.981929);
\draw (0.189251, 0.981929) -- (-0.189251, 0.981929);
\draw (-0.189251, 0.981929) -- (-0.281733, 0.959493);
\draw (-0.281733, 0.959493) -- (-0.371662, 0.928368);
\draw (0.189251, 0.981929) -- (0.095056, 0.995472);
\draw (0.095056, 0.995472) -- (0.000000, 1.000000);
\draw (0.000000, 1.000000) -- (-0.095056, 0.995472);
\draw (-0.095056, 0.995472) -- (-0.189251, 0.981929);
\draw (-0.540641, 0.841254) -- (-0.618159, 0.786053);
\draw (-0.618159, 0.786053) -- (-0.690079, 0.723734);
\draw (-0.690079, 0.723734) -- (-0.755750, 0.654861);
\draw (-0.755750, 0.654861) -- (-0.945001, 0.327068);
\draw (-0.945001, 0.327068) -- (-0.971812, 0.235759);
\draw (-0.971812, 0.235759) -- (-0.989821, 0.142315);
\draw (-0.755750, 0.654861) -- (-0.814576, 0.580057);
\draw (-0.814576, 0.580057) -- (-0.866025, 0.500000);
\draw (-0.866025, 0.500000) -- (-0.909632, 0.415415);
\draw (-0.909632, 0.415415) -- (-0.945001, 0.327068);
\draw (-0.998867, -0.047582) -- (-0.989821, -0.142315);
\draw (-0.989821, -0.142315) -- (-0.971812, -0.235759);
\draw (-0.971812, -0.235759) -- (-0.945001, -0.327068);
\draw (-0.945001, -0.327068) -- (-0.755750, -0.654861);
\draw (-0.755750, -0.654861) -- (-0.690079, -0.723734);
\draw (-0.690079, -0.723734) -- (-0.618159, -0.786053);
\draw (-0.945001, -0.327068) -- (-0.909632, -0.415415);
\draw (-0.909632, -0.415415) -- (-0.866025, -0.500000);
\draw (-0.866025, -0.500000) -- (-0.814576, -0.580057);
\draw (-0.814576, -0.580057) -- (-0.755750, -0.654861);
\draw (-0.458227, -0.888835) -- (-0.371662, -0.928368);
\draw (-0.371662, -0.928368) -- (-0.281733, -0.959493);
\draw (-0.281733, -0.959493) -- (-0.189251, -0.981929);
\draw (-0.189251, -0.981929) -- (0.189251, -0.981929);
\draw (0.189251, -0.981929) -- (0.281733, -0.959493);
\draw (0.281733, -0.959493) -- (0.371662, -0.928368);
\draw (-0.189251, -0.981929) -- (-0.095056, -0.995472);
\draw (-0.095056, -0.995472) -- (-0.000000, -1.000000);
\draw (-0.000000, -1.000000) -- (0.095056, -0.995472);
\draw (0.095056, -0.995472) -- (0.189251, -0.981929);
\draw (0.540641, -0.841254) -- (0.618159, -0.786053);
\draw (0.618159, -0.786053) -- (0.690079, -0.723734);
\draw (0.690079, -0.723734) -- (0.755750, -0.654861);
\draw (0.755750, -0.654861) -- (0.945001, -0.327068);
\draw (0.945001, -0.327068) -- (0.971812, -0.235759);
\draw (0.971812, -0.235759) -- (0.989821, -0.142315);
\draw (0.755750, -0.654861) -- (0.814576, -0.580057);
\draw (0.814576, -0.580057) -- (0.866025, -0.500000);
\draw (0.866025, -0.500000) -- (0.909632, -0.415415);
\draw (0.909632, -0.415415) -- (0.945001, -0.327068);

% \node at (0.228530, 0.078419) {0};
% \node at (-0.534718, 0.343535) {1};
% \node at (-0.321307, 0.725920) {2};
% \node at (-0.571273, 0.645673) {3};
% \node at (-0.824672, 0.251338) {4};
% \node at (-0.793813, -0.009354) {5};
% \node at (-0.211296, 0.135769) {6};
% \node at (0.211273, -0.135787) {7};
% \node at (0.534625, -0.343637) {8};
% \node at (0.321242, -0.725937) {9};
% \node at (0.571198, -0.645726) {10};
% \node at (0.824637, -0.251416) {11};
% \node at (0.793807, 0.009314) {12};
% \node at (0.833101, 0.403069) {13};
% \node at (0.858197, 0.495480) {14};
% \node at (0.056096, 0.923559) {15};
% \node at (0.000000, 0.990960) {16};
% \node at (-0.783521, 0.524690) {17};
% \node at (-0.858197, 0.495480) {18};
% \node at (-0.833103, -0.403068) {19};
% \node at (-0.858197, -0.495480) {20};
% \node at (-0.056103, -0.923562) {21};
% \node at (-0.000000, -0.990960) {22};
% \node at (0.783520, -0.524691) {23};
% \node at (0.858197, -0.495480) {24};

\fill[black] (-0.297003, 0.445473) circle (0.2pt);
\fill[black] (-0.470182, 0.606872) circle (0.2pt);
\fill[black] (-0.630409, 0.404927) circle (0.2pt);
\fill[black] (-0.747357, 0.175386) circle (0.2pt);
\fill[black] (-0.528641, 0.085018) circle (0.2pt);
\fill[black] (-0.009460, 0.760053) circle (0.2pt);
\fill[black] (-0.371662, 0.928368) circle (0.2pt);
\fill[black] (-0.458227, 0.888835) circle (0.2pt);
\fill[black] (-0.540641, 0.841254) circle (0.2pt);
\fill[black] (-0.756908, 0.486478) circle (0.2pt);
\fill[black] (-0.989821, 0.142315) circle (0.2pt);
\fill[black] (-0.998867, 0.047582) circle (0.2pt);
\fill[black] (-0.998867, -0.047582) circle (0.2pt);
\fill[black] (-0.695335, -0.307174) circle (0.2pt);
\fill[black] (-0.618159, -0.786053) circle (0.2pt);
\fill[black] (-0.540641, -0.841254) circle (0.2pt);
\fill[black] (0.540641, 0.841254) circle (0.2pt);
\fill[black] (0.458227, 0.888835) circle (0.2pt);
\fill[black] (-0.458227, -0.888835) circle (0.2pt);
\fill[black] (0.009400, -0.760084) circle (0.2pt);
\fill[black] (0.296922, -0.445520) circle (0.2pt);
\fill[black] (0.528605, -0.085087) circle (0.2pt);
\fill[black] (0.695323, 0.307178) circle (0.2pt);
\fill[black] (0.618159, 0.786053) circle (0.2pt);
\fill[black] (0.469997, -0.606880) circle (0.2pt);
\fill[black] (0.630228, -0.405177) circle (0.2pt);
\fill[black] (0.747371, -0.175522) circle (0.2pt);
\fill[black] (0.371662, -0.928368) circle (0.2pt);
\fill[black] (0.458227, -0.888835) circle (0.2pt);
\fill[black] (0.540641, -0.841254) circle (0.2pt);
\fill[black] (0.756899, -0.486486) circle (0.2pt);
\fill[black] (0.989821, -0.142315) circle (0.2pt);
\fill[black] (0.998867, -0.047582) circle (0.2pt);
\fill[black] (0.998867, 0.047582) circle (0.2pt);
\fill[black] (0.989821, 0.142315) circle (0.2pt);
\fill[black] (0.971812, 0.235759) circle (0.2pt);
\fill[black] (0.945001, 0.327068) circle (0.2pt);
\fill[black] (0.755750, 0.654861) circle (0.2pt);
\fill[black] (0.690079, 0.723734) circle (0.2pt);
\fill[black] (0.909632, 0.415415) circle (0.2pt);
\fill[black] (0.866025, 0.500000) circle (0.2pt);
\fill[black] (0.814576, 0.580057) circle (0.2pt);
\fill[black] (0.371662, 0.928368) circle (0.2pt);
\fill[black] (0.281733, 0.959493) circle (0.2pt);
\fill[black] (0.189251, 0.981929) circle (0.2pt);
\fill[black] (-0.189251, 0.981929) circle (0.2pt);
\fill[black] (-0.281733, 0.959493) circle (0.2pt);
\fill[black] (0.095056, 0.995472) circle (0.2pt);
\fill[black] (0.000000, 1.000000) circle (0.2pt);
\fill[black] (-0.095056, 0.995472) circle (0.2pt);
\fill[black] (-0.618159, 0.786053) circle (0.2pt);
\fill[black] (-0.690079, 0.723734) circle (0.2pt);
\fill[black] (-0.755750, 0.654861) circle (0.2pt);
\fill[black] (-0.945001, 0.327068) circle (0.2pt);
\fill[black] (-0.971812, 0.235759) circle (0.2pt);
\fill[black] (-0.814576, 0.580057) circle (0.2pt);
\fill[black] (-0.866025, 0.500000) circle (0.2pt);
\fill[black] (-0.909632, 0.415415) circle (0.2pt);
\fill[black] (-0.989821, -0.142315) circle (0.2pt);
\fill[black] (-0.971812, -0.235759) circle (0.2pt);
\fill[black] (-0.945001, -0.327068) circle (0.2pt);
\fill[black] (-0.755750, -0.654861) circle (0.2pt);
\fill[black] (-0.690079, -0.723734) circle (0.2pt);
\fill[black] (-0.909632, -0.415415) circle (0.2pt);
\fill[black] (-0.866025, -0.500000) circle (0.2pt);
\fill[black] (-0.814576, -0.580057) circle (0.2pt);
\fill[black] (-0.371662, -0.928368) circle (0.2pt);
\fill[black] (-0.281733, -0.959493) circle (0.2pt);
\fill[black] (-0.189251, -0.981929) circle (0.2pt);
\fill[black] (0.189251, -0.981929) circle (0.2pt);
\fill[black] (0.281733, -0.959493) circle (0.2pt);
\fill[black] (-0.095056, -0.995472) circle (0.2pt);
\fill[black] (-0.000000, -1.000000) circle (0.2pt);
\fill[black] (0.095056, -0.995472) circle (0.2pt);
\fill[black] (0.618159, -0.786053) circle (0.2pt);
\fill[black] (0.690079, -0.723734) circle (0.2pt);
\fill[black] (0.755750, -0.654861) circle (0.2pt);
\fill[black] (0.945001, -0.327068) circle (0.2pt);
\fill[black] (0.971812, -0.235759) circle (0.2pt);
\fill[black] (0.814576, -0.580057) circle (0.2pt);
\fill[black] (0.866025, -0.500000) circle (0.2pt);
\fill[black] (0.909632, -0.415415) circle (0.2pt);

\node at (-0.534718, 0.343535) {5};
\node at (-0.321307, 0.725920) {5};
\node at (-0.571273, 0.645673) {5};
\node at (-0.824672, 0.251338) {5};
\node at (-0.793813, -0.009354) {5};
\node at (-0.211296, 0.135769) {8};
\node at (0.211273, -0.135787) {8};
\node at (0.534625, -0.343637) {5};
\node at (0.321242, -0.725937) {5};
\node at (0.571198, -0.645726) {5};
\node at (0.824637, -0.251416) {5};
\node at (0.793807, 0.009314) {5};
\node at (0.833101, 0.403069) {8};
\node[anchor=210] at (0.858197, 0.495480) {5};
\node at (0.056096, 0.923559) {8};
\node[anchor=270] at (0.000000, 0.990960) {5};
\node at (-0.783521, 0.524690) {8};
\node[anchor=330] at (-0.858197, 0.495480) {5};
\node at (-0.833103, -0.403068) {8};
\node[anchor= 30] at (-0.858197, -0.495480) {5};
\node at (-0.056103, -0.923562) {8};
\node[anchor= 90] at (-0.000000, -0.990960) {5};
\node at (0.783520, -0.524691) {8};
\node[anchor=150] at (0.858197, -0.495480) {5};

    \end{scope}
  \end{tikzsubfigure}
\end{tikzfigure2}
\clearpage
\begin{theorem}
  Let $p$ and $v$ be a pair of admissible sequences for an orientable closed $2$-manifold $S$. Then $(p, v)$ is $[(5k + 3) \times 5, (5k+9)]$-$[3]$-realizable for all $k \in \nats$.
  \begin{proof}
    An expansion $3$-patch $\mathcal{P}_N$ with outer tuple $o = (1, 1, 2, 2, 2, 2, 1, 2, 1, 2, 1, 1, 1, 1, 2, 2)$ is shown in \autoref{fig:expansion:patch:5:9:a} and a corresponding $o$-$6$-gonal $3$-patch $\mathcal{P}_F$ is shown in \autoref{fig:expansion:patch:5:9:c}, both consisting of pentagons and enneagons. By using \autoref{const:edge:replacement:5:1} as indicated we get $3$-patches consisting of only pentagons and $(5k+9)$-gons, $k \in \nats$. We can see in \autoref{fig:expansion:patch:5:9:b} that $\mathcal{P}_N$ has the polyhedral property, thus we can apply \autoref{thm:main:const} with $\mathcal{P}_P \defeq \mathcal{P}_N$.
  \end{proof}
\end{theorem}
\begin{tikzfigure2}
  \begin{tikzsubfigure}{\label{fig:expansion:patch:5:9:a}}{$\mathcal{P}_N = \mathcal{P}_P$}{1.0}
    \begin{scope}[yscale=0.866, scale=0.8]
      \draw (0.5,1) -- (-0.5,1) -- (-5,10) -- (-2.5,17) -- (-2,17.666) -- (-0.5,16.5) -- (0.5,17.5) -- (2,16.333) -- (2.5,17) -- (4,18) -- (5,18) -- (6,17.666) -- (6.5,16.333) -- (8,16) -- (8.5,17) -- (9.5,17) -- (10,15.333) -- (9.25,14.167) -- (9.5,13) -- (10,12) -- (11.5,11) -- (5,10) -- (0.5,1);
      \draw (-0.5,16.5)--(1.5,15)--(4.5,11)--(5,10);
      \draw (4.5,11)--(4.5,13.5)--(10,12);
      \draw (4.5,13.5)--(4,15)--(5.5,15)--(9.5,13);
      \draw (4,15)--(2.5,15.5)--(2,16.333);
      \draw[lsquare] (1.5,15)--(2.5,15.5);
      \draw (5.5,15)--(6.5,16.333);
      \draw (8,16)--(9.25,14.167);

      \node (k1) at (0,10) {$9$};
      \node (k2) at (4.5,16.5) {$9$};
      \node at (8,11.5) {$5$};
      \node at (6,14) {$5$};
      \node at (7.5,15) {$5$};
      \node at (9,16) {$5$};
      \node at (3,14) {$5$};
      \node at (1,16.5) {$5$};

      \draw[lface] (1.5,15) -- (k1);
      \draw[lface] (2.5,15.5) -- (k2);

      \fill[black] (0.5,1)       circle(2.5pt);
      \fill[black] (-0.5,1)      circle(2.5pt);
      \fill[black] (-5,10)       circle(2.5pt);
      \fill[black] (-2.5,17)     circle(2.5pt);
      \fill[black] (-2,17.666)   circle(2.5pt);
      \fill[black] (-0.5,16.5)   circle(2.5pt);
      \fill[black] (0.5,17.5)    circle(2.5pt);
      \fill[black] (2,16.333)    circle(2.5pt);
      \fill[black] (2.5,17)      circle(2.5pt);
      \fill[black] (4,18)        circle(2.5pt);
      \fill[black] (5,18)        circle(2.5pt);
      \fill[black] (6,17.666)    circle(2.5pt);
      \fill[black] (6.5,16.333)  circle(2.5pt);
      \fill[black] (8,16)        circle(2.5pt);
      \fill[black] (8.5,17)      circle(2.5pt);
      \fill[black] (9.5,17)      circle(2.5pt);
      \fill[black] (10,15.333)   circle(2.5pt);
      \fill[black] (9.25,14.167) circle(2.5pt);
      \fill[black] (9.5,13)      circle(2.5pt);
      \fill[black] (10,12)       circle(2.5pt);
      \fill[black] (11.5,11)     circle(2.5pt);
      \fill[black] (5,10)        circle(2.5pt);
      \fill[black] (4.5,11)      circle(2.5pt);
      \fill[black] (4.5,13.5)    circle(2.5pt);
      \fill[black] (4,15)        circle(2.5pt);
      \fill[black] (1.5,15)      circle(2.5pt);
      \fill[black] (2.5,15.5)    circle(2.5pt);
      \fill[black] (5.5,15)      circle(2.5pt);

      \node[anchor= 90] at (0.5,1)       {$i_{0}'$};
      \node[anchor= 90] at (-0.5,1)      {$i_{0}$};
      \node[anchor=  0] at (-5,10)       {$i_{1}$};
      \node[anchor=  0] at (-2.5,17)     {$i_{2}=o_{0}$};
      \node[anchor=270] at (-2,17.666)   {$o_{1}$};
      \node[anchor=270] at (-0.5,16.5)   {$o_{2}$};
      \node[anchor=270] at (0.5,17.5)    {$o_{3}$};
      \node[anchor=270] at (2,16.333)    {$o_{4}$};
      \node[anchor=300] at (2.5,17)      {$o_{5}$};
      \node[anchor=270] at (4,18)        {$o_{6}$};
      \node[anchor=270] at (5,18)        {$o_{7}$};
      \node[anchor=240] at (6,17.666)    {$o_{8}$};
      \node[anchor=220] at (6.5,16.333)  {$o_{9}$};
      \node[anchor=300] at (8,16)        {$o_{10}$};
      \node[anchor=270] at (8.5,17)      {$\mathbf{o_{11}}$};
      \node[anchor=270] at (9.5,17)      {$o_{12}$};
      \node[anchor=180] at (10,15.333)   {$o_{13}$};
      \node[anchor=180] at (9.25,14.167) {$o_{14}$};
      \node[anchor=180] at (9.5,13)      {$o_{15}$};
      \node[anchor=200] at (10,12)       {$o_{16}$};
      \node[anchor=180] at (11.5,11)     {$i_{2}'=o_{17}$};
      \node[anchor=120] at (5,10)        {$i_{1}'$};
    \end{scope}
  \end{tikzsubfigure}
\end{tikzfigure2}
\begin{figure}
  \ContinuedFloat
  \begin{tikzsubfigure}{\label{fig:expansion:patch:5:9:b}}{Edge patch of $\mathcal{P}_N$}{1.0}
    \begin{scope}[scale=0.30]
      \begin{scope}[yscale=0.866]
        \draw[very thick] (0.5,1)--(-0.5,1)--(-5,10)--(-2.5,17)--(-2,17.666)--(-0.5,16.5)--(0.5,17.5)--(2,16.333)--(2.5,17)--(4,18)--(5,18)--(6,17.666)--(6.5,16.333)--(8,16)--(8.5,17)--(9.5,17)--(10,15.333)--(9.25,14.167)--(9.5,13)--(10,12)--(11.5,11)--(5,10)--(0.5,1);
        \draw (-0.5,16.5)--(1.5,15)--(4.5,11)--(5,10);
        \draw (4.5,11)--(4.5,13.5)--(10,12);
        \draw (4.5,13.5)--(4,15)--(5.5,15)--(9.5,13);
        \draw (4,15)--(2.5,15.5)--(2,16.333);
        \draw (1.5,15)--(2.5,15.5);
        \draw (5.5,15)--(6.5,16.333);
        \draw (8,16)--(9.25,14.167);
        \fill[black] (0.5,1)       circle(6pt);
        \fill[black] (-0.5,1)      circle(6pt);
        \fill[black] (-5,10)       circle(6pt);
        \fill[black] (-2.5,17)     circle(6pt);
        \fill[black] (-2,17.666)   circle(6pt);
        \fill[black] (-0.5,16.5)   circle(6pt);
        \fill[black] (0.5,17.5)    circle(6pt);
        \fill[black] (2,16.333)    circle(6pt);
        \fill[black] (2.5,17)      circle(6pt);
        \fill[black] (4,18)        circle(6pt);
        \fill[black] (5,18)        circle(6pt);
        \fill[black] (6,17.666)    circle(6pt);
        \fill[black] (6.5,16.333)  circle(6pt);
        \fill[black] (8,16)        circle(6pt);
        \fill[black] (8.5,17)      circle(6pt);
        \fill[black] (9.5,17)      circle(6pt);
        \fill[black] (10,15.333)   circle(6pt);
        \fill[black] (9.25,14.167) circle(6pt);
        \fill[black] (9.5,13)      circle(6pt);
        \fill[black] (10,12)       circle(6pt);
        \fill[black] (11.5,11)     circle(6pt);
        \fill[black] (5,10)        circle(6pt);
        \fill[black] (4.5,11)      circle(6pt);
        \fill[black] (4.5,13.5)    circle(6pt);
        \fill[black] (4,15)        circle(6pt);
        \fill[black] (1.5,15)      circle(6pt);
        \fill[black] (2.5,15.5)    circle(6pt);
        \fill[black] (5.5,15)      circle(6pt);
      \end{scope}
      \begin{scope}[rotate=60,yscale=0.866]
        \draw[very thick] (0.5,1)--(-0.5,1)--(-5,10)--(-2.5,17)--(-2,17.666)--(-0.5,16.5)--(0.5,17.5)--(2,16.333)--(2.5,17)--(4,18)--(5,18)--(6,17.666)--(6.5,16.333)--(8,16)--(8.5,17)--(9.5,17)--(10,15.333)--(9.25,14.167)--(9.5,13)--(10,12)--(11.5,11)--(5,10)--(0.5,1);
        \draw (-0.5,16.5)--(1.5,15)--(4.5,11)--(5,10);
        \draw (4.5,11)--(4.5,13.5)--(10,12);
        \draw (4.5,13.5)--(4,15)--(5.5,15)--(9.5,13);
        \draw (4,15)--(2.5,15.5)--(2,16.333);
        \draw (1.5,15)--(2.5,15.5);
        \draw (5.5,15)--(6.5,16.333);
        \draw (8,16)--(9.25,14.167);
        \fill[black] (0.5,1)       circle(6pt);
        \fill[black] (-0.5,1)      circle(6pt);
        \fill[black] (-5,10)       circle(6pt);
        \fill[black] (-2.5,17)     circle(6pt);
        \fill[black] (-2,17.666)   circle(6pt);
        \fill[black] (-0.5,16.5)   circle(6pt);
        \fill[black] (0.5,17.5)    circle(6pt);
        \fill[black] (2,16.333)    circle(6pt);
        \fill[black] (2.5,17)      circle(6pt);
        \fill[black] (4,18)        circle(6pt);
        \fill[black] (5,18)        circle(6pt);
        \fill[black] (6,17.666)    circle(6pt);
        \fill[black] (6.5,16.333)  circle(6pt);
        \fill[black] (8,16)        circle(6pt);
        \fill[black] (8.5,17)      circle(6pt);
        \fill[black] (9.5,17)      circle(6pt);
        \fill[black] (10,15.333)   circle(6pt);
        \fill[black] (9.25,14.167) circle(6pt);
        \fill[black] (9.5,13)      circle(6pt);
        \fill[black] (10,12)       circle(6pt);
        \fill[black] (11.5,11)     circle(6pt);
        \fill[black] (5,10)        circle(6pt);
        \fill[black] (4.5,11)      circle(6pt);
        \fill[black] (4.5,13.5)    circle(6pt);
        \fill[black] (4,15)        circle(6pt);
        \fill[black] (1.5,15)      circle(6pt);
        \fill[black] (2.5,15.5)    circle(6pt);
        \fill[black] (5.5,15)      circle(6pt);
      \end{scope}
      \begin{scope}[yscale=0.866,shift={(0cm,34cm)},rotate=180]
        \draw[very thick] (0.5,1)--(-0.5,1)--(-5,10)--(-2.5,17)--(-2,17.666)--(-0.5,16.5)--(0.5,17.5)--(2,16.333)--(2.5,17)--(4,18)--(5,18)--(6,17.666)--(6.5,16.333)--(8,16)--(8.5,17)--(9.5,17)--(10,15.333)--(9.25,14.167)--(9.5,13)--(10,12)--(11.5,11)--(5,10)--(0.5,1);
        \draw (-0.5,16.5)--(1.5,15)--(4.5,11)--(5,10);
        \draw (4.5,11)--(4.5,13.5)--(10,12);
        \draw (4.5,13.5)--(4,15)--(5.5,15)--(9.5,13);
        \draw (4,15)--(2.5,15.5)--(2,16.333);
        \draw (1.5,15)--(2.5,15.5);
        \draw (5.5,15)--(6.5,16.333);
        \draw (8,16)--(9.25,14.167);
        \fill[black] (0.5,1)       circle(6pt);
        \fill[black] (-0.5,1)      circle(6pt);
        \fill[black] (-5,10)       circle(6pt);
        \fill[black] (-2.5,17)     circle(6pt);
        \fill[black] (-2,17.666)   circle(6pt);
        \fill[black] (-0.5,16.5)   circle(6pt);
        \fill[black] (0.5,17.5)    circle(6pt);
        \fill[black] (2,16.333)    circle(6pt);
        \fill[black] (2.5,17)      circle(6pt);
        \fill[black] (4,18)        circle(6pt);
        \fill[black] (5,18)        circle(6pt);
        \fill[black] (6,17.666)    circle(6pt);
        \fill[black] (6.5,16.333)  circle(6pt);
        \fill[black] (8,16)        circle(6pt);
        \fill[black] (8.5,17)      circle(6pt);
        \fill[black] (9.5,17)      circle(6pt);
        \fill[black] (10,15.333)   circle(6pt);
        \fill[black] (9.25,14.167) circle(6pt);
        \fill[black] (9.5,13)      circle(6pt);
        \fill[black] (10,12)       circle(6pt);
        \fill[black] (11.5,11)     circle(6pt);
        \fill[black] (5,10)        circle(6pt);
        \fill[black] (4.5,11)      circle(6pt);
        \fill[black] (4.5,13.5)    circle(6pt);
        \fill[black] (4,15)        circle(6pt);
        \fill[black] (1.5,15)      circle(6pt);
        \fill[black] (2.5,15.5)    circle(6pt);
        \fill[black] (5.5,15)      circle(6pt);
      \end{scope}
      \begin{scope}[shift={(0cm, 29.444cm)},rotate=240,yscale=0.866]
        \draw[very thick] (0.5,1)--(-0.5,1)--(-5,10)--(-2.5,17)--(-2,17.666)--(-0.5,16.5)--(0.5,17.5)--(2,16.333)--(2.5,17)--(4,18)--(5,18)--(6,17.666)--(6.5,16.333)--(8,16)--(8.5,17)--(9.5,17)--(10,15.333)--(9.25,14.167)--(9.5,13)--(10,12)--(11.5,11)--(5,10)--(0.5,1);
        \draw (-0.5,16.5)--(1.5,15)--(4.5,11)--(5,10);
        \draw (4.5,11)--(4.5,13.5)--(10,12);
        \draw (4.5,13.5)--(4,15)--(5.5,15)--(9.5,13);
        \draw (4,15)--(2.5,15.5)--(2,16.333);
        \draw (1.5,15)--(2.5,15.5);
        \draw (5.5,15)--(6.5,16.333);
        \draw (8,16)--(9.25,14.167);
        \fill[black] (0.5,1)       circle(6pt);
        \fill[black] (-0.5,1)      circle(6pt);
        \fill[black] (-5,10)       circle(6pt);
        \fill[black] (-2.5,17)     circle(6pt);
        \fill[black] (-2,17.666)   circle(6pt);
        \fill[black] (-0.5,16.5)   circle(6pt);
        \fill[black] (0.5,17.5)    circle(6pt);
        \fill[black] (2,16.333)    circle(6pt);
        \fill[black] (2.5,17)      circle(6pt);
        \fill[black] (4,18)        circle(6pt);
        \fill[black] (5,18)        circle(6pt);
        \fill[black] (6,17.666)    circle(6pt);
        \fill[black] (6.5,16.333)  circle(6pt);
        \fill[black] (8,16)        circle(6pt);
        \fill[black] (8.5,17)      circle(6pt);
        \fill[black] (9.5,17)      circle(6pt);
        \fill[black] (10,15.333)   circle(6pt);
        \fill[black] (9.25,14.167) circle(6pt);
        \fill[black] (9.5,13)      circle(6pt);
        \fill[black] (10,12)       circle(6pt);
        \fill[black] (11.5,11)     circle(6pt);
        \fill[black] (5,10)        circle(6pt);
        \fill[black] (4.5,11)      circle(6pt);
        \fill[black] (4.5,13.5)    circle(6pt);
        \fill[black] (4,15)        circle(6pt);
        \fill[black] (1.5,15)      circle(6pt);
        \fill[black] (2.5,15.5)    circle(6pt);
        \fill[black] (5.5,15)      circle(6pt);
      \end{scope}
    \end{scope}
  \end{tikzsubfigure}
  \begin{tikzsubfigure}{\label{fig:expansion:patch:5:9:c}}{$\mathcal{P}_F$}{1.0}
    \begin{scope}[scale=7]
      \node (n1) at (0.499123, -0.772923) {5};
\node (n2) at (0.464846, -0.562659) {5};
\node (n3) at (0.374599, -0.795809) {5};
\node (n4) at (0.273172, -0.902640) {5};
\node (n5) at (0.140980, -0.938064) {5};
\node (n6) at (0.051086, -0.644725) {9};
\node (n7) at (-0.150686, -0.222981) {5};
\node (n8) at (-0.433245, -0.617560) {5};
\node (n9) at (-0.397591, -0.804168) {5};
\node (n10) at (0.549039, -0.243337) {9};
\node (n11) at (0.870248, -0.131194) {5};
\node (n12) at (0.526441, 0.156718) {5};
\node (n13) at (-0.028352, 0.518585) {9};
\node (n14) at (0.356853, 0.538604) {5};
\node (n15) at (0.452804, 0.759670) {5};
\node (n16) at (0.735924, 0.333398) {9};
\node (n17) at (0.821384, 0.407072) {5};
\node (n18) at (0.707480, 0.655627) {5};
\node (n19) at (0.941498, 0.110064) {5};
\node (n20) at (0.774029, 0.592021) {5};
\node (n21) at (-0.326202, 0.777162) {5};
\node (n22) at (-0.517643, -0.187393) {9};
\node (n23) at (-0.741565, -0.525117) {5};
\node (n24) at (-0.723062, -0.144552) {5};
\node (n25) at (-0.906961, -0.037307) {5};
\node (n26) at (-0.667326, 0.422478) {9};
\node (n27) at (-0.931473, 0.264926) {5};
\node (n28) at (-0.763086, 0.507938) {5};
\node (n29) at (-0.893975, 0.376887) {5};
\node (n30) at (-0.565980, 0.760422) {5};
\node (n31) at (-0.609294, 0.784040) {5};
\node (n32) at (-0.801623, 0.553568) {9};
\node (n33) at (-0.978743, -0.151229) {5};
\node (n34) at (-0.864629, -0.392925) {9};
\node (n35) at (-0.351682, -0.918442) {5};
\node (n36) at (-0.080533, -0.947288) {9};
\node (n37) at (0.622163, -0.773019) {5};
\node (n38) at (0.762003, -0.547065) {9};
\node (n39) at (0.981310, 0.154356) {5};
\node (n40) at (0.885188, 0.410524) {9};
\node (n41) at (0.106177, 0.913540) {9};
\node (n42) at (0.353181, 0.908552) {5};
\node[anchor=250] (n43) at ($1.02*(0.359872, 0.928938)$) {5};
\node[anchor=310] (n44) at ($1.02*(-0.624547, 0.776127)$) {5};
\node[anchor= 10] (n45) at ($1.02*(-0.984420, -0.152810)$) {5};
\node[anchor= 70] (n46) at ($1.02*(-0.359872, -0.928938)$) {5};
\node[anchor=130] (n47) at ($1.02*(0.624547, -0.776127)$) {5};
\node[anchor=190] (n48) at ($1.02*(0.984420, 0.152810)$) {5};


% \node (n0) at (0.354176, -0.358206) {0};
% \node (n1) at (0.499123, -0.772923) {1};
% \node (n2) at (0.464846, -0.562659) {2};
% \node (n3) at (0.374599, -0.795809) {3};
% \node (n4) at (0.273172, -0.902640) {4};
% \node (n5) at (0.140980, -0.938064) {5};
% \node (n6) at (0.051086, -0.644725) {6};
% \node (n7) at (-0.150686, -0.222981) {7};
% \node (n8) at (-0.433245, -0.617560) {8};
% \node (n9) at (-0.397591, -0.804168) {9};
% \node (n10) at (0.549039, -0.243337) {10};
% \node (n11) at (0.870248, -0.131194) {11};
% \node (n12) at (0.526441, 0.156718) {12};
% \node (n13) at (-0.028352, 0.518585) {13};
% \node (n14) at (0.356853, 0.538604) {14};
% \node (n15) at (0.452804, 0.759670) {15};
% \node (n16) at (0.735924, 0.333398) {16};
% \node (n17) at (0.821384, 0.407072) {17};
% \node (n18) at (0.707480, 0.655627) {18};
% \node (n19) at (0.941498, 0.110064) {19};
% \node (n20) at (0.774029, 0.597021) {20};
% \node (n21) at (-0.326202, 0.777162) {21};
% \node (n22) at (-0.517643, -0.187393) {22};
% \node (n23) at (-0.741565, -0.525117) {23};
% \node (n24) at (-0.723062, -0.144552) {24};
% \node (n25) at (-0.906961, -0.037307) {25};
% \node (n26) at (-0.667326, 0.422478) {26};
% \node (n27) at (-0.921473, 0.284926) {27};
% \node (n28) at (-0.763086, 0.507938) {28};
% \node (n29) at (-0.903975, 0.371887) {29};
% \node (n30) at (-0.565980, 0.760422) {30};
% \node (n31) at (-0.619294, 0.764040) {31};
% \node (n32) at (-0.801623, 0.553568) {32};
% \node (n33) at (-0.968743, -0.151229) {33};
% \node (n34) at (-0.864629, -0.392925) {34};
% \node (n35) at (-0.351682, -0.913442) {35};
% \node (n36) at (-0.080533, -0.947288) {36};
% \node (n37) at (0.615163, -0.763019) {37};
% \node (n38) at (0.762003, -0.547065) {38};
% \node (n39) at (0.971310, 0.154356) {39};
% \node (n40) at (0.880188, 0.417524) {40};
% \node (n41) at (0.106177, 0.913540) {41};
% \node (n42) at (0.353181, 0.908552) {42};
% \node[anchor=250] (n43) at (0.359872, 0.928938) {43};
% \node[anchor=310] (n44) at (-0.624547, 0.776127) {44};
% \node[anchor= 10] (n45) at (-0.984420, -0.152810) {45};
% \node[anchor= 70] (n46) at (-0.359872, -0.928938) {46};
% \node[anchor=130] (n47) at (0.624547, -0.776127) {47};
% \node[anchor=190] (n48) at (0.984420, 0.152810) {48};

\coordinate (x0) at (0.558544, -0.607764);
\coordinate (x1) at (0.447042, -0.730262);
\coordinate (x2) at (0.417960, -0.908465);
\coordinate (x3) at (0.473094, -0.881012);
\coordinate (x4) at (0.598972, -0.737113);
\coordinate (x5) at (0.627389, -0.437562);
\coordinate (x6) at (0.340926, -0.419204);
\coordinate (x7) at (0.350327, -0.618505);
\coordinate (x8) at (0.296422, -0.789339);
\coordinate (x9) at (0.361242, -0.932472);
\coordinate (x10) at (0.161131, -0.868652);
\coordinate (x11) at (0.243914, -0.969797);
\coordinate (x12) at (0.303153, -0.952942);
\coordinate (x13) at (-0.006782, -0.876478);
\coordinate (x14) at (0.122888, -0.992421);
\coordinate (x15) at (0.183750, -0.982973);
\coordinate (x16) at (0.066202, -0.310804);
\coordinate (x17) at (-0.195930, -0.448331);
\coordinate (x18) at (-0.349482, -0.669337);
\coordinate (x19) at (-0.203045, -0.801879);
\coordinate (x20) at (0.053517, -0.032816);
\coordinate (x21) at (-0.261105, 0.051902);
\coordinate (x22) at (-0.416114, -0.374856);
\coordinate (x23) at (-0.626924, -0.779081);
\coordinate (x24) at (-0.577774, -0.816197);
\coordinate (x25) at (-0.526432, -0.850217);
\coordinate (x26) at (-0.331222, -0.883212);
\coordinate (x27) at (0.920906, -0.389786);
\coordinate (x28) at (0.943154, -0.332355);
\coordinate (x29) at (0.961826, -0.273663);
\coordinate (x30) at (0.669413, -0.078589);
\coordinate (x31) at (0.358022, 0.084742);
\coordinate (x32) at (0.976848, -0.213933);
\coordinate (x33) at (0.988165, -0.153392);
\coordinate (x34) at (0.754985, 0.063606);
\coordinate (x35) at (0.550244, 0.312857);
\coordinate (x36) at (0.299539, 0.400973);
\coordinate (x37) at (0.113387, 0.706016);
\coordinate (x38) at (-0.122888, 0.992421);
\coordinate (x39) at (-0.183750, 0.982973);
\coordinate (x40) at (-0.243914, 0.969797);
\coordinate (x41) at (-0.267979, 0.511254);
\coordinate (x42) at (0.501351, 0.580110);
\coordinate (x43) at (0.319744, 0.693063);
\coordinate (x44) at (0.577774, 0.816197);
\coordinate (x45) at (0.526432, 0.850217);
\coordinate (x46) at (0.338717, 0.858763);
\coordinate (x47) at (0.995734, -0.092268);
\coordinate (x48) at (0.875745, 0.202370);
\coordinate (x49) at (0.752395, 0.492021);
\coordinate (x50) at (0.626924, 0.779081);
\coordinate (x51) at (0.883969, 0.305251);
\coordinate (x52) at (0.828338, 0.463825);
\coordinate (x53) at (0.756476, 0.561892);
\coordinate (x54) at (0.717912, 0.696134);
\coordinate (x55) at (0.673696, 0.739009);
\coordinate (x56) at (0.999526, -0.030795);
\coordinate (x57) at (0.952517, 0.165764);
\coordinate (x58) at (0.798017, 0.602635);
\coordinate (x59) at (0.759405, 0.650618);
\coordinate (x60) at (-0.303153, 0.952942);
\coordinate (x61) at (-0.361242, 0.932472);
\coordinate (x62) at (-0.454724, 0.519343);
\coordinate (x63) at (-0.582960, 0.112906);
\coordinate (x64) at (-0.657372, -0.292861);
\coordinate (x65) at (-0.717912, -0.696134);
\coordinate (x66) at (-0.673696, -0.739009);
\coordinate (x67) at (-0.775121, -0.383335);
\coordinate (x68) at (-0.798017, -0.602635);
\coordinate (x69) at (-0.759405, -0.650618);
\coordinate (x70) at (-0.789870, 0.020509);
\coordinate (x71) at (-0.809988, -0.179980);
\coordinate (x72) at (-0.995734, 0.092268);
\coordinate (x73) at (-0.999526, 0.030795);
\coordinate (x74) at (-0.939685, -0.150129);
\coordinate (x75) at (-0.417960, 0.908465);
\coordinate (x76) at (-0.613053, 0.657298);
\coordinate (x77) at (-0.802225, 0.405647);
\coordinate (x78) at (-0.988165, 0.153392);
\coordinate (x79) at (-0.878299, 0.347994);
\coordinate (x80) at (-0.961826, 0.273663);
\coordinate (x81) at (-0.976848, 0.213933);
\coordinate (x82) at (-0.706166, 0.613116);
\coordinate (x83) at (-0.815690, 0.485636);
\coordinate (x84) at (-0.920906, 0.389786);
\coordinate (x85) at (-0.943154, 0.332355);
\coordinate (x86) at (-0.473094, 0.881012);
\coordinate (x87) at (-0.619630, 0.742217);
\coordinate (x88) at ($1.03*(-0.526432, 0.850217)$);
\coordinate (x89) at ($1.03*(-0.717912, 0.696134)$);
\coordinate (x90) at (-0.759405, 0.650618);
\coordinate (x91) at (-0.798017, 0.602635);
\coordinate (x92) at (-0.833602, 0.552365);
\coordinate (x93) at (-0.866025, 0.500000);
\coordinate (x94) at (-0.895163, 0.445738);
\coordinate (x95) at ($1.03*(-0.999526, -0.030795)$);
\coordinate (x96) at ($1.03*(-0.961826, -0.273663)$);
\coordinate (x97) at (-0.943154, -0.332355);
\coordinate (x98) at (-0.920906, -0.389786);
\coordinate (x99) at (-0.895163, -0.445738);
\coordinate (x100) at (-0.866025, -0.500000);
\coordinate (x101) at (-0.833602, -0.552365);
\coordinate (x102) at ($1.03*(-0.473094, -0.881012)$);
\coordinate (x103) at ($1.03*(-0.243914, -0.969797)$);
\coordinate (x104) at (-0.183750, -0.982973);
\coordinate (x105) at (-0.122888, -0.992421);
\coordinate (x106) at (-0.061561, -0.998103);
\coordinate (x107) at (-0.000000, -1.000000);
\coordinate (x108) at (0.061561, -0.998103);
\coordinate (x109) at ($1.03*(0.526432, -0.850217)$);
\coordinate (x110) at ($1.03*(0.717912, -0.696134)$);
\coordinate (x111) at (0.759405, -0.650618);
\coordinate (x112) at (0.798017, -0.602635);
\coordinate (x113) at (0.833602, -0.552365);
\coordinate (x114) at (0.866025, -0.500000);
\coordinate (x115) at (0.895163, -0.445738);
\coordinate (x116) at ($1.03*(0.999526, 0.030795)$);
\coordinate (x117) at ($1.03*(0.961826, 0.273663)$);
\coordinate (x118) at (0.943154, 0.332355);
\coordinate (x119) at (0.920906, 0.389786);
\coordinate (x120) at (0.895163, 0.445738);
\coordinate (x121) at (0.866025, 0.500000);
\coordinate (x122) at (0.833602, 0.552365);
\coordinate (x123) at (0.183750, 0.982973);
\coordinate (x124) at (0.122888, 0.992421);
\coordinate (x125) at (0.061561, 0.998103);
\coordinate (x126) at (0.000000, 1.000000);
\coordinate (x127) at (-0.061561, 0.998103);

\coordinate (x128) at ($1.03*(0.473094, 0.881012)$);
\coordinate (x129) at ($1.03*(0.243914, 0.969797)$);

\coordinate (x130) at ($1.1*(0.417960, 0.908465)$);
\coordinate (x131) at ($1.15*(0.361242, 0.932472)$);
\coordinate (x132) at ($1.1*(0.303153, 0.952942)$);

\coordinate (x133) at ($1.1*(-0.577774, 0.816197)$);
\coordinate (x134) at ($1.15*(-0.626924, 0.779081)$);
\coordinate (x135) at ($1.1*(-0.673696, 0.739009)$);

\coordinate (x136) at ($1.1*(-0.995734, -0.092268)$);
\coordinate (x137) at ($1.15*(-0.988165, -0.153392)$);
\coordinate (x138) at ($1.1*(-0.976848, -0.213933)$);

\coordinate (x139) at ($1.1*(-0.417960, -0.908465)$);
\coordinate (x140) at ($1.15*(-0.361242, -0.932472)$);
\coordinate (x141) at ($1.1*(-0.303153, -0.952942)$);

\coordinate (x142) at ($1.1*(0.577774, -0.816197)$);
\coordinate (x143) at ($1.15*(0.626924, -0.779081)$);
\coordinate (x144) at ($1.1*(0.673696, -0.739009)$);

\coordinate (x145) at ($1.1*(0.995734, 0.092268)$);
\coordinate (x146) at ($1.15*(0.988165, 0.153392)$);
\coordinate (x147) at ($1.1*(0.976848, 0.213933)$);

\draw (x0) -- (x1);
\draw (x1) -- (x2);
\draw (x2) -- (x3);
\draw (x3) -- (x4);
\draw (x4) -- (x0);
\draw (x0) -- (x5);
\draw (x5) -- (x6);
\draw (x6) -- (x7);
\draw (x7) -- (x1);
\draw (x7) -- (x8);
\draw (x8) -- (x9);
\draw (x9) -- (x2);
\draw (x8) -- (x10);
\draw (x10) -- (x11);
\draw (x11) -- (x12);
\draw (x12) -- (x9);
\draw (x10) -- (x13);
\draw (x13) -- (x14);
\draw (x14) -- (x15);
\draw (x15) -- (x11);
\draw (x6) -- (x16);
\draw (x16) -- (x17);
\draw (x17) -- (x18);
\draw (x18) -- (x19);
\draw[ldiamond] (x19) -- (x13) node[midway] (m1) {};
\draw[lface] (m1) -- (n6);
\draw[lface] (m1) -- (n36);
\draw (x16) -- (x20);
\draw (x20) -- (x21);
\draw (x21) -- (x22);
\draw (x22) -- (x17);
\draw (x22) -- (x23);
\draw (x23) -- (x24);
\draw (x24) -- (x18);
\draw (x24) -- (x25);
\draw (x25) -- (x26);
\draw (x26) -- (x19);
\draw[ldiamond] (x5) -- (x27) node[midway] (m2) {};
\draw[lface] (m2) -- (n10);
\draw[lface] (m2) -- (n38);
\draw (x27) -- (x28);
\draw (x28) -- (x29);
\draw (x29) -- (x30);
\draw (x30) -- (x31);
\draw (x31) -- (x20);
\draw (x29) -- (x32);
\draw (x32) -- (x33);
\draw (x33) -- (x34);
\draw (x34) -- (x30);
\draw (x34) -- (x35);
\draw (x35) -- (x36);
\draw (x36) -- (x31);
\draw (x36) -- (x37);
\draw[ldiamond] (x37) -- (x38) node[midway] (m3) {};
\draw[lface] (m3) -- (n13);
\draw[lface] (m3) -- (n41);
\draw (x38) -- (x39);
\draw (x39) -- (x40);
\draw (x40) -- (x41);
\draw (x41) -- (x21);
\draw (x35) -- (x42);
\draw (x42) -- (x43);
\draw (x43) -- (x37);
\draw (x42) -- (x44);
\draw (x44) -- (x45);
\draw (x45) -- (x46);
\draw (x46) -- (x43);
\draw (x33) -- (x47);
\draw (x47) -- (x48);
\draw (x48) -- (x49);
\draw (x49) -- (x50);
\draw (x50) -- (x44);
\draw[lsquare] (x48) -- (x51) node[midway] (m4) {};
\draw[lface] (x48) -- (n16);
\draw[lface] (x51) -- (n40);
\draw (x51) -- (x52);
\draw (x52) -- (x53);
\draw (x53) -- (x49);
\draw (x53) -- (x54);
\draw (x54) -- (x55);
\draw (x55) -- (x50);
\draw (x47) -- (x56);
\draw (x56) -- (x57);
\draw (x57) -- (x51);
\draw (x52) -- (x58);
\draw (x58) -- (x59);
\draw (x59) -- (x54);
\draw (x40) -- (x60);
\draw (x60) -- (x61);
\draw (x61) -- (x62);
\draw (x62) -- (x41);
\draw (x62) -- (x63);
\draw (x63) -- (x64);
\draw (x64) -- (x65);
\draw (x65) -- (x66);
\draw (x66) -- (x23);
\draw[lsquare] (x64) -- (x67) node[midway] (m5) {};
\draw[lface] (x64) -- (n22);
\draw[lface] (x67) -- (n34);
\draw (x67) -- (x68);
\draw (x68) -- (x69);
\draw (x69) -- (x65);
\draw (x63) -- (x70);
\draw (x70) -- (x71);
\draw (x71) -- (x67);
\draw (x70) -- (x72);
\draw (x72) -- (x73);
\draw (x73) -- (x74);
\draw (x74) -- (x71);
\draw (x61) -- (x75);
\draw (x75) -- (x76);
\draw (x76) -- (x77);
\draw (x77) -- (x78);
\draw (x78) -- (x72);
\draw (x77) -- (x79);
\draw (x79) -- (x80);
\draw (x80) -- (x81);
\draw (x81) -- (x78);
\draw[lsquare] (x76) -- (x82) node[midway] (m6) {};
\draw[lface] (x76) -- (n26);
\draw[lface] (x82) -- (n32);
\draw (x82) -- (x83);
\draw (x83) -- (x79);
\draw (x83) -- (x84);
\draw (x84) -- (x85);
\draw (x85) -- (x80);
\draw (x75) -- (x86);
\draw (x86) -- (x87);
\draw (x87) -- (x82);
\draw (x86) -- (x88);
\draw (x88) -- (x89);
\draw (x89) -- (x90);
\draw (x90) -- (x87);
\draw (x90) -- (x91);
\draw (x91) -- (x92);
\draw (x92) -- (x93);
\draw (x93) -- (x94);
\draw (x94) -- (x84);
\draw (x73) -- (x95);
\draw (x95) -- (x96);
\draw (x96) -- (x97);
\draw (x97) -- (x74);
\draw (x97) -- (x98);
\draw (x98) -- (x99);
\draw (x99) -- (x100);
\draw (x100) -- (x101);
\draw (x101) -- (x68);
\draw (x25) -- (x102);
\draw (x102) -- (x103);
\draw (x103) -- (x104);
\draw (x104) -- (x26);
\draw (x104) -- (x105);
\draw (x105) -- (x106);
\draw (x106) -- (x107);
\draw (x107) -- (x108);
\draw (x108) -- (x14);
\draw (x3) -- (x109);
\draw (x109) -- (x110);
\draw (x110) -- (x111);
\draw (x111) -- (x4);
\draw (x111) -- (x112);
\draw (x112) -- (x113);
\draw (x113) -- (x114);
\draw (x114) -- (x115);
\draw (x115) -- (x27);
\draw (x56) -- (x116);
\draw (x116) -- (x117);
\draw (x117) -- (x118);
\draw (x118) -- (x57);
\draw (x118) -- (x119);
\draw (x119) -- (x120);
\draw (x120) -- (x121);
\draw (x121) -- (x122);
\draw (x122) -- (x58);
\draw (x46) -- (x123);
\draw (x123) -- (x124);
\draw (x124) -- (x125);
\draw (x125) -- (x126);
\draw (x126) -- (x127);
\draw (x127) -- (x38);
\draw (x45) -- (x128);
\draw (x128) -- (x129);
\draw (x129) -- (x123);
\draw (x128) -- (x130);
\draw (x130) -- (x131);
\draw (x131) -- (x132);
\draw (x132) -- (x129);
\draw (x88) -- (x133);
\draw (x133) -- (x134);
\draw (x134) -- (x135);
\draw (x135) -- (x89);
\draw (x95) -- (x136);
\draw (x136) -- (x137);
\draw (x137) -- (x138);
\draw (x138) -- (x96);
\draw (x102) -- (x139);
\draw (x139) -- (x140);
\draw (x140) -- (x141);
\draw (x141) -- (x103);
\draw (x109) -- (x142);
\draw (x142) -- (x143);
\draw (x143) -- (x144);
\draw (x144) -- (x110);
\draw (x116) -- (x145);
\draw (x145) -- (x146);
\draw (x146) -- (x147);
\draw (x147) -- (x117);

\fill[black] (x0) circle (0.3pt);
\fill[black] (x1) circle (0.3pt);
\fill[black] (x2) circle (0.3pt);
\fill[black] (x3) circle (0.3pt);
\fill[black] (x4) circle (0.3pt);
\fill[black] (x5) circle (0.3pt);
\fill[black] (x6) circle (0.3pt);
\fill[black] (x7) circle (0.3pt);
\fill[black] (x8) circle (0.3pt);
\fill[black] (x9) circle (0.3pt);
\fill[black] (x10) circle (0.3pt);
\fill[black] (x11) circle (0.3pt);
\fill[black] (x12) circle (0.3pt);
\fill[black] (x13) circle (0.3pt);
\fill[black] (x14) circle (0.3pt);
\fill[black] (x15) circle (0.3pt);
\fill[black] (x16) circle (0.3pt);
\fill[black] (x17) circle (0.3pt);
\fill[black] (x18) circle (0.3pt);
\fill[black] (x19) circle (0.3pt);
\fill[black] (x20) circle (0.3pt);
\fill[black] (x21) circle (0.3pt);
\fill[black] (x22) circle (0.3pt);
\fill[black] (x23) circle (0.3pt);
\fill[black] (x24) circle (0.3pt);
\fill[black] (x25) circle (0.3pt);
\fill[black] (x26) circle (0.3pt);
\fill[black] (x27) circle (0.3pt);
\fill[black] (x28) circle (0.3pt);
\fill[black] (x29) circle (0.3pt);
\fill[black] (x30) circle (0.3pt);
\fill[black] (x31) circle (0.3pt);
\fill[black] (x32) circle (0.3pt);
\fill[black] (x33) circle (0.3pt);
\fill[black] (x34) circle (0.3pt);
\fill[black] (x35) circle (0.3pt);
\fill[black] (x36) circle (0.3pt);
\fill[black] (x37) circle (0.3pt);
\fill[black] (x38) circle (0.3pt);
\fill[black] (x39) circle (0.3pt);
\fill[black] (x40) circle (0.3pt);
\fill[black] (x41) circle (0.3pt);
\fill[black] (x42) circle (0.3pt);
\fill[black] (x43) circle (0.3pt);
\fill[black] (x44) circle (0.3pt);
\fill[black] (x45) circle (0.3pt);
\fill[black] (x46) circle (0.3pt);
\fill[black] (x47) circle (0.3pt);
\fill[black] (x48) circle (0.3pt);
\fill[black] (x49) circle (0.3pt);
\fill[black] (x50) circle (0.3pt);
\fill[black] (x51) circle (0.3pt);
\fill[black] (x52) circle (0.3pt);
\fill[black] (x53) circle (0.3pt);
\fill[black] (x54) circle (0.3pt);
\fill[black] (x55) circle (0.3pt);
\fill[black] (x56) circle (0.3pt);
\fill[black] (x57) circle (0.3pt);
\fill[black] (x58) circle (0.3pt);
\fill[black] (x59) circle (0.3pt);
\fill[black] (x60) circle (0.3pt);
\fill[black] (x61) circle (0.3pt);
\fill[black] (x62) circle (0.3pt);
\fill[black] (x63) circle (0.3pt);
\fill[black] (x64) circle (0.3pt);
\fill[black] (x65) circle (0.3pt);
\fill[black] (x66) circle (0.3pt);
\fill[black] (x67) circle (0.3pt);
\fill[black] (x68) circle (0.3pt);
\fill[black] (x69) circle (0.3pt);
\fill[black] (x70) circle (0.3pt);
\fill[black] (x71) circle (0.3pt);
\fill[black] (x72) circle (0.3pt);
\fill[black] (x73) circle (0.3pt);
\fill[black] (x74) circle (0.3pt);
\fill[black] (x75) circle (0.3pt);
\fill[black] (x76) circle (0.3pt);
\fill[black] (x77) circle (0.3pt);
\fill[black] (x78) circle (0.3pt);
\fill[black] (x79) circle (0.3pt);
\fill[black] (x80) circle (0.3pt);
\fill[black] (x81) circle (0.3pt);
\fill[black] (x82) circle (0.3pt);
\fill[black] (x83) circle (0.3pt);
\fill[black] (x84) circle (0.3pt);
\fill[black] (x85) circle (0.3pt);
\fill[black] (x86) circle (0.3pt);
\fill[black] (x87) circle (0.3pt);
\fill[black] (x88) circle (0.3pt);
\fill[black] (x89) circle (0.3pt);
\fill[black] (x90) circle (0.3pt);
\fill[black] (x91) circle (0.3pt);
\fill[black] (x92) circle (0.3pt);
\fill[black] (x93) circle (0.3pt);
\fill[black] (x94) circle (0.3pt);
\fill[black] (x95) circle (0.3pt);
\fill[black] (x96) circle (0.3pt);
\fill[black] (x97) circle (0.3pt);
\fill[black] (x98) circle (0.3pt);
\fill[black] (x99) circle (0.3pt);
\fill[black] (x100) circle (0.3pt);
\fill[black] (x101) circle (0.3pt);
\fill[black] (x102) circle (0.3pt);
\fill[black] (x103) circle (0.3pt);
\fill[black] (x104) circle (0.3pt);
\fill[black] (x105) circle (0.3pt);
\fill[black] (x106) circle (0.3pt);
\fill[black] (x107) circle (0.3pt);
\fill[black] (x108) circle (0.3pt);
\fill[black] (x109) circle (0.3pt);
\fill[black] (x110) circle (0.3pt);
\fill[black] (x111) circle (0.3pt);
\fill[black] (x112) circle (0.3pt);
\fill[black] (x113) circle (0.3pt);
\fill[black] (x114) circle (0.3pt);
\fill[black] (x115) circle (0.3pt);
\fill[black] (x116) circle (0.3pt);
\fill[black] (x117) circle (0.3pt);
\fill[black] (x118) circle (0.3pt);
\fill[black] (x119) circle (0.3pt);
\fill[black] (x120) circle (0.3pt);
\fill[black] (x121) circle (0.3pt);
\fill[black] (x122) circle (0.3pt);
\fill[black] (x123) circle (0.3pt);
\fill[black] (x124) circle (0.3pt);
\fill[black] (x125) circle (0.3pt);
\fill[black] (x126) circle (0.3pt);
\fill[black] (x127) circle (0.3pt);
\fill[black] (x128) circle (0.3pt);
\fill[black] (x129) circle (0.3pt);
\fill[black] (x130) circle (0.3pt);
\fill[black] (x131) circle (0.3pt);
\fill[black] (x132) circle (0.3pt);
\fill[black] (x133) circle (0.3pt);
\fill[black] (x134) circle (0.3pt);
\fill[black] (x135) circle (0.3pt);
\fill[black] (x136) circle (0.3pt);
\fill[black] (x137) circle (0.3pt);
\fill[black] (x138) circle (0.3pt);
\fill[black] (x139) circle (0.3pt);
\fill[black] (x140) circle (0.3pt);
\fill[black] (x141) circle (0.3pt);
\fill[black] (x142) circle (0.3pt);
\fill[black] (x143) circle (0.3pt);
\fill[black] (x144) circle (0.3pt);
\fill[black] (x145) circle (0.3pt);
\fill[black] (x146) circle (0.3pt);
\fill[black] (x147) circle (0.3pt);

    \end{scope}
  \end{tikzsubfigure}
\end{figure}
\clearpage
\begin{theorem}
  Let $p$ and $v$ be a pair of admissible sequences for an orientable closed $2$-manifold $S$. Then $(p, v)$ is $[(5k + 5) \times 5, (5k+11)]$-$[3]$-realizable for all $k \in \nats$.
  \begin{proof}
    An expansion $3$-patch $\mathcal{P}_N$ with outer tuple $o = (1, 1, 2, 1, 1, 2, 1, 2, 1, 2, 1, 2, 2, 1, 2, 2)$ is shown in \autoref{fig:expansion:patch:5:11:c}, a corresponding $o$-$6$-gonal $3$-patch $\mathcal{P}_F$ is shown in \autoref{fig:expansion:patch:5:11:d} and an expansion $3$-patch is shown in \autoref{fig:expansion:patch:5:11:a}, which has the polyhedral property as is seen \autoref{fig:expansion:patch:5:11:b}, all consisting of only pentagons and hendecagons. By using \autoref{const:edge:replacement:5:1} and \autoref{const:edge:replacement:5:3} as indicated we get $3$-patches consisting of only pentagons and $(5k+11)$-gons, $k \in \nats$. Therefore we can apply \autoref{thm:main:const}.
  \end{proof}
\end{theorem}
\begin{tikzfigure2}{}
  \begin{tikzsubfigure}{\label{fig:expansion:patch:5:11:a}}{$\mathcal{P}_P$}{0.5}
    \begin{scope}[yscale=0.866,scale=1]
      \draw (-0.5, 1) -- (-1, 2) -- (-1.625, 2.75) -- (-2.5, 3) -- (-3.5, 3) -- (-4, 2.666) -- (-4.5, 5) -- (-4, 6) -- (-4, 8) -- (-2.5, 9) -- (-1.5, 9) -- (-0.5, 8.75) -- (0.5, 7.25) -- (1.5, 7) -- (0, 5.333) -- (0.5, 5) -- (1, 4) -- (1.25, 3) -- (1, 2) -- (0.5, 1) -- (-0.5, 1); 
      \draw (-1,2) -- (-1.25, 3) -- (-1, 4) -- (-0.5, 5) -- (0, 5.333);
      \draw (-4, 6) -- (-3.5, 5) -- (-2.5, 5) -- (-1.833, 5.2) -- (-1.166, 5.5) -- (-0.5, 6) -- (0.5, 7.25);

      \node (n1) at (-1.5,7.25) {$11$};
      \node (n2) at (0,4) {$11$};
      \node at (-4,4) {$5$};
      \node at (-3,4) {$5$};
      \node at (-2.1,3.4) {$5$};
      \node at (-2.1,4.5) {$5$};
      \node at (-1.7,4) {$5$};
      \node at (-1.2,4) {$5$};
      \node at (-1.5,3) {$5$};
      \node at (-1.5,5) {$5$};
      \node at (0.5,6.5) {$5$};
      \node at (-0.8,5) {$5$};

      \node[anchor= 90] at (-0.5, 1)      {$i_{0}'$};
      \node[anchor= 90] at (-1, 2)        {$i_{1}$};   
      \node[anchor= 60] at (-1.625, 2.75) {$i_{2}$};   
      \node[anchor= 90] at (-2.5, 3)      {$i_{3}$};   
      \node[anchor= 90] at (-3.5, 3)      {$i_{4}$};   
      \node[anchor= 45] at (-4, 2.666)    {$i_{5}$};   
      \node[anchor=  0] at (-4.5, 5)      {$i_{6}=o_0$};   
      \node[anchor=  0] at (-4, 6)        {$o_1$}; 
      \node[anchor=335] at (-4, 8)        {$\mathbf{o_2}$}; 
      \node[anchor=270] at (-2.5, 9)      {$o_3$}; 
      \node[anchor=210] at (-1.5, 9)      {$o_4$}; 
      \node[anchor=180] at (-0.5, 8.75)   {$o_5$}; 
      \node[anchor=230] at (0.5, 7.25)    {$o_6$}; 
      \node[anchor=250] at (1.5, 7)       {$i_{6}'=o_7$};   
      \node[anchor=270] at (0, 5.333)     {$i_{5}'$};   
      \node[anchor=180] at (0.5, 5)       {$i_{4}'$};   
      \node[anchor=180] at (1, 4)         {$i_{3}'$};   
      \node[anchor=180] at (1.25, 3)      {$i_{2}'$};   
      \node[anchor=180] at (1, 2)         {$i_{1}'$};   
      \node[anchor= 90] at (0.5, 1)       {$i_{0}'$};   

      \fill[black] (-0.5, 1)      circle(2pt);
      \fill[black] (-1, 2)        circle(2pt);
      \fill[black] (-1.625, 2.75) circle(2pt);
      \fill[black] (-2.5, 3)      circle(2pt);
      \fill[black] (-3.5, 3)      circle(2pt);
      \fill[black] (-4, 2.666)    circle(2pt);
      \fill[black] (-4.5, 5)      circle(2pt);
      \fill[black] (-4, 6)        circle(2pt);
      \fill[black] (-4, 8)        circle(2pt);
      \fill[black] (-2.5, 9)      circle(2pt);
      \fill[black] (-1.5, 9)      circle(2pt);
      \fill[black] (-0.5, 8.75)   circle(2pt);
      \fill[black] (0.5, 7.25)    circle(2pt);
      \fill[black] (1.5, 7)       circle(2pt);
      \fill[black] (0, 5.333)     circle(2pt);
      \fill[black] (0.5, 5)       circle(2pt);
      \fill[black] (1, 4)         circle(2pt);
      \fill[black] (1.25, 3)      circle(2pt);
      \fill[black] (1, 2)         circle(2pt);
      \fill[black] (0.5, 1)       circle(2pt);
      
      \fill[black] (-0.5, 6)      circle(2pt);
      \fill[black] (-0.5, 5)      circle(2pt);
      \fill[black] (-3.5, 5)      circle(2pt);
      \fill[black] (-2.5, 5)      circle(2pt);
      \fill[black] (-2.25, 4   )  circle(2pt);
      \fill[black] (-1.833, 4.75) circle(2pt);
      \fill[black] (-1.833, 5.2)  circle(2pt);
      \fill[black] (-1.25, 3)     circle(2pt);
      \fill[black] (-1.5, 3.5)    circle(2pt);
      \fill[black] (-1.626, 2.75) circle(2pt);
      \fill[black] (-2, 4)        circle(2pt);
      \fill[black] (-1.5, 4.5)    circle(2pt);
      \fill[black] (-1.25, 4.75)  circle(2pt);
      \fill[black] (-1.166, 5.5)  circle(2pt);
      \fill[black] (-1, 4)        circle(2pt);
      
      \draw[lsquare] (-0.5, 6) -- (-0.5, 5);
      \draw[lface] (-0.5, 6) -- (n1);
      \draw[lface] (-0.5, 5) -- (n2);
      \draw (-3.5, 3) -- (-3.5, 5);
      \draw (-2.5, 3) -- (-2.25, 4) -- (-2.5, 5);
      \draw (-1.626, 2.75) -- (-1.75, 3.5) -- (-1.5, 3.5) -- (-1.25, 3);
      \draw (-1.75, 3.5) -- (-2, 4) -- (-2.25, 4);
      \draw (-1.5, 3.5) -- (-1.5, 4.5) -- (-1.833, 4.75) -- (-1.833, 5.2);
      \draw (-2, 4) -- (-1.833, 4.75);
      \draw (-1, 4) -- (-1.25, 4.75) -- (-1.5, 4.5);
      \draw (-1.25, 4.75) -- (-1.166, 5.5);
    \end{scope}
  \end{tikzsubfigure}~
  \begin{tikzsubfigure}{\label{fig:expansion:patch:5:11:b}}{Edge patch of $\mathcal{P}_P$}{0.5}
    \begin{scope}[scale=0.5]
      \begin{scope}[yscale=0.866]
        \draw[very thick] (-0.5, 1) -- (-1, 2) -- (-1.625, 2.75) -- (-2.5, 3) -- (-3.5, 3) -- (-4, 2.666) -- (-3.4, 5) -- (-4, 6) -- (-4, 8) -- (-2.5, 9) -- (-1.5, 9) -- (-0.5, 8.75) -- (0.5, 7.25) -- (1.5, 7) -- (0, 5.333) -- (0.5, 5) -- (1, 4) -- (1.25, 3) -- (1, 2) -- (0.5, 1) -- (-0.5, 1); 
        \draw (-1,2) -- (-1.25, 3) -- (-1, 4) -- (-0.5, 5) -- (0, 5.333);
        \draw (-4, 6) -- (-3.5, 5) -- (-2.5, 5) -- (-1.833, 5.2) -- (-1.166, 5.5) -- (-0.5, 6) -- (0.5, 7.25);
        \draw (-0.5, 6) -- (-0.5, 5);
        \draw (-3.5, 3) -- (-3.5, 5);
        \draw (-2.5, 3) -- (-2.25, 4) -- (-2.5, 5);
        \draw (-1.626, 2.75) -- (-1.75, 3.5) -- (-1.5, 3.5) -- (-1.25, 3);
        \draw (-1.75, 3.5) -- (-2, 4) -- (-2.25, 4);
        \draw (-1.5, 3.5) -- (-1.5, 3.4) -- (-1.833, 4.75) -- (-1.833, 5.2);
        \draw (-2, 4) -- (-1.833, 4.75);
        \draw (-1, 4) -- (-1.25, 4.75) -- (-1.5, 3.4);
        \draw (-1.25, 4.75) -- (-1.166, 5.5);

        \fill[black] (-0.5, 1)      circle(3.4pt);
        \fill[black] (-1, 2)        circle(3.4pt);
        \fill[black] (-1.625, 2.75) circle(3.4pt);
        \fill[black] (-2.5, 3)      circle(3.4pt);
        \fill[black] (-3.5, 3)      circle(3.4pt);
        \fill[black] (-4, 2.666)    circle(3.4pt);
        \fill[black] (-3.4, 5)      circle(3.4pt);
        \fill[black] (-4, 6)        circle(3.4pt);
        \fill[black] (-4, 8)        circle(3.4pt);
        \fill[black] (-2.5, 9)      circle(3.4pt);
        \fill[black] (-1.5, 9)      circle(3.4pt);
        \fill[black] (-0.5, 8.75)   circle(3.4pt);
        \fill[black] (0.5, 7.25)    circle(3.4pt);
        \fill[black] (1.5, 7)       circle(3.4pt);
        \fill[black] (0, 5.333)     circle(3.4pt);
        \fill[black] (0.5, 5)       circle(3.4pt);
        \fill[black] (1, 4)         circle(3.4pt);
        \fill[black] (1.25, 3)      circle(3.4pt);
        \fill[black] (1, 2)         circle(3.4pt);
        \fill[black] (0.5, 1)       circle(3.4pt);
        
        \fill[black] (-0.5, 6)      circle(3.4pt);
        \fill[black] (-0.5, 5)      circle(3.4pt);
        \fill[black] (-3.5, 5)      circle(3.4pt);
        \fill[black] (-2.5, 5)      circle(3.4pt);
        \fill[black] (-2.25, 4)     circle(3.4pt);
        \fill[black] (-1.833, 4.75) circle(3.4pt);
        \fill[black] (-1.833, 5.2)  circle(3.4pt);
        \fill[black] (-1.25, 3)     circle(3.4pt);
        \fill[black] (-1.5, 3.5)    circle(3.4pt);
        \fill[black] (-1.626, 2.75) circle(3.4pt);
        \fill[black] (-2, 4)        circle(3.4pt);
        \fill[black] (-1.5, 3.4)    circle(3.4pt);
        \fill[black] (-1.25, 4.75)  circle(3.4pt);
        \fill[black] (-1.166, 5.5)  circle(3.4pt);
        \fill[black] (-1, 4)        circle(3.4pt);


      \end{scope}
      \begin{scope}[rotate=-60, yscale=0.866]
        \draw[very thick] (-0.5, 1) -- (-1, 2) -- (-1.625, 2.75) -- (-2.5, 3) -- (-3.5, 3) -- (-4, 2.666) -- (-3.4, 5) -- (-4, 6) -- (-4, 8) -- (-2.5, 9) -- (-1.5, 9) -- (-0.5, 8.75) -- (0.5, 7.25) -- (1.5, 7) -- (0, 5.333) -- (0.5, 5) -- (1, 4) -- (1.25, 3) -- (1, 2) -- (0.5, 1) -- (-0.5, 1); 
        \draw (-1,2) -- (-1.25, 3) -- (-1, 4) -- (-0.5, 5) -- (0, 5.333);
        \draw (-4, 6) -- (-3.5, 5) -- (-2.5, 5) -- (-1.833, 5.2) -- (-1.166, 5.5) -- (-0.5, 6) -- (0.5, 7.25);
        \draw (-0.5, 6) -- (-0.5, 5);
        \draw (-3.5, 3) -- (-3.5, 5);
        \draw (-2.5, 3) -- (-2.25, 4) -- (-2.5, 5);
        \draw (-1.626, 2.75) -- (-1.75, 3.5) -- (-1.5, 3.5) -- (-1.25, 3);
        \draw (-1.75, 3.5) -- (-2, 4) -- (-2.25, 4);
        \draw (-1.5, 3.5) -- (-1.5, 3.4) -- (-1.833, 4.75) -- (-1.833, 5.2);
        \draw (-2, 4) -- (-1.833, 4.75);
        \draw (-1, 4) -- (-1.25, 4.75) -- (-1.5, 3.4);
        \draw (-1.25, 4.75) -- (-1.166, 5.5);

        \fill[black] (-0.5, 1)      circle(3.4pt);
        \fill[black] (-1, 2)        circle(3.4pt);
        \fill[black] (-1.625, 2.75) circle(3.4pt);
        \fill[black] (-2.5, 3)      circle(3.4pt);
        \fill[black] (-3.5, 3)      circle(3.4pt);
        \fill[black] (-4, 2.666)    circle(3.4pt);
        \fill[black] (-3.4, 5)      circle(3.4pt);
        \fill[black] (-4, 6)        circle(3.4pt);
        \fill[black] (-4, 8)        circle(3.4pt);
        \fill[black] (-2.5, 9)      circle(3.4pt);
        \fill[black] (-1.5, 9)      circle(3.4pt);
        \fill[black] (-0.5, 8.75)   circle(3.4pt);
        \fill[black] (0.5, 7.25)    circle(3.4pt);
        \fill[black] (1.5, 7)       circle(3.4pt);
        \fill[black] (0, 5.333)     circle(3.4pt);
        \fill[black] (0.5, 5)       circle(3.4pt);
        \fill[black] (1, 4)         circle(3.4pt);
        \fill[black] (1.25, 3)      circle(3.4pt);
        \fill[black] (1, 2)         circle(3.4pt);
        \fill[black] (0.5, 1)       circle(3.4pt);
        
        \fill[black] (-0.5, 6)      circle(3.4pt);
        \fill[black] (-0.5, 5)      circle(3.4pt);
        \fill[black] (-3.5, 5)      circle(3.4pt);
        \fill[black] (-2.5, 5)      circle(3.4pt);
        \fill[black] (-2.25, 4)     circle(3.4pt);
        \fill[black] (-1.833, 4.75) circle(3.4pt);
        \fill[black] (-1.833, 5.2)  circle(3.4pt);
        \fill[black] (-1.25, 3)     circle(3.4pt);
        \fill[black] (-1.5, 3.5)    circle(3.4pt);
        \fill[black] (-1.626, 2.75) circle(3.4pt);
        \fill[black] (-2, 4)        circle(3.4pt);
        \fill[black] (-1.5, 3.4)    circle(3.4pt);
        \fill[black] (-1.25, 4.75)  circle(3.4pt);
        \fill[black] (-1.166, 5.5)  circle(3.4pt);
        \fill[black] (-1, 4)        circle(3.4pt);

      \end{scope}
      \begin{scope}[yscale=0.866,shift={(0 cm,16 cm)},rotate=180]
        \draw[very thick] (-0.5, 1) -- (-1, 2) -- (-1.625, 2.75) -- (-2.5, 3) -- (-3.5, 3) -- (-4, 2.666) -- (-3.4, 5) -- (-4, 6) -- (-4, 8) -- (-2.5, 9) -- (-1.5, 9) -- (-0.5, 8.75) -- (0.5, 7.25) -- (1.5, 7) -- (0, 5.333) -- (0.5, 5) -- (1, 4) -- (1.25, 3) -- (1, 2) -- (0.5, 1) -- (-0.5, 1); 
        \draw (-1,2) -- (-1.25, 3) -- (-1, 4) -- (-0.5, 5) -- (0, 5.333);
        \draw (-4, 6) -- (-3.5, 5) -- (-2.5, 5) -- (-1.833, 5.2) -- (-1.166, 5.5) -- (-0.5, 6) -- (0.5, 7.25);
        \draw (-0.5, 6) -- (-0.5, 5);
        \draw (-3.5, 3) -- (-3.5, 5);
        \draw (-2.5, 3) -- (-2.25, 4) -- (-2.5, 5);
        \draw (-1.626, 2.75) -- (-1.75, 3.5) -- (-1.5, 3.5) -- (-1.25, 3);
        \draw (-1.75, 3.5) -- (-2, 4) -- (-2.25, 4);
        \draw (-1.5, 3.5) -- (-1.5, 3.4) -- (-1.833, 4.75) -- (-1.833, 5.2);
        \draw (-2, 4) -- (-1.833, 4.75);
        \draw (-1, 4) -- (-1.25, 4.75) -- (-1.5, 3.4);
        \draw (-1.25, 4.75) -- (-1.166, 5.5);

        \fill[black] (-0.5, 1)      circle(3.4pt);
        \fill[black] (-1, 2)        circle(3.4pt);
        \fill[black] (-1.625, 2.75) circle(3.4pt);
        \fill[black] (-2.5, 3)      circle(3.4pt);
        \fill[black] (-3.5, 3)      circle(3.4pt);
        \fill[black] (-4, 2.666)    circle(3.4pt);
        \fill[black] (-3.4, 5)      circle(3.4pt);
        \fill[black] (-4, 6)        circle(3.4pt);
        \fill[black] (-4, 8)        circle(3.4pt);
        \fill[black] (-2.5, 9)      circle(3.4pt);
        \fill[black] (-1.5, 9)      circle(3.4pt);
        \fill[black] (-0.5, 8.75)   circle(3.4pt);
        \fill[black] (0.5, 7.25)    circle(3.4pt);
        \fill[black] (1.5, 7)       circle(3.4pt);
        \fill[black] (0, 5.333)     circle(3.4pt);
        \fill[black] (0.5, 5)       circle(3.4pt);
        \fill[black] (1, 4)         circle(3.4pt);
        \fill[black] (1.25, 3)      circle(3.4pt);
        \fill[black] (1, 2)         circle(3.4pt);
        \fill[black] (0.5, 1)       circle(3.4pt);
        
        \fill[black] (-0.5, 6)      circle(3.4pt);
        \fill[black] (-0.5, 5)      circle(3.4pt);
        \fill[black] (-3.5, 5)      circle(3.4pt);
        \fill[black] (-2.5, 5)      circle(3.4pt);
        \fill[black] (-2.25, 4)     circle(3.4pt);
        \fill[black] (-1.833, 4.75) circle(3.4pt);
        \fill[black] (-1.833, 5.2)  circle(3.4pt);
        \fill[black] (-1.25, 3)     circle(3.4pt);
        \fill[black] (-1.5, 3.5)    circle(3.4pt);
        \fill[black] (-1.626, 2.75) circle(3.4pt);
        \fill[black] (-2, 4)        circle(3.4pt);
        \fill[black] (-1.5, 3.4)    circle(3.4pt);
        \fill[black] (-1.25, 4.75)  circle(3.4pt);
        \fill[black] (-1.166, 5.5)  circle(3.4pt);
        \fill[black] (-1, 4)        circle(3.4pt);

      \end{scope}
      \begin{scope}[shift={(0cm, 13.856cm)},rotate=120,yscale=0.866]
        \draw[very thick] (-0.5, 1) -- (-1, 2) -- (-1.625, 2.75) -- (-2.5, 3) -- (-3.5, 3) -- (-4, 2.666) -- (-3.4, 5) -- (-4, 6) -- (-4, 8) -- (-2.5, 9) -- (-1.5, 9) -- (-0.5, 8.75) -- (0.5, 7.25) -- (1.5, 7) -- (0, 5.333) -- (0.5, 5) -- (1, 4) -- (1.25, 3) -- (1, 2) -- (0.5, 1) -- (-0.5, 1); 
        \draw (-1,2) -- (-1.25, 3) -- (-1, 4) -- (-0.5, 5) -- (0, 5.333);
        \draw (-4, 6) -- (-3.5, 5) -- (-2.5, 5) -- (-1.833, 5.2) -- (-1.166, 5.5) -- (-0.5, 6) -- (0.5, 7.25);
        \draw (-0.5, 6) -- (-0.5, 5);
        \draw (-3.5, 3) -- (-3.5, 5);
        \draw (-2.5, 3) -- (-2.25, 4) -- (-2.5, 5);
        \draw (-1.626, 2.75) -- (-1.75, 3.5) -- (-1.5, 3.5) -- (-1.25, 3);
        \draw (-1.75, 3.5) -- (-2, 4) -- (-2.25, 4);
        \draw (-1.5, 3.5) -- (-1.5, 3.4) -- (-1.833, 4.75) -- (-1.833, 5.2);
        \draw (-2, 4) -- (-1.833, 4.75);
        \draw (-1, 4) -- (-1.25, 4.75) -- (-1.5, 3.4);
        \draw (-1.25, 4.75) -- (-1.166, 5.5);

        \fill[black] (-0.5, 1)      circle(3.4pt);
        \fill[black] (-1, 2)        circle(3.4pt);
        \fill[black] (-1.625, 2.75) circle(3.4pt);
        \fill[black] (-2.5, 3)      circle(3.4pt);
        \fill[black] (-3.5, 3)      circle(3.4pt);
        \fill[black] (-4, 2.666)    circle(3.4pt);
        \fill[black] (-3.4, 5)      circle(3.4pt);
        \fill[black] (-4, 6)        circle(3.4pt);
        \fill[black] (-4, 8)        circle(3.4pt);
        \fill[black] (-2.5, 9)      circle(3.4pt);
        \fill[black] (-1.5, 9)      circle(3.4pt);
        \fill[black] (-0.5, 8.75)   circle(3.4pt);
        \fill[black] (0.5, 7.25)    circle(3.4pt);
        \fill[black] (1.5, 7)       circle(3.4pt);
        \fill[black] (0, 5.333)     circle(3.4pt);
        \fill[black] (0.5, 5)       circle(3.4pt);
        \fill[black] (1, 4)         circle(3.4pt);
        \fill[black] (1.25, 3)      circle(3.4pt);
        \fill[black] (1, 2)         circle(3.4pt);
        \fill[black] (0.5, 1)       circle(3.4pt);
        
        \fill[black] (-0.5, 6)      circle(3.4pt);
        \fill[black] (-0.5, 5)      circle(3.4pt);
        \fill[black] (-3.5, 5)      circle(3.4pt);
        \fill[black] (-2.5, 5)      circle(3.4pt);
        \fill[black] (-2.25, 4)     circle(3.4pt);
        \fill[black] (-1.833, 4.75) circle(3.4pt);
        \fill[black] (-1.833, 5.2)  circle(3.4pt);
        \fill[black] (-1.25, 3)     circle(3.4pt);
        \fill[black] (-1.5, 3.5)    circle(3.4pt);
        \fill[black] (-1.626, 2.75) circle(3.4pt);
        \fill[black] (-2, 4)        circle(3.4pt);
        \fill[black] (-1.5, 3.4)    circle(3.4pt);
        \fill[black] (-1.25, 4.75)  circle(3.4pt);
        \fill[black] (-1.166, 5.5)  circle(3.4pt);
        \fill[black] (-1, 4)        circle(3.4pt);

      \end{scope}
    \end{scope}
  \end{tikzsubfigure}
  \begin{tikzsubfigure}{\label{fig:expansion:patch:5:11:c}}{$\mathcal{P}_N$}{1.0}
    \begin{scope}[scale=0.40]
      \draw (13,6)--(12,7)--(11,6)--(11,3)--(8,0)--(3,0)--(0,5)--(3,8)--(4,9)--(3,10)--(3,12)--(4,13)--(6,14)--(6,16)--(8,18)--(10,17)--(12,16)--(13,14)--(16,14)--(16,12)--(20,11)--(21,9)--(20,7)--(19,6)--(17,5)--(15,5)--(13,6);
      \draw (3,0)--(4,3)--(4,6)--(3,8);
      \draw (4,3)--(8,4)--(11,3);
      \draw (8,4)--(8,5)--(6,6)--(4,6);
      \draw (8,5)--(10,6)--(11,6);
      \draw (10,6)--(9,7)--(6,8)--(6,6);
      \draw (6,8)--(4,9);
      \draw (9,7)--(11,9)--(12,7);
      \draw[ldiamond] (11,9)--(12,11);
      \draw (16,12)--(12,11)--(11,13)--(13,14);
      \draw (11,13)--(9,15)--(6,14);
      \draw (9,15)--(10,17);

      \node[anchor= 90] at (13,6)  {$i_{0}$};
      \node[anchor= 90] at (12,7)  {$i_{1}$};
      \node[anchor=160] at (11,6)  {$i_{2}$};
      \node[anchor= 90] at (11,3)  {$i_{3}$};
      \node[anchor=135] at (8,0)   {$i_{4}=o_0$};
      \node[anchor= 90] at (3,0)   {$o_{1}$};
      \node[anchor=  0] at (0,5)   {$o_{2}$};
      \node[anchor=  0] at (3,8)   {$o_{3}$};
      \node[anchor=  0] at (4,9)   {$o_{4}$};
      \node[anchor=  0] at (3,10)  {$\mathbf{o_{5}}$};
      \node[anchor=  0] at (3,12)  {$o_{6}$};
      \node[anchor=335] at (4,13)  {$o_{7}$};
      \node[anchor=335] at (6,14)  {$o_{8}$};
      \node[anchor=335] at (6,16)  {$o_{9}$};
      \node[anchor=270] at (8,18)  {$o_{10}$};
      \node[anchor=240] at (10,17) {$o_{11}$};
      \node[anchor=235] at (12,16) {$o_{12}$};
      \node[anchor=235] at (13,14) {$o_{13}$};
      \node[anchor=235] at (16,14) {$o_{14}$};
      \node[anchor=200] at (16,12) {$o_{15}$};
      \node[anchor=235] at (20,11) {$o_{16}$};
      \node[anchor=180] at (21,9)  {$i_{4}'=o_{17}$};
      \node[anchor=135] at (20,7)  {$i_{3}'$};
      \node[anchor= 90] at (19,6)  {$i_{2}'$};
      \node[anchor= 90] at (17,5)  {$i_{1}'$};
      \node[anchor= 90] at (15,5)  {$i_{0}'$};

      \fill[black]  (13,6)  circle (3.5pt);
      \fill[black]  (12,7)  circle (3.5pt);
      \fill[black]  (11,6)  circle (3.5pt);
      \fill[black]  (11,3)  circle (3.5pt);
      \fill[black]  (8,0)   circle (3.5pt);
      \fill[black]  (3,0)   circle (3.5pt);
      \fill[black]  (0,5)   circle (3.5pt);
      \fill[black]  (3,8)   circle (3.5pt);
      \fill[black]  (4,9)   circle (3.5pt);
      \fill[black]  (4,13)  circle (3.5pt);
      \fill[black]  (3,10)  circle (3.5pt);
      \fill[black]  (3,12)  circle (3.5pt);
      \fill[black]  (6,14)  circle (3.5pt);
      \fill[black]  (6,16)  circle (3.5pt);
      \fill[black]  (8,18)  circle (3.5pt);
      \fill[black]  (10,17) circle (3.5pt);
      \fill[black]  (12,16) circle (3.5pt);
      \fill[black]  (13,14) circle (3.5pt);
      \fill[black]  (16,14) circle (3.5pt);
      \fill[black]  (16,12) circle (3.5pt);
      \fill[black]  (20,11) circle (3.5pt);
      \fill[black]  (21,9)  circle (3.5pt);
      \fill[black]  (20,7)  circle (3.5pt);
      \fill[black]  (19,6)  circle (3.5pt);
      \fill[black]  (17,5)  circle (3.5pt);
      \fill[black]  (15,5)  circle (3.5pt);

      \fill[black]  (4,3)   circle (3.5pt);
      \fill[black]  (4,6)   circle (3.5pt);
      \fill[black]  (6,6)   circle (3.5pt);
      \fill[black]  (8,4)   circle (3.5pt);
      \fill[black]  (8,5)   circle (3.5pt);
      \fill[black]  (10,6)  circle (3.5pt);
      \fill[black]  (9,7)   circle (3.5pt);
      \fill[black]  (6,8)   circle (3.5pt);
      \fill[black]  (11,9)  circle (3.5pt);
      \fill[black]  (12,11) circle (3.5pt);
      \fill[black]  (11,13) circle (3.5pt);
      \fill[black]  (9,15)  circle (3.5pt);

      \node (n1) at (9,12)  {$11$};
      \node (n2) at (14,9)  {$11$};
      \node at (2,4)   {$5$};
      \node at (7,2)   {$5$};
      \node at (6,4.5) {$5$};
      \node at (10,5)  {$5$};
      \node at (8,6)   {$5$};
      \node at (5,7)   {$5$};
      \node at (10.5,7){$5$};
      \node at (13,12) {$5$};
      \node at (11,15) {$5$};
      \node at (8,16)  {$5$};

      \draw[lface] (11.5,10)--(n1);
      \draw[lface] (11.5,10)--(n2);
      
    \end{scope}
  \end{tikzsubfigure}
\end{tikzfigure2}
\begin{figure}
  \ContinuedFloat
  \begin{tikzsubfigure}{\label{fig:expansion:patch:5:11:d}}{$\mathcal{P}_F$}{1.0}
    \begin{scope}[scale=8]
      \node (n1) at (-0.699424, 0.122136) {5};
\node (n2) at (-0.645896, 0.089428) {5};
\node (n3) at (-0.722974, 0.251192) {5};
\node (n4) at (-0.747331, 0.148123) {5};
\node (n5) at (-0.707681, 0.026769) {5};
\node (n6) at (-0.670715, 0.027041) {5};
\node (n7) at (-0.616530, -0.028285) {5};
\node (n8) at (-0.668559, -0.080133) {5};
\node (n9) at (-0.604220, -0.187435) {5};
\node (n10) at (-0.829480, -0.138309) {11};
\node (n11) at (-0.962817, -0.183831) {5};
\node (n12) at (-0.935208, -0.081614) {5};
\node (n13) at (-0.972512, -0.090582) {5};
\node (n14) at (-0.967553, -0.008036) {5};
\node (n15) at (-0.913031, 0.022375) {5};
\node (n16) at (-0.971996, 0.081203) {5};
\node (n17) at (-0.858365, 0.310900) {11};
\node (n18) at (-0.586910, 0.231319) {11};
\node (n19) at (-0.590756, -0.510035) {11};
\node (n20) at (-0.488525, -0.795333) {5};
\node (n21) at (-0.422337, -0.863757) {5};
\node (n22) at (-0.311337, -0.916641) {5};
\node (n23)[anchor=70] at (-0.396407, -0.907809) {5};
\node (n24)[anchor=70] at (-0.305245, -0.944399) {5};
\node (n25)[anchor=70] at (-0.211224, -0.965788) {5};
\node (n26) at (-0.064717, -0.897820) {11};
\node (n27) at (-0.624121, 0.644906) {5};
\node (n28) at (-0.464174, 0.425073) {5};
\node (n29) at (-0.553069, 0.732793) {5};
\node (n30) at (-0.442730, 0.674000) {5};
\node (n31) at (0.010798, -0.219834) {11};
\node (n32) at (-0.353975, 0.767150) {5};
\node (n33) at (0.258788, -0.136434) {5};
\node (n34) at (0.425806, -0.617112) {5};
\node (n35) at (0.432353, -0.312981) {5};
\node (n36) at (0.546494, -0.694544) {5};
\node (n37) at (0.330055, 0.126970) {5};
\node (n38) at (0.004399, 0.697075) {11};
\node (n39) at (0.294845, 0.555036) {5};
\node (n40) at (0.205634, 0.886304) {5};
\node (n41) at (0.310906, 0.848768) {5};
\node (n42) at (0.509040, 0.085933) {11};
\node (n43) at (0.714942, -0.393420) {5};
\node (n44) at (0.583545, 0.417738) {5};
\node (n45) at (0.607059, 0.670930) {5};
\node (n46) at (0.707560, 0.577224) {5};
\node (n47) at (0.843752, 0.273756) {5};
\node (n48) at (0.822384, -0.113448) {11};
\node (n49) at (0.959670, -0.010016) {5};
\node (n50) at (0.940701, 0.123684) {5};
\node (n51)[anchor=180] at (0.981802, 0.137126) {5};
\node (n52) at (0.834612, 0.436302) {11};
\node (n53)[anchor=200] at (0.958759, 0.242490) {5};
\node (n54)[anchor=200] at (0.925371, 0.354526) {5};
\node (n55)[anchor=200] at (0.879309, 0.465948) {5};
\node (n56)[anchor=280] at (-0.183053, 0.979247) {5};
\node (n57)[anchor=340] at (-0.939579, 0.331095) {5};
\node (n58)[anchor= 40] at (-0.756526, -0.648152) {5};
\node (n59)[anchor=100] at (0.183053, -0.979247) {5};
\node (n60)[anchor=160] at (0.939579, -0.331095) {5};


% \node (n0) at (-0.923074, 0.096932) {0};
% \node (n1) at (-0.699424, 0.122136) {1};
% \node (n2) at (-0.660896, 0.119428) {2};
% \node (n3) at (-0.722974, 0.251192) {3};
% \node (n4) at (-0.747331, 0.148123) {4};
% \node (n5) at (-0.707681, 0.026769) {5};
% \node (n6) at (-0.665715, 0.027041) {6};
% \node (n7) at (-0.616530, -0.028285) {7};
% \node (n8) at (-0.668559, -0.080133) {8};
% \node (n9) at (-0.604220, -0.187435) {9};
% \node (n10) at (-0.829480, -0.138309) {10};
% \node (n11) at (-0.962817, -0.183831) {11};
% \node (n12) at (-0.935208, -0.081614) {12};
% \node (n13) at (-0.979512, -0.090582) {13};
% \node (n14) at (-0.967553, -0.008036) {14};
% \node (n15) at (-0.913031, 0.022375) {15};
% \node (n16) at (-0.971996, 0.081203) {16};
% \node (n17) at (-0.858365, 0.310900) {17};
% \node (n18) at (-0.586910, 0.231319) {18};
% \node (n19) at (-0.590756, -0.510035) {19};
% \node (n20) at (-0.488525, -0.795333) {20};
% \node (n21) at (-0.422337, -0.863757) {21};
% \node (n22) at (-0.311337, -0.916641) {22};
% \node (n23)[anchor=70] at (-0.396407, -0.907809) {23};
% \node (n24)[anchor=70] at (-0.305245, -0.944399) {24};
% \node (n25)[anchor=70] at (-0.211224, -0.965788) {25};
% \node (n26) at (-0.064717, -0.897820) {26};
% \node (n27) at (-0.624121, 0.644906) {27};
% \node (n28) at (-0.464174, 0.425073) {28};
% \node (n29) at (-0.553069, 0.732793) {29};
% \node (n30) at (-0.442730, 0.674000) {30};
% \node (n31) at (0.010798, -0.219834) {31};
% \node (n32) at (-0.353975, 0.767150) {32};
% \node (n33) at (0.258788, -0.136434) {33};
% \node (n34) at (0.425806, -0.617112) {34};
% \node (n35) at (0.432353, -0.312981) {35};
% \node (n36) at (0.546494, -0.694544) {36};
% \node (n37) at (0.330055, 0.126970) {37};
% \node (n38) at (0.004399, 0.697075) {38};
% \node (n39) at (0.294845, 0.555036) {39};
% \node (n40) at (0.205634, 0.886304) {40};
% \node (n41) at (0.310906, 0.848768) {41};
% \node (n42) at (0.509040, 0.085933) {42};
% \node (n43) at (0.714942, -0.393420) {43};
% \node (n44) at (0.583545, 0.417738) {44};
% \node (n45) at (0.607059, 0.670930) {45};
% \node (n46) at (0.707560, 0.577224) {46};
% \node (n47) at (0.843752, 0.273756) {47};
% \node (n48) at (0.822384, -0.113448) {48};
% \node (n49) at (0.959670, -0.010016) {49};
% \node (n50) at (0.940701, 0.123684) {50};
% \node (n51)[anchor=180] at (0.981802, 0.137126) {51};
% \node (n52) at (0.824612, 0.476302) {52};
% \node (n53)[anchor=200] at (0.958759, 0.242490) {53};
% \node (n54)[anchor=200] at (0.925371, 0.354526) {54};
% \node (n55)[anchor=200] at (0.879309, 0.465948) {55};
% \node (n56)[anchor=280] at (-0.183053, 0.979247) {56};
% \node (n57)[anchor=340] at (-0.939579, 0.331095) {57};
% \node (n58)[anchor= 40] at (-0.756526, -0.648152) {58};
% \node (n59)[anchor=100] at (0.183053, -0.979247) {59};
% \node (n60)[anchor=160] at (0.939579, -0.331095) {60};

\coordinate (x0) at (-0.668668, 0.095025);
\coordinate (x1) at (-0.685642, 0.165465);
\coordinate (x2) at (-0.731296, 0.186654);
\coordinate (x3) at (-0.720998, 0.108076);
\coordinate (x4) at (-0.690517, 0.055461);
\coordinate (x5) at (-0.645580, 0.032069);
\coordinate (x6) at (-0.618618, 0.069871);
\coordinate (x7) at (-0.685971, 0.234711);
\coordinate (x8) at (-0.751087, 0.404079);
\coordinate (x9) at (-0.760876, 0.265049);
\coordinate (x10) at (-0.784845, 0.137658);
\coordinate (x11) at (-0.738643, 0.043176);
\coordinate (x12) at (-0.706519, -0.063820);
\coordinate (x13) at (-0.681728, -0.009051);
\coordinate (x14) at (-0.642082, -0.038298);
\coordinate (x15) at (-0.627357, -0.111755);
\coordinate (x16) at (-0.549015, -0.093311);
\coordinate (x17) at (-0.685107, -0.177740);
\coordinate (x18) at (-0.695335, -0.303794);
\coordinate (x19) at (-0.464287, -0.250577);
\coordinate (x20) at (-0.858468, 0.052688);
\coordinate (x21) at (-0.897711, -0.052524);
\coordinate (x22) at (-0.931770, -0.161242);
\coordinate (x23) at (-0.961826, -0.273663);
\coordinate (x24) at (-0.943154, -0.332355);
\coordinate (x25) at (-0.920906, -0.389786);
\coordinate (x26) at (-0.955475, -0.116925);
\coordinate (x27) at (-0.988165, -0.153392);
\coordinate (x28) at (-0.976848, -0.213933);
\coordinate (x29) at (-0.932421, -0.017849);
\coordinate (x30) at (-0.958660, -0.059532);
\coordinate (x31) at (-0.999526, -0.030795);
\coordinate (x32) at (-0.995734, -0.092268);
\coordinate (x33) at (-0.947632, 0.037202);
\coordinate (x34) at (-0.999526, 0.030795);
\coordinate (x35) at (-0.928921, 0.092359);
\coordinate (x36) at (-0.988165, 0.153392);
\coordinate (x37) at (-0.995734, 0.092268);
\coordinate (x38) at (-0.798017, 0.602635);
\coordinate (x39) at (-0.833602, 0.552365);
\coordinate (x40) at (-0.866025, 0.500000);
\coordinate (x41) at (-0.895163, 0.445738);
\coordinate (x42) at (-0.976848, 0.213933);
\coordinate (x43) at (-0.215374, -0.282460);
\coordinate (x44) at (-0.361140, 0.101243);
\coordinate (x45) at (-0.535184, 0.411565);
\coordinate (x46) at (-0.717912, 0.696134);
\coordinate (x47) at (-0.759405, 0.650618);
\coordinate (x48) at (-0.895163, -0.445738);
\coordinate (x49) at (-0.866025, -0.500000);
\coordinate (x50) at (-0.833602, -0.552365);
\coordinate (x51) at (-0.673696, -0.739009);
\coordinate (x52) at (-0.626924, -0.779081);
\coordinate (x53) at (-0.312288, -0.719588);
\coordinate (x54) at (0.005287, -0.647988);
\coordinate (x55) at (-0.577774, -0.816197);
\coordinate (x56) at (-0.526432, -0.850217);
\coordinate (x57) at (-0.399205, -0.811581);
\coordinate (x58) at (-0.473094, -0.881012);
\coordinate (x59) at (-0.391582, -0.893607);
\coordinate (x60) at (-0.321374, -0.882366);
\coordinate (x61) at (-0.338158, -0.923487);
\coordinate (x62) at (-0.279761, -0.943295);
\coordinate (x63) at (-0.225808, -0.940452);
\coordinate (x64) at (-0.417960, -0.908465);
\coordinate (x65) at (-0.361242, -0.932472);
\coordinate (x66) at (-0.303153, -0.952942);
\coordinate (x67) at (-0.243914, -0.969797);
\coordinate (x68) at (-0.183750, -0.982973);
\coordinate (x69) at (-0.122888, -0.992421);
\coordinate (x70) at (-0.061561, -0.998103);
\coordinate (x71) at (-0.000000, -1.000000);
\coordinate (x72) at (0.061561, -0.998103);
\coordinate (x73) at (0.303153, -0.952942);
\coordinate (x74) at (0.361242, -0.932472);
\coordinate (x75) at (-0.566888, 0.598741);
\coordinate (x76) at (-0.626924, 0.779081);
\coordinate (x77) at (-0.673696, 0.739009);
\coordinate (x78) at (-0.390331, 0.394090);
\coordinate (x79) at (-0.467326, 0.619728);
\coordinate (x80) at (-0.526432, 0.850217);
\coordinate (x81) at (-0.577774, 0.816197);
\coordinate (x82) at (-0.356467, 0.624952);
\coordinate (x83) at (-0.473094, 0.881012);
\coordinate (x84) at (0.417960, -0.908465);
\coordinate (x85) at (0.473094, -0.881012);
\coordinate (x86) at (0.275335, -0.417427);
\coordinate (x87) at (0.070285, 0.042522);
\coordinate (x88) at (-0.161112, 0.488847);
\coordinate (x89) at (-0.361242, 0.932472);
\coordinate (x90) at (-0.417960, 0.908465);
\coordinate (x91) at (0.383668, -0.349580);
\coordinate (x92) at (0.325770, -0.100235);
\coordinate (x93) at (0.238884, 0.142552);
\coordinate (x94) at (0.526432, -0.850217);
\coordinate (x95) at (0.470500, -0.587325);
\coordinate (x96) at (0.530838, -0.439900);
\coordinate (x97) at (0.450990, -0.087865);
\coordinate (x98) at (0.577774, -0.816197);
\coordinate (x99) at (0.626924, -0.779081);
\coordinate (x100) at (0.386167, 0.266693);
\coordinate (x101) at (0.248462, 0.413706);
\coordinate (x102) at (0.193374, 0.706155);
\coordinate (x103) at (0.122888, 0.992421);
\coordinate (x104) at (0.061561, 0.998103);
\coordinate (x105) at (0.000000, 1.000000);
\coordinate (x106) at (-0.061561, 0.998103);
\coordinate (x107) at (-0.303153, 0.952942);
\coordinate (x108) at (0.361979, 0.608454);
\coordinate (x109) at (0.284243, 0.780174);
\coordinate (x110) at (0.243914, 0.969797);
\coordinate (x111) at (0.183750, 0.982973);
\coordinate (x112) at (0.361242, 0.932472);
\coordinate (x113) at (0.303153, 0.952942);
\coordinate (x114) at (0.673696, -0.739009);
\coordinate (x115) at (0.717912, -0.696134);
\coordinate (x116) at (0.598640, 0.090154);
\coordinate (x117) at (0.473094, 0.881012);
\coordinate (x118) at (0.417960, 0.908465);
\coordinate (x119) at (0.759405, -0.650618);
\coordinate (x120) at (0.798017, -0.602635);
\coordinate (x121) at (0.700737, -0.107865);
\coordinate (x122) at (0.618822, 0.375172);
\coordinate (x123) at (0.526432, 0.850217);
\coordinate (x124) at (0.685341, 0.533981);
\coordinate (x125) at (0.626924, 0.779081);
\coordinate (x126) at (0.577774, 0.816197);
\coordinate (x127) at (0.792281, 0.329848);
\coordinate (x128) at (0.759556, 0.504203);
\coordinate (x129) at (0.673696, 0.739009);
\coordinate (x130) at (0.891217, 0.088999);
\coordinate (x131) at (0.913710, 0.127376);
\coordinate (x132) at (0.851999, 0.308357);
\coordinate (x133) at (0.833602, -0.552365);
\coordinate (x134) at (0.866025, -0.500000);
\coordinate (x135) at (0.895163, -0.445738);
\coordinate (x136) at (0.976848, -0.213933);
\coordinate (x137) at (0.988165, -0.153392);
\coordinate (x138) at (0.995734, -0.092268);
\coordinate (x139) at (0.999526, -0.030795);
\coordinate (x140) at (0.999526, 0.030795);
\coordinate (x141) at (0.938746, 0.172687);
\coordinate (x142) at (0.995734, 0.092268);
\coordinate (x143) at (0.988165, 0.153392);
\coordinate (x144) at (0.946836, 0.236487);
\coordinate (x145) at (0.920118, 0.334975);
\coordinate (x146) at (0.890850, 0.416851);
\coordinate (x147) at (0.833602, 0.552365);
\coordinate (x148) at (0.798017, 0.602635);
\coordinate (x149) at (0.759405, 0.650618);
\coordinate (x150) at (0.717912, 0.696134);
\coordinate (x151) at (0.976848, 0.213933);
\coordinate (x152) at (0.961826, 0.273663);
\coordinate (x153) at (0.943154, 0.332355);
\coordinate (x154) at (0.920906, 0.389786);
\coordinate (x155) at (0.895163, 0.445738);
\coordinate (x156) at (0.866025, 0.500000);
\coordinate (x157) at ($1.05*(-0.122888, 0.992421)$);
\coordinate (x158) at ($1.07*(-0.183750, 0.982973)$);
\coordinate (x159) at ($1.05*(-0.243914, 0.969797)$);
\coordinate (x160) at ($1.05*(-0.920906, 0.389786)$);
\coordinate (x161) at ($1.07*(-0.943154, 0.332355)$);
\coordinate (x162) at ($1.05*(-0.961826, 0.273663)$);
\coordinate (x163) at ($1.05*(-0.798017, -0.602635)$);
\coordinate (x164) at ($1.07*(-0.759405, -0.650618)$);
\coordinate (x165) at ($1.05*(-0.717912, -0.696134)$);
\coordinate (x166) at ($1.05*(0.122888, -0.992421)$);
\coordinate (x167) at ($1.07*(0.183750, -0.982973)$);
\coordinate (x168) at ($1.05*(0.243914, -0.969797)$);
\coordinate (x169) at ($1.05*(0.920906, -0.389786)$);
\coordinate (x170) at ($1.07*(0.943154, -0.332355)$);
\coordinate (x171) at ($1.05*(0.961826, -0.273663)$);

\draw (-0.668668, 0.095025) -- (-0.685642, 0.165465);
\draw (-0.685642, 0.165465) -- (-0.731296, 0.186654);
\draw (-0.731296, 0.186654) -- (-0.720998, 0.108076);
\draw (-0.720998, 0.108076) -- (-0.690517, 0.055461);
\draw (-0.690517, 0.055461) -- (-0.668668, 0.095025);
\draw (-0.668668, 0.095025) -- (-0.645580, 0.032069);
\draw (-0.645580, 0.032069) -- (-0.618618, 0.069871);
\draw (-0.618618, 0.069871) -- (-0.685971, 0.234711);
\draw (-0.685971, 0.234711) -- (-0.685642, 0.165465);
\draw (-0.685971, 0.234711) -- (-0.751087, 0.404079);
\draw (-0.751087, 0.404079) -- (-0.760876, 0.265049);
\draw (-0.760876, 0.265049) -- (-0.731296, 0.186654);
\draw (-0.760876, 0.265049) -- (-0.784845, 0.137658);
\draw (-0.784845, 0.137658) -- (-0.738643, 0.043176);
\draw (-0.738643, 0.043176) -- (-0.720998, 0.108076);
\draw (-0.738643, 0.043176) -- (-0.706519, -0.063820);
\draw (-0.706519, -0.063820) -- (-0.681728, -0.009051);
\draw (-0.681728, -0.009051) -- (-0.690517, 0.055461);
\draw (-0.681728, -0.009051) -- (-0.642082, -0.038298);
\draw (-0.642082, -0.038298) -- (-0.645580, 0.032069);
\draw (-0.642082, -0.038298) -- (-0.627357, -0.111755);
\draw (-0.627357, -0.111755) -- (-0.549015, -0.093311);
\draw (-0.549015, -0.093311) -- (-0.618618, 0.069871);
\draw (-0.706519, -0.063820) -- (-0.685107, -0.177740);
\draw (-0.685107, -0.177740) -- (-0.627357, -0.111755);
\draw (-0.685107, -0.177740) -- (-0.695335, -0.303794);
\draw (-0.695335, -0.303794) -- (-0.464287, -0.250577);
\draw (-0.464287, -0.250577) -- (-0.549015, -0.093311);
\draw (-0.784845, 0.137658) -- (-0.858468, 0.052688);
\draw (-0.858468, 0.052688) -- (-0.897711, -0.052524);
\draw (-0.897711, -0.052524) -- (-0.931770, -0.161242);
\draw (-0.931770, -0.161242) -- (-0.961826, -0.273663);
\draw (-0.961826, -0.273663) -- (-0.943154, -0.332355);
\draw (-0.943154, -0.332355) -- (-0.920906, -0.389786);
\draw[ldiamond] (-0.920906, -0.389786) -- (-0.695335, -0.303794) node[midway] (m1) {};
\draw[lface] (m1) -- (n10);
\draw[lface] (m1) -- (n19);
\draw (-0.931770, -0.161242) -- (-0.955475, -0.116925);
\draw (-0.955475, -0.116925) -- (-0.988165, -0.153392);
\draw (-0.988165, -0.153392) -- (-0.976848, -0.213933);
\draw (-0.976848, -0.213933) -- (-0.961826, -0.273663);
\draw (-0.897711, -0.052524) -- (-0.932421, -0.017849);
\draw (-0.932421, -0.017849) -- (-0.958660, -0.059532);
\draw (-0.958660, -0.059532) -- (-0.955475, -0.116925);
\draw (-0.958660, -0.059532) -- (-0.999526, -0.030795);
\draw (-0.999526, -0.030795) -- (-0.995734, -0.092268);
\draw (-0.995734, -0.092268) -- (-0.988165, -0.153392);
\draw (-0.932421, -0.017849) -- (-0.947632, 0.037202);
\draw (-0.947632, 0.037202) -- (-0.999526, 0.030795);
\draw (-0.999526, 0.030795) -- (-0.999526, -0.030795);
\draw (-0.858468, 0.052688) -- (-0.928921, 0.092359);
\draw (-0.928921, 0.092359) -- (-0.947632, 0.037202);
\draw (-0.928921, 0.092359) -- (-0.988165, 0.153392);
\draw (-0.988165, 0.153392) -- (-0.995734, 0.092268);
\draw (-0.995734, 0.092268) -- (-0.999526, 0.030795);
\draw[ldiamond] (-0.751087, 0.404079) -- (-0.798017, 0.602635) node[midway] (m2) {};
\draw[lface] (m2) -- (n17);
\draw[lface] (m2) -- (n18);
\draw (-0.798017, 0.602635) -- (-0.833602, 0.552365);
\draw (-0.833602, 0.552365) -- (-0.866025, 0.500000);
\draw (-0.866025, 0.500000) -- (-0.895163, 0.445738);
\draw (-0.895163, 0.445738) -- (-0.976848, 0.213933);
\draw (-0.976848, 0.213933) -- (-0.988165, 0.153392);
\draw (-0.464287, -0.250577) -- (-0.215374, -0.282460);
\draw (-0.215374, -0.282460) -- (-0.361140, 0.101243);
\draw (-0.361140, 0.101243) -- (-0.535184, 0.411565);
\draw (-0.535184, 0.411565) -- (-0.717912, 0.696134);
\draw (-0.717912, 0.696134) -- (-0.759405, 0.650618);
\draw (-0.759405, 0.650618) -- (-0.798017, 0.602635);
\draw (-0.920906, -0.389786) -- (-0.895163, -0.445738);
\draw (-0.895163, -0.445738) -- (-0.866025, -0.500000);
\draw (-0.866025, -0.500000) -- (-0.833602, -0.552365);
\draw (-0.833602, -0.552365) -- (-0.673696, -0.739009);
\draw (-0.673696, -0.739009) -- (-0.626924, -0.779081);
\draw (-0.626924, -0.779081) -- (-0.312288, -0.719588);
\draw (-0.312288, -0.719588) -- (0.005287, -0.647988);
\draw (0.005287, -0.647988) -- (-0.215374, -0.282460);
\draw (-0.626924, -0.779081) -- (-0.577774, -0.816197);
\draw (-0.577774, -0.816197) -- (-0.526432, -0.850217);
\draw (-0.526432, -0.850217) -- (-0.399205, -0.811581);
\draw (-0.399205, -0.811581) -- (-0.312288, -0.719588);
\draw (-0.526432, -0.850217) -- (-0.473094, -0.881012);
\draw (-0.473094, -0.881012) -- (-0.391582, -0.893607);
\draw (-0.391582, -0.893607) -- (-0.321374, -0.882366);
\draw (-0.321374, -0.882366) -- (-0.399205, -0.811581);
\draw (-0.391582, -0.893607) -- (-0.338158, -0.923487);
\draw (-0.338158, -0.923487) -- (-0.279761, -0.943295);
\draw (-0.279761, -0.943295) -- (-0.225808, -0.940452);
\draw (-0.225808, -0.940452) -- (-0.321374, -0.882366);
\draw (-0.473094, -0.881012) -- (-0.417960, -0.908465);
\draw (-0.417960, -0.908465) -- (-0.361242, -0.932472);
\draw (-0.361242, -0.932472) -- (-0.338158, -0.923487);
\draw (-0.361242, -0.932472) -- (-0.303153, -0.952942);
\draw (-0.303153, -0.952942) -- (-0.243914, -0.969797);
\draw (-0.243914, -0.969797) -- (-0.279761, -0.943295);
\draw (-0.243914, -0.969797) -- (-0.183750, -0.982973);
\draw (-0.183750, -0.982973) -- (-0.122888, -0.992421);
\draw (-0.122888, -0.992421) -- (-0.225808, -0.940452);
\draw (-0.122888, -0.992421) -- (-0.061561, -0.998103);
\draw (-0.061561, -0.998103) -- (-0.000000, -1.000000);
\draw (-0.000000, -1.000000) -- (0.061561, -0.998103);
\draw (0.061561, -0.998103) -- (0.303153, -0.952942);
\draw (0.303153, -0.952942) -- (0.361242, -0.932472);
\draw[ldiamond] (0.361242, -0.932472) -- (0.005287, -0.647988) node[midway] (m3) {};
\draw[lface] (m3) -- (n26);
\draw[lface] (m3) -- (n31);
\draw (-0.535184, 0.411565) -- (-0.566888, 0.598741);
\draw (-0.566888, 0.598741) -- (-0.626924, 0.779081);
\draw (-0.626924, 0.779081) -- (-0.673696, 0.739009);
\draw (-0.673696, 0.739009) -- (-0.717912, 0.696134);
\draw (-0.361140, 0.101243) -- (-0.390331, 0.394090);
\draw (-0.390331, 0.394090) -- (-0.467326, 0.619728);
\draw (-0.467326, 0.619728) -- (-0.566888, 0.598741);
\draw (-0.467326, 0.619728) -- (-0.526432, 0.850217);
\draw (-0.526432, 0.850217) -- (-0.577774, 0.816197);
\draw (-0.577774, 0.816197) -- (-0.626924, 0.779081);
\draw (-0.390331, 0.394090) -- (-0.356467, 0.624952);
\draw (-0.356467, 0.624952) -- (-0.473094, 0.881012);
\draw (-0.473094, 0.881012) -- (-0.526432, 0.850217);
\draw (0.361242, -0.932472) -- (0.417960, -0.908465);
\draw (0.417960, -0.908465) -- (0.473094, -0.881012);
\draw (0.473094, -0.881012) -- (0.275335, -0.417427);
\draw (0.275335, -0.417427) -- (0.070285, 0.042522);
\draw (0.070285, 0.042522) -- (-0.161112, 0.488847);
\draw (-0.161112, 0.488847) -- (-0.356467, 0.624952);
\draw (-0.161112, 0.488847) -- (-0.361242, 0.932472);
\draw (-0.361242, 0.932472) -- (-0.417960, 0.908465);
\draw (-0.417960, 0.908465) -- (-0.473094, 0.881012);
\draw (0.275335, -0.417427) -- (0.383668, -0.349580);
\draw (0.383668, -0.349580) -- (0.325770, -0.100235);
\draw (0.325770, -0.100235) -- (0.238884, 0.142552);
\draw (0.238884, 0.142552) -- (0.070285, 0.042522);
\draw (0.473094, -0.881012) -- (0.526432, -0.850217);
\draw (0.526432, -0.850217) -- (0.470500, -0.587325);
\draw (0.470500, -0.587325) -- (0.383668, -0.349580);
\draw (0.470500, -0.587325) -- (0.530838, -0.439900);
\draw (0.530838, -0.439900) -- (0.450990, -0.087865);
\draw (0.450990, -0.087865) -- (0.325770, -0.100235);
\draw (0.526432, -0.850217) -- (0.577774, -0.816197);
\draw (0.577774, -0.816197) -- (0.626924, -0.779081);
\draw (0.626924, -0.779081) -- (0.530838, -0.439900);
\draw (0.450990, -0.087865) -- (0.386167, 0.266693);
\draw[lsquare] (0.386167, 0.266693) -- (0.248462, 0.413706);
\draw[lface] (0.386167, 0.266693) -- (n42);
\draw[lface] (0.248462, 0.413706) -- (n38);
\draw (0.248462, 0.413706) -- (0.238884, 0.142552);
\draw (0.248462, 0.413706) -- (0.193374, 0.706155);
\draw (0.193374, 0.706155) -- (0.122888, 0.992421);
\draw (0.122888, 0.992421) -- (0.061561, 0.998103);
\draw (0.061561, 0.998103) -- (0.000000, 1.000000);
\draw (0.000000, 1.000000) -- (-0.061561, 0.998103);
\draw (-0.061561, 0.998103) -- (-0.303153, 0.952942);
\draw (-0.303153, 0.952942) -- (-0.361242, 0.932472);
\draw (0.386167, 0.266693) -- (0.361979, 0.608454);
\draw (0.361979, 0.608454) -- (0.284243, 0.780174);
\draw (0.284243, 0.780174) -- (0.193374, 0.706155);
\draw (0.284243, 0.780174) -- (0.243914, 0.969797);
\draw (0.243914, 0.969797) -- (0.183750, 0.982973);
\draw (0.183750, 0.982973) -- (0.122888, 0.992421);
\draw (0.361979, 0.608454) -- (0.361242, 0.932472);
\draw (0.361242, 0.932472) -- (0.303153, 0.952942);
\draw (0.303153, 0.952942) -- (0.243914, 0.969797);
\draw (0.626924, -0.779081) -- (0.673696, -0.739009);
\draw (0.673696, -0.739009) -- (0.717912, -0.696134);
\draw (0.717912, -0.696134) -- (0.598640, 0.090154);
\draw (0.598640, 0.090154) -- (0.473094, 0.881012);
\draw (0.473094, 0.881012) -- (0.417960, 0.908465);
\draw (0.417960, 0.908465) -- (0.361242, 0.932472);
\draw (0.717912, -0.696134) -- (0.759405, -0.650618);
\draw (0.759405, -0.650618) -- (0.798017, -0.602635);
\draw (0.798017, -0.602635) -- (0.700737, -0.107865);
\draw (0.700737, -0.107865) -- (0.598640, 0.090154);
\draw (0.700737, -0.107865) -- (0.618822, 0.375172);
\draw (0.618822, 0.375172) -- (0.526432, 0.850217);
\draw (0.526432, 0.850217) -- (0.473094, 0.881012);
\draw (0.618822, 0.375172) -- (0.685341, 0.533981);
\draw (0.685341, 0.533981) -- (0.626924, 0.779081);
\draw (0.626924, 0.779081) -- (0.577774, 0.816197);
\draw (0.577774, 0.816197) -- (0.526432, 0.850217);
\draw (0.685341, 0.533981) -- (0.792281, 0.329848);
\draw[lsquare] (0.792281, 0.329848) -- (0.759556, 0.504203);
\draw[lface] (0.792281, 0.329848) -- (n48);
\draw[lface] (0.759556, 0.504203) -- (n52);
\draw (0.759556, 0.504203) -- (0.673696, 0.739009);
\draw (0.673696, 0.739009) -- (0.626924, 0.779081);
\draw (0.792281, 0.329848) -- (0.891217, 0.088999);
\draw (0.891217, 0.088999) -- (0.913710, 0.127376);
\draw (0.913710, 0.127376) -- (0.851999, 0.308357);
\draw (0.851999, 0.308357) -- (0.759556, 0.504203);
\draw (0.798017, -0.602635) -- (0.833602, -0.552365);
\draw (0.833602, -0.552365) -- (0.866025, -0.500000);
\draw (0.866025, -0.500000) -- (0.895163, -0.445738);
\draw (0.895163, -0.445738) -- (0.976848, -0.213933);
\draw (0.976848, -0.213933) -- (0.988165, -0.153392);
\draw (0.988165, -0.153392) -- (0.891217, 0.088999);
\draw (0.988165, -0.153392) -- (0.995734, -0.092268);
\draw (0.995734, -0.092268) -- (0.999526, -0.030795);
\draw (0.999526, -0.030795) -- (0.913710, 0.127376);
\draw (0.999526, -0.030795) -- (0.999526, 0.030795);
\draw (0.999526, 0.030795) -- (0.938746, 0.172687);
\draw (0.938746, 0.172687) -- (0.851999, 0.308357);
\draw (0.999526, 0.030795) -- (0.995734, 0.092268);
\draw (0.995734, 0.092268) -- (0.988165, 0.153392);
\draw (0.988165, 0.153392) -- (0.946836, 0.236487);
\draw (0.946836, 0.236487) -- (0.938746, 0.172687);
\draw (0.946836, 0.236487) -- (0.920118, 0.334975);
\draw (0.920118, 0.334975) -- (0.890850, 0.416851);
\draw (0.890850, 0.416851) -- (0.833602, 0.552365);
\draw (0.833602, 0.552365) -- (0.798017, 0.602635);
\draw (0.798017, 0.602635) -- (0.759405, 0.650618);
\draw (0.759405, 0.650618) -- (0.717912, 0.696134);
\draw (0.717912, 0.696134) -- (0.673696, 0.739009);
\draw (0.988165, 0.153392) -- (0.976848, 0.213933);
\draw (0.976848, 0.213933) -- (0.961826, 0.273663);
\draw (0.961826, 0.273663) -- (0.920118, 0.334975);
\draw (0.961826, 0.273663) -- (0.943154, 0.332355);
\draw (0.943154, 0.332355) -- (0.920906, 0.389786);
\draw (0.920906, 0.389786) -- (0.890850, 0.416851);
\draw (0.920906, 0.389786) -- (0.895163, 0.445738);
\draw (0.895163, 0.445738) -- (0.866025, 0.500000);
\draw (0.866025, 0.500000) -- (0.833602, 0.552365);
\draw (-0.061561, 0.998103) -- (x157);
\draw (x157) -- (x158);
\draw (x158) -- (x159);
\draw (x159) -- (-0.303153, 0.952942);
\draw (-0.895163, 0.445738) -- (x160);
\draw (x160) -- (x161);
\draw (x161) -- (x162);
\draw (x162) -- (-0.976848, 0.213933);
\draw (-0.833602, -0.552365) -- (x163);
\draw (x163) -- (x164);
\draw (x164) -- (x165);
\draw (x165) -- (-0.673696, -0.739009);
\draw (0.061561, -0.998103) -- (x166);
\draw (x166) -- (x167);
\draw (x167) -- (x168);
\draw (x168) -- (0.303153, -0.952942);
\draw (0.895163, -0.445738) -- (x169);
\draw (x169) -- (x170);
\draw (x170) -- (x171);
\draw (x171) -- (0.976848, -0.213933);


\fill[black] (x0) circle (0.3pt);
\fill[black] (x1) circle (0.3pt);
\fill[black] (x2) circle (0.3pt);
\fill[black] (x3) circle (0.3pt);
\fill[black] (x4) circle (0.3pt);
\fill[black] (x5) circle (0.3pt);
\fill[black] (x6) circle (0.3pt);
\fill[black] (x7) circle (0.3pt);
\fill[black] (x8) circle (0.3pt);
\fill[black] (x9) circle (0.3pt);
\fill[black] (x10) circle (0.3pt);
\fill[black] (x11) circle (0.3pt);
\fill[black] (x12) circle (0.3pt);
\fill[black] (x13) circle (0.3pt);
\fill[black] (x14) circle (0.3pt);
\fill[black] (x15) circle (0.3pt);
\fill[black] (x16) circle (0.3pt);
\fill[black] (x17) circle (0.3pt);
\fill[black] (x18) circle (0.3pt);
\fill[black] (x19) circle (0.3pt);
\fill[black] (x20) circle (0.3pt);
\fill[black] (x21) circle (0.3pt);
\fill[black] (x22) circle (0.3pt);
\fill[black] (x23) circle (0.3pt);
\fill[black] (x24) circle (0.3pt);
\fill[black] (x25) circle (0.3pt);
\fill[black] (x26) circle (0.3pt);
\fill[black] (x27) circle (0.3pt);
\fill[black] (x28) circle (0.3pt);
\fill[black] (x29) circle (0.3pt);
\fill[black] (x30) circle (0.3pt);
\fill[black] (x31) circle (0.3pt);
\fill[black] (x32) circle (0.3pt);
\fill[black] (x33) circle (0.3pt);
\fill[black] (x34) circle (0.3pt);
\fill[black] (x35) circle (0.3pt);
\fill[black] (x36) circle (0.3pt);
\fill[black] (x37) circle (0.3pt);
\fill[black] (x38) circle (0.3pt);
\fill[black] (x39) circle (0.3pt);
\fill[black] (x40) circle (0.3pt);
\fill[black] (x41) circle (0.3pt);
\fill[black] (x42) circle (0.3pt);
\fill[black] (x43) circle (0.3pt);
\fill[black] (x44) circle (0.3pt);
\fill[black] (x45) circle (0.3pt);
\fill[black] (x46) circle (0.3pt);
\fill[black] (x47) circle (0.3pt);
\fill[black] (x48) circle (0.3pt);
\fill[black] (x49) circle (0.3pt);
\fill[black] (x50) circle (0.3pt);
\fill[black] (x51) circle (0.3pt);
\fill[black] (x52) circle (0.3pt);
\fill[black] (x53) circle (0.3pt);
\fill[black] (x54) circle (0.3pt);
\fill[black] (x55) circle (0.3pt);
\fill[black] (x56) circle (0.3pt);
\fill[black] (x57) circle (0.3pt);
\fill[black] (x58) circle (0.3pt);
\fill[black] (x59) circle (0.3pt);
\fill[black] (x60) circle (0.3pt);
\fill[black] (x61) circle (0.3pt);
\fill[black] (x62) circle (0.3pt);
\fill[black] (x63) circle (0.3pt);
\fill[black] (x64) circle (0.3pt);
\fill[black] (x65) circle (0.3pt);
\fill[black] (x66) circle (0.3pt);
\fill[black] (x67) circle (0.3pt);
\fill[black] (x68) circle (0.3pt);
\fill[black] (x69) circle (0.3pt);
\fill[black] (x70) circle (0.3pt);
\fill[black] (x71) circle (0.3pt);
\fill[black] (x72) circle (0.3pt);
\fill[black] (x73) circle (0.3pt);
\fill[black] (x74) circle (0.3pt);
\fill[black] (x75) circle (0.3pt);
\fill[black] (x76) circle (0.3pt);
\fill[black] (x77) circle (0.3pt);
\fill[black] (x78) circle (0.3pt);
\fill[black] (x79) circle (0.3pt);
\fill[black] (x80) circle (0.3pt);
\fill[black] (x81) circle (0.3pt);
\fill[black] (x82) circle (0.3pt);
\fill[black] (x83) circle (0.3pt);
\fill[black] (x84) circle (0.3pt);
\fill[black] (x85) circle (0.3pt);
\fill[black] (x86) circle (0.3pt);
\fill[black] (x87) circle (0.3pt);
\fill[black] (x88) circle (0.3pt);
\fill[black] (x89) circle (0.3pt);
\fill[black] (x90) circle (0.3pt);
\fill[black] (x91) circle (0.3pt);
\fill[black] (x92) circle (0.3pt);
\fill[black] (x93) circle (0.3pt);
\fill[black] (x94) circle (0.3pt);
\fill[black] (x95) circle (0.3pt);
\fill[black] (x96) circle (0.3pt);
\fill[black] (x97) circle (0.3pt);
\fill[black] (x98) circle (0.3pt);
\fill[black] (x99) circle (0.3pt);
\fill[black] (x100) circle (0.3pt);
\fill[black] (x101) circle (0.3pt);
\fill[black] (x102) circle (0.3pt);
\fill[black] (x103) circle (0.3pt);
\fill[black] (x104) circle (0.3pt);
\fill[black] (x105) circle (0.3pt);
\fill[black] (x106) circle (0.3pt);
\fill[black] (x107) circle (0.3pt);
\fill[black] (x108) circle (0.3pt);
\fill[black] (x109) circle (0.3pt);
\fill[black] (x110) circle (0.3pt);
\fill[black] (x111) circle (0.3pt);
\fill[black] (x112) circle (0.3pt);
\fill[black] (x113) circle (0.3pt);
\fill[black] (x114) circle (0.3pt);
\fill[black] (x115) circle (0.3pt);
\fill[black] (x116) circle (0.3pt);
\fill[black] (x117) circle (0.3pt);
\fill[black] (x118) circle (0.3pt);
\fill[black] (x119) circle (0.3pt);
\fill[black] (x120) circle (0.3pt);
\fill[black] (x121) circle (0.3pt);
\fill[black] (x122) circle (0.3pt);
\fill[black] (x123) circle (0.3pt);
\fill[black] (x124) circle (0.3pt);
\fill[black] (x125) circle (0.3pt);
\fill[black] (x126) circle (0.3pt);
\fill[black] (x127) circle (0.3pt);
\fill[black] (x128) circle (0.3pt);
\fill[black] (x129) circle (0.3pt);
\fill[black] (x130) circle (0.3pt);
\fill[black] (x131) circle (0.3pt);
\fill[black] (x132) circle (0.3pt);
\fill[black] (x133) circle (0.3pt);
\fill[black] (x134) circle (0.3pt);
\fill[black] (x135) circle (0.3pt);
\fill[black] (x136) circle (0.3pt);
\fill[black] (x137) circle (0.3pt);
\fill[black] (x138) circle (0.3pt);
\fill[black] (x139) circle (0.3pt);
\fill[black] (x140) circle (0.3pt);
\fill[black] (x141) circle (0.3pt);
\fill[black] (x142) circle (0.3pt);
\fill[black] (x143) circle (0.3pt);
\fill[black] (x144) circle (0.3pt);
\fill[black] (x145) circle (0.3pt);
\fill[black] (x146) circle (0.3pt);
\fill[black] (x147) circle (0.3pt);
\fill[black] (x148) circle (0.3pt);
\fill[black] (x149) circle (0.3pt);
\fill[black] (x150) circle (0.3pt);
\fill[black] (x151) circle (0.3pt);
\fill[black] (x152) circle (0.3pt);
\fill[black] (x153) circle (0.3pt);
\fill[black] (x154) circle (0.3pt);
\fill[black] (x155) circle (0.3pt);
\fill[black] (x156) circle (0.3pt);
\fill[black] (x157) circle (0.3pt);
\fill[black] (x158) circle (0.3pt);
\fill[black] (x159) circle (0.3pt);
\fill[black] (x160) circle (0.3pt);
\fill[black] (x161) circle (0.3pt);
\fill[black] (x162) circle (0.3pt);
\fill[black] (x163) circle (0.3pt);
\fill[black] (x164) circle (0.3pt);
\fill[black] (x165) circle (0.3pt);
\fill[black] (x166) circle (0.3pt);
\fill[black] (x167) circle (0.3pt);
\fill[black] (x168) circle (0.3pt);
\fill[black] (x169) circle (0.3pt);
\fill[black] (x170) circle (0.3pt);
\fill[black] (x171) circle (0.3pt);


    \end{scope}
  \end{tikzsubfigure}
\end{figure}
