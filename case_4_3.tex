\mysection{$3$-valent {\sc Eberhard}-like theorems with quadrangles}

In this section we want to prove $3$-valent {\sc Eberhard}-like theorems with quadrangles, i.e. for $q = [q_4 \times 4, q_l \times l]$, $w = [w_3 \times 3]$, $l > 6$, $\gcd(q_4, q_l) = 1$. As it turns out in \autoref{sec:negative:results}, this is only possible for all admissible pairs of sequences on all closed orientable $2$-manifolds, if $2 \nmid l$. We split the remaining cases in two series, each with a residual class of $l \mod 4$ and additionally prove an {\sc Eberhard}-like theorem for $l = 10 + 4k$, $k \in \nats$ in all cases where we can apply \autoref{thm:main:const}. 
\clearpage
For all the proofs, we want to use \autoref{thm:main:const}, therefore we need to have a construction scheme for patches with arbitrary large $l$-gons. These we get by the next two constructions:
\begin{construction}\label{const:edge:replacement:4:1} When we want to use this construction in this section we label an edge (the specified edge) with a square and point with arrows to a $k_1$-gon and a $k_2$-gon.
  \begin{cinput}
  \item A $3$-patch with $p$-vector $p$ and a specified edge with exactly one vertex incident to some $k_1$-gon and the other vertex incident to some $k_2$-gon.
  \end{cinput}
  \begin{coutput}
  \item A $3$-patch with $p$-vector $p - [k_1, k_2] + [k \times 4] + [k_1 + k, k_2 + k]$ for all $k \in \nats$.
  \item If every two faces of the $3$-patch meet properly, then this is carried over to the new patch.
  \end{coutput}
  \begin{cdescription}
    Using the replacement of the single edge as seen in \autoref{fig:const:edge:replacement:4:1} results in a new $3$-patch with $p$-vector $p - [k_1, k_2] + [4] + [k_1 + 1, k_2 + 1]$. The line on the left labeled with a square is the specified edge and the line on the right labeled with a square is a new edge which we can use to repeat the construction. Every time we use this construction we add a new quadrangle while increasing the number of vertices of the left and right polygon by one; doing this $k$ times gives the desired $3$-patch. That all faces meet properly follows by induction as this property is preserved in each step.
    \begin{tikzfigure}{\label{fig:const:edge:replacement:4:1}}{}
      \matrix (m) [column sep=1cm] {
        \begin{scope}
          \draw[lsquare] (-1, 0) -- (1, 0);
          \draw (-1.2, 0.5) -- (-1, 0) -- (-1.2, -0.5);
          \draw (1.2, 0.5) -- (1, 0) -- (1.2, -0.5);
          \node (k1) at (-2.5, 0) {$k_1$};
          \node (k2) at (2.5, 0) {$k_2$};
          \draw[lface] (-1, 0) -- (k1);
          \draw[lface] ( 1, 0) -- (k2);
          \fill[black] (-1,0) circle(2pt);
          \fill[black] (1,0) circle(2pt);
        \end{scope}
        &
        \begin{scope}
          \draw[lsquare] (-1, 0.5) -- (1, 0.5);
          \draw (-1, -0.5) -- (1, -0.5);
          \draw (-1.2, 1) -- (-1, 0.5);
          \draw (-1.2, -1) -- (-1, -0.5);
          \draw (1.2, 1) -- (1, 0.5);
          \draw (1.2, -1) -- (1, -0.5);
          \draw (1, 0.5) -- (1, -0.5);
          \draw (-1, 0.5) -- (-1, -0.5);

          \fill[black] (-1,0.5) circle(2pt);
          \fill[black] (1,0.5) circle(2pt);
          \fill[black] (-1,-0.5) circle(2pt);
          \fill[black] (1,-0.5) circle(2pt);


          \node (k1) at (-2.8, 0) {$k_1 + 1$};
          \node (k2) at ( 2.8, 0) {$k_2 + 1$};
          \node      at (   0, 0) {$4$};
          \draw[lface] (-1, 0.5) -- (k1);
          \draw[lface] ( 1, 0.5) -- (k2);
        \end{scope}
        \\
      };
    \end{tikzfigure}
  \end{cdescription}
\end{construction}
\begin{construction}\label{const:edge:replacement:4:2} When we want to use this construction in this section we label an edge (the specified edge) with a diamond and point with arrows to a $k_1$-gon and a $k_2$-gon.
  \begin{cinput}
  \item A $3$-patch with $p$-vector $p$ and a specified edge incident to both some $k_1$-gon and some $k_2$-gon.
  \item We want to label the edge with a diamond and the two polygons by arrows beginning at the edge which point to them. 
  \end{cinput}
  \begin{coutput}
  \item A $3$-patch with $p$-vector $p - [k_1, k_2] + [(4k) \times 4] + [4k + k_1 , 4k + k_2]$ for all $k \in \nats$.
  \end{coutput}
  \begin{cdescription}
    Using the replacement of the single edge as seen in \autoref{fig:const:edge:replacement:4:2} results in a new $3$-patch with $p$-vector $p - [k_1, k_2] + [4 \times 4] + [k_1 + 4, k_2 + 4]$. The line on the left labeled with a diamond is the specified edge and the line on the right labeled with a diamond is a new edge which we can use to repeat the construction. Every time we use this construction we add four new quadrangles while increasing the number of vertices of the left and right polygon by four; doing this $k$ times gives the desired $3$-patch. 
    \begin{tikzfigure}{\label{fig:const:edge:replacement:4:2}}{}
      \matrix (m) [column sep=1cm] {
        \begin{scope}
          \draw[ldiamond] (0, -1) -- (0, 1);
          \draw (0.5, -1.2) -- (0, -1) -- (-0.5, -1.2);
          \draw (0.5, 1.2) -- (0, 1) -- (-0.5, 1.2);
          \node (k1) at (-2, 0) {$k_1$};
          \node (k2) at ( 2, 0) {$k_2$};
          \draw[lface] (0, 0) -- (k1);
          \draw[lface] (0, 0) -- (k2);

          \fill[black] (0,-1) circle(2pt);
          \fill[black] (0,1) circle(2pt);

        \end{scope}
        &
        \begin{scope}
          \draw (0.5, -2.2) -- (0, -2) -- (-0.5, -2.2);
          \draw (0.5, 2.7) -- (0, 2.5) -- (-0.5, 2.7);
          \draw[ldiamond] (0, 2.5) -- (0, 1.5);
          \draw (0, 1.5) -- (-0.5, 1) -- (-0.5, -1) -- (0, -1.5) -- (0.5, -1) -- (0.5, 1) -- (0, 1.5);
          \draw (0, -2) -- (0, -1.5);
          \draw (-0.5, -1) -- (0, -0.5) -- (0.5, -1);
          \draw (-0.5, 1) -- (0, 0.5) -- (0.5, 1);
          \draw (0, -0.5) -- (0, 0.5);
          \node (k1) at (-2, 0) {$k_1 + 4$};
          \node (k2) at (2, 0) {$k_2 + 4$};
          \draw[lface] (0, 2) -- (k1);
          \draw[lface] (0, 2) -- (k2);

          \fill[black] (0,-2) circle(2pt);
          \fill[black] (0,2.5) circle(2pt);
          \fill[black] (0,1.5) circle(2pt);
          \fill[black] (-0.5,1) circle(2pt);
          \fill[black] (-0.5,-1) circle(2pt);
          \fill[black] (0,-1.5) circle(2pt);
          \fill[black] (0.5,-1) circle(2pt);
          \fill[black] (0.5,1) circle(2pt);
          \fill[black] (0,-0.5) circle(2pt);
          \fill[black] (0,0.5) circle(2pt);

          \node at (-0.25, 0) {$4$};
          \node at (0.25, 0) {$4$};
          \node at (0, -1) {$4$};
          \node at (0, 1) {$4$};

        \end{scope}
        \\
      };
    \end{tikzfigure}
  \end{cdescription}
\end{construction}
\begin{lemma}\label{thm:expansion:patch:poly:4:k}
  There is an expansion $3$-patch $\mathcal{P}_P$ with the polyhedral property consisting of only quadrangles and $k$-gons, $k \geq 7$.
  \begin{proof}
    The $3$-patch $\mathcal{P}_P$ in \autoref{fig:expansion:patch:poly:4:a} consists of quadrangles and heptagons, whereas we wanted quadrangles and $k$-gons. To fix this, we use \autoref{const:edge:replacement:4:1} on all the edges labeled with squares to transform the heptagons to $k$-gons while increasing the number of quadrangles. $\mathcal{P}_P$ has the polyhedral property, which does not change after the transformations. Thus we yield a $3$-patch with the polyhedral property consisting of only quadrangles and $k$-gons.
    \begin{tikzfigure2}
      \begin{tikzsubfigure}{\label{fig:expansion:patch:poly:4:a}}{$\mathcal{P}_P$}{0.5}
        \begin{scope}[yscale=0.866]
          \draw (-0.5, 1) -- (-0.5, 3) -- (-2, 4) -- (-3.5, 7) -- (-3.5, 8) -- (-2.5, 7.666) -- (-1.5, 8) -- (-0.5, 9) -- (0.5, 9) -- (1.5, 10) -- (2.5, 10.333) -- (3.5, 10) -- (4.5, 9) -- (4.25, 7.5) -- (3.5, 7)  (2, 4) -- (2, 2) -- (0.5, 1) -- (-0.5, 1);
          \draw (3.5, 7) -- (0.5, 9);
          \draw[lsquare] (-1, 6) -- (-0.5, 6);
          \draw (-0.5, 1) -- (-0.5, 3) -- (0, 4) -- (-1, 6) -- (-2.5, 7.666);
          \draw (2, 4) -- (1, 4.666) -- (-0.5, 6) -- (-1.5, 8);
          \draw (0, 4) -- (1, 4.666);
          \draw[lsquare] (3.5, 7) -- (2, 4);

          \node (n1) at (-2,5.5)  {$7$};
          \node (n2) at (0.5,6.5) {$7$};
          \node (n3) at (2.5,8.5) {$7$};
          \node (n4) at (1,3)     {$7$};
          \node at (-1.5,7) {$4$};
          \node at (0.1,4.7)  {$4$};

          
          \draw[lface] (-1, 6) -- (n1);
          \draw[lface] (-0.5, 6) -- (n2);
          \draw[lface] (3.5, 7) -- (n3);
          \draw[lface] (2, 4) -- (n4);

          \node[anchor= 90] at (-0.5, 1)    {$i_0$};
          \node[anchor= 45] at (-0.5, 3)    {$i_1$};
          \node[anchor= 30] at (-2, 4)      {$i_2$};
          \node[anchor=  0] at (-3.5, 7)    {$i_3$};
          \node[anchor=270] at (-3.5, 8)    {$i_4=o_0$};
          \node[anchor=270] at (-2.5, 7.666){$o_{1}$};
          \node[anchor=300] at (-1.5, 8)    {$o_{2}$};
          \node[anchor=270] at (-0.5, 9)    {$o_{3}$};
          \node[anchor=300] at (0.5, 9)     {$o_{4}$};
          \node[anchor=300] at (1.5, 10)    {$o_{5}$};
          \node[anchor=270] at (2.5, 10.333){$o_{6}$};
          \node[anchor=245] at (3.5, 10)    {$o_{7}$};
          \node[anchor=180] at (4.5, 9)     {$\bm{o_{s}}$};
          \node[anchor=180] at (4.25, 7.5)  {$i_4'=o_{9}$};
          \node[anchor=160] at (3.5, 7)     {$i_3'$};
          \node[anchor=180] at (2, 4)       {$i_2'$};
          \node[anchor= 90] at (2, 2)       {$i_1'$};
          \node[anchor= 90] at (0.5, 1)     {$i_0'$};

          \fill[black] (-0.5, 1)    circle (2pt);
          \fill[black] (-0.5, 3)    circle (2pt);
          \fill[black] (-2, 4)      circle (2pt);
          \fill[black] (-3.5, 7)    circle (2pt);
          \fill[black] (-3.5, 8)    circle (2pt);
          \fill[black] (-2.5, 7.666)circle (2pt);
          \fill[black] (-1.5, 8)    circle (2pt);
          \fill[black] (-0.5, 9)    circle (2pt);
          \fill[black] (0.5, 9)     circle (2pt);
          \fill[black] (1.5, 10)    circle (2pt);
          \fill[black] (2.5, 10.333)circle (2pt);
          \fill[black] (3.5, 10)    circle (2pt);
          \fill[black] (4.5, 9)     circle (2pt);
          \fill[black] (4.25, 7.5)  circle (2pt);
          \fill[black] (3.5, 7)     circle (2pt);
          \fill[black] (2, 4)       circle (2pt);
          \fill[black] (2, 2)       circle (2pt);
          \fill[black] (0.5, 1)     circle (2pt);
          \fill[black] (-1,6)       circle (2pt);
          \fill[black] (-0.5,6)     circle (2pt);
          \fill[black] (0,4)        circle (2pt);
          \fill[black] (1, 4.666)   circle (2pt);
          
          
        \end{scope}
      \end{tikzsubfigure}~
      \begin{tikzsubfigure}{\label{fig:expansion:patch:poly:4:b}}{Edge patch of $\mathcal{P}_P$}{0.5}
        \begin{scope}[scale=0.5]
          \begin{scope}[yscale=0.866]
            \draw[very thick] (-0.5, 1) -- (-0.5, 3) -- (-2, 4) -- (-3.5, 7) -- (-3.5, 8) -- (-2.5, 7.666) -- (-1.5, 8) -- (-0.5, 9) -- (0.5, 9) -- (1.5, 10) -- (2.5, 10.333) -- (3.5, 10) -- (4.5, 9) -- (4.25, 7.5) -- (3.5, 7) -- (2, 4) -- (2, 2) -- (0.5, 1) -- (-0.5, 1);
            \draw (3.5, 7) -- (0.5, 9);
            \draw (-1, 6) -- (-0.5, 6);
            \draw (-0.5, 1) -- (-0.5, 3) -- (0, 4) -- (-1, 6) -- (-2.5, 7.666);
            \draw (2, 4) -- (1, 4.666) -- (-0.5, 6) -- (-1.5, 8);
            \draw (0, 4) -- (1, 4.666);

            \fill[black] (-0.5, 1)    circle (3pt);
            \fill[black] (-0.5, 3)    circle (3pt);
            \fill[black] (-2, 4)      circle (3pt);
            \fill[black] (-3.5, 7)    circle (3pt);
            \fill[black] (-3.5, 8)    circle (3pt);
            \fill[black] (-2.5, 7.666)circle (3pt);
            \fill[black] (-1.5, 8)    circle (3pt);
            \fill[black] (-0.5, 9)    circle (3pt);
            \fill[black] (0.5, 9)     circle (3pt);
            \fill[black] (1.5, 10)    circle (3pt);
            \fill[black] (2.5, 10.333)circle (3pt);
            \fill[black] (3.5, 10)    circle (3pt);
            \fill[black] (4.5, 9)     circle (3pt);
            \fill[black] (4.25, 7.5)  circle (3pt);
            \fill[black] (3.5, 7)     circle (3pt);
            \fill[black] (2, 4)       circle (3pt);
            \fill[black] (2, 2)       circle (3pt);
            \fill[black] (0.5, 1)     circle (3pt);
            \fill[black] (-1,6)       circle (3pt);
            \fill[black] (-0.5,6)     circle (3pt);
            \fill[black] (0,4)        circle (3pt);
            \fill[black] (1, 4.666)   circle (3pt);
            
          \end{scope}
          \begin{scope}[rotate=60,yscale=0.866]
            \draw[very thick] (-0.5, 1) -- (-0.5, 3) -- (-2, 4) -- (-3.5, 7) -- (-3.5, 8) -- (-2.5, 7.666) -- (-1.5, 8) -- (-0.5, 9) -- (0.5, 9) -- (1.5, 10) -- (2.5, 10.333) -- (3.5, 10) -- (4.5, 9) -- (4.25, 7.5) -- (3.5, 7) -- (2, 4) -- (2, 2) -- (0.5, 1) -- (-0.5, 1);
            \draw (3.5, 7) -- (0.5, 9);
            \draw (-1, 6) -- (-0.5, 6);
            \draw (-0.5, 1) -- (-0.5, 3) -- (0, 4) -- (-1, 6) -- (-2.5, 7.666);
            \draw (2, 4) -- (1, 4.666) -- (-0.5, 6) -- (-1.5, 8);
            \draw (0, 4) -- (1, 4.666);

            \fill[black] (-0.5, 1)    circle (3pt);
            \fill[black] (-0.5, 3)    circle (3pt);
            \fill[black] (-2, 4)      circle (3pt);
            \fill[black] (-3.5, 7)    circle (3pt);
            \fill[black] (-3.5, 8)    circle (3pt);
            \fill[black] (-2.5, 7.666)circle (3pt);
            \fill[black] (-1.5, 8)    circle (3pt);
            \fill[black] (-0.5, 9)    circle (3pt);
            \fill[black] (0.5, 9)     circle (3pt);
            \fill[black] (1.5, 10)    circle (3pt);
            \fill[black] (2.5, 10.333)circle (3pt);
            \fill[black] (3.5, 10)    circle (3pt);
            \fill[black] (4.5, 9)     circle (3pt);
            \fill[black] (4.25, 7.5)  circle (3pt);
            \fill[black] (3.5, 7)     circle (3pt);
            \fill[black] (2, 4)       circle (3pt);
            \fill[black] (2, 2)       circle (3pt);
            \fill[black] (0.5, 1)     circle (3pt);
            \fill[black] (-1,6)       circle (3pt);
            \fill[black] (-0.5,6)     circle (3pt);
            \fill[black] (0,4)        circle (3pt);
            \fill[black] (1, 4.666)   circle (3pt);
          \end{scope}
          \begin{scope}[yscale=0.866,shift={(0 cm,18 cm)},rotate=180]
            \draw[very thick] (-0.5, 1) -- (-0.5, 3) -- (-2, 4) -- (-3.5, 7) -- (-3.5, 8) -- (-2.5, 7.666) -- (-1.5, 8) -- (-0.5, 9) -- (0.5, 9) -- (1.5, 10) -- (2.5, 10.333) -- (3.5, 10) -- (4.5, 9) -- (4.25, 7.5) -- (3.5, 7) -- (2, 4) -- (2, 2) -- (0.5, 1) -- (-0.5, 1);
            \draw (3.5, 7) -- (0.5, 9);
            \draw (-1, 6) -- (-0.5, 6);
            \draw (-0.5, 1) -- (-0.5, 3) -- (0, 4) -- (-1, 6) -- (-2.5, 7.666);
            \draw (2, 4) -- (1, 4.666) -- (-0.5, 6) -- (-1.5, 8);
            \draw (0, 4) -- (1, 4.666);

            \fill[black] (-0.5, 1)    circle (3pt);
            \fill[black] (-0.5, 3)    circle (3pt);
            \fill[black] (-2, 4)      circle (3pt);
            \fill[black] (-3.5, 7)    circle (3pt);
            \fill[black] (-3.5, 8)    circle (3pt);
            \fill[black] (-2.5, 7.666)circle (3pt);
            \fill[black] (-1.5, 8)    circle (3pt);
            \fill[black] (-0.5, 9)    circle (3pt);
            \fill[black] (0.5, 9)     circle (3pt);
            \fill[black] (1.5, 10)    circle (3pt);
            \fill[black] (2.5, 10.333)circle (3pt);
            \fill[black] (3.5, 10)    circle (3pt);
            \fill[black] (4.5, 9)     circle (3pt);
            \fill[black] (4.25, 7.5)  circle (3pt);
            \fill[black] (3.5, 7)     circle (3pt);
            \fill[black] (2, 4)       circle (3pt);
            \fill[black] (2, 2)       circle (3pt);
            \fill[black] (0.5, 1)     circle (3pt);
            \fill[black] (-1,6)       circle (3pt);
            \fill[black] (-0.5,6)     circle (3pt);
            \fill[black] (0,4)        circle (3pt);
            \fill[black] (1, 4.666)   circle (3pt);
          \end{scope}
          \begin{scope}[shift={(0 cm,15.588 cm)},rotate=240,yscale=0.866]
            \draw[very thick] (-0.5, 1) -- (-0.5, 3) -- (-2, 4) -- (-3.5, 7) -- (-3.5, 8) -- (-2.5, 7.666) -- (-1.5, 8) -- (-0.5, 9) -- (0.5, 9) -- (1.5, 10) -- (2.5, 10.333) -- (3.5, 10) -- (4.5, 9) -- (4.25, 7.5) -- (3.5, 7) -- (2, 4) -- (2, 2) -- (0.5, 1) -- (-0.5, 1);
            \draw (3.5, 7) -- (0.5, 9);
            \draw (-1, 6) -- (-0.5, 6);
            \draw (-0.5, 1) -- (-0.5, 3) -- (0, 4) -- (-1, 6) -- (-2.5, 7.666);
            \draw (2, 4) -- (1, 4.666) -- (-0.5, 6) -- (-1.5, 8);
            \draw (0, 4) -- (1, 4.666);

            \fill[black] (-0.5, 1)    circle (3pt);
            \fill[black] (-0.5, 3)    circle (3pt);
            \fill[black] (-2, 4)      circle (3pt);
            \fill[black] (-3.5, 7)    circle (3pt);
            \fill[black] (-3.5, 8)    circle (3pt);
            \fill[black] (-2.5, 7.666)circle (3pt);
            \fill[black] (-1.5, 8)    circle (3pt);
            \fill[black] (-0.5, 9)    circle (3pt);
            \fill[black] (0.5, 9)     circle (3pt);
            \fill[black] (1.5, 10)    circle (3pt);
            \fill[black] (2.5, 10.333)circle (3pt);
            \fill[black] (3.5, 10)    circle (3pt);
            \fill[black] (4.5, 9)     circle (3pt);
            \fill[black] (4.25, 7.5)  circle (3pt);
            \fill[black] (3.5, 7)     circle (3pt);
            \fill[black] (2, 4)       circle (3pt);
            \fill[black] (2, 2)       circle (3pt);
            \fill[black] (0.5, 1)     circle (3pt);
            \fill[black] (-1,6)       circle (3pt);
            \fill[black] (-0.5,6)     circle (3pt);
            \fill[black] (0,4)        circle (3pt);
            \fill[black] (1, 4.666)   circle (3pt);
          \end{scope}
        \end{scope}
      \end{tikzsubfigure}
    \end{tikzfigure2}
  \end{proof}
\end{lemma}
\clearpage
\begin{theorem}
  Let $p$ and $v$ be a pair of admissible sequences for an orientable closed $2$-manifold $S$. Then $(p, v)$ is $[(4k + 1) \times 4, 2 \times (4k+7)]$-$[3]$-realizable on $S$ for all $k \in \nats$.
  \begin{proof}
    An expansion $3$-patch $\mathcal{P}_N$ with outer tuple $o = (1, 1, 1, 2, 2, 2)$ is shown in \autoref{fig:expansion:patch:4:7:a} and a corresponding $o$-$6$-gonal $3$-patch $\mathcal{P}_F$ is shown in \autoref{fig:expansion:patch:4:7:b}, both consisting of only quadrangles and heptagons. By using \autoref{const:edge:replacement:4:1} and \autoref{const:edge:replacement:4:2} as indicated we get $3$-patches consisting of only quadrangles and $(4k+7)$-gons, $k \in \nats$. By \autoref{thm:expansion:patch:poly:4:k} there is an expansion $3$-patch $\mathcal{P}_P$ with the polyhedral property consisting also of only quadrangles and $(4k+7)$-gons, thus we can apply \autoref{thm:main:const}.
  \end{proof}
\end{theorem}
{\par\vspace*{\fill}}
\begin{tikzfigure2}
  \begin{tikzsubfigure}{\label{fig:expansion:patch:4:7:a}}{$\mathcal{P}_N$}{1.0}
    \begin{scope}[scale=1]
      \node (n1) at (1.5,3) {$7$};
      \node (n2) at (0.5,1.5){$7$};
      \node at (2.2,1.8) {$4$};
      
      \draw (0.5, 2.5) -- (-0.5, 2) -- (-1, 1) -- (-0.1, 0) -- (1.1, 0) -- (2, 1) -- (1.5, 2);
      \draw[ldiamond] (0.5, 2.5) -- (1.5, 2);
      \draw (0.5, 2.5) -- (0, 3.5) -- (0.9, 4.5) -- (2.1, 4.5) -- (3, 3.5) -- (2.5, 2.5) -- (1.5, 2);
      \draw (2, 1) -- (3, 1.5) -- (2.5, 2.5);
      \node[anchor=90] at (-0.1, 0) {$i_0$};
      \node[anchor=90] at (1.2, 0) {$i'_0$};
      \node[anchor=120] at (2, 1) {$i'_1$};
      \node[anchor=180] at (3, 1.5) {$i'_2 = o_7$};
      \node[anchor=180] at (2.5, 2.5) {$o_6$};
      \node[anchor=180] at (3, 3.5) {$o_5$};
      \node[anchor=-120] at (2.1, 4.5) {$o_4$};
      \node[anchor=-60] at (0.9, 4.5) {$o_3$};
      \node[anchor=0] at (0, 3.5) {$\bm{o_s}$};
      \node[anchor=-20] at (0.5, 2.5) {$o_{1}$};
      \node[anchor=0] at (-0.5, 2) {$i_2 = o_{0}$};
      \node[anchor=45] at (-1, 1) {$i_1$};
      \draw[lface] (1, 2.25) -- (n1);
      \draw[lface] (1, 2.25) -- (n2);

      \fill[black] (-0.1, 0)  circle(2pt);
      \fill[black] (1.2, 0)   circle(2pt);
      \fill[black] (2, 1)     circle(2pt);
      \fill[black] (3, 1.5)   circle(2pt);
      \fill[black] (2.5, 2.5) circle(2pt);
      \fill[black] (3, 3.5)   circle(2pt);
      \fill[black] (2.1, 4.5) circle(2pt);
      \fill[black] (0.9, 4.5) circle(2pt);
      \fill[black] (0, 3.5)   circle(2pt);
      \fill[black] (0.5, 2.5) circle(2pt);
      \fill[black] (-0.5, 2)  circle(2pt);
      \fill[black] (-1, 1)    circle(2pt);
      \fill[black] (1.5,2)    circle(2pt);
      
    \end{scope}
  \end{tikzsubfigure}
  \begin{tikzsubfigure}{\label{fig:expansion:patch:4:7:b}}{$\mathcal{P}_F$}{1.0}
    \begin{scope}[scale=5]
      % \node (n0) at (0.036639, 0.243081) {0};
% \node (n1) at (-0.674601, -0.379082) {1};
% \node (n2) at (-0.612496, -0.412918) {2};
% \node (n3) at (-0.654897, -0.322729) {3};
% \node (n4) at (-0.805662, -0.206688) {4};
% \node (n5) at (-0.637394, -0.555550) {5};
% \node (n6) at (-0.542496, -0.355371) {6};
% \node (n7) at (-0.257551, -0.369593) {7};
% \node (n8) at (-0.546626, -0.003502) {8};
% \node (n9) at (-0.255121, -0.709897) {9};
% \node (n10) at (0.187725, -0.519467) {10};
% \node (n11) at (0.202364, -0.877406) {11};
% \node (n12) at (0.134227, 0.196853) {12};
% \node (n13) at (0.368263, 0.540115) {13};
% \node (n14) at (0.608390, 0.659794) {14};
% \node (n15) at (0.391955, 0.807363) {15};
% \node (n16) at (0.079413, 0.947736) {16};
% \node (n17) at (-0.753667, 0.512095) {17};
% \node (n18) at (-0.866518, -0.417367) {18};
% \node (n19) at (-0.069790, -0.955266) {19};
% \node (n20) at (0.752043, -0.514482) {20};
% \node (n21) at (0.853213, 0.420164) {21};

\node (n1) at (-0.674601, -0.379082) {4};
\node (n2) at (-0.612496, -0.412918) {4};
\node (n3) at (-0.654897, -0.322729) {4};
\node (n4) at (-0.805662, -0.206688) {7};
\node (n5) at (-0.637394, -0.555550) {7};
\node (n6) at (-0.542496, -0.355371) {7};
\node (n7) at (-0.257551, -0.369593) {4};
\node (n8) at (-0.546626, -0.003502) {7};
\node (n9) at (-0.255121, -0.709897) {7};
\node (n10) at (0.187725, -0.519467) {7};
\node (n11) at (0.202364, -0.877406) {4};
\node (n12) at (0.134227, 0.196853) {7};
\node (n13) at (0.368263, 0.540115) {7};
\node (n14) at (0.608390, 0.659794) {4};
\node (n15) at (0.391955, 0.807363) {4};
\node (n16) at (0.079413, 0.947736) {7};
\node (n17) at (-0.753667, 0.512095) {7};
\node (n18) at (-0.866518, -0.417367) {7};
\node (n19) at (-0.069790, -0.955266) {7};
\node (n20) at (0.752043, -0.514482) {7};
\node (n21) at (0.853213, 0.420164) {7};

\fill[black] (-0.636833, -0.430484) circle (0.3pt);
\fill[black] (-0.648324, -0.371449) circle (0.3pt);
\fill[black] (-0.685325, -0.322172) circle (0.3pt);
\fill[black] (-0.727924, -0.392224) circle (0.3pt);
\fill[black] (-0.552087, -0.487859) circle (0.3pt);
\fill[black] (-0.612740, -0.361879) circle (0.3pt);
\fill[black] (-0.673200, -0.235415) circle (0.3pt);
\fill[black] (-0.724186, -0.094568) circle (0.3pt);
\fill[black] (-0.997204, 0.074730) circle (0.3pt);
\fill[black] (-0.997204, -0.074730) circle (0.3pt);
\fill[black] (-0.834593, -0.402435) circle (0.3pt);
\fill[black] (-0.680173, -0.733052) circle (0.3pt);
\fill[black] (-0.563320, -0.826239) circle (0.3pt);
\fill[black] (-0.466831, -0.616557) circle (0.3pt);
\fill[black] (-0.320629, -0.450392) circle (0.3pt);
\fill[black] (-0.447803, -0.240925) circle (0.3pt);
\fill[black] (-0.082005, -0.522040) circle (0.3pt);
\fill[black] (-0.179768, -0.265015) circle (0.3pt);
\fill[black] (-0.023974, -0.035278) circle (0.3pt);
\fill[black] (-0.478520, 0.314020) circle (0.3pt);
\fill[black] (-0.974928, 0.222521) circle (0.3pt);
\fill[black] (-0.433884, -0.900969) circle (0.3pt);
\fill[black] (-0.054646, -0.897083) circle (0.3pt);
\fill[black] (0.135465, -0.756001) circle (0.3pt);
\fill[black] (0.433884, -0.900969) circle (0.3pt);
\fill[black] (0.563320, -0.826239) circle (0.3pt);
\fill[black] (0.467152, -0.330729) circle (0.3pt);
\fill[black] (0.294755, -0.955573) circle (0.3pt);
\fill[black] (0.974928, -0.222521) circle (0.3pt);
\fill[black] (0.997204, -0.074730) circle (0.3pt);
\fill[black] (-0.433884, 0.900969) circle (0.3pt);
\fill[black] (-0.563320, 0.826239) circle (0.3pt);
\fill[black] (0.997204, 0.074730) circle (0.3pt);
\fill[black] (0.741454, 0.422017) circle (0.3pt);
\fill[black] (0.448614, 0.657866) circle (0.3pt);
\fill[black] (0.122004, 0.844377) circle (0.3pt);
\fill[black] (-0.294755, 0.955573) circle (0.3pt);
\fill[black] (0.680173, 0.733052) circle (0.3pt);
\fill[black] (0.563320, 0.826239) circle (0.3pt);
\fill[black] (0.433884, 0.900969) circle (0.3pt);
\fill[black] (0.294755, 0.955573) circle (0.3pt);
\fill[black] (0.149042, 0.988831) circle (0.3pt);
\fill[black] (0.000000, 1.000000) circle (0.3pt);
\fill[black] (-0.149042, 0.988831) circle (0.3pt);
\fill[black] (-0.680173, 0.733052) circle (0.3pt);
\fill[black] (-0.781831, 0.623490) circle (0.3pt);
\fill[black] (-0.866025, 0.500000) circle (0.3pt);
\fill[black] (-0.930874, 0.365341) circle (0.3pt);
\fill[black] (-0.974928, -0.222521) circle (0.3pt);
\fill[black] (-0.930874, -0.365341) circle (0.3pt);
\fill[black] (-0.866025, -0.500000) circle (0.3pt);
\fill[black] (-0.781831, -0.623490) circle (0.3pt);
\fill[black] (-0.294755, -0.955573) circle (0.3pt);
\fill[black] (-0.149042, -0.988831) circle (0.3pt);
\fill[black] (-0.000000, -1.000000) circle (0.3pt);
\fill[black] (0.149042, -0.988831) circle (0.3pt);
\fill[black] (0.680173, -0.733052) circle (0.3pt);
\fill[black] (0.781831, -0.623490) circle (0.3pt);
\fill[black] (0.866025, -0.500000) circle (0.3pt);
\fill[black] (0.930874, -0.365341) circle (0.3pt);
\fill[black] (0.974928, 0.222521) circle (0.3pt);
\fill[black] (0.930874, 0.365341) circle (0.3pt);
\fill[black] (0.866025, 0.500000) circle (0.3pt);
\fill[black] (0.781831, 0.623490) circle (0.3pt);

\coordinate (x0) at (-0.636833, -0.430484);
\coordinate (x1) at (-0.648324, -0.371449);
\coordinate (x2) at (-0.685325, -0.322172);
\coordinate (x3) at (-0.727924, -0.392224);
\coordinate (x4) at (-0.552087, -0.487859);
\coordinate (x5) at (-0.612740, -0.361879);
\coordinate (x6) at (-0.673200, -0.235415);
\coordinate (x7) at (-0.724186, -0.094568);
\coordinate (x8) at (-0.997204, 0.074730);
\coordinate (x9) at (-0.997204, -0.074730);
\coordinate (x10) at (-0.834593, -0.402435);
\coordinate (x11) at (-0.680173, -0.733052);
\coordinate (x12) at (-0.563320, -0.826239);
\coordinate (x13) at (-0.466831, -0.616557);
\coordinate (x14) at (-0.320629, -0.450392);
\coordinate (x15) at (-0.447803, -0.240925);
\coordinate (x16) at (-0.082005, -0.522040);
\coordinate (x17) at (-0.179768, -0.265015);
\coordinate (x18) at (-0.023974, -0.035278);
\coordinate (x19) at (-0.478520, 0.314020);
\coordinate (x20) at (-0.974928, 0.222521);
\coordinate (x21) at (-0.433884, -0.900969);
\coordinate (x22) at (-0.054646, -0.897083);
\coordinate (x23) at (0.135465, -0.756001);
\coordinate (x24) at (0.433884, -0.900969);
\coordinate (x25) at (0.563320, -0.826239);
\coordinate (x26) at (0.467152, -0.330729);
\coordinate (x27) at (0.294755, -0.955573);
\coordinate (x28) at (0.974928, -0.222521);
\coordinate (x29) at (0.997204, -0.074730);
\coordinate (x30) at (-0.433884, 0.900969);
\coordinate (x31) at (-0.563320, 0.826239);
\coordinate (x32) at (0.997204, 0.074730);
\coordinate (x33) at (0.741454, 0.422017);
\coordinate (x34) at (0.448614, 0.657866);
\coordinate (x35) at (0.122004, 0.844377);
\coordinate (x36) at (-0.294755, 0.955573);
\coordinate (x37) at (0.680173, 0.733052);
\coordinate (x38) at (0.563320, 0.826239);
\coordinate (x39) at (0.433884, 0.900969);
\coordinate (x40) at (0.294755, 0.955573);
\coordinate (x41) at (0.149042, 0.988831);
\coordinate (x42) at (0.000000, 1.000000);
\coordinate (x43) at (-0.149042, 0.988831);
\coordinate (x44) at (-0.680173, 0.733052);
\coordinate (x45) at (-0.781831, 0.623490);
\coordinate (x46) at (-0.866025, 0.500000);
\coordinate (x47) at (-0.930874, 0.365341);
\coordinate (x48) at (-0.974928, -0.222521);
\coordinate (x49) at (-0.930874, -0.365341);
\coordinate (x50) at (-0.866025, -0.500000);
\coordinate (x51) at (-0.781831, -0.623490);
\coordinate (x52) at (-0.294755, -0.955573);
\coordinate (x53) at (-0.149042, -0.988831);
\coordinate (x54) at (-0.000000, -1.000000);
\coordinate (x55) at (0.149042, -0.988831);
\coordinate (x56) at (0.680173, -0.733052);
\coordinate (x57) at (0.781831, -0.623490);
\coordinate (x58) at (0.866025, -0.500000);
\coordinate (x59) at (0.930874, -0.365341);
\coordinate (x60) at (0.974928, 0.222521);
\coordinate (x61) at (0.930874, 0.365341);
\coordinate (x62) at (0.866025, 0.500000);
\coordinate (x63) at (0.781831, 0.623490);

\draw (x0) -- (x1);
\draw (x1) -- (x2);
\draw (x2) -- (x3);
\draw (x3) -- (x0);
\draw (x0) -- (x4);
\draw (x4) -- (x5);
\draw (x5) -- (x1);
\draw (x5) -- (x6);
\draw (x6) -- (x2);
\draw (x6) -- (x7);
\draw (x7) -- (x8);
\draw (x8) -- (x9);
\draw[ldiamond] (x9) -- (x10) node[midway] (m1) {};
\draw[lface] (m1) -- (n4);
\draw[lface] (m1) -- (n18);
\draw (x10) -- (x3);
\draw (x10) -- (x11);
\draw (x11) -- (x12);
\draw (x12) -- (x13);
\draw[ldiamond] (x13) -- (x4) node[midway] (m2) {};
\draw[lface] (m2) -- (n5);
\draw[lface] (m2) -- (n6);
\draw (x13) -- (x14);
\draw (x14) -- (x15);
\draw (x15) -- (x7);
\draw (x14) -- (x16);
\draw (x16) -- (x17);
\draw (x17) -- (x15);
\draw (x17) -- (x18);
\draw (x18) -- (x19);
\draw[ldiamond] (x19) -- (x20) node[midway] (m3) {};
\draw[lface] (m3) -- (n8);
\draw[lface] (m3) -- (n17);
\draw (x20) -- (x8);
\draw (x12) -- (x21);
\draw[ldiamond] (x21) -- (x22) node[midway] (m4) {};
\draw[lface] (m4) -- (n9);
\draw[lface] (m4) -- (n19);
\draw (x22) -- (x23);
\draw (x23) -- (x16);
\draw (x23) -- (x24);
\draw (x24) -- (x25);
\draw[ldiamond] (x25) -- (x26) node[midway] (m5) {};
\draw[lface] (m5) -- (n10);
\draw[lface] (m5) -- (n20);
\draw (x26) -- (x18);
\draw (x22) -- (x27);
\draw (x27) -- (x24);
\draw (x26) -- (x28);
\draw (x28) -- (x29);
\draw (x29) -- (x30);
\draw (x30) -- (x31);
\draw (x31) -- (x19);
\draw[lsquare] (x29) -- (x32);
\draw[lface] (x29) -- (n12);
\draw[lface] (x32) -- (n21);
\draw (x32) -- (x33);
\draw (x33) -- (x34);
\draw (x34) -- (x35);
\draw[ldiamond] (x35) -- (x36) node[midway] (m6) {};
\draw[lface] (m6) -- (n13);
\draw[lface] (m6) -- (n16);
\draw (x36) -- (x30);
\draw (x33) -- (x37);
\draw (x37) -- (x38);
\draw (x38) -- (x34);
\draw (x38) -- (x39);
\draw (x39) -- (x35);
\draw (x39) -- (x40);
\draw (x40) -- (x41);
\draw (x41) -- (x42);
\draw (x42) -- (x43);
\draw (x43) -- (x36);
\draw (x31) -- (x44);
\draw (x44) -- (x45);
\draw (x45) -- (x46);
\draw (x46) -- (x47);
\draw (x47) -- (x20);
\draw (x9) -- (x48);
\draw (x48) -- (x49);
\draw (x49) -- (x50);
\draw (x50) -- (x51);
\draw (x51) -- (x11);
\draw (x21) -- (x52);
\draw (x52) -- (x53);
\draw (x53) -- (x54);
\draw (x54) -- (x55);
\draw (x55) -- (x27);
\draw (x25) -- (x56);
\draw (x56) -- (x57);
\draw (x57) -- (x58);
\draw (x58) -- (x59);
\draw (x59) -- (x28);
\draw (x32) -- (x60);
\draw (x60) -- (x61);
\draw (x61) -- (x62);
\draw (x62) -- (x63);
\draw (x63) -- (x37);


    \end{scope}
  \end{tikzsubfigure}
\end{tikzfigure2}
\clearpage
\begin{theorem}
  Let $p$ and $v$ be a pair of admissible sequences for an orientable closed $2$-manifold $S$. Then $(p, v)$ is $[(4k + 3) \times 4, 2 \times (4k+9)]$-$[3]$-realizable on $S$ for all $k \in \nats$.
  \begin{proof}
    An expansion $3$-patch $\mathcal{P}_N$ with outer tuple $o = (2, 2, 2, 2, 1, 2, 1, 1, 1, 1)$ is shown in \autoref{fig:expansion:patch:4:9:a} and a corresponding $o$-$6$-gonal $3$-patch $\mathcal{P}_F$ is shown in \autoref{fig:expansion:patch:4:9:b}, both consisting of only quadrangles and enneagons. By using \autoref{const:edge:replacement:4:1} and \autoref{const:edge:replacement:4:2} as indicated we get $3$-patches consisting of only quadrangles and $(4k+9)$-gons, $k \in \nats$. By \autoref{thm:expansion:patch:poly:4:k} there is a expansion $3$-patch $\mathcal{P}_P$ with the polyhedral property consisting also of only quadrangles and $(4k+9)$-gons, thus we can apply \autoref{thm:main:const}.
  \end{proof}
\end{theorem}
{\par\vspace*{\fill}}
\begin{tikzfigure2}
  \begin{tikzsubfigure}{\label{fig:expansion:patch:4:9:a}}{$\mathcal{P}_N$}{1.0}
    \begin{scope}[scale=0.7, xscale=-1]
      \node (n1) at (-0.5,4.2) {$9$};
      \node (n2) at (0.8,2.2)  {$9$};
      \node at (1.25,3.75)   {$4$};
      \node at (2,3)   {$4$};
      \node at (-1.25,2.75)   {$4$};
      
      \draw (-0.5, 3) -- (-1, 2) -- (-1, 1) -- (0, 0) -- (1, 0) -- (2, 1) -- (2, 2) -- (1.5, 3) -- (0.5, 3.5);
      \draw[ldiamond] (0.5, 3.5) -- (-0.5, 3);
      \draw (1,4.5) -- (2,4) -- (2.5, 3)--(2,2);
      \draw (2,4) -- (1.5, 3);
      \draw (-1,2) -- (-2,2.5) -- (-1.5, 3.5);
     % \draw (3, 2) -- (2.5, 3) -- (1.5, 3);
     % \draw (2.5, 3) -- (1, 4.5);
      \draw (0.5, 3.5) -- (1, 4.5) -- (1, 5.5) -- (0, 6.5) -- (-1, 6.5) -- (-2, 5.5) -- (-2, 4.5) -- (-1.5, 3.5) -- (-0.5, 3);

      \node[anchor= 90] at (0, 0) {$i_1'$};
      \node[anchor= 90] at (1, 0) {$i_0'$};
      \node[anchor= 60] at (2, 1) {$i_0$};
      \node[anchor=  0] at (2,2) {$i_1$};
      \node[anchor=300] at (2,4) {$o_{1}$};
      \node[anchor=  0] at (2.5, 3) {$i_{2}'=o_{0}$};
      \node[anchor=320] at (1, 4.5) {$o_2$};
      \node[anchor=300] at (1, 5.5) {$\bm{o_s}$};
      \node[anchor=270] at (0, 6.5) {$o_4$};
      \node[anchor=270] at (-1, 6.5) {$o_5$};
      \node[anchor=200] at (-2, 5.5) {$o_6$};
      \node[anchor=180] at (-2, 4.5) {$o_7$};
      \node[anchor=180] at (-1.5, 3.5) {$o_8$};
      \node[anchor=100] at (-2,2.5) {$o_{9}$};
      \node[anchor=150] at (-1, 2) {$o_{10}$};
      \node[anchor=150] at (-1, 1) {$i_2'=o_{11}$};
      \draw[lface] (0, 3.25) -- (n1);
      \draw[lface] (0, 3.25) -- (n2);


      \fill[black]  (0, 0) circle (2pt);
      \fill[black]  (1, 0) circle (2pt);
      \fill[black]  (2, 1) circle (2pt);
      \fill[black]  (2,2) circle (2pt);
      \fill[black]  (2,4) circle (2pt);
      \fill[black]  (2.5, 3) circle (2pt);
      \fill[black]  (1, 4.5) circle (2pt);
      \fill[black]  (1, 5.5) circle (2pt);
      \fill[black]  (0, 6.5) circle (2pt);
      \fill[black]  (-1, 6.5) circle (2pt);
      \fill[black]  (-2, 5.5) circle (2pt);
      \fill[black]  (-2, 4.5) circle (2pt);
      \fill[black]  (-1.5, 3.5) circle (2pt);
      \fill[black]  (-2,2.5) circle (2pt);
      \fill[black]  (-1, 2) circle (2pt);
      \fill[black]  (-1, 1) circle (2pt);
      \fill[black]  (0.5,3.5) circle (2pt);
      \fill[black]  (-0.5, 3) circle (2pt);
      \fill[black]  (1.5, 3) circle (2pt);

      % \node (n1) at (-0.5,4.2) {$9$};
      % \node (n2) at (0.8,2.2)  {$9$};
      % \node at (1.1,3.8)   {$4$};
      % \node at (2.2,2.6)   {$4$};
      % \node at (2.4,1.6)   {$4$};
      
      % \draw (-0.5, 3) -- (-1, 2) -- (-1, 1) -- (0, 0) -- (1, 0) -- (2, 1) -- (2, 2) -- (1.5, 3) -- (0.5, 3.5);
      % \draw[ldiamond] (0.5, 3.5) -- (-0.5, 3);
      % \draw (2, 1) -- (3, 1) -- (3, 2) -- (2, 2);
      % \draw (3, 2) -- (2.5, 3) -- (1.5, 3);
      % \draw (2.5, 3) -- (1, 4.5);
      % \draw (0.5, 3.5) -- (1, 4.5) -- (1, 5.5) -- (0, 6.5) -- (-1, 6.5) -- (-2, 5.5) -- (-2, 4.5) -- (-1.5, 3.5) -- (-0.5, 3);
      % \node[anchor=90] at (0, 0) {$i_0$};
      % \node[anchor=90] at (1, 0) {$i'_0$};
      % \node[anchor=120] at (2, 1) {$i'_1$};
      % \node[anchor=180] at (3, 1) {$i'_2 = o_{11}$};
      % \node[anchor=180] at (3, 2) {$o_{10}$};
      % \node[anchor=-150] at (2.5, 3) {$o_9$};
      % \node[anchor=-150] at (1, 4.5) {$o_8$};
      % \node[anchor=-150] at (1, 5.5) {$o_7$};
      % \node[anchor=-110] at (0, 6.5) {$o_6$};
      % \node[anchor=-70] at (-1, 6.5) {$o_5$};
      % \node[anchor=-20] at (-2, 5.5) {$o_4$};
      % \node[anchor=0] at (-2, 4.5) {$o_3$};
      % \node[anchor=60] at (-1.5, 3.5) {$\bm{o_s}$};
      % \node[anchor=20] at (-0.5, 3) {$o_{1}$};
      % \node[anchor=0] at (-1, 2) {$i_2 = o_{0}$};
      % \node[anchor=45] at (-1, 1) {$i_1$};
      % \draw[lface] (0, 3.25) -- (n1);
      % \draw[lface] (0, 3.25) -- (n2);


      % \fill[black] (0, 0)   circle (2pt);
      % \fill[black] (1, 0)   circle (2pt);
      % \fill[black] (2, 1) circle (2pt);
      % \fill[black] (3, 1) circle (2pt);
      % \fill[black] (3, 2) circle (2pt);
      % \fill[black] (2.5, 3) circle (2pt);
      % \fill[black] (1, 4.5) circle (2pt);
      % \fill[black] (1, 5.5) circle (2pt);
      % \fill[black] (0, 6.5) circle (2pt);
      % \fill[black] (-1, 6.5) circle (2pt);
      % \fill[black] (-2, 5.5) circle (2pt);
      % \fill[black] (-2, 4.5) circle (2pt);
      % \fill[black] (-1.5, 3.5) circle (2pt);
      % \fill[black] (-0.5, 3) circle (2pt);
      % \fill[black] (-1, 2) circle (2pt);
      % \fill[black] (-1, 1) circle (2pt);
      % \fill[black] (0.5,3.5) circle (2pt);
      % \fill[black] (2,2)   circle (2pt);
      % \fill[black] (1.5,3) circle (2pt);
      
    \end{scope}
  \end{tikzsubfigure}
  \begin{tikzsubfigure}{\label{fig:expansion:patch:4:9:b}}{$\mathcal{P}_F$}{1.0}
    \begin{scope}[scale=5]
      % \node (n0) at (0.046353, 0.223914) {0};
% \node (n1) at (0.105373, -0.516742) {1};
% \node (n2) at (0.233589, -0.585638) {2};
% \node (n3) at (0.279100, -0.447089) {3};
% \node (n4) at (0.104419, -0.261603) {4};
% \node (n5) at (-0.000193, -0.000318) {5};
% \node (n6) at (-0.229106, -0.091916) {6};
% \node (n7) at (0.228736, 0.091350) {7};
% \node (n8) at (-0.104822, 0.261129) {8};
% \node (n9) at (-0.280129, 0.446457) {9};
% \node (n10) at (-0.106378, 0.515909) {10};
% \node (n11) at (-0.234887, 0.584836) {11};
% \node (n12) at (-0.574528, 0.494873) {12};
% \node (n13) at (-0.648331, -0.259117) {13};
% \node (n14) at (-0.075205, -0.754579) {14};
% \node (n15) at (0.574179, -0.495066) {15};
% \node (n16) at (0.648196, 0.258946) {16};
% \node (n17) at (0.074870, 0.754333) {17};
% \node[anchor= 90] (n18) at (0.357068, 0.882723) {18};
% \node[anchor=240] (n19) at (0.351837, 0.919028) {19};
% \node[anchor=150] (n20) at (-0.585927, 0.750591) {20};
% \node[anchor=300] (n21) at (-0.619983, 0.764214) {21};
% \node[anchor=210] (n22) at (-0.943005, -0.132105) {22};
% \node[anchor=  0] (n23) at (-0.971823, -0.154806) {23};
% \node[anchor=270] (n24) at (-0.357096, -0.882719) {24};
% \node[anchor= 60] (n25) at (-0.351846, -0.919026) {25};
% \node[anchor=330] (n26) at (0.585927, -0.750591) {26};
% \node[anchor=120] (n27) at (0.619983, -0.764214) {27};
% \node[anchor= 30] (n28) at (0.942994, 0.132133) {28};
% \node[anchor=180] (n29) at (0.971820, 0.154815) {29};
% \node (n30) at (0.858568, 0.440603) {30};
% \node (n31) at (0.047711, 0.963844) {31};
% \node (n32) at (-0.810857, 0.523241) {32};
% \node (n33) at (-0.858573, -0.440590) {33};
% \node (n34) at (-0.047724, -0.963841) {34};
% \node (n35) at (0.810857, -0.523241) {35};

\node (n1) at (0.105373, -0.516742) {4};
\node (n2) at (0.233589, -0.585638) {4};
\node (n3) at (0.279100, -0.447089) {4};
\node (n4) at (0.104419, -0.261603) {9};
\node (n5) at (-0.000193, -0.000318) {4};
\node (n6) at (-0.229106, -0.091916) {4};
\node (n7) at (0.228736, 0.091350) {4};
\node (n8) at (-0.104822, 0.261129) {9};
\node (n9) at (-0.280129, 0.446457) {4};
\node (n10) at (-0.106378, 0.515909) {4};
\node (n11) at (-0.234887, 0.584836) {4};
\node (n12) at (-0.574528, 0.494873) {9};
\node (n13) at (-0.648331, -0.259117) {9};
\node (n14) at (-0.075205, -0.754579) {9};
\node (n15) at (0.574179, -0.495066) {9};
\node (n16) at (0.648196, 0.258946) {9};
\node (n17) at (0.074870, 0.754333) {9};
\node[anchor= 90] (n18) at (0.357068, 0.882723) {4};
\node[anchor=240] (n19) at (0.351837, 0.919028) {4};
\node[anchor=150] (n20) at (-0.585927, 0.750591) {4};
\node[anchor=300] (n21) at (-0.619983, 0.764214) {4};
\node[anchor=210] (n22) at (-0.943005, -0.132105) {4};
\node[anchor=  0] (n23) at (-0.971823, -0.154806) {4};
\node[anchor=270] (n24) at (-0.357096, -0.882719) {4};
\node[anchor= 60] (n25) at (-0.351846, -0.919026) {4};
\node[anchor=330] (n26) at (0.585927, -0.750591) {4};
\node[anchor=120] (n27) at (0.619983, -0.764214) {4};
\node[anchor= 30] (n28) at (0.942994, 0.132133) {4};
\node[anchor=180] (n29) at (0.971820, 0.154815) {4};
\node (n30) at (0.858568, 0.440603) {9};
\node (n31) at (0.047711, 0.963844) {9};
\node (n32) at (-0.810857, 0.523241) {9};
\node (n33) at (-0.858573, -0.440590) {9};
\node (n34) at (-0.047724, -0.963841) {9};
\node (n35) at (0.810857, -0.523241) {9};

\fill[black] (0.115923, -0.598153) circle (0.3pt);
\fill[black] (0.207990, -0.522493) circle (0.3pt);
\fill[black] (0.171064, -0.428303) circle (0.3pt);
\fill[black] (-0.073483, -0.518020) circle (0.3pt);
\fill[black] (0.283060, -0.708777) circle (0.3pt);
\fill[black] (0.327382, -0.513128) circle (0.3pt);
\fill[black] (0.409965, -0.324431) circle (0.3pt);
\fill[black] (0.641900, -0.147518) circle (0.3pt);
\fill[black] (0.391100, 0.014595) circle (0.3pt);
\fill[black] (0.143454, -0.024633) circle (0.3pt);
\fill[black] (-0.087375, -0.116976) circle (0.3pt);
\fill[black] (-0.293670, -0.259430) circle (0.3pt);
\fill[black] (-0.363182, -0.549714) circle (0.3pt);
\fill[black] (0.087018, 0.116378) circle (0.3pt);
\fill[black] (-0.143867, 0.023957) circle (0.3pt);
\fill[black] (-0.391513, -0.015213) circle (0.3pt);
\fill[black] (0.293373, 0.259058) circle (0.3pt);
\fill[black] (0.363000, 0.549564) circle (0.3pt);
\fill[black] (0.072947, 0.517548) circle (0.3pt);
\fill[black] (-0.171622, 0.427790) circle (0.3pt);
\fill[black] (-0.410541, 0.324026) circle (0.3pt);
\fill[black] (-0.642196, 0.147049) circle (0.3pt);
\fill[black] (-0.209734, 0.521265) circle (0.3pt);
\fill[black] (-0.328619, 0.512747) circle (0.3pt);
\fill[black] (-0.117104, 0.597035) circle (0.3pt);
\fill[black] (-0.284091, 0.708296) circle (0.3pt);
\fill[black] (-0.371662, 0.928368) circle (0.3pt);
\fill[black] (-0.458227, 0.888835) circle (0.3pt);
\fill[black] (-0.713787, 0.566461) circle (0.3pt);
\fill[black] (-0.971812, 0.235759) circle (0.3pt);
\fill[black] (-0.989821, 0.142315) circle (0.3pt);
\fill[black] (-0.998867, 0.047582) circle (0.3pt);
\fill[black] (-0.847493, -0.334854) circle (0.3pt);
\fill[black] (-0.690079, -0.723734) circle (0.3pt);
\fill[black] (-0.618159, -0.786053) circle (0.3pt);
\fill[black] (-0.540641, -0.841254) circle (0.3pt);
\fill[black] (-0.133754, -0.901378) circle (0.3pt);
\fill[black] (0.281733, -0.959493) circle (0.3pt);
\fill[black] (0.371662, -0.928368) circle (0.3pt);
\fill[black] (0.458227, -0.888835) circle (0.3pt);
\fill[black] (0.713787, -0.566461) circle (0.3pt);
\fill[black] (0.971812, -0.235759) circle (0.3pt);
\fill[black] (0.989821, -0.142315) circle (0.3pt);
\fill[black] (0.998867, -0.047582) circle (0.3pt);
\fill[black] (0.847463, 0.334927) circle (0.3pt);
\fill[black] (0.690079, 0.723734) circle (0.3pt);
\fill[black] (0.618159, 0.786053) circle (0.3pt);
\fill[black] (0.540641, 0.841254) circle (0.3pt);
\fill[black] (0.133676, 0.901388) circle (0.3pt);
\fill[black] (-0.281733, 0.959493) circle (0.3pt);
\fill[black] (0.458227, 0.888835) circle (0.3pt);
\fill[black] (0.295727, 0.899417) circle (0.3pt);
\fill[black] (0.371662, 0.928368) circle (0.3pt);
\fill[black] (0.281733, 0.959493) circle (0.3pt);
\fill[black] (-0.540641, 0.841254) circle (0.3pt);
\fill[black] (-0.631054, 0.705815) circle (0.3pt);
\fill[black] (-0.618159, 0.786053) circle (0.3pt);
\fill[black] (-0.690079, 0.723734) circle (0.3pt);
\fill[black] (-0.998867, -0.047582) circle (0.3pt);
\fill[black] (-0.926791, -0.193567) circle (0.3pt);
\fill[black] (-0.989821, -0.142315) circle (0.3pt);
\fill[black] (-0.971812, -0.235759) circle (0.3pt);
\fill[black] (-0.458227, -0.888835) circle (0.3pt);
\fill[black] (-0.295763, -0.899408) circle (0.3pt);
\fill[black] (-0.371662, -0.928368) circle (0.3pt);
\fill[black] (-0.281733, -0.959493) circle (0.3pt);
\fill[black] (0.540641, -0.841254) circle (0.3pt);
\fill[black] (0.631054, -0.705815) circle (0.3pt);
\fill[black] (0.618159, -0.786053) circle (0.3pt);
\fill[black] (0.690079, -0.723734) circle (0.3pt);
\fill[black] (0.998867, 0.047582) circle (0.3pt);
\fill[black] (0.926779, 0.193606) circle (0.3pt);
\fill[black] (0.989821, 0.142315) circle (0.3pt);
\fill[black] (0.971812, 0.235759) circle (0.3pt);
\fill[black] (0.945001, 0.327068) circle (0.3pt);
\fill[black] (0.909632, 0.415415) circle (0.3pt);
\fill[black] (0.866025, 0.500000) circle (0.3pt);
\fill[black] (0.814576, 0.580057) circle (0.3pt);
\fill[black] (0.755750, 0.654861) circle (0.3pt);
\fill[black] (0.189251, 0.981929) circle (0.3pt);
\fill[black] (0.095056, 0.995472) circle (0.3pt);
\fill[black] (0.000000, 1.000000) circle (0.3pt);
\fill[black] (-0.095056, 0.995472) circle (0.3pt);
\fill[black] (-0.189251, 0.981929) circle (0.3pt);
\fill[black] (-0.755750, 0.654861) circle (0.3pt);
\fill[black] (-0.814576, 0.580057) circle (0.3pt);
\fill[black] (-0.866025, 0.500000) circle (0.3pt);
\fill[black] (-0.909632, 0.415415) circle (0.3pt);
\fill[black] (-0.945001, 0.327068) circle (0.3pt);
\fill[black] (-0.945001, -0.327068) circle (0.3pt);
\fill[black] (-0.909632, -0.415415) circle (0.3pt);
\fill[black] (-0.866025, -0.500000) circle (0.3pt);
\fill[black] (-0.814576, -0.580057) circle (0.3pt);
\fill[black] (-0.755750, -0.654861) circle (0.3pt);
\fill[black] (-0.189251, -0.981929) circle (0.3pt);
\fill[black] (-0.095056, -0.995472) circle (0.3pt);
\fill[black] (-0.000000, -1.000000) circle (0.3pt);
\fill[black] (0.095056, -0.995472) circle (0.3pt);
\fill[black] (0.189251, -0.981929) circle (0.3pt);
\fill[black] (0.755750, -0.654861) circle (0.3pt);
\fill[black] (0.814576, -0.580057) circle (0.3pt);
\fill[black] (0.866025, -0.500000) circle (0.3pt);
\fill[black] (0.909632, -0.415415) circle (0.3pt);
\fill[black] (0.945001, -0.327068) circle (0.3pt);


\coordinate (x0) at (0.115923, -0.598153);
\coordinate (x1) at (0.207990, -0.522493);
\coordinate (x2) at (0.171064, -0.428303);
\coordinate (x3) at (-0.073483, -0.518020);
\coordinate (x4) at (0.283060, -0.708777);
\coordinate (x5) at (0.327382, -0.513128);
\coordinate (x6) at (0.409965, -0.324431);
\coordinate (x7) at (0.641900, -0.147518);
\coordinate (x8) at (0.391100, 0.014595);
\coordinate (x9) at (0.143454, -0.024633);
\coordinate (x10) at (-0.087375, -0.116976);
\coordinate (x11) at (-0.293670, -0.259430);
\coordinate (x12) at (-0.363182, -0.549714);
\coordinate (x13) at (0.087018, 0.116378);
\coordinate (x14) at (-0.143867, 0.023957);
\coordinate (x15) at (-0.391513, -0.015213);
\coordinate (x16) at (0.293373, 0.259058);
\coordinate (x17) at (0.363000, 0.549564);
\coordinate (x18) at (0.072947, 0.517548);
\coordinate (x19) at (-0.171622, 0.427790);
\coordinate (x20) at (-0.410541, 0.324026);
\coordinate (x21) at (-0.642196, 0.147049);
\coordinate (x22) at (-0.209734, 0.521265);
\coordinate (x23) at (-0.328619, 0.512747);
\coordinate (x24) at (-0.117104, 0.597035);
\coordinate (x25) at (-0.284091, 0.708296);
\coordinate (x26) at (-0.371662, 0.928368);
\coordinate (x27) at (-0.458227, 0.888835);
\coordinate (x28) at (-0.713787, 0.566461);
\coordinate (x29) at (-0.971812, 0.235759);
\coordinate (x30) at (-0.989821, 0.142315);
\coordinate (x31) at (-0.998867, 0.047582);
\coordinate (x32) at (-0.847493, -0.334854);
\coordinate (x33) at (-0.690079, -0.723734);
\coordinate (x34) at (-0.618159, -0.786053);
\coordinate (x35) at (-0.540641, -0.841254);
\coordinate (x36) at (-0.133754, -0.901378);
\coordinate (x37) at (0.281733, -0.959493);
\coordinate (x38) at (0.371662, -0.928368);
\coordinate (x39) at (0.458227, -0.888835);
\coordinate (x40) at (0.713787, -0.566461);
\coordinate (x41) at (0.971812, -0.235759);
\coordinate (x42) at (0.989821, -0.142315);
\coordinate (x43) at (0.998867, -0.047582);
\coordinate (x44) at (0.847463, 0.334927);
\coordinate (x45) at (0.690079, 0.723734);
\coordinate (x46) at (0.618159, 0.786053);
\coordinate (x47) at (0.540641, 0.841254);
\coordinate (x48) at (0.133676, 0.901388);
\coordinate (x49) at (-0.281733, 0.959493);
\coordinate (x50) at (0.458227, 0.888835);
\coordinate (x51) at (0.295727, 0.899417);
\coordinate (x52) at (0.371662, 0.928368);
\coordinate (x53) at (0.281733, 0.959493);
\coordinate (x54) at (-0.540641, 0.841254);
\coordinate (x55) at (-0.631054, 0.705815);
\coordinate (x56) at (-0.618159, 0.786053);
\coordinate (x57) at (-0.690079, 0.723734);
\coordinate (x58) at (-0.998867, -0.047582);
\coordinate (x59) at (-0.926791, -0.193567);
\coordinate (x60) at (-0.989821, -0.142315);
\coordinate (x61) at (-0.971812, -0.235759);
\coordinate (x62) at (-0.458227, -0.888835);
\coordinate (x63) at (-0.295763, -0.899408);
\coordinate (x64) at (-0.371662, -0.928368);
\coordinate (x65) at (-0.281733, -0.959493);
\coordinate (x66) at (0.540641, -0.841254);
\coordinate (x67) at (0.631054, -0.705815);
\coordinate (x68) at (0.618159, -0.786053);
\coordinate (x69) at (0.690079, -0.723734);
\coordinate (x70) at (0.998867, 0.047582);
\coordinate (x71) at (0.926779, 0.193606);
\coordinate (x72) at (0.989821, 0.142315);
\coordinate (x73) at (0.971812, 0.235759);
\coordinate (x74) at (0.945001, 0.327068);
\coordinate (x75) at (0.909632, 0.415415);
\coordinate (x76) at (0.866025, 0.500000);
\coordinate (x77) at (0.814576, 0.580057);
\coordinate (x78) at (0.755750, 0.654861);
\coordinate (x79) at (0.189251, 0.981929);
\coordinate (x80) at (0.095056, 0.995472);
\coordinate (x81) at (0.000000, 1.000000);
\coordinate (x82) at (-0.095056, 0.995472);
\coordinate (x83) at (-0.189251, 0.981929);
\coordinate (x84) at (-0.755750, 0.654861);
\coordinate (x85) at (-0.814576, 0.580057);
\coordinate (x86) at (-0.866025, 0.500000);
\coordinate (x87) at (-0.909632, 0.415415);
\coordinate (x88) at (-0.945001, 0.327068);
\coordinate (x89) at (-0.945001, -0.327068);
\coordinate (x90) at (-0.909632, -0.415415);
\coordinate (x91) at (-0.866025, -0.500000);
\coordinate (x92) at (-0.814576, -0.580057);
\coordinate (x93) at (-0.755750, -0.654861);
\coordinate (x94) at (-0.189251, -0.981929);
\coordinate (x95) at (-0.095056, -0.995472);
\coordinate (x96) at (-0.000000, -1.000000);
\coordinate (x97) at (0.095056, -0.995472);
\coordinate (x98) at (0.189251, -0.981929);
\coordinate (x99) at (0.755750, -0.654861);
\coordinate (x100) at (0.814576, -0.580057);
\coordinate (x101) at (0.866025, -0.500000);
\coordinate (x102) at (0.909632, -0.415415);
\coordinate (x103) at (0.945001, -0.327068);

\draw (x0) -- (x1);
\draw (x1) -- (x2);
\draw (x2) -- (x3);
\draw (x3) -- (x0);
\draw (x0) -- (x4);
\draw (x4) -- (x5);
\draw (x5) -- (x1);
\draw (x5) -- (x6);
\draw (x6) -- (x2);
\draw (x6) -- (x7);
\draw (x7) -- (x8);
\draw (x8) -- (x9);
\draw (x9) -- (x10);
\draw (x10) -- (x11);
\draw (x11) -- (x12);
\draw (x12) -- (x3);
\draw (x9) -- (x13);
\draw (x13) -- (x14);
\draw[lsquare] (x14) -- (x10);
\draw[lface] (x14) -- (n8);
\draw[lface] (x10) -- (n4);
\draw (x14) -- (x15);
\draw (x15) -- (x11);
\draw (x8) -- (x16);
\draw (x16) -- (x13);
\draw (x16) -- (x17);
\draw (x17) -- (x18);
\draw (x18) -- (x19);
\draw (x19) -- (x20);
\draw (x20) -- (x21);
\draw (x21) -- (x15);
\draw (x19) -- (x22);
\draw (x22) -- (x23);
\draw (x23) -- (x20);
\draw (x18) -- (x24);
\draw (x24) -- (x22);
\draw (x24) -- (x25);
\draw (x25) -- (x23);
\draw (x25) -- (x26);
\draw (x26) -- (x27);
\draw (x27) -- (x28);
\draw[ldiamond] (x28) -- (x29) node[midway] (m1) {};
\draw[lface] (m1) -- (n12);
\draw[lface] (m1) -- (n32);
\draw (x29) -- (x30);
\draw (x30) -- (x21);
\draw (x30) -- (x31);
\draw (x31) -- (x32);
\draw[ldiamond] (x32) -- (x33) node[midway] (m2) {};
\draw[lface] (m2) -- (n13);
\draw[lface] (m2) -- (n33);
\draw (x33) -- (x34);
\draw (x34) -- (x12);
\draw (x34) -- (x35);
\draw (x35) -- (x36);
\draw[ldiamond] (x36) -- (x37) node[midway] (m3) {};
\draw[lface] (m3) -- (n14);
\draw[lface] (m3) -- (n34);
\draw (x37) -- (x38);
\draw (x38) -- (x4);
\draw (x38) -- (x39);
\draw (x39) -- (x40);
\draw[ldiamond] (x40) -- (x41) node[midway] (m4) {};
\draw[lface] (m4) -- (n15);
\draw[lface] (m4) -- (n35);
\draw (x41) -- (x42);
\draw (x42) -- (x7);
\draw (x42) -- (x43);
\draw (x43) -- (x44);
\draw[ldiamond] (x44) -- (x45) node[midway] (m5) {};
\draw[lface] (m5) -- (n16);
\draw[lface] (m5) -- (n30);
\draw (x45) -- (x46);
\draw (x46) -- (x17);
\draw (x46) -- (x47);
\draw (x47) -- (x48);
\draw[ldiamond] (x48) -- (x49) node[midway] (m6) {};
\draw[lface] (m6) -- (n17);
\draw[lface] (m6) -- (n31);
\draw (x49) -- (x26);
\draw (x47) -- (x50);
\draw (x50) -- (x51);
\draw (x51) -- (x48);
\draw (x50) -- (x52);
\draw (x52) -- (x53);
\draw (x53) -- (x51);
\draw (x27) -- (x54);
\draw (x54) -- (x55);
\draw (x55) -- (x28);
\draw (x54) -- (x56);
\draw (x56) -- (x57);
\draw (x57) -- (x55);
\draw (x31) -- (x58);
\draw (x58) -- (x59);
\draw (x59) -- (x32);
\draw (x58) -- (x60);
\draw (x60) -- (x61);
\draw (x61) -- (x59);
\draw (x35) -- (x62);
\draw (x62) -- (x63);
\draw (x63) -- (x36);
\draw (x62) -- (x64);
\draw (x64) -- (x65);
\draw (x65) -- (x63);
\draw (x39) -- (x66);
\draw (x66) -- (x67);
\draw (x67) -- (x40);
\draw (x66) -- (x68);
\draw (x68) -- (x69);
\draw (x69) -- (x67);
\draw (x43) -- (x70);
\draw (x70) -- (x71);
\draw (x71) -- (x44);
\draw (x70) -- (x72);
\draw (x72) -- (x73);
\draw (x73) -- (x71);
\draw (x73) -- (x74);
\draw (x74) -- (x75);
\draw (x75) -- (x76);
\draw (x76) -- (x77);
\draw (x77) -- (x78);
\draw (x78) -- (x45);
\draw (x53) -- (x79);
\draw (x79) -- (x80);
\draw (x80) -- (x81);
\draw (x81) -- (x82);
\draw (x82) -- (x83);
\draw (x83) -- (x49);
\draw (x57) -- (x84);
\draw (x84) -- (x85);
\draw (x85) -- (x86);
\draw (x86) -- (x87);
\draw (x87) -- (x88);
\draw (x88) -- (x29);
\draw (x61) -- (x89);
\draw (x89) -- (x90);
\draw (x90) -- (x91);
\draw (x91) -- (x92);
\draw (x92) -- (x93);
\draw (x93) -- (x33);
\draw (x65) -- (x94);
\draw (x94) -- (x95);
\draw (x95) -- (x96);
\draw (x96) -- (x97);
\draw (x97) -- (x98);
\draw (x98) -- (x37);
\draw (x69) -- (x99);
\draw (x99) -- (x100);
\draw (x100) -- (x101);
\draw (x101) -- (x102);
\draw (x102) -- (x103);
\draw (x103) -- (x41);


    \end{scope}
  \end{tikzsubfigure}
\end{tikzfigure2}
\clearpage
\begin{theorem}
  Let $p$ and $v$ be a pair of admissible sequences for an orientable closed $2$-manifold $S$ with {\sc Euler} characteristic $\chi$. If $\chi = 2$ we exclude the case
  \begin{align*}
    \sum_{k=3 \atop 2 \nmid k}^{m} p_k = 0 \text{\quad and\quad}\sum_{k=4 \atop 3 \nmid k}^n v_k = 1.
  \end{align*}
  Then $(p, v)$ is $[(2k + 2) \times 4, 4k+10]$-$[3]$-realizable on $S$ for all $k \in \nats$.
  \begin{proof}
    An expansion $3$-patch $\mathcal{P}_N$ with outer tuple $o = (2, 2, 1, 1)$ is shown in \autoref{fig:expansion:patch:4:10:a} and a corresponding $o$-$6$-gonal $3$-patch $\mathcal{P}_F$ is shown in \autoref{fig:expansion:patch:4:10:b}, both consisting of only quadrangles and decagons. By using \autoref{const:edge:replacement:4:1} and \autoref{const:edge:replacement:4:2} as indicated we get $3$-patches consisting of only quadrangles and $(4k+10)$-gons, $k \in \nats$. By \autoref{thm:expansion:patch:poly:4:k} there is a expansion $3$-patch $\mathcal{P}_P$ with the polyhedral property consisting also of only quadrangles and $(4k+10)$-gons, thus we can apply \autoref{thm:main:const}.
  \end{proof}
\end{theorem}
{\par\vspace*{\fill}}
\begin{tikzfigure2}
  \begin{tikzsubfigure}{\label{fig:expansion:patch:4:10:a}}{$\mathcal{P}_N$}{1.0}
    \begin{scope}[yscale=0.866, scale=0.8]
      \node[anchor= 90] at (-1, 2) {$i_0$};
      \node[anchor= 45] at (-2, 4)   {$i_1$};
      \node[anchor= 45] at (-4, 4)   {$i_2=o_{0}$};
      \node[anchor=  0] at (-5, 6) {$o_1$};
      \node[anchor=300] at (-4, 8)   {$o_2$};
      \node[anchor=240] at (-2, 8)   {$\bm{o_s}$};
      \node[anchor=240] at (-1, 6) {$o_4$};
      \node[anchor=240] at (1, 6)  {$i_{2}'=o_5$};
      \node[anchor=180] at (2, 4)    {$i_1'$};
      \node[anchor=120] at (1,2)   {$i_{0}'$};

      \fill[black] (-1, 2) circle(1.5pt);
      \fill[black] (-2, 4)   circle(1.5pt);
      \fill[black] (-4, 4)   circle(1.5pt);
      \fill[black] (-5, 6) circle(1.5pt);
      \fill[black] (-4, 8)   circle(1.5pt);
      \fill[black] (-2, 8)   circle(1.5pt);
      \fill[black] (-1, 6) circle(1.5pt);
      \fill[black] (1, 6)  circle(1.5pt);
      \fill[black] (2, 4)    circle(1.5pt);
      \fill[black] (1,2)   circle(1.5pt);
      \fill[black] (-1.5, 4.25)  circle(1.5pt);
      \fill[black] (-2, 4.5)   circle(1.5pt);
      \fill[black] (-2, 5.5)   circle(1.5pt);
      \fill[black] (-1, 4.5)   circle(1.5pt);
      \fill[black] (-1, 5.5)   circle(1.5pt);
      \fill[black] (-1.5, 5.75)  circle(1.5pt);
      \fill[black] (-1.5, 4.666)  circle(1.5pt);
      \fill[black] (-1.5, 5.333)  circle(1.5pt);

      \draw (-1,2)--(-2,4)--(-4,4)--(-5,6)--(-4,8)--(-2,8)--(-1,6)--(1,6)--(2,4)--(1,2)--(-1,2);
      \draw (-2,4)--(-1.5,4.25)--(-2,4.5)--(-2,5.5)--(-1.5,5.75)--(-1,5.5)--(-1,4.5)--(-1.5,4.25);
      \draw (-2,5.5)--(-1.5, 5.333)--(-1,5.5);
      \draw (-2,4.5)--(-1.5, 4.666)--(-1,4.5);
      \draw (-1.5,4.666)--(-1.5, 5.333);
      \draw[ldiamond] (-1,6)--(-1.5, 5.75);         
      
      \node (k1) at (-3,7)   {$10$}; 
      \node (k2) at (0,5)    {$10$};
      \node at (-1.2,5)      {$4$}; 
      \node at (-1.8,5)    {$4$}; 
      \node (h1) at (-1,3.6) {$4$}; 
      \node (h2) at (-2.5,6) {$4$};
      \draw[lface] (-1.25, 5.875) -- (k1);
      \draw[lface] (-1.25, 5.875) -- (k2);

      \draw[dotted] (-1.5,4.5)--(h1);
      \draw[dotted] (-1.5,5.5)--(h2);
    \end{scope}
  \end{tikzsubfigure}
  \begin{tikzsubfigure}{\label{fig:expansion:patch:4:10:b}}{$\mathcal{P}_F$}{1.0}
    \begin{scope}[scale=5]
      \coordinate (x0) at (-0.365385, 0.717713);
\coordinate (x1) at (-0.406737, 0.913545);
\coordinate (x2) at (-0.587785, 0.809017);
\coordinate (x3) at (-0.470998, 0.659679);
\coordinate (x4) at (-0.283612, 0.537937);
\coordinate (x5) at (-0.159018, 0.381951);
\coordinate (x6) at (0.097077, 0.360698);
\coordinate (x7) at (0.297413, 0.535869);
\coordinate (x8) at (0.406737, 0.913545);
\coordinate (x9) at (0.207912, 0.978148);
\coordinate (x10) at (0.000000, 1.000000);
\coordinate (x11) at (-0.207912, 0.978148);
\coordinate (x12) at (-0.397680, 0.485286);
\coordinate (x13) at (-0.376360, 0.293964);
\coordinate (x14) at (0.143888, 0.123784);
\coordinate (x15) at (0.276362, 0.178024);
\coordinate (x16) at (0.144049, -0.122644);
\coordinate (x17) at (0.276426, -0.177430);
\coordinate (x18) at (0.097642, -0.359681);
\coordinate (x19) at (0.297616, -0.535539);
\coordinate (x20) at (-0.549960, 0.108715);
\coordinate (x21) at (-0.549754, -0.108214);
\coordinate (x22) at (-0.375875, -0.293275);
\coordinate (x23) at (-0.158333, -0.380911);
\coordinate (x24) at (-0.772107, 0.074295);
\coordinate (x25) at (-0.772023, -0.074053);
\coordinate (x26) at (-0.743145, 0.669131);
\coordinate (x27) at (-0.866025, 0.500000);
\coordinate (x28) at (-0.951057, 0.309017);
\coordinate (x29) at (-0.994522, 0.104528);
\coordinate (x30) at (-0.994522, -0.104528);
\coordinate (x31) at (-0.397102, -0.484666);
\coordinate (x32) at (-0.282908, -0.537009);
\coordinate (x33) at (-0.470672, -0.659283);
\coordinate (x34) at (-0.364844, -0.717213);
\coordinate (x35) at (-0.951057, -0.309017);
\coordinate (x36) at (-0.866025, -0.500000);
\coordinate (x37) at (-0.743145, -0.669131);
\coordinate (x38) at (-0.587785, -0.809017);
\coordinate (x39) at (-0.406737, -0.913545);
\coordinate (x40) at (-0.207912, -0.978148);
\coordinate (x41) at (-0.000000, -1.000000);
\coordinate (x42) at (0.207912, -0.978148);
\coordinate (x43) at (0.406737, -0.913545);
\coordinate (x44) at (0.587785, -0.809017);
\coordinate (x45) at (0.590971, -0.267400);
\coordinate (x46) at (0.590963, 0.267438);
\coordinate (x47) at (0.587785, 0.809017);
\coordinate (x48) at (0.619285, -0.087671);
\coordinate (x49) at (0.619165, 0.088090);
\coordinate (x50) at (0.708487, -0.036660);
\coordinate (x51) at (0.708521, 0.039158);
\coordinate (x52) at (0.806694, -0.030935);
\coordinate (x53) at (0.806761, 0.028476);
\coordinate (x54) at (0.905832, -0.043686);
\coordinate (x55) at (0.906158, 0.040967);
\coordinate (x56) at (0.743145, -0.669131);
\coordinate (x57) at (0.866025, -0.500000);
\coordinate (x58) at (0.951057, -0.309017);
\coordinate (x59) at (0.994522, -0.104528);
\coordinate (x60) at (0.994522, 0.104528);
\coordinate (x61) at (0.951057, 0.309017);
\coordinate (x62) at (0.866025, 0.500000);
\coordinate (x63) at (0.743145, 0.669131);
\draw (x0) -- (x1);
\draw (x1) -- (x2);
\draw (x2) -- (x3);
\draw (x3) -- (x0);
\draw (x0) -- (x4);
\draw (x4) -- (x5);
\draw (x5) -- (x6);
\draw (x6) -- (x7);
\draw (x7) -- (x8);
\draw (x8) -- (x9);
\draw (x9) -- (x10);
\draw (x10) -- (x11);
\draw (x11) -- (x1);
\draw (x3) -- (x12);
\draw (x12) -- (x4);
\draw (x12) -- (x13);
\draw (x13) -- (x5);
\draw (x6) -- (x14);
\draw (x14) -- (x15);
\draw (x15) -- (x7);
\draw (x14) -- (x16);
\draw (x16) -- (x17);
\draw (x17) -- (x15);
\draw (x16) -- (x18);
\draw (x18) -- (x19);
\draw (x19) -- (x17);
\draw (x13) -- (x20);
\draw (x20) -- (x21);
\draw (x21) -- (x22);
\draw (x22) -- (x23);
\draw (x23) -- (x18);
\draw (x20) -- (x24);
\draw (x24) -- (x25);
\draw (x25) -- (x21);
\draw (x2) -- (x26);
\draw (x26) -- (x27);
\draw (x27) -- (x28);
\draw (x28) -- (x29);
\draw (x29) -- (x24);
\draw (x29) -- (x30);
\draw (x30) -- (x25);
\draw (x22) -- (x31);
\draw (x31) -- (x32);
\draw (x32) -- (x23);
\draw (x31) -- (x33);
\draw (x33) -- (x34);
\draw (x34) -- (x32);
\draw (x30) -- (x35);
\draw (x35) -- (x36);
\draw (x36) -- (x37);
\draw (x37) -- (x38);
\draw (x38) -- (x33);
\draw (x38) -- (x39);
\draw (x39) -- (x34);
\draw (x39) -- (x40);
\draw (x40) -- (x41);
\draw (x41) -- (x42);
\draw (x42) -- (x43);
\draw (x43) -- (x19);
\draw (x43) -- (x44);
\draw (x44) -- (x45);
\draw (x45) -- (x46);
\draw (x46) -- (x47);
\draw (x47) -- (x8);
\draw (x45) -- (x48);
\draw (x48) -- (x49);
\draw (x49) -- (x46);
\draw (x48) -- (x50);
\draw (x50) -- (x51);
\draw (x51) -- (x49);
\draw (x50) -- (x52);
\draw (x52) -- (x53);
\draw (x53) -- (x51);
\draw (x52) -- (x54);
\draw (x54) -- (x55);
\draw (x55) -- (x53);
\draw (x44) -- (x56);
\draw (x56) -- (x57);
\draw (x57) -- (x58);
\draw (x58) -- (x59);
\draw (x59) -- (x54);
\draw (x59) -- (x60);
\draw (x60) -- (x55);
\draw (x60) -- (x61);
\draw (x61) -- (x62);
\draw (x62) -- (x63);
\draw (x63) -- (x47);
\node at (-0.562446, 0.182750) {0};
\node at (-0.457726, 0.774989) {1};
\node at (-0.041352, 0.731755) {2};
\node at (-0.379419, 0.600154) {3};
\node at (-0.304168, 0.424784) {4};
\node at (0.203685, 0.299594) {5};
\node at (0.210181, 0.000433) {6};
\node at (0.203933, -0.298824) {7};
\node at (-0.168664, 0.000439) {8};
\node at (-0.660961, 0.000186) {9};
\node at (-0.670964, 0.401363) {10};
\node at (-0.883293, 0.000060) {11};
\node at (-0.303555, -0.423965) {12};
\node at (-0.378882, -0.599543) {13};
\node at (-0.670796, -0.401118) {14};
\node at (-0.457510, -0.774765) {15};
\node at (-0.041083, -0.731374) {16};
\node at (0.431879, 0.000096) {17};
\node at (0.605096, 0.000114) {18};
\node at (0.663864, 0.000729) {19};
\node at (0.757616, 0.000010) {20};
\node at (0.856362, -0.001295) {21};
\node at (0.777380, -0.285805) {22};
\node at (0.950259, -0.000680) {23};
\node at (0.777410, 0.285582) {24};

    \end{scope}
  \end{tikzsubfigure}
\end{tikzfigure2}