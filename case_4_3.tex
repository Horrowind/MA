\section{$3$-valent {\sc Eberhard}-like theorems with quadrangles}

In this section we want to prove $3$-valent {\sc Eberhard}-like theorems with quadrangles, i.e. for $q = [q_4 \times 4, q_l \times l]$, $w = [w_3 \times 3]$, $l > 6$, $\gcd(q_4, q_l) = 1$. As it turns out in \autoref{sec:negative:results}, this is only possible for all admissible pairs of sequences on all closed orientable $2$-manifolds, if $4 \nmid l$. For all the proofs, we want to use \autoref{thm:main:const}, therefore we need to have a construction scheme for patches with arbitrary large $l$-gons. These we get by the next two constructions:

\begin{construction}\label{const:edge:replacement:4:1}
  \begin{cinput}
  \item A $3$-patch with $p$-vector $p$ and a specified edge with exactly one vertex incident to some $k_1$-gon and the other vertex  incident to some $k_2$-gon.
  \item We want to label the edge with a square and the two polygons by arrows beginning at the edge which point to them. 
  \end{cinput}
  \begin{coutput}
  \item A new $3$-patch with $p$-vector $p - [k_1, k_2] + [k \times 4] + [k_1 + k, k_2 + k]$ for all $k \in \nats$.
  \item If every two faces of the $3$-patch meet proper, then this is carried over to the new patch.
  \end{coutput}
  \begin{cdescription}
    Using the replacement of the single edge as seen in \autoref{fig:const:edge:replacement:4:1} results in a new map with $p$-vector $p - [k_1, k_2] + [4] + [k_1 + 1, k_2 + 1]$. The line on the left labeled with a square is the specified edge and the line on the right labeled with a square is a new edge which we can use to repeat the construction. Every time we use this construction we add a new quadrangle while increasing the number of vertices of the left and right polygon; doing this $k$ times has the desired result, since we only inserted $3$-valent vertices. That all faces meet proper follows by induction, as this property is preserved in each step.
    \begin{tikzfigure}{\label{fig:const:edge:replacement:4:1}}{}
      \matrix (m) [column sep=1cm] {
        \begin{scope}
          \draw[lsquare] (-1, 0) -- (1, 0);
          \draw (-1.2, 0.5) -- (-1, 0) -- (-1.2, -0.5);
          \draw (1.2, 0.5) -- (1, 0) -- (1.2, -0.5);
          \node (k1) at (-2.5, 0) {$k_1$};
          \node (k2) at (2.5, 0) {$k_2$};
          \draw[lface] (-1, 0) -- (k1);
          \draw[lface] ( 1, 0) -- (k2);
        \end{scope}
        &
        \begin{scope}
          \draw[lsquare] (-1, 0.5) -- (1, 0.5);
          \draw (-1, -0.5) -- (1, -0.5);
          \draw (-1.2, 1) -- (-1, 0.5);
          \draw (-1.2, -1) -- (-1, -0.5);
          \draw (1.2, 1) -- (1, 0.5);
          \draw (1.2, -1) -- (1, -0.5);
          \draw (1, 0.5) -- (1, -0.5);
          \draw (-1, 0.5) -- (-1, -0.5);

          \node (k1) at (-2.8, 0) {$k_1 + 1$};
          \node (k2) at ( 2.8, 0) {$k_2 + 1$};
          \node      at (   0, 0) {$4$};
          \draw[lface] (-1, 0.5) -> (k1);
          \draw[lface] ( 1, 0.5) -> (k2);
        \end{scope}
        \\
      };
    \end{tikzfigure}
  \end{cdescription}
\end{construction}

\begin{construction}\label{const:edge:replacement:4:2}
  \begin{cinput}
  \item A $3$-patch with $p$-vector $p$ and a specified edge incident to both some $k_1$-gon and some $k_2$-gon.
  \item We want to label the edge with a diamond and the two polygons by arrows beginning at the edge which point to them. 
  \end{cinput}
  \begin{coutput}
  \item A new $3$-patch with $p$-vector $p - [k_1, k_2] + [(4k) \times 4] + [4k + k_1 , 4k + k_2]$ for all $k \in \nats$.
  \end{coutput}
  \begin{cdescription}
    Using the replacement of the single edge as seen in \autoref{fig:const:edge:replacement:4:2} results in a new map with $p$-vector $p - [k_1, k_2] + [4 \times 4] + [k_1 + 4, k_2 + 4]$. The thick line on the left is the specified edge and the thick line on the right is a new edge which we can use to repeat the construction. Every time we use this construction we add four new quadrangles while increasing the number of vertices of the left and right polygon; doing this $k$ times has the desired result, since we only insert $3$-valent vertices.
    \begin{tikzfigure}{\label{fig:const:edge:replacement:4:2}}{}
      \matrix (m) [column sep=1cm] {
        \begin{scope}
          \draw[ldiamond] (0, -1) -- (0, 1);
          \draw (0.5, -1.2) -- (0, -1) -- (-0.5, -1.2);
          \draw (0.5, 1.2) -- (0, 1) -- (-0.5, 1.2);
          \node (k1) at (-2, 0) {$k_1$};
          \node (k2) at ( 2, 0) {$k_2$};
          \draw[lface] (0, 0) -- (k1);
          \draw[lface] (0, 0) -- (k2);
        \end{scope}
        &
        \begin{scope}
          \draw (0.5, -2.2) -- (0, -2) -- (-0.5, -2.2);
          \draw (0.5, 2.7) -- (0, 2.5) -- (-0.5, 2.7);
          \draw[ldiamond] (0, 2.5) -- (0, 1.5);
          \draw (0, 1.5) -- (-0.5, 1) -- (-0.5, -1) -- (0, -1.5) -- (0.5, -1) -- (0.5, 1) -- (0, 1.5);
          \draw (0, -2) -- (0, -1.5);
          \draw (-0.5, -1) -- (0, -0.5) -- (0.5, -1);
          \draw (-0.5, 1) -- (0, 0.5) -- (0.5, 1);
          \draw (0, -0.5) -- (0, 0.5);
          \node (k1) at (-2, 0) {$k_1 + 4$};
          \node (k2) at (2, 0) {$k_2 + 4$};
          \draw[lface] (0, 2) -- (k1);
          \draw[lface] (0, 2) -- (k2);

          \node at (-0.25, 0) {$4$};
          \node at (0.25, 0) {$4$};
          \node at (0, -1) {$4$};
          \node at (0, 1) {$4$};

        \end{scope}
        \\
      };
    \end{tikzfigure}
  \end{cdescription}
\end{construction}

\begin{lemma}\label{thm:expansion:patch:poly:4:k}
  There is an expansion $3$-patch with the polyhedral property consisting of $4$-gons and $k$-gons, $k \geq 7$.
  \begin{proof}
    The $3$-patch in \autoref{fig:expansion:patch:poly:4} has the polyhedral property. This does not change, when we use \autoref{const:edge:replacement:4:1} on all the edges drawn thick in this figure, since \todo{} and we yield a patch in which all polygons are either $4$-gonal or $k$-gonal.
    \begin{tikzfigure}{\label{fig:expansion:patch:poly:4}}{A expansion patch with the polyhedral property consisting of $4$-gons and $7$-gons}
      \matrix (m) [column sep=1cm] {
        \begin{scope}[yscale=0.866]
          \draw (-0.5, 1) -- (-0.5, 3) -- (-2, 4) -- (-3.5, 7) -- (-3.5, 8) -- (-2.5, 7.666) -- (-1.5, 8) -- (-0.5, 9) -- (0.5, 9) -- (1.5, 10) -- (2.5, 10.333) -- (3.5, 10) -- (4.5, 9) -- (4.25, 7.5) -- (3.5, 7)  (2, 4) -- (2, 2) -- (0.5, 1) -- (-0.5, 1);
          \draw (3.5, 7) -- (0.5, 9);
          \draw[lsquare] (-1, 6) -- (-0.5, 6);
          \draw (-0.5, 1) -- (-0.5, 3) -- (0, 4) -- (-1, 6) -- (-2.5, 7.666);
          \draw (2, 4) -- (1, 4.666) -- (-0.5, 6) -- (-1.5, 8);
          \draw (0, 4) -- (1, 4.666);
          \draw[lsquare] (3.5, 7) -- (2, 4);

          \node (n1) at (-2,5.5)  {$7$};
          \node (n2) at (0.5,6.5) {$7$};
          \node (n3) at (2.5,8.5) {$7$};
          \node (n4) at (1,3)     {$7$};
          \node at (-1.5,7) {$4$};
          \node at (0.1,4.7)  {$4$};

          
          \draw[lface] (-1, 6) -- (n1);
          \draw[lface] (-0.5, 6) -- (n2);
          \draw[lface] (3.5, 7) -- (n3);
          \draw[lface] (2, 4) -- (n4);

          \node[anchor= 90] at (-0.5, 1)    {$i_0$};
          \node[anchor= 45] at (-0.5, 3)    {$i_1$};
          \node[anchor= 30] at (-2, 4)      {$i_2$};
          \node[anchor=  0] at (-3.5, 7)    {$i_3$};
          \node[anchor=270] at (-3.5, 8)    {$i_4=o_0$};
          \node[anchor=270] at (-2.5, 7.666){$o_{1}$};
          \node[anchor=300] at (-1.5, 8)    {$o_{2}$};
          \node[anchor=270] at (-0.5, 9)    {$o_{3}$};
          \node[anchor=300] at (0.5, 9)     {$o_{4}$};
          \node[anchor=300] at (1.5, 10)    {$o_{5}$};
          \node[anchor=270] at (2.5, 10.333){$o_{6}$};
          \node[anchor=245] at (3.5, 10)    {$o_{7}$};
          \node[anchor=180] at (4.5, 9)     {$o_{8}$};
          \node[anchor=180] at (4.25, 7.5)  {$o_{10}$};
          \node[anchor=160] at (3.5, 7)     {$i_3'=o_{11}$};
          \node[anchor=180] at (2, 4)       {$i_2'$};
          \node[anchor=180] at (2, 2)       {$i_1'$};
          \node[anchor=180] at (0.5, 1)     {$i_0'$};

          \fill[black] (-0.5, 1)    circle (2pt);
          \fill[black] (-0.5, 3)    circle (2pt);
          \fill[black] (-2, 4)      circle (2pt);
          \fill[black] (-3.5, 7)    circle (2pt);
          \fill[black] (-3.5, 8)    circle (2pt);
          \fill[black] (-2.5, 7.666)circle (2pt);
          \fill[black] (-1.5, 8)    circle (2pt);
          \fill[black] (-0.5, 9)    circle (2pt);
          \fill[black] (0.5, 9)     circle (2pt);
          \fill[black] (1.5, 10)    circle (2pt);
          \fill[black] (2.5, 10.333)circle (2pt);
          \fill[black] (3.5, 10)    circle (2pt);
          \fill[black] (4.5, 9)     circle (2pt);
          \fill[black] (4.25, 7.5)  circle (2pt);
          \fill[black] (3.5, 7)     circle (2pt);
          \fill[black] (2, 4)       circle (2pt);
          \fill[black] (2, 2)       circle (2pt);
          \fill[black] (0.5, 1)     circle (2pt);
          \fill[black] (-1,6)       circle (2pt);
          \fill[black] (-0.5,6)     circle (2pt);
          \fill[black] (0,4)        circle (2pt);
          \fill[black] (1, 4.666)   circle (2pt);
          
       
        \end{scope}
        &
        \begin{scope}[scale=0.5]
          \begin{scope}[yscale=0.866]
            \draw[very thick] (-0.5, 1) -- (-0.5, 3) -- (-2, 4) -- (-3.5, 7) -- (-3.5, 8) -- (-2.5, 7.666) -- (-1.5, 8) -- (-0.5, 9) -- (0.5, 9) -- (1.5, 10) -- (2.5, 10.333) -- (3.5, 10) -- (4.5, 9) -- (4.25, 7.5) -- (3.5, 7) -- (2, 4) -- (2, 2) -- (0.5, 1) -- (-0.5, 1);
            \draw (3.5, 7) -- (0.5, 9);
            \draw (-1, 6) -- (-0.5, 6);
            \draw (-0.5, 1) -- (-0.5, 3) -- (0, 4) -- (-1, 6) -- (-2.5, 7.666);
            \draw (2, 4) -- (1, 4.666) -- (-0.5, 6) -- (-1.5, 8);
            \draw (0, 4) -- (1, 4.666);

            \fill[black] (-0.5, 1)    circle (3pt);
          \fill[black] (-0.5, 3)    circle (3pt);
          \fill[black] (-2, 4)      circle (3pt);
          \fill[black] (-3.5, 7)    circle (3pt);
          \fill[black] (-3.5, 8)    circle (3pt);
          \fill[black] (-2.5, 7.666)circle (3pt);
          \fill[black] (-1.5, 8)    circle (3pt);
          \fill[black] (-0.5, 9)    circle (3pt);
          \fill[black] (0.5, 9)     circle (3pt);
          \fill[black] (1.5, 10)    circle (3pt);
          \fill[black] (2.5, 10.333)circle (3pt);
          \fill[black] (3.5, 10)    circle (3pt);
          \fill[black] (4.5, 9)     circle (3pt);
          \fill[black] (4.25, 7.5)  circle (3pt);
          \fill[black] (3.5, 7)     circle (3pt);
          \fill[black] (2, 4)       circle (3pt);
          \fill[black] (2, 2)       circle (3pt);
          \fill[black] (0.5, 1)     circle (3pt);
          \fill[black] (-1,6)       circle (3pt);
          \fill[black] (-0.5,6)     circle (3pt);
          \fill[black] (0,4)        circle (3pt);
          \fill[black] (1, 4.666)   circle (3pt);
          
          \end{scope}
          \begin{scope}[rotate=60,yscale=0.866]
            \draw[very thick] (-0.5, 1) -- (-0.5, 3) -- (-2, 4) -- (-3.5, 7) -- (-3.5, 8) -- (-2.5, 7.666) -- (-1.5, 8) -- (-0.5, 9) -- (0.5, 9) -- (1.5, 10) -- (2.5, 10.333) -- (3.5, 10) -- (4.5, 9) -- (4.25, 7.5) -- (3.5, 7) -- (2, 4) -- (2, 2) -- (0.5, 1) -- (-0.5, 1);
            \draw (3.5, 7) -- (0.5, 9);
            \draw (-1, 6) -- (-0.5, 6);
            \draw (-0.5, 1) -- (-0.5, 3) -- (0, 4) -- (-1, 6) -- (-2.5, 7.666);
            \draw (2, 4) -- (1, 4.666) -- (-0.5, 6) -- (-1.5, 8);
            \draw (0, 4) -- (1, 4.666);

                 \fill[black] (-0.5, 1)    circle (3pt);
          \fill[black] (-0.5, 3)    circle (3pt);
          \fill[black] (-2, 4)      circle (3pt);
          \fill[black] (-3.5, 7)    circle (3pt);
          \fill[black] (-3.5, 8)    circle (3pt);
          \fill[black] (-2.5, 7.666)circle (3pt);
          \fill[black] (-1.5, 8)    circle (3pt);
          \fill[black] (-0.5, 9)    circle (3pt);
          \fill[black] (0.5, 9)     circle (3pt);
          \fill[black] (1.5, 10)    circle (3pt);
          \fill[black] (2.5, 10.333)circle (3pt);
          \fill[black] (3.5, 10)    circle (3pt);
          \fill[black] (4.5, 9)     circle (3pt);
          \fill[black] (4.25, 7.5)  circle (3pt);
          \fill[black] (3.5, 7)     circle (3pt);
          \fill[black] (2, 4)       circle (3pt);
          \fill[black] (2, 2)       circle (3pt);
          \fill[black] (0.5, 1)     circle (3pt);
          \fill[black] (-1,6)       circle (3pt);
          \fill[black] (-0.5,6)     circle (3pt);
          \fill[black] (0,4)        circle (3pt);
          \fill[black] (1, 4.666)   circle (3pt);
          \end{scope}
          \begin{scope}[yscale=0.866,shift={(0 cm,18 cm)},rotate=180]
            \draw[very thick] (-0.5, 1) -- (-0.5, 3) -- (-2, 4) -- (-3.5, 7) -- (-3.5, 8) -- (-2.5, 7.666) -- (-1.5, 8) -- (-0.5, 9) -- (0.5, 9) -- (1.5, 10) -- (2.5, 10.333) -- (3.5, 10) -- (4.5, 9) -- (4.25, 7.5) -- (3.5, 7) -- (2, 4) -- (2, 2) -- (0.5, 1) -- (-0.5, 1);
            \draw (3.5, 7) -- (0.5, 9);
            \draw (-1, 6) -- (-0.5, 6);
            \draw (-0.5, 1) -- (-0.5, 3) -- (0, 4) -- (-1, 6) -- (-2.5, 7.666);
            \draw (2, 4) -- (1, 4.666) -- (-0.5, 6) -- (-1.5, 8);
            \draw (0, 4) -- (1, 4.666);

                 \fill[black] (-0.5, 1)    circle (3pt);
          \fill[black] (-0.5, 3)    circle (3pt);
          \fill[black] (-2, 4)      circle (3pt);
          \fill[black] (-3.5, 7)    circle (3pt);
          \fill[black] (-3.5, 8)    circle (3pt);
          \fill[black] (-2.5, 7.666)circle (3pt);
          \fill[black] (-1.5, 8)    circle (3pt);
          \fill[black] (-0.5, 9)    circle (3pt);
          \fill[black] (0.5, 9)     circle (3pt);
          \fill[black] (1.5, 10)    circle (3pt);
          \fill[black] (2.5, 10.333)circle (3pt);
          \fill[black] (3.5, 10)    circle (3pt);
          \fill[black] (4.5, 9)     circle (3pt);
          \fill[black] (4.25, 7.5)  circle (3pt);
          \fill[black] (3.5, 7)     circle (3pt);
          \fill[black] (2, 4)       circle (3pt);
          \fill[black] (2, 2)       circle (3pt);
          \fill[black] (0.5, 1)     circle (3pt);
          \fill[black] (-1,6)       circle (3pt);
          \fill[black] (-0.5,6)     circle (3pt);
          \fill[black] (0,4)        circle (3pt);
          \fill[black] (1, 4.666)   circle (3pt);
          \end{scope}
          \begin{scope}[shift={(0 cm,15.588 cm)},rotate=240,yscale=0.866]
            \draw[very thick] (-0.5, 1) -- (-0.5, 3) -- (-2, 4) -- (-3.5, 7) -- (-3.5, 8) -- (-2.5, 7.666) -- (-1.5, 8) -- (-0.5, 9) -- (0.5, 9) -- (1.5, 10) -- (2.5, 10.333) -- (3.5, 10) -- (4.5, 9) -- (4.25, 7.5) -- (3.5, 7) -- (2, 4) -- (2, 2) -- (0.5, 1) -- (-0.5, 1);
            \draw (3.5, 7) -- (0.5, 9);
            \draw (-1, 6) -- (-0.5, 6);
            \draw (-0.5, 1) -- (-0.5, 3) -- (0, 4) -- (-1, 6) -- (-2.5, 7.666);
            \draw (2, 4) -- (1, 4.666) -- (-0.5, 6) -- (-1.5, 8);
            \draw (0, 4) -- (1, 4.666);

                 \fill[black] (-0.5, 1)    circle (3pt);
          \fill[black] (-0.5, 3)    circle (3pt);
          \fill[black] (-2, 4)      circle (3pt);
          \fill[black] (-3.5, 7)    circle (3pt);
          \fill[black] (-3.5, 8)    circle (3pt);
          \fill[black] (-2.5, 7.666)circle (3pt);
          \fill[black] (-1.5, 8)    circle (3pt);
          \fill[black] (-0.5, 9)    circle (3pt);
          \fill[black] (0.5, 9)     circle (3pt);
          \fill[black] (1.5, 10)    circle (3pt);
          \fill[black] (2.5, 10.333)circle (3pt);
          \fill[black] (3.5, 10)    circle (3pt);
          \fill[black] (4.5, 9)     circle (3pt);
          \fill[black] (4.25, 7.5)  circle (3pt);
          \fill[black] (3.5, 7)     circle (3pt);
          \fill[black] (2, 4)       circle (3pt);
          \fill[black] (2, 2)       circle (3pt);
          \fill[black] (0.5, 1)     circle (3pt);
          \fill[black] (-1,6)       circle (3pt);
          \fill[black] (-0.5,6)     circle (3pt);
          \fill[black] (0,4)        circle (3pt);
          \fill[black] (1, 4.666)   circle (3pt);
          \end{scope}
        \end{scope}
        \\
      };
      % \draw[very thick] (0.5, 2.5) -- (0, 2);
      % \draw[very thick] (1, 2) -- (1.5, 2.5);
      % \draw (0, 2) -- (-0.5, 1) -- (0, 0) -- (1, 0) -- (1.5, 1) -- (1, 2) -- (0.5, 2.5);
      % \draw (1.5, 2.5) -- (2, 3.5) -- (1.5, 4.5) -- (0.5, 4.5) -- (0, 3.5) -- (0.5, 2.5);
      % \draw (1.5, 2.5) -- (2, 1.5) -- (3, 1.5) -- (3.5, 2.5) -- (3, 3.5) -- (2.5, 4) -- (2, 3.5);
      % \draw (-0.5, 1) -- (-1.5, 1) -- (-2, 2) -- (-1.5, 3) -- (-1, 3.5) -- (-0.5, 3) -- (0, 2);
      % \draw (-0.5, 3) -- (0, 3.5);
      % \draw (-1, 3.5) -- (0.5, 4.5);
    \end{tikzfigure}
  \end{proof}
\end{lemma}

We now can split the infinitely many cases in three series, each with a different residual class of $l \mod 4$:

\begin{theorem}
  Let $p$ and $v$ be a pair of admissible sequences for a orientable closed $2$-manifold $S$. Then $(p, v)$ is $[(4k + 1) \times 4, 2 \times (4k+7)]$-$[3]$-realizable for all $k \in \nats$.
  \begin{proof}
    An expansion $3$-patch with outer tuple $o = (1, 1, 1, 2, 2, 2)$ and a corresponding $o$-$6$-gonal $3$-patch consisting only of quadrangles and heptagons is shown in \autoref{fig:expansion:patch:4:7}. Using \autoref{const:edge:replacement:4:2} on the edges as indicated in both $3$-patches we get new $3$-patches consisting only of quadrangles and $4k + 7$-gons. Therefore we can apply \autoref{thm:main:const} with these patches together with the patch in \autoref{thm:expansion:patch:poly:4:k}.
    \begin{tikzfigure}{\label{fig:expansion:patch:4:7}}{}
      \matrix (m) [column sep=1cm] {
        \begin{scope}[scale=1.2]

          \node (n1) at (1.5,3) {$7$};
          \node (n2) at (0.5,1.5){$7$};
          \node at (2.2,1.8) {$4$};
          
          \draw (0.5, 2.5) -- (-0.5, 2) -- (-1, 1) -- (-0.1, 0) -- (1.1, 0) -- (2, 1) -- (1.5, 2);
          \draw[ldiamond] (0.5, 2.5) -- (1.5, 2);
          \draw (0.5, 2.5) -- (0, 3.5) -- (0.9, 4.5) -- (2.1, 4.5) -- (3, 3.5) -- (2.5, 2.5) -- (1.5, 2);
          \draw (2, 1) -- (3, 1.5) -- (2.5, 2.5);
          \node[anchor=90] at (-0.1, 0) {$i_0$};
          \node[anchor=90] at (1.2, 0) {$i'_0$};
          \node[anchor=120] at (2, 1) {$i'_1$};
          \node[anchor=180] at (3, 1.5) {$i'_2 = o_7$};
          \node[anchor=180] at (2.5, 2.5) {$o_6$};
          \node[anchor=180] at (3, 3.5) {$o_5$};
          \node[anchor=-120] at (2.1, 4.5) {$o_4$};
          \node[anchor=-60] at (0.9, 4.5) {$o_3$};
          \node[anchor=0] at (0, 3.5) {$\mathbf{o_2}$};
          \node[anchor=-20] at (0.5, 2.5) {$o_{1}$};
          \node[anchor=0] at (-0.5, 2) {$i_2 = o_{0}$};
          \node[anchor=45] at (-1, 1) {$i_1$};
          \draw[lface] (1, 2.25) -- (n1);
          \draw[lface] (1, 2.25) -- (n2);

          \fill[black] (-0.1, 0)  circle(2pt);
          \fill[black] (1.2, 0)   circle(2pt);
          \fill[black] (2, 1)     circle(2pt);
          \fill[black] (3, 1.5)   circle(2pt);
          \fill[black] (2.5, 2.5) circle(2pt);
          \fill[black] (3, 3.5)   circle(2pt);
          \fill[black] (2.1, 4.5) circle(2pt);
          \fill[black] (0.9, 4.5) circle(2pt);
          \fill[black] (0, 3.5)   circle(2pt);
          \fill[black] (0.5, 2.5) circle(2pt);
          \fill[black] (-0.5, 2)  circle(2pt);
          \fill[black] (-1, 1)    circle(2pt);
          \fill[black] (1.5,2)    circle(2pt);
          
        \end{scope}
        &
        \begin{scope}[scale=3, yshift=25]
          \coordinate (x0) at (-0.636833, -0.430484);
\coordinate (x1) at (-0.648324, -0.371449);
\coordinate (x2) at (-0.685325, -0.322172);
\coordinate (x3) at (-0.727924, -0.392224);
\coordinate (x4) at (-0.552087, -0.487859);
\coordinate (x5) at (-0.612740, -0.361879);
\coordinate (x6) at (-0.673200, -0.235415);
\coordinate (x7) at (-0.724186, -0.094568);
\coordinate (x8) at (-0.997204, 0.074730);
\coordinate (x9) at (-0.997204, -0.074730);
\coordinate (x10) at (-0.834593, -0.402435);
\coordinate (x11) at (-0.680173, -0.733052);
\coordinate (x12) at (-0.563320, -0.826239);
\coordinate (x13) at (-0.466831, -0.616557);
\coordinate (x14) at (-0.320629, -0.450392);
\coordinate (x15) at (-0.447803, -0.240925);
\coordinate (x16) at (-0.082005, -0.522040);
\coordinate (x17) at (-0.179768, -0.265015);
\coordinate (x18) at (-0.023974, -0.035278);
\coordinate (x19) at (-0.478520, 0.314020);
\coordinate (x20) at (-0.974928, 0.222521);
\coordinate (x21) at (-0.433884, -0.900969);
\coordinate (x22) at (-0.054646, -0.897083);
\coordinate (x23) at (0.135465, -0.756001);
\coordinate (x24) at (0.433884, -0.900969);
\coordinate (x25) at (0.563320, -0.826239);
\coordinate (x26) at (0.467152, -0.330729);
\coordinate (x27) at (0.294755, -0.955573);
\coordinate (x28) at (0.974928, -0.222521);
\coordinate (x29) at (0.997204, -0.074730);
\coordinate (x30) at (-0.433884, 0.900969);
\coordinate (x31) at (-0.563320, 0.826239);
\coordinate (x32) at (0.997204, 0.074730);
\coordinate (x33) at (0.741454, 0.422017);
\coordinate (x34) at (0.448614, 0.657866);
\coordinate (x35) at (0.122004, 0.844377);
\coordinate (x36) at (-0.294755, 0.955573);
\coordinate (x37) at (0.680173, 0.733052);
\coordinate (x38) at (0.563320, 0.826239);
\coordinate (x39) at (0.433884, 0.900969);
\coordinate (x40) at (0.294755, 0.955573);
\coordinate (x41) at (0.149042, 0.988831);
\coordinate (x42) at (0.000000, 1.000000);
\coordinate (x43) at (-0.149042, 0.988831);
\coordinate (x44) at (-0.680173, 0.733052);
\coordinate (x45) at (-0.781831, 0.623490);
\coordinate (x46) at (-0.866025, 0.500000);
\coordinate (x47) at (-0.930874, 0.365341);
\coordinate (x48) at (-0.974928, -0.222521);
\coordinate (x49) at (-0.930874, -0.365341);
\coordinate (x50) at (-0.866025, -0.500000);
\coordinate (x51) at (-0.781831, -0.623490);
\coordinate (x52) at (-0.294755, -0.955573);
\coordinate (x53) at (-0.149042, -0.988831);
\coordinate (x54) at (-0.000000, -1.000000);
\coordinate (x55) at (0.149042, -0.988831);
\coordinate (x56) at (0.680173, -0.733052);
\coordinate (x57) at (0.781831, -0.623490);
\coordinate (x58) at (0.866025, -0.500000);
\coordinate (x59) at (0.930874, -0.365341);
\coordinate (x60) at (0.974928, 0.222521);
\coordinate (x61) at (0.930874, 0.365341);
\coordinate (x62) at (0.866025, 0.500000);
\coordinate (x63) at (0.781831, 0.623490);
\draw (-0.636833, -0.430484) -- (-0.648324, -0.371449);
\draw (-0.648324, -0.371449) -- (-0.685325, -0.322172);
\draw (-0.685325, -0.322172) -- (-0.727924, -0.392224);
\draw (-0.727924, -0.392224) -- (-0.636833, -0.430484);
\draw (-0.636833, -0.430484) -- (-0.552087, -0.487859);
\draw (-0.552087, -0.487859) -- (-0.612740, -0.361879);
\draw (-0.612740, -0.361879) -- (-0.648324, -0.371449);
\draw (-0.612740, -0.361879) -- (-0.673200, -0.235415);
\draw (-0.673200, -0.235415) -- (-0.685325, -0.322172);
\draw (-0.673200, -0.235415) -- (-0.724186, -0.094568);
\draw (-0.724186, -0.094568) -- (-0.997204, 0.074730);
\draw (-0.997204, 0.074730) -- (-0.997204, -0.074730);
\draw[very thick] (-0.997204, -0.074730) -- (-0.834593, -0.402435);
\draw (-0.834593, -0.402435) -- (-0.727924, -0.392224);
\draw (-0.834593, -0.402435) -- (-0.680173, -0.733052);
\draw (-0.680173, -0.733052) -- (-0.563320, -0.826239);
\draw (-0.563320, -0.826239) -- (-0.466831, -0.616557);
\draw[very thick] (-0.466831, -0.616557) -- (-0.552087, -0.487859);
\draw (-0.466831, -0.616557) -- (-0.320629, -0.450392);
\draw (-0.320629, -0.450392) -- (-0.447803, -0.240925);
\draw (-0.447803, -0.240925) -- (-0.724186, -0.094568);
\draw (-0.320629, -0.450392) -- (-0.082005, -0.522040);
\draw (-0.082005, -0.522040) -- (-0.179768, -0.265015);
\draw (-0.179768, -0.265015) -- (-0.447803, -0.240925);
\draw (-0.179768, -0.265015) -- (-0.023974, -0.035278);
\draw (-0.023974, -0.035278) -- (-0.478520, 0.314020);
\draw[very thick] (-0.478520, 0.314020) -- (-0.974928, 0.222521);
\draw (-0.974928, 0.222521) -- (-0.997204, 0.074730);
\draw (-0.563320, -0.826239) -- (-0.433884, -0.900969);
\draw[very thick] (-0.433884, -0.900969) -- (-0.054646, -0.897083);
\draw (-0.054646, -0.897083) -- (0.135465, -0.756001);
\draw (0.135465, -0.756001) -- (-0.082005, -0.522040);
\draw (0.135465, -0.756001) -- (0.433884, -0.900969);
\draw (0.433884, -0.900969) -- (0.563320, -0.826239);
\draw[very thick] (0.563320, -0.826239) -- (0.467152, -0.330729);
\draw (0.467152, -0.330729) -- (-0.023974, -0.035278);
\draw (-0.054646, -0.897083) -- (0.294755, -0.955573);
\draw (0.294755, -0.955573) -- (0.433884, -0.900969);
\draw (0.467152, -0.330729) -- (0.974928, -0.222521);
\draw (0.974928, -0.222521) -- (0.997204, -0.074730);
\draw (0.997204, -0.074730) -- (-0.433884, 0.900969);
\draw (-0.433884, 0.900969) -- (-0.563320, 0.826239);
\draw (-0.563320, 0.826239) -- (-0.478520, 0.314020);
\draw (0.997204, -0.074730) -- (0.997204, 0.074730);
\draw[very thick] (0.997204, 0.074730) -- (0.741454, 0.422017);
\draw (0.741454, 0.422017) -- (0.448614, 0.657866);
\draw (0.448614, 0.657866) -- (0.122004, 0.844377);
\draw (0.122004, 0.844377) -- (-0.294755, 0.955573);
\draw[very thick] (-0.294755, 0.955573) -- (-0.433884, 0.900969);
\draw (0.741454, 0.422017) -- (0.680173, 0.733052);
\draw (0.680173, 0.733052) -- (0.563320, 0.826239);
\draw (0.563320, 0.826239) -- (0.448614, 0.657866);
\draw (0.563320, 0.826239) -- (0.433884, 0.900969);
\draw (0.433884, 0.900969) -- (0.122004, 0.844377);
\draw (0.433884, 0.900969) -- (0.294755, 0.955573);
\draw (0.294755, 0.955573) -- (0.149042, 0.988831);
\draw (0.149042, 0.988831) -- (0.000000, 1.000000);
\draw (0.000000, 1.000000) -- (-0.149042, 0.988831);
\draw (-0.149042, 0.988831) -- (-0.294755, 0.955573);
\draw (-0.563320, 0.826239) -- (-0.680173, 0.733052);
\draw (-0.680173, 0.733052) -- (-0.781831, 0.623490);
\draw (-0.781831, 0.623490) -- (-0.866025, 0.500000);
\draw (-0.866025, 0.500000) -- (-0.930874, 0.365341);
\draw (-0.930874, 0.365341) -- (-0.974928, 0.222521);
\draw (-0.997204, -0.074730) -- (-0.974928, -0.222521);
\draw (-0.974928, -0.222521) -- (-0.930874, -0.365341);
\draw (-0.930874, -0.365341) -- (-0.866025, -0.500000);
\draw (-0.866025, -0.500000) -- (-0.781831, -0.623490);
\draw (-0.781831, -0.623490) -- (-0.680173, -0.733052);
\draw (-0.433884, -0.900969) -- (-0.294755, -0.955573);
\draw (-0.294755, -0.955573) -- (-0.149042, -0.988831);
\draw (-0.149042, -0.988831) -- (-0.000000, -1.000000);
\draw (-0.000000, -1.000000) -- (0.149042, -0.988831);
\draw (0.149042, -0.988831) -- (0.294755, -0.955573);
\draw (0.563320, -0.826239) -- (0.680173, -0.733052);
\draw (0.680173, -0.733052) -- (0.781831, -0.623490);
\draw (0.781831, -0.623490) -- (0.866025, -0.500000);
\draw (0.866025, -0.500000) -- (0.930874, -0.365341);
\draw (0.930874, -0.365341) -- (0.974928, -0.222521);
\draw (0.997204, 0.074730) -- (0.974928, 0.222521);
\draw (0.974928, 0.222521) -- (0.930874, 0.365341);
\draw (0.930874, 0.365341) -- (0.866025, 0.500000);
\draw (0.866025, 0.500000) -- (0.781831, 0.623490);
\draw (0.781831, 0.623490) -- (0.680173, 0.733052);


        \end{scope}
        \\
      };
    \end{tikzfigure}
  \end{proof}
\end{theorem}

\begin{theorem}
 Let $p$ and $v$ be a pair of admissible sequences for a orientable closed $2$-manifold $S$. Then $(p, v)$ is $[(4k + 3) \times 4, 2 \times (4k+9)]$-$[3]$-$[3]$-realizable for all $k \in \nats$.
  \begin{proof}
    An expansion $3$-patch with outer tuple $o = (1, 1, 1, 1, 1, 2, 2, 2, 2, 2)$ and a corresponding $o$-$6$-gonal $3$-patch consisting only of quadrangles and nonagons is shown in \autoref{fig:expansion:patch:4:9}. Using \autoref{const:edge:replacement:4:2} on the edges as indicated in both $3$-patches we get new $3$-patches consisting only of quadrangles and $4k + 9$-gons. Therefore we can apply \autoref{thm:main:const} with these patches together with the patch in \autoref{thm:expansion:patch:poly:4:k}.
    \begin{tikzfigure}{\label{fig:expansion:patch:4:9}}{}
      \matrix (m) [column sep=1cm] {
        \begin{scope}[scale=0.8]

          \node (n1) at (-0.5,4.2) {$9$};
          \node (n2) at (0.8,2.2)  {$9$};
          \node at (1.1,3.8)   {$4$};
          \node at (2.2,2.6)   {$4$};
          \node at (2.4,1.6)   {$4$};
          
          \draw (-0.5, 3) -- (-1, 2) -- (-1, 1) -- (0, 0) -- (1, 0) -- (2, 1) -- (2, 2) -- (1.5, 3) -- (0.5, 3.5);
          \draw[ldiamond] (0.5, 3.5) -- (-0.5, 3);
          \draw (2, 1) -- (3, 1) -- (3, 2) -- (2, 2);
          \draw (3, 2) -- (2.5, 3) -- (1.5, 3);
          \draw (2.5, 3) -- (1, 4.5);
          \draw (0.5, 3.5) -- (1, 4.5) -- (1, 5.5) -- (0, 6.5) -- (-1, 6.5) -- (-2, 5.5) -- (-2, 4.5) -- (-1.5, 3.5) -- (-0.5, 3);
          \node[anchor=90] at (0, 0) {$i_0$};
          \node[anchor=90] at (1, 0) {$i'_0$};
          \node[anchor=120] at (2, 1) {$i'_1$};
          \node[anchor=180] at (3, 1) {$i'_2 = o_{11}$};
          \node[anchor=180] at (3, 2) {$o_{10}$};
          \node[anchor=-150] at (2.5, 3) {$o_9$};
          \node[anchor=-150] at (1, 4.5) {$o_8$};
          \node[anchor=-150] at (1, 5.5) {$o_7$};
          \node[anchor=-110] at (0, 6.5) {$o_6$};
          \node[anchor=-70] at (-1, 6.5) {$o_5$};
          \node[anchor=-20] at (-2, 5.5) {$o_4$};
          \node[anchor=0] at (-2, 4.5) {$o_3$};
          \node[anchor=60] at (-1.5, 3.5) {$\mathbf{o_2}$};
          \node[anchor=20] at (-0.5, 3) {$o_{1}$};
          \node[anchor=0] at (-1, 2) {$i_2 = o_{0}$};
          \node[anchor=45] at (-1, 1) {$i_1$};
          \draw[lface] (0, 3.25) -- (n1);
          \draw[lface] (0, 3.25) -- (n2);


          \fill[black] (0, 0)   circle (2pt);
          \fill[black] (1, 0)   circle (2pt);
          \fill[black] (2, 1) circle (2pt);
          \fill[black] (3, 1) circle (2pt);
          \fill[black] (3, 2) circle (2pt);
          \fill[black] (2.5, 3) circle (2pt);
          \fill[black] (1, 4.5) circle (2pt);
          \fill[black] (1, 5.5) circle (2pt);
          \fill[black] (0, 6.5) circle (2pt);
          \fill[black] (-1, 6.5) circle (2pt);
          \fill[black] (-2, 5.5) circle (2pt);
          \fill[black] (-2, 4.5) circle (2pt);
          \fill[black] (-1.5, 3.5) circle (2pt);
          \fill[black] (-0.5, 3) circle (2pt);
          \fill[black] (-1, 2) circle (2pt);
          \fill[black] (-1, 1) circle (2pt);
          \fill[black] (0.5,3.5) circle (2pt);
          \fill[black] (2,2)   circle (2pt);
          \fill[black] (1.5,3) circle (2pt);
          
        \end{scope}
        &
        \begin{scope}[scale=3, yshift=25]
          \coordinate (x0) at (0.575695, -0.197192);
\coordinate (x1) at (0.556573, -0.078916);
\coordinate (x2) at (0.456424, -0.065169);
\coordinate (x3) at (0.411774, -0.321955);
\coordinate (x4) at (0.755417, -0.108068);
\coordinate (x5) at (0.608514, 0.028443);
\coordinate (x6) at (0.486005, 0.193713);
\coordinate (x7) at (0.448511, 0.482721);
\coordinate (x8) at (0.182722, 0.346782);
\coordinate (x9) at (0.092916, 0.112802);
\coordinate (x10) at (0.057548, -0.133347);
\coordinate (x11) at (0.077844, -0.383380);
\coordinate (x12) at (0.294498, -0.589016);
\coordinate (x13) at (-0.057365, 0.134413);
\coordinate (x14) at (-0.092752, -0.111739);
\coordinate (x15) at (-0.182716, -0.345924);
\coordinate (x16) at (-0.077701, 0.384190);
\coordinate (x17) at (-0.294379, 0.589529);
\coordinate (x18) at (-0.411671, 0.322705);
\coordinate (x19) at (-0.456219, 0.065992);
\coordinate (x20) at (-0.485771, -0.192942);
\coordinate (x21) at (-0.448469, -0.482209);
\coordinate (x22) at (-0.556027, 0.081036);
\coordinate (x23) at (-0.607771, -0.026861);
\coordinate (x24) at (-0.575798, 0.198657);
\coordinate (x25) at (-0.755220, 0.109252);
\coordinate (x26) at (-0.989821, 0.142315);
\coordinate (x27) at (-0.998867, 0.047582);
\coordinate (x28) at (-0.844574, -0.338160);
\coordinate (x29) at (-0.690079, -0.723734);
\coordinate (x30) at (-0.618159, -0.786053);
\coordinate (x31) at (-0.540641, -0.841254);
\coordinate (x32) at (-0.129512, -0.900491);
\coordinate (x33) at (0.281733, -0.959493);
\coordinate (x34) at (0.371662, -0.928368);
\coordinate (x35) at (0.458227, -0.888835);
\coordinate (x36) at (0.715092, -0.562406);
\coordinate (x37) at (0.971812, -0.235759);
\coordinate (x38) at (0.989821, -0.142315);
\coordinate (x39) at (0.998867, -0.047582);
\coordinate (x40) at (0.844604, 0.338085);
\coordinate (x41) at (0.690079, 0.723734);
\coordinate (x42) at (0.618159, 0.786053);
\coordinate (x43) at (0.540641, 0.841254);
\coordinate (x44) at (0.129432, 0.900502);
\coordinate (x45) at (-0.281733, 0.959493);
\coordinate (x46) at (-0.371662, 0.928368);
\coordinate (x47) at (-0.458227, 0.888835);
\coordinate (x48) at (-0.715142, 0.562342);
\coordinate (x49) at (-0.971812, 0.235759);
\coordinate (x50) at (-0.540641, 0.841254);
\coordinate (x51) at (-0.742622, 0.584088);
\coordinate (x52) at (-0.945001, 0.327068);
\coordinate (x53) at (-0.998867, -0.047582);
\coordinate (x54) at (-0.877098, -0.351207);
\coordinate (x55) at (-0.755750, -0.654861);
\coordinate (x56) at (-0.458227, -0.888835);
\coordinate (x57) at (-0.134523, -0.935174);
\coordinate (x58) at (0.189251, -0.981929);
\coordinate (x59) at (0.540641, -0.841254);
\coordinate (x60) at (0.742703, -0.583985);
\coordinate (x61) at (0.945001, -0.327068);
\coordinate (x62) at (0.998867, 0.047582);
\coordinate (x63) at (0.877146, 0.351086);
\coordinate (x64) at (0.755750, 0.654861);
\coordinate (x65) at (0.134394, 0.935192);
\coordinate (x66) at (-0.189251, 0.981929);
\coordinate (x67) at (0.458227, 0.888835);
\coordinate (x68) at (0.371662, 0.928368);
\coordinate (x69) at (0.281733, 0.959493);
\coordinate (x70) at (0.189251, 0.981929);
\coordinate (x71) at (0.095056, 0.995472);
\coordinate (x72) at (0.000000, 1.000000);
\coordinate (x73) at (-0.095056, 0.995472);
\coordinate (x74) at (-0.618159, 0.786053);
\coordinate (x75) at (-0.690079, 0.723734);
\coordinate (x76) at (-0.755750, 0.654861);
\coordinate (x77) at (-0.814576, 0.580057);
\coordinate (x78) at (-0.866025, 0.500000);
\coordinate (x79) at (-0.909632, 0.415415);
\coordinate (x80) at (-0.989821, -0.142315);
\coordinate (x81) at (-0.971812, -0.235759);
\coordinate (x82) at (-0.945001, -0.327068);
\coordinate (x83) at (-0.909632, -0.415415);
\coordinate (x84) at (-0.866025, -0.500000);
\coordinate (x85) at (-0.814576, -0.580057);
\coordinate (x86) at (-0.371662, -0.928368);
\coordinate (x87) at (-0.281733, -0.959493);
\coordinate (x88) at (-0.189251, -0.981929);
\coordinate (x89) at (-0.095056, -0.995472);
\coordinate (x90) at (-0.000000, -1.000000);
\coordinate (x91) at (0.095056, -0.995472);
\coordinate (x92) at (0.618159, -0.786053);
\coordinate (x93) at (0.690079, -0.723734);
\coordinate (x94) at (0.755750, -0.654861);
\coordinate (x95) at (0.814576, -0.580057);
\coordinate (x96) at (0.866025, -0.500000);
\coordinate (x97) at (0.909632, -0.415415);
\coordinate (x98) at (0.989821, 0.142315);
\coordinate (x99) at (0.971812, 0.235759);
\coordinate (x100) at (0.945001, 0.327068);
\coordinate (x101) at (0.909632, 0.415415);
\coordinate (x102) at (0.866025, 0.500000);
\coordinate (x103) at (0.814576, 0.580057);
\draw (0.575695, -0.197192) -- (0.556573, -0.078916);
\draw (0.556573, -0.078916) -- (0.456424, -0.065169);
\draw (0.456424, -0.065169) -- (0.411774, -0.321955);
\draw (0.411774, -0.321955) -- (0.575695, -0.197192);
\draw (0.575695, -0.197192) -- (0.755417, -0.108068);
\draw (0.755417, -0.108068) -- (0.608514, 0.028443);
\draw (0.608514, 0.028443) -- (0.556573, -0.078916);
\draw (0.608514, 0.028443) -- (0.486005, 0.193713);
\draw (0.486005, 0.193713) -- (0.456424, -0.065169);
\draw (0.486005, 0.193713) -- (0.448511, 0.482721);
\draw (0.448511, 0.482721) -- (0.182722, 0.346782);
\draw (0.182722, 0.346782) -- (0.092916, 0.112802);
\draw (0.092916, 0.112802) -- (0.057548, -0.133347);
\draw (0.057548, -0.133347) -- (0.077844, -0.383380);
\draw (0.077844, -0.383380) -- (0.294498, -0.589016);
\draw (0.294498, -0.589016) -- (0.411774, -0.321955);
\draw (0.092916, 0.112802) -- (-0.057365, 0.134413);
\draw (-0.057365, 0.134413) -- (-0.092752, -0.111739);
\draw (-0.092752, -0.111739) -- (0.057548, -0.133347);
\draw (-0.092752, -0.111739) -- (-0.182716, -0.345924);
\draw (-0.182716, -0.345924) -- (0.077844, -0.383380);
\draw (0.182722, 0.346782) -- (-0.077701, 0.384190);
\draw (-0.077701, 0.384190) -- (-0.057365, 0.134413);
\draw (-0.077701, 0.384190) -- (-0.294379, 0.589529);
\draw (-0.294379, 0.589529) -- (-0.411671, 0.322705);
\draw (-0.411671, 0.322705) -- (-0.456219, 0.065992);
\draw (-0.456219, 0.065992) -- (-0.485771, -0.192942);
\draw (-0.485771, -0.192942) -- (-0.448469, -0.482209);
\draw (-0.448469, -0.482209) -- (-0.182716, -0.345924);
\draw (-0.456219, 0.065992) -- (-0.556027, 0.081036);
\draw (-0.556027, 0.081036) -- (-0.607771, -0.026861);
\draw (-0.607771, -0.026861) -- (-0.485771, -0.192942);
\draw (-0.411671, 0.322705) -- (-0.575798, 0.198657);
\draw (-0.575798, 0.198657) -- (-0.556027, 0.081036);
\draw (-0.575798, 0.198657) -- (-0.755220, 0.109252);
\draw (-0.755220, 0.109252) -- (-0.607771, -0.026861);
\draw (-0.755220, 0.109252) -- (-0.989821, 0.142315);
\draw (-0.989821, 0.142315) -- (-0.998867, 0.047582);
\draw (-0.998867, 0.047582) -- (-0.844574, -0.338160);
\draw (-0.844574, -0.338160) -- (-0.690079, -0.723734);
\draw (-0.690079, -0.723734) -- (-0.618159, -0.786053);
\draw (-0.618159, -0.786053) -- (-0.448469, -0.482209);
\draw (-0.618159, -0.786053) -- (-0.540641, -0.841254);
\draw (-0.540641, -0.841254) -- (-0.129512, -0.900491);
\draw (-0.129512, -0.900491) -- (0.281733, -0.959493);
\draw (0.281733, -0.959493) -- (0.371662, -0.928368);
\draw (0.371662, -0.928368) -- (0.294498, -0.589016);
\draw (0.371662, -0.928368) -- (0.458227, -0.888835);
\draw (0.458227, -0.888835) -- (0.715092, -0.562406);
\draw (0.715092, -0.562406) -- (0.971812, -0.235759);
\draw (0.971812, -0.235759) -- (0.989821, -0.142315);
\draw (0.989821, -0.142315) -- (0.755417, -0.108068);
\draw (0.989821, -0.142315) -- (0.998867, -0.047582);
\draw (0.998867, -0.047582) -- (0.844604, 0.338085);
\draw (0.844604, 0.338085) -- (0.690079, 0.723734);
\draw (0.690079, 0.723734) -- (0.618159, 0.786053);
\draw (0.618159, 0.786053) -- (0.448511, 0.482721);
\draw (0.618159, 0.786053) -- (0.540641, 0.841254);
\draw (0.540641, 0.841254) -- (0.129432, 0.900502);
\draw (0.129432, 0.900502) -- (-0.281733, 0.959493);
\draw (-0.281733, 0.959493) -- (-0.371662, 0.928368);
\draw (-0.371662, 0.928368) -- (-0.294379, 0.589529);
\draw (-0.371662, 0.928368) -- (-0.458227, 0.888835);
\draw (-0.458227, 0.888835) -- (-0.715142, 0.562342);
\draw (-0.715142, 0.562342) -- (-0.971812, 0.235759);
\draw (-0.971812, 0.235759) -- (-0.989821, 0.142315);
\draw (-0.458227, 0.888835) -- (-0.540641, 0.841254);
\draw (-0.540641, 0.841254) -- (-0.742622, 0.584088);
\draw (-0.742622, 0.584088) -- (-0.715142, 0.562342);
\draw (-0.742622, 0.584088) -- (-0.945001, 0.327068);
\draw (-0.945001, 0.327068) -- (-0.971812, 0.235759);
\draw (-0.998867, 0.047582) -- (-0.998867, -0.047582);
\draw (-0.998867, -0.047582) -- (-0.877098, -0.351207);
\draw (-0.877098, -0.351207) -- (-0.844574, -0.338160);
\draw (-0.877098, -0.351207) -- (-0.755750, -0.654861);
\draw (-0.755750, -0.654861) -- (-0.690079, -0.723734);
\draw (-0.540641, -0.841254) -- (-0.458227, -0.888835);
\draw (-0.458227, -0.888835) -- (-0.134523, -0.935174);
\draw (-0.134523, -0.935174) -- (-0.129512, -0.900491);
\draw (-0.134523, -0.935174) -- (0.189251, -0.981929);
\draw (0.189251, -0.981929) -- (0.281733, -0.959493);
\draw (0.458227, -0.888835) -- (0.540641, -0.841254);
\draw (0.540641, -0.841254) -- (0.742703, -0.583985);
\draw (0.742703, -0.583985) -- (0.715092, -0.562406);
\draw (0.742703, -0.583985) -- (0.945001, -0.327068);
\draw (0.945001, -0.327068) -- (0.971812, -0.235759);
\draw (0.998867, -0.047582) -- (0.998867, 0.047582);
\draw (0.998867, 0.047582) -- (0.877146, 0.351086);
\draw (0.877146, 0.351086) -- (0.844604, 0.338085);
\draw (0.877146, 0.351086) -- (0.755750, 0.654861);
\draw (0.755750, 0.654861) -- (0.690079, 0.723734);
\draw (0.129432, 0.900502) -- (0.134394, 0.935192);
\draw (0.134394, 0.935192) -- (-0.189251, 0.981929);
\draw (-0.189251, 0.981929) -- (-0.281733, 0.959493);
\draw (0.540641, 0.841254) -- (0.458227, 0.888835);
\draw (0.458227, 0.888835) -- (0.134394, 0.935192);
\draw (0.458227, 0.888835) -- (0.371662, 0.928368);
\draw (0.371662, 0.928368) -- (0.281733, 0.959493);
\draw (0.281733, 0.959493) -- (0.189251, 0.981929);
\draw (0.189251, 0.981929) -- (0.095056, 0.995472);
\draw (0.095056, 0.995472) -- (0.000000, 1.000000);
\draw (0.000000, 1.000000) -- (-0.095056, 0.995472);
\draw (-0.095056, 0.995472) -- (-0.189251, 0.981929);
\draw (-0.540641, 0.841254) -- (-0.618159, 0.786053);
\draw (-0.618159, 0.786053) -- (-0.690079, 0.723734);
\draw (-0.690079, 0.723734) -- (-0.755750, 0.654861);
\draw (-0.755750, 0.654861) -- (-0.814576, 0.580057);
\draw (-0.814576, 0.580057) -- (-0.866025, 0.500000);
\draw (-0.866025, 0.500000) -- (-0.909632, 0.415415);
\draw (-0.909632, 0.415415) -- (-0.945001, 0.327068);
\draw (-0.998867, -0.047582) -- (-0.989821, -0.142315);
\draw (-0.989821, -0.142315) -- (-0.971812, -0.235759);
\draw (-0.971812, -0.235759) -- (-0.945001, -0.327068);
\draw (-0.945001, -0.327068) -- (-0.909632, -0.415415);
\draw (-0.909632, -0.415415) -- (-0.866025, -0.500000);
\draw (-0.866025, -0.500000) -- (-0.814576, -0.580057);
\draw (-0.814576, -0.580057) -- (-0.755750, -0.654861);
\draw (-0.458227, -0.888835) -- (-0.371662, -0.928368);
\draw (-0.371662, -0.928368) -- (-0.281733, -0.959493);
\draw (-0.281733, -0.959493) -- (-0.189251, -0.981929);
\draw (-0.189251, -0.981929) -- (-0.095056, -0.995472);
\draw (-0.095056, -0.995472) -- (-0.000000, -1.000000);
\draw (-0.000000, -1.000000) -- (0.095056, -0.995472);
\draw (0.095056, -0.995472) -- (0.189251, -0.981929);
\draw (0.540641, -0.841254) -- (0.618159, -0.786053);
\draw (0.618159, -0.786053) -- (0.690079, -0.723734);
\draw (0.690079, -0.723734) -- (0.755750, -0.654861);
\draw (0.755750, -0.654861) -- (0.814576, -0.580057);
\draw (0.814576, -0.580057) -- (0.866025, -0.500000);
\draw (0.866025, -0.500000) -- (0.909632, -0.415415);
\draw (0.909632, -0.415415) -- (0.945001, -0.327068);
\draw (0.998867, 0.047582) -- (0.989821, 0.142315);
\draw (0.989821, 0.142315) -- (0.971812, 0.235759);
\draw (0.971812, 0.235759) -- (0.945001, 0.327068);
\draw (0.945001, 0.327068) -- (0.909632, 0.415415);
\draw (0.909632, 0.415415) -- (0.866025, 0.500000);
\draw (0.866025, 0.500000) -- (0.814576, 0.580057);
\draw (0.814576, 0.580057) -- (0.755750, 0.654861);

\fill[black] (0.575695, -0.197192) circle (0.4pt);
\fill[black] (0.556573, -0.078916) circle (0.4pt);
\fill[black] (0.456424, -0.065169) circle (0.4pt);
\fill[black] (0.411774, -0.321955) circle (0.4pt);
\fill[black] (0.755417, -0.108068) circle (0.4pt);
\fill[black] (0.608514, 0.028443) circle (0.4pt);
\fill[black] (0.486005, 0.193713) circle (0.4pt);
\fill[black] (0.448511, 0.482721) circle (0.4pt);
\fill[black] (0.182722, 0.346782) circle (0.4pt);
\fill[black] (0.092916, 0.112802) circle (0.4pt);
\fill[black] (0.057548, -0.133347) circle (0.4pt);
\fill[black] (0.077844, -0.383380) circle (0.4pt);
\fill[black] (0.294498, -0.589016) circle (0.4pt);
\fill[black] (-0.057365, 0.134413) circle (0.4pt);
\fill[black] (-0.092752, -0.111739) circle (0.4pt);
\fill[black] (-0.182716, -0.345924) circle (0.4pt);
\fill[black] (-0.077701, 0.384190) circle (0.4pt);
\fill[black] (-0.294379, 0.589529) circle (0.4pt);
\fill[black] (-0.411671, 0.322705) circle (0.4pt);
\fill[black] (-0.456219, 0.065992) circle (0.4pt);
\fill[black] (-0.485771, -0.192942) circle (0.4pt);
\fill[black] (-0.448469, -0.482209) circle (0.4pt);
\fill[black] (-0.556027, 0.081036) circle (0.4pt);
\fill[black] (-0.607771, -0.026861) circle (0.4pt);
\fill[black] (-0.575798, 0.198657) circle (0.4pt);
\fill[black] (-0.755220, 0.109252) circle (0.4pt);
\fill[black] (-0.989821, 0.142315) circle (0.4pt);
\fill[black] (-0.998867, 0.047582) circle (0.4pt);
\fill[black] (-0.844574, -0.338160) circle (0.4pt);
\fill[black] (-0.690079, -0.723734) circle (0.4pt);
\fill[black] (-0.618159, -0.786053) circle (0.4pt);
\fill[black] (-0.540641, -0.841254) circle (0.4pt);
\fill[black] (-0.129512, -0.900491) circle (0.4pt);
\fill[black] (0.281733, -0.959493) circle (0.4pt);
\fill[black] (0.371662, -0.928368) circle (0.4pt);
\fill[black] (0.458227, -0.888835) circle (0.4pt);
\fill[black] (0.715092, -0.562406) circle (0.4pt);
\fill[black] (0.971812, -0.235759) circle (0.4pt);
\fill[black] (0.989821, -0.142315) circle (0.4pt);
\fill[black] (0.998867, -0.047582) circle (0.4pt);
\fill[black] (0.844604, 0.338085) circle (0.4pt);
\fill[black] (0.690079, 0.723734) circle (0.4pt);
\fill[black] (0.618159, 0.786053) circle (0.4pt);
\fill[black] (0.540641, 0.841254) circle (0.4pt);
\fill[black] (0.129432, 0.900502) circle (0.4pt);
\fill[black] (-0.281733, 0.959493) circle (0.4pt);
\fill[black] (-0.371662, 0.928368) circle (0.4pt);
\fill[black] (-0.458227, 0.888835) circle (0.4pt);
\fill[black] (-0.715142, 0.562342) circle (0.4pt);
\fill[black] (-0.971812, 0.235759) circle (0.4pt);
\fill[black] (-0.540641, 0.841254) circle (0.4pt);
\fill[black] (-0.742622, 0.584088) circle (0.4pt);
\fill[black] (-0.945001, 0.327068) circle (0.4pt);
\fill[black] (-0.998867, -0.047582) circle (0.4pt);
\fill[black] (-0.877098, -0.351207) circle (0.4pt);
\fill[black] (-0.755750, -0.654861) circle (0.4pt);
\fill[black] (-0.458227, -0.888835) circle (0.4pt);
\fill[black] (-0.134523, -0.935174) circle (0.4pt);
\fill[black] (0.189251, -0.981929) circle (0.4pt);
\fill[black] (0.540641, -0.841254) circle (0.4pt);
\fill[black] (0.742703, -0.583985) circle (0.4pt);
\fill[black] (0.945001, -0.327068) circle (0.4pt);
\fill[black] (0.998867, 0.047582) circle (0.4pt);
\fill[black] (0.877146, 0.351086) circle (0.4pt);
\fill[black] (0.755750, 0.654861) circle (0.4pt);
\fill[black] (0.134394, 0.935192) circle (0.4pt);
\fill[black] (-0.189251, 0.981929) circle (0.4pt);
\fill[black] (0.458227, 0.888835) circle (0.4pt);
\fill[black] (0.371662, 0.928368) circle (0.4pt);
\fill[black] (0.281733, 0.959493) circle (0.4pt);
\fill[black] (0.189251, 0.981929) circle (0.4pt);
\fill[black] (0.095056, 0.995472) circle (0.4pt);
\fill[black] (0.000000, 1.000000) circle (0.4pt);
\fill[black] (-0.095056, 0.995472) circle (0.4pt);
\fill[black] (-0.618159, 0.786053) circle (0.4pt);
\fill[black] (-0.690079, 0.723734) circle (0.4pt);
\fill[black] (-0.755750, 0.654861) circle (0.4pt);
\fill[black] (-0.814576, 0.580057) circle (0.4pt);
\fill[black] (-0.866025, 0.500000) circle (0.4pt);
\fill[black] (-0.909632, 0.415415) circle (0.4pt);
\fill[black] (-0.989821, -0.142315) circle (0.4pt);
\fill[black] (-0.971812, -0.235759) circle (0.4pt);
\fill[black] (-0.945001, -0.327068) circle (0.4pt);
\fill[black] (-0.909632, -0.415415) circle (0.4pt);
\fill[black] (-0.866025, -0.500000) circle (0.4pt);
\fill[black] (-0.814576, -0.580057) circle (0.4pt);
\fill[black] (-0.371662, -0.928368) circle (0.4pt);
\fill[black] (-0.281733, -0.959493) circle (0.4pt);
\fill[black] (-0.189251, -0.981929) circle (0.4pt);
\fill[black] (-0.095056, -0.995472) circle (0.4pt);
\fill[black] (-0.000000, -1.000000) circle (0.4pt);
\fill[black] (0.095056, -0.995472) circle (0.4pt);
\fill[black] (0.618159, -0.786053) circle (0.4pt);
\fill[black] (0.690079, -0.723734) circle (0.4pt);
\fill[black] (0.755750, -0.654861) circle (0.4pt);
\fill[black] (0.814576, -0.580057) circle (0.4pt);
\fill[black] (0.866025, -0.500000) circle (0.4pt);
\fill[black] (0.909632, -0.415415) circle (0.4pt);
\fill[black] (0.989821, 0.142315) circle (0.4pt);
\fill[black] (0.971812, 0.235759) circle (0.4pt);
\fill[black] (0.945001, 0.327068) circle (0.4pt);
\fill[black] (0.909632, 0.415415) circle (0.4pt);
\fill[black] (0.866025, 0.500000) circle (0.4pt);
\fill[black] (0.814576, 0.580057) circle (0.4pt);

        \end{scope}
        \\
      };
    \end{tikzfigure}
  \end{proof}
\end{theorem}

\begin{theorem}
 Let $p$ and $v$ be a pair of admissible sequences for a orientable closed $2$-manifold $S$. Then $(p, v)$ is $[(4k + 4) \times 4, 2 \times (4k+10)]$-$[3]$-realizable for all $k \in \nats$.
  \begin{proof}
    An expansion $3$-patch with outer tuple $o = (1, 1, 2, 1, 2, 2)$ and a corresponding $o$-$6$-gonal $3$-patch consisting only of quadrangles and decagons is shown in \autoref{fig:expansion:patch:4:10}. Using \autoref{const:edge:replacement:4:2} on the edges as indicated in both $3$-patches we get new $3$-patches consisting only of quadrangles and $4k + 10$-gons. Therefore we can apply \autoref{thm:main:const} with these patches together with the patch in \autoref{thm:expansion:patch:poly:4:k}.
    \begin{tikzfigure}{\label{fig:expansion:patch:4:10}}{\todo{better picture}}
      \matrix (m) [column sep=1cm] {
        \begin{scope}[yscale=0.866, scale=1.5]
          \node[anchor= 90] at (-0.5, 1) {$i_0$};
          \node[anchor= 45] at (-1, 2)   {$i_1$};
          \node[anchor= 45] at (-2, 2)   {$i_2=o_{0}$};
          \node[anchor=  0] at (-2.5, 3) {$o_1$};
          \node[anchor=300] at (-2, 4)   {$o_2$};
          \node[anchor=240] at (-1, 4)   {$o_3$};
          \node[anchor=240] at (-0.5, 3) {$o_4$};
          \node[anchor=240] at (0.5, 3)  {$i_{2}'=o_5$};
          \node[anchor=180] at (1, 2)    {$i_1'$};
          \node[anchor=120] at (0.5,1)   {$i_{0}'$};

          \fill[black] (-0.5, 1) circle(1pt);
          \fill[black] (-1, 2)   circle(1pt);
          \fill[black] (-2, 2)   circle(1pt);
          \fill[black] (-2.5, 3) circle(1pt);
          \fill[black] (-2, 4)   circle(1pt);
          \fill[black] (-1, 4)   circle(1pt);
          \fill[black] (-0.5, 3) circle(1pt);
          \fill[black] (0.5, 3)  circle(1pt);
          \fill[black] (1, 2)    circle(1pt);
          \fill[black] (0.5,1)   circle(1pt);
          \fill[black] (-0.75, 2.165)  circle(1pt);
          \fill[black] (-1, 2.25)   circle(1pt);
          \fill[black] (-1, 2.75)   circle(1pt);
          \fill[black] (-0.5, 2.25)   circle(1pt);
          \fill[black] (-0.5, 2.75)   circle(1pt);
          \fill[black] (-0.75, 2.835)  circle(1pt);
          \fill[black] (-0.75, 2.333)  circle(1pt);
          \fill[black] (-0.75, 2.666)  circle(1pt);

          \draw(-0.5,1)--(-1,2)--(-2,2)--(-2.5,3)--(-2,4)--(-1,4)--(-0.5,3)--(0.5,3)--(1,2)--(0.5,1)--(-0.5,1);
          \draw(-1,2)--(-0.75,2.165)--(-1,2.25)--(-1,2.75)--(-0.75,2.835)--(-0.5,2.75)--(-0.5,2.25)--(-0.75,2.165);
          \draw (-1,2.75)--(-0.75, 2.666)--(-0.5,2.75);
          \draw (-1,2.25)--(-0.75, 2.333)--(-0.5,2.25);
          \draw (-0.75,2.333)--(-0.75, 2.666);
          \draw (-0.5,3)--(-0.75, 2.835);         
 

        \end{scope}
        %  \begin{scope}[scale=3]
        %    \coordinate (x0) at (0.900969, -0.433884);
\coordinate (x1) at (0.529927, 0.030827);
\coordinate (x2) at (0.158789, 0.495608);
\coordinate (x3) at (-0.623490, 0.781831);
\coordinate (x4) at (-1.000000, 0.000000);
\coordinate (x5) at (-0.623490, -0.781831);
\coordinate (x6) at (0.222521, -0.974928);
\coordinate (x7) at (0.900969, 0.433884);
\coordinate (x8) at (0.222521, 0.974928);
\draw (0.900969, -0.433884) -- (0.529927, 0.030827);
\draw (0.529927, 0.030827) -- (0.158789, 0.495608);
\draw (0.158789, 0.495608) -- (-0.623490, 0.781831);
\draw (-0.623490, 0.781831) -- (-1.000000, 0.000000);
\draw (-1.000000, 0.000000) -- (-0.623490, -0.781831);
\draw (-0.623490, -0.781831) -- (0.222521, -0.974928);
\draw (0.222521, -0.974928) -- (0.900969, -0.433884);
\draw (0.900969, -0.433884) -- (0.900969, 0.433884);
\draw (0.900969, 0.433884) -- (0.158789, 0.495608);
\draw (0.900969, 0.433884) -- (0.222521, 0.974928);
\draw (0.222521, 0.974928) -- (-0.623490, 0.781831);

        %  \end{scope}
        &
        %  \begin{scope}[scale=3]
        %    \coordinate (x0) at (-0.489667, 0.574133);
\coordinate (x1) at (-0.582787, 0.425731);
\coordinate (x2) at (-0.460723, 0.497361);
\coordinate (x3) at (-0.342392, 0.574507);
\coordinate (x4) at (-0.503762, 0.739269);
\coordinate (x5) at (-0.639636, 0.713636);
\coordinate (x6) at (-0.781831, 0.623490);
\coordinate (x7) at (-0.866025, 0.500000);
\coordinate (x8) at (-0.930874, 0.365341);
\coordinate (x9) at (-0.974928, 0.222521);
\coordinate (x10) at (-0.997204, 0.074730);
\coordinate (x11) at (-0.715461, 0.275987);
\coordinate (x12) at (-0.442911, 0.449633);
\coordinate (x13) at (-0.169681, 0.622248);
\coordinate (x14) at (-0.433884, 0.900969);
\coordinate (x15) at (-0.563320, 0.826239);
\coordinate (x16) at (-0.680173, 0.733052);
\coordinate (x17) at (0.125157, 0.791453);
\coordinate (x18) at (0.149042, 0.988831);
\coordinate (x19) at (0.000000, 1.000000);
\coordinate (x20) at (-0.149042, 0.988831);
\coordinate (x21) at (-0.294755, 0.955573);
\coordinate (x22) at (0.433884, 0.900969);
\coordinate (x23) at (0.294755, 0.955573);
\coordinate (x24) at (-0.997204, -0.074730);
\coordinate (x25) at (-0.518526, 0.030249);
\coordinate (x26) at (0.046678, 0.379833);
\coordinate (x27) at (0.563320, 0.826239);
\coordinate (x28) at (-0.974928, -0.222521);
\coordinate (x29) at (-0.930874, -0.365341);
\coordinate (x30) at (-0.866025, -0.500000);
\coordinate (x31) at (-0.781831, -0.623490);
\coordinate (x32) at (-0.680173, -0.733052);
\coordinate (x33) at (-0.563320, -0.826239);
\coordinate (x34) at (-0.278180, -0.603149);
\coordinate (x35) at (-0.018597, -0.327896);
\coordinate (x36) at (0.107571, 0.005772);
\coordinate (x37) at (0.255728, -0.367789);
\coordinate (x38) at (0.381386, -0.119020);
\coordinate (x39) at (-0.433884, -0.900969);
\coordinate (x40) at (-0.273134, -0.819539);
\coordinate (x41) at (-0.294755, -0.955573);
\coordinate (x42) at (-0.149042, -0.988831);
\coordinate (x43) at (-0.000000, -1.000000);
\coordinate (x44) at (0.149042, -0.988831);
\coordinate (x45) at (0.294755, -0.955573);
\coordinate (x46) at (0.433884, -0.900969);
\coordinate (x47) at (0.387886, -0.629052);
\coordinate (x48) at (0.563320, -0.826239);
\coordinate (x49) at (0.583270, -0.671976);
\coordinate (x50) at (0.680173, -0.733052);
\coordinate (x51) at (0.781831, -0.623490);
\coordinate (x52) at (0.866025, -0.500000);
\coordinate (x53) at (0.930874, -0.365341);
\coordinate (x54) at (0.974928, -0.222521);
\coordinate (x55) at (0.997204, -0.074730);
\coordinate (x56) at (0.701510, -0.044622);
\coordinate (x57) at (0.849717, 0.142774);
\coordinate (x58) at (0.930874, 0.365341);
\coordinate (x59) at (0.866025, 0.500000);
\coordinate (x60) at (0.781831, 0.623490);
\coordinate (x61) at (0.680173, 0.733052);
\coordinate (x62) at (0.997204, 0.074730);
\coordinate (x63) at (0.974928, 0.222521);
\draw (-0.489667, 0.574133) -- (-0.582787, 0.425731);
\draw (-0.582787, 0.425731) -- (-0.460723, 0.497361);
\draw (-0.460723, 0.497361) -- (-0.342392, 0.574507);
\draw (-0.342392, 0.574507) -- (-0.489667, 0.574133);
\draw (-0.489667, 0.574133) -- (-0.503762, 0.739269);
\draw (-0.503762, 0.739269) -- (-0.639636, 0.713636);
\draw (-0.639636, 0.713636) -- (-0.781831, 0.623490);
\draw (-0.781831, 0.623490) -- (-0.866025, 0.500000);
\draw (-0.866025, 0.500000) -- (-0.930874, 0.365341);
\draw (-0.930874, 0.365341) -- (-0.974928, 0.222521);
\draw (-0.974928, 0.222521) -- (-0.997204, 0.074730);
\draw (-0.997204, 0.074730) -- (-0.715461, 0.275987);
\draw (-0.715461, 0.275987) -- (-0.582787, 0.425731);
\draw (-0.715461, 0.275987) -- (-0.442911, 0.449633);
\draw (-0.442911, 0.449633) -- (-0.460723, 0.497361);
\draw (-0.442911, 0.449633) -- (-0.169681, 0.622248);
\draw (-0.169681, 0.622248) -- (-0.342392, 0.574507);
\draw (-0.503762, 0.739269) -- (-0.433884, 0.900969);
\draw (-0.433884, 0.900969) -- (-0.563320, 0.826239);
\draw (-0.563320, 0.826239) -- (-0.639636, 0.713636);
\draw (-0.563320, 0.826239) -- (-0.680173, 0.733052);
\draw (-0.680173, 0.733052) -- (-0.781831, 0.623490);
\draw (-0.169681, 0.622248) -- (0.125157, 0.791453);
\draw (0.125157, 0.791453) -- (0.149042, 0.988831);
\draw (0.149042, 0.988831) -- (0.000000, 1.000000);
\draw (0.000000, 1.000000) -- (-0.149042, 0.988831);
\draw (-0.149042, 0.988831) -- (-0.294755, 0.955573);
\draw (-0.294755, 0.955573) -- (-0.433884, 0.900969);
\draw (0.125157, 0.791453) -- (0.433884, 0.900969);
\draw (0.433884, 0.900969) -- (0.294755, 0.955573);
\draw (0.294755, 0.955573) -- (0.149042, 0.988831);
\draw (-0.997204, 0.074730) -- (-0.997204, -0.074730);
\draw (-0.997204, -0.074730) -- (-0.518526, 0.030249);
\draw (-0.518526, 0.030249) -- (0.046678, 0.379833);
\draw (0.046678, 0.379833) -- (0.563320, 0.826239);
\draw (0.563320, 0.826239) -- (0.433884, 0.900969);
\draw (-0.997204, -0.074730) -- (-0.974928, -0.222521);
\draw (-0.974928, -0.222521) -- (-0.930874, -0.365341);
\draw (-0.930874, -0.365341) -- (-0.518526, 0.030249);
\draw (-0.930874, -0.365341) -- (-0.866025, -0.500000);
\draw (-0.866025, -0.500000) -- (-0.781831, -0.623490);
\draw (-0.781831, -0.623490) -- (-0.680173, -0.733052);
\draw (-0.680173, -0.733052) -- (-0.563320, -0.826239);
\draw (-0.563320, -0.826239) -- (-0.278180, -0.603149);
\draw (-0.278180, -0.603149) -- (-0.018597, -0.327896);
\draw (-0.018597, -0.327896) -- (0.107571, 0.005772);
\draw (0.107571, 0.005772) -- (0.046678, 0.379833);
\draw (-0.018597, -0.327896) -- (0.255728, -0.367789);
\draw (0.255728, -0.367789) -- (0.381386, -0.119020);
\draw (0.381386, -0.119020) -- (0.107571, 0.005772);
\draw (-0.563320, -0.826239) -- (-0.433884, -0.900969);
\draw (-0.433884, -0.900969) -- (-0.273134, -0.819539);
\draw (-0.273134, -0.819539) -- (-0.278180, -0.603149);
\draw (-0.433884, -0.900969) -- (-0.294755, -0.955573);
\draw (-0.294755, -0.955573) -- (-0.149042, -0.988831);
\draw (-0.149042, -0.988831) -- (-0.273134, -0.819539);
\draw (-0.149042, -0.988831) -- (-0.000000, -1.000000);
\draw (-0.000000, -1.000000) -- (0.149042, -0.988831);
\draw (0.149042, -0.988831) -- (0.294755, -0.955573);
\draw (0.294755, -0.955573) -- (0.433884, -0.900969);
\draw (0.433884, -0.900969) -- (0.387886, -0.629052);
\draw (0.387886, -0.629052) -- (0.255728, -0.367789);
\draw (0.433884, -0.900969) -- (0.563320, -0.826239);
\draw (0.563320, -0.826239) -- (0.583270, -0.671976);
\draw (0.583270, -0.671976) -- (0.387886, -0.629052);
\draw (0.563320, -0.826239) -- (0.680173, -0.733052);
\draw (0.680173, -0.733052) -- (0.781831, -0.623490);
\draw (0.781831, -0.623490) -- (0.583270, -0.671976);
\draw (0.781831, -0.623490) -- (0.866025, -0.500000);
\draw (0.866025, -0.500000) -- (0.930874, -0.365341);
\draw (0.930874, -0.365341) -- (0.974928, -0.222521);
\draw (0.974928, -0.222521) -- (0.997204, -0.074730);
\draw (0.997204, -0.074730) -- (0.701510, -0.044622);
\draw (0.701510, -0.044622) -- (0.381386, -0.119020);
\draw (0.701510, -0.044622) -- (0.849717, 0.142774);
\draw (0.849717, 0.142774) -- (0.930874, 0.365341);
\draw (0.930874, 0.365341) -- (0.866025, 0.500000);
\draw (0.866025, 0.500000) -- (0.781831, 0.623490);
\draw (0.781831, 0.623490) -- (0.680173, 0.733052);
\draw (0.680173, 0.733052) -- (0.563320, 0.826239);
\draw (0.997204, -0.074730) -- (0.997204, 0.074730);
\draw (0.997204, 0.074730) -- (0.849717, 0.142774);
\draw (0.997204, 0.074730) -- (0.974928, 0.222521);
\draw (0.974928, 0.222521) -- (0.930874, 0.365341);

        %  \end{scope}
        \begin{scope}[scale=1]
          \begin{scope}[yscale=0.866]
            \draw[very thick](-0.5,1)--(-1,2)--(-2,2)--(-2.5,3)--(-2,4)--(-1,4)--(-0.5,3)--(0.5,3)--(1,2)--(0.5,1)--(-0.5,1);
          \end{scope}
          \begin{scope}[rotate=-60,yscale=0.866]
            \draw[very thick](-0.5,1)--(-1,2)--(-2,2)--(-2.5,3)--(-2,4)--(-1,4)--(-0.5,3)--(0.5,3)--(1,2)--(0.5,1)--(-0.5,1);
          \end{scope}
          \begin{scope}[yscale=0.866,shift={(-1.5 cm,7 cm)},rotate=180]
            \draw[very thick](-0.5,1)--(-1,2)--(-2,2)--(-2.5,3)--(-2,4)--(-1,4)--(-0.5,3)--(0.5,3)--(1,2)--(0.5,1)--(-0.5,1);
          \end{scope}
          \begin{scope}[shift={(-1.5 cm,6.062 cm)},rotate=120,yscale=0.866]
            \draw[very thick](-0.5,1)--(-1,2)--(-2,2)--(-2.5,3)--(-2,4)--(-1,4)--(-0.5,3)--(0.5,3)--(1,2)--(0.5,1)--(-0.5,1);
          \end{scope}
        \end{scope}\\
        \begin{scope}[scale=3]
          \coordinate (x0) at (0.900969, -0.433884);
\coordinate (x1) at (0.529927, 0.030827);
\coordinate (x2) at (0.158789, 0.495608);
\coordinate (x3) at (-0.623490, 0.781831);
\coordinate (x4) at (-1.000000, 0.000000);
\coordinate (x5) at (-0.623490, -0.781831);
\coordinate (x6) at (0.222521, -0.974928);
\coordinate (x7) at (0.900969, 0.433884);
\coordinate (x8) at (0.222521, 0.974928);
\draw (0.900969, -0.433884) -- (0.529927, 0.030827);
\draw (0.529927, 0.030827) -- (0.158789, 0.495608);
\draw (0.158789, 0.495608) -- (-0.623490, 0.781831);
\draw (-0.623490, 0.781831) -- (-1.000000, 0.000000);
\draw (-1.000000, 0.000000) -- (-0.623490, -0.781831);
\draw (-0.623490, -0.781831) -- (0.222521, -0.974928);
\draw (0.222521, -0.974928) -- (0.900969, -0.433884);
\draw (0.900969, -0.433884) -- (0.900969, 0.433884);
\draw (0.900969, 0.433884) -- (0.158789, 0.495608);
\draw (0.900969, 0.433884) -- (0.222521, 0.974928);
\draw (0.222521, 0.974928) -- (-0.623490, 0.781831);

        \end{scope}
        &        
        \begin{scope}[scale=3]
          \coordinate (x0) at (-0.365385, 0.717713);
\coordinate (x1) at (-0.406737, 0.913545);
\coordinate (x2) at (-0.587785, 0.809017);
\coordinate (x3) at (-0.470998, 0.659679);
\coordinate (x4) at (-0.283612, 0.537937);
\coordinate (x5) at (-0.159018, 0.381951);
\coordinate (x6) at (0.097077, 0.360698);
\coordinate (x7) at (0.297413, 0.535869);
\coordinate (x8) at (0.406737, 0.913545);
\coordinate (x9) at (0.207912, 0.978148);
\coordinate (x10) at (0.000000, 1.000000);
\coordinate (x11) at (-0.207912, 0.978148);
\coordinate (x12) at (-0.397680, 0.485286);
\coordinate (x13) at (-0.376360, 0.293964);
\coordinate (x14) at (0.143888, 0.123784);
\coordinate (x15) at (0.276362, 0.178024);
\coordinate (x16) at (0.144049, -0.122644);
\coordinate (x17) at (0.276426, -0.177430);
\coordinate (x18) at (0.097642, -0.359681);
\coordinate (x19) at (0.297616, -0.535539);
\coordinate (x20) at (-0.549960, 0.108715);
\coordinate (x21) at (-0.549754, -0.108214);
\coordinate (x22) at (-0.375875, -0.293275);
\coordinate (x23) at (-0.158333, -0.380911);
\coordinate (x24) at (-0.772107, 0.074295);
\coordinate (x25) at (-0.772023, -0.074053);
\coordinate (x26) at (-0.743145, 0.669131);
\coordinate (x27) at (-0.866025, 0.500000);
\coordinate (x28) at (-0.951057, 0.309017);
\coordinate (x29) at (-0.994522, 0.104528);
\coordinate (x30) at (-0.994522, -0.104528);
\coordinate (x31) at (-0.397102, -0.484666);
\coordinate (x32) at (-0.282908, -0.537009);
\coordinate (x33) at (-0.470672, -0.659283);
\coordinate (x34) at (-0.364844, -0.717213);
\coordinate (x35) at (-0.951057, -0.309017);
\coordinate (x36) at (-0.866025, -0.500000);
\coordinate (x37) at (-0.743145, -0.669131);
\coordinate (x38) at (-0.587785, -0.809017);
\coordinate (x39) at (-0.406737, -0.913545);
\coordinate (x40) at (-0.207912, -0.978148);
\coordinate (x41) at (-0.000000, -1.000000);
\coordinate (x42) at (0.207912, -0.978148);
\coordinate (x43) at (0.406737, -0.913545);
\coordinate (x44) at (0.587785, -0.809017);
\coordinate (x45) at (0.590971, -0.267400);
\coordinate (x46) at (0.590963, 0.267438);
\coordinate (x47) at (0.587785, 0.809017);
\coordinate (x48) at (0.619285, -0.087671);
\coordinate (x49) at (0.619165, 0.088090);
\coordinate (x50) at (0.708487, -0.036660);
\coordinate (x51) at (0.708521, 0.039158);
\coordinate (x52) at (0.806694, -0.030935);
\coordinate (x53) at (0.806761, 0.028476);
\coordinate (x54) at (0.905832, -0.043686);
\coordinate (x55) at (0.906158, 0.040967);
\coordinate (x56) at (0.743145, -0.669131);
\coordinate (x57) at (0.866025, -0.500000);
\coordinate (x58) at (0.951057, -0.309017);
\coordinate (x59) at (0.994522, -0.104528);
\coordinate (x60) at (0.994522, 0.104528);
\coordinate (x61) at (0.951057, 0.309017);
\coordinate (x62) at (0.866025, 0.500000);
\coordinate (x63) at (0.743145, 0.669131);
\draw (x0) -- (x1);
\draw (x1) -- (x2);
\draw (x2) -- (x3);
\draw (x3) -- (x0);
\draw (x0) -- (x4);
\draw (x4) -- (x5);
\draw (x5) -- (x6);
\draw (x6) -- (x7);
\draw (x7) -- (x8);
\draw (x8) -- (x9);
\draw (x9) -- (x10);
\draw (x10) -- (x11);
\draw (x11) -- (x1);
\draw (x3) -- (x12);
\draw (x12) -- (x4);
\draw (x12) -- (x13);
\draw (x13) -- (x5);
\draw (x6) -- (x14);
\draw (x14) -- (x15);
\draw (x15) -- (x7);
\draw (x14) -- (x16);
\draw (x16) -- (x17);
\draw (x17) -- (x15);
\draw (x16) -- (x18);
\draw (x18) -- (x19);
\draw (x19) -- (x17);
\draw (x13) -- (x20);
\draw (x20) -- (x21);
\draw (x21) -- (x22);
\draw (x22) -- (x23);
\draw (x23) -- (x18);
\draw (x20) -- (x24);
\draw (x24) -- (x25);
\draw (x25) -- (x21);
\draw (x2) -- (x26);
\draw (x26) -- (x27);
\draw (x27) -- (x28);
\draw (x28) -- (x29);
\draw (x29) -- (x24);
\draw (x29) -- (x30);
\draw (x30) -- (x25);
\draw (x22) -- (x31);
\draw (x31) -- (x32);
\draw (x32) -- (x23);
\draw (x31) -- (x33);
\draw (x33) -- (x34);
\draw (x34) -- (x32);
\draw (x30) -- (x35);
\draw (x35) -- (x36);
\draw (x36) -- (x37);
\draw (x37) -- (x38);
\draw (x38) -- (x33);
\draw (x38) -- (x39);
\draw (x39) -- (x34);
\draw (x39) -- (x40);
\draw (x40) -- (x41);
\draw (x41) -- (x42);
\draw (x42) -- (x43);
\draw (x43) -- (x19);
\draw (x43) -- (x44);
\draw (x44) -- (x45);
\draw (x45) -- (x46);
\draw (x46) -- (x47);
\draw (x47) -- (x8);
\draw (x45) -- (x48);
\draw (x48) -- (x49);
\draw (x49) -- (x46);
\draw (x48) -- (x50);
\draw (x50) -- (x51);
\draw (x51) -- (x49);
\draw (x50) -- (x52);
\draw (x52) -- (x53);
\draw (x53) -- (x51);
\draw (x52) -- (x54);
\draw (x54) -- (x55);
\draw (x55) -- (x53);
\draw (x44) -- (x56);
\draw (x56) -- (x57);
\draw (x57) -- (x58);
\draw (x58) -- (x59);
\draw (x59) -- (x54);
\draw (x59) -- (x60);
\draw (x60) -- (x55);
\draw (x60) -- (x61);
\draw (x61) -- (x62);
\draw (x62) -- (x63);
\draw (x63) -- (x47);
\node at (-0.562446, 0.182750) {0};
\node at (-0.457726, 0.774989) {1};
\node at (-0.041352, 0.731755) {2};
\node at (-0.379419, 0.600154) {3};
\node at (-0.304168, 0.424784) {4};
\node at (0.203685, 0.299594) {5};
\node at (0.210181, 0.000433) {6};
\node at (0.203933, -0.298824) {7};
\node at (-0.168664, 0.000439) {8};
\node at (-0.660961, 0.000186) {9};
\node at (-0.670964, 0.401363) {10};
\node at (-0.883293, 0.000060) {11};
\node at (-0.303555, -0.423965) {12};
\node at (-0.378882, -0.599543) {13};
\node at (-0.670796, -0.401118) {14};
\node at (-0.457510, -0.774765) {15};
\node at (-0.041083, -0.731374) {16};
\node at (0.431879, 0.000096) {17};
\node at (0.605096, 0.000114) {18};
\node at (0.663864, 0.000729) {19};
\node at (0.757616, 0.000010) {20};
\node at (0.856362, -0.001295) {21};
\node at (0.777380, -0.285805) {22};
\node at (0.950259, -0.000680) {23};
\node at (0.777410, 0.285582) {24};

        \end{scope}
        \\
      };
    \end{tikzfigure}
  \end{proof}
\end{theorem}
