\mysection{Closing remarks}\label{sec:closing:remarks}

We have seen that in our setting, $[q_s \times s, q_l \times l]$-$[r]$-realization with $s < l$, $\gcd(q_s, q_l) = 1$ we could establish all {\sc Eberhard}-like theorems which realize all admissible sequences $p$ and $q$. The most interesting question emerging from this work is of course what the situation is in the cases where some admissible sequences are not $[q_s \times s, q_l \times l]$-$[r]$-realizable. 
\clearpage
We want to propose the following conjectures, which state that $q$-$w$-realization only fails on the sphere $\sphere^2$ with a single exceptional $k$-gon or a single exceptional $k$-valent vertex:

\begin{conjecture}
  Let $(s, r) \in \set{(3, 4), (3, 5), (4, 3), (5, 3)}$, $l = s \cdot c$ for some $c \in \nats$ and \linebreak $q = [q_s \times s, q_l \times l]$ and $w = [r]$ be flat with $\gcd(q_s, q_l) = 1$. Let $p = (p_3, \dots, p_m)$ and $q = (q_3, \dots, q_m)$ be admissible sequences for a closed $2$-manifold $S \not\cong \sphere^2$. Then $(p, v)$ is $q$-$w$-realizable on $S$.
\end{conjecture}

\begin{conjecture}
  Let $(s, r) \in \set{(3, 4), (3, 5), (4, 3), (5, 3)}$, $l = s \cdot c$ for some $c \in \nats$ and \linebreak $q = [q_s \times s, q_l \times l]$ and $w = [r]$ be flat with $\gcd(q_s, q_l) = 1$. Let $p = (p_3, \dots, p_m)$ and $q = (q_3, \dots, q_m)$ be admissible sequences on the sphere $\sphere^2$. If neither
  \begin{align*}
    \sum_{k = 3 \atop k \neq s, ~k \neq l}^m p_k = 1, \qquad&\sum_{k = 3 \atop k \neq r}^n v_k = 0, \quad \text{~nor} \\
    \sum_{k = 3 \atop k \neq s, ~k \neq l}^m p_k = 0, \qquad&\sum_{k = 3 \atop k \neq r}^n v_k = 1
  \end{align*}
then $(p, v)$ is $q$-$w$-realizable on $\sphere^2$.
\end{conjecture}

Which of those sequences excluded in the last conjecture are actually realizable is a difficult question. We have seen a number of examples where we could show that they are not realizable, but the author assumes that there could be more parity conditions missing.

An oddity we found is the case $(r, s) = (3, 3)$. Proving an {\sc Eberhard} theorem for $l \geq 13$ is probably as hard as determining the set of all $p$-vectors and $v$-vectors of polyhedral maps. If $l = 11$ or $l = 12$ we have problems with spacing the triangles far enough away so that we can insert all of them without creating \autoref{fig:adjacent:triangles}. Usually those things work better if $\sum_{k \geq 4} 2k \cdot v_k + \sum_{k \geq 4} \floor{\tfrac{k}{2}} p_k \gg 3p_3$ (we have fewer restrictions to place our triangles) or if we have large $p$- and $v$-vectors (we have more possibilities to place other polygons).

Another interesting question would be determining the set $C(p, v, q, w, S)$ we defined in the previous section. Note that by \eqref{eq:handshake} only the value of $c$ in each pair $(c, d) \in C(p, v, q, w, S)$ is needed, since it uniquely defines $d$. In view of the classical {\sc Eberhard} theorems \cite{jendrol1993face} and \cite{jucovivc1976face} established for almost all sequences $p$ and $v$ which values of $c$ the pairs $(c, d)$ in the sets $C(p, v, [6], [3], S)$ and $C(p, v, [4], [4], S)$ contain after some threshold. They found that there is a parity condition on $c$, either these sets are empty, they contain a pair $(c, d)$ for each $c$ greater than some limit depending on $p$ and $v$, or they contain either all even or all odd values of $c$ after that threshold. In light of this, knowing those parity conditions on the pairs $(c, d)$ in the set $C(p, v, [q_s \times s, q_l \times l], [r], S)$ for $\gcd(q_s, q_l) = 1$ would also determine the set $C(p, v, [q_s \times s, q_l \times l], [r], S)$ when the restriction $\gcd(q_s, q_l) = 1$ is lifted. For example: if $C(p, v, [q_s \times s, q_l \times l], [r], S)$ contains only pairs $(c, d)$ with odd values of $c$, then $C(p, v, 2 \cdot [q_s \times s, q_l \times l], [r], S)$ is empty.

A final question we would like to raise is extending our results to non-orientable closed $2$-manifolds. Our construction scheme works well on orientable manifolds, ceases to work for non-orientable ones. This is because of \autoref{const:map}, where we choose an orientation to fit all patches $\mathcal{P}(f)$ together. If we do not have an orientation we have to have stricter requirements on $\mathcal{P}(f)$, i.e. that the parts of the boundary not only fit together, but that they fit together even when mirrored. This requirement is quite strong and is unfortunately only possible in expansion $r$-patches with an outer tuple $(\frac{r}{2}, \frac{r}{2}, \dots, \frac{r}{2})$, thus only in the case $r = 4$. Another difficulty is that \autoref{thm:eberhard:extended:3} and \autoref{thm:eberhard:extended:4} are only stated for orientable $2$-manifolds. It seems therefore more reasonable to attack this problem with additional replacement patches, an approach also taken in \cite{devos2010eberhard}. We could for example use a patch with a hole in it, where we can identify the edges on the boundary of the hole to form a cross-cap. We could then add enough of these patches into map using the normal construction step in \autoref{const:map} and close the holes to form the cross-caps which creates a map on any non-orientable surface.
