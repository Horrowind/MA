\usepackage[utf8]{inputenc}
%\usepackage[ngerman]{babel}
\usepackage[english]{babel}
\usepackage{hyperref}
\usepackage{amsmath}
\usepackage{amsthm}
\usepackage{amssymb}
\usepackage{mathrsfs}
\usepackage{amstext}
\usepackage{amsfonts}
\usepackage{mathtools}
\usepackage{enumitem}
\usepackage{chngcntr}
\usepackage{caption}
\usepackage{setspace}
\usepackage{array}
\usepackage{graphicx}
\usepackage{makeidx}
\usepackage{aliascnt}
%\usepackage[super,square]{natbib}
\usepackage[labelfont=bf]{caption}
\usepackage[inner=2cm,outer=2cm,top=1.5cm,bottom=1.5cm,includeheadfoot]{geometry}
\counterwithin{figure}{section}

%\usepackage{extarrows}

\usepackage{tikz}
\usetikzlibrary{%
    shapes.geometric,
    matrix,
    intersections,
    positioning
}
\usepackage{shapepar}

\setlist{nosep}
\setlength{\parindent}{0pt}
\renewcommand{\baselinestretch}{1.2}
\theoremstyle{definition}

% figure enviroments
\captionsetup[FLOAT_TYPE]{labelformat=simple, labelsep=none}
\newenvironment{tikzfigure}[2]
    {\begin{figure}[h!] \caption{#2} #1 \centering  \begin{tikzpicture}[line cap=round, line join=round]}
    {\end{tikzpicture} \end{figure}}

\newcommand{\todo}[1]{(TODO)}

\newcommand{\newthm}[3]{%
  \newaliascnt{#2}{#1}%
  \newtheorem{#2}[#2]{#3}%
  \aliascntresetthe{#2}%
  \expandafter\def\csname#2autorefname\endcsname{#3}%
}



% math environments
\newtheorem{theorem}{Theorem}[section]
\newthm{theorem}{lemma}{Lemma}
\newthm{theorem}{problem}{Problem}
\newthm{theorem}{proposition}{Proposition}
\newthm{theorem}{notation}{Notation}
\newthm{theorem}{example}{Example}
\newthm{theorem}{remark}{Remark}
\newthm{theorem}{conjecture}{Conjecture}
\newthm{theorem}{definition}{Definition}
\newthm{theorem}{corollary}{Corollary}
\newthm{theorem}{construction}{Construction}

\newlength{\tempitemindent}

\newenvironment{cinput}
    {\leavevmode \par \begin{minipage}[t]{.1\textwidth} {\it Input: } \end{minipage}\begin{minipage}[t]{.9\textwidth} \begin{itemize} }
    {\end{itemize} \vspace{0.2em} \end{minipage}}
\newenvironment{coutput}
    {\leavevmode \par \begin{minipage}[t]{.1\textwidth} {\it Output: } \end{minipage}\begin{minipage}[t]{.9\textwidth} \begin{itemize} }
    {\setlength{\global\tempitemindent}{\leftmargin}\vspace{0.2em}\end{itemize} \end{minipage}}
\newenvironment{cdescription}
    {\leavevmode \par \begin{minipage}[t]{\dimexpr .1\textwidth + \tempitemindent}{\it Description: }\end{minipage}\ignorespaces}
    {}

\renewcommand*{\theenumi}{\thetheorem\,(\roman{enumi})}%
\renewcommand*{\labelenumi}{\thetheorem\,(\roman{enumi})}%
\newcommand{\defeq}{\vcentcolon=}
\newcommand{\eqdef}{=\vcentcolon}
% Operatoren

\newcommand{\piece}[1]{\textcircled{\tiny{#1}}}
%\newcommand{\deg}{\operatorname{deg}}
\renewcommand{\gcd}{\operatorname{gcd}}
\newcommand{\set}[1]{\{ #1 \}}
\newcommand{\nats}{\mathbb{N}}
\newcommand{\wholes}{\mathbb{Z}}
\newcommand{\reals}{\mathbb{R}}
\newcommand{\sphere}{\mathbb{S}}
\newcommand{\floor}[1]{\ensuremath{\left \lfloor #1 \right \rfloor}}
\newcommand{\ceil}[1]{\ensuremath{\left \lceil #1 \right \rceil}}
\renewcommand{\mod}{\operatorname{mod}}
\newcommand{\cyc}[1]{{#1}_{\operatorname{cyc}}}
\newcommand{\cyctup}[1]{{(#1)}_{\operatorname{cyc}}}

\newcommand{\eps}{\varepsilon}
\renewcommand{\phi}{\varphi}

% Things for equations  
\numberwithin{equation}{section}
\setcounter{subsection}{0}

%\renewcommand{\labelenumi}{(\alph{enumi})}

% Rewriting
\newcommand{\rewrite}{\rightarrow}
\newcommand{\cycsim}{\stackrel{\circ}{\sim}}
\newcommand{\srewrite}[1]{\stackrel{#1}{\rewrite}}

