\mysection{Construction}\label{sec:construction}

Let $r \in \nats$. Within this section we think of $r$ as valence of those vertices we are free to add to a polyhedral map. Many constructions in this thesis will utilize the concept of replacing each face of a polyhedral map with a larger patch. It can be quite challenging to see whether the resulting structures fit as well together as the previous faces did. This section introduces the necessary formalism for these kinds of constructions. Note that throughout this section we allow $2$-valent vertices in special maps we call patches.
\clearpage
\begin{definition}[Patch] For a planar graph embedding we want to call the single unbounded face the \hdef{outer face} and every other face \hdef{inner face}. The \hdef{boundary $\partial M$ of a planar map $M$} is the induces subgraph of all vertices and edges adjacent to the outer face. A map $\mathcal{P}$ on the Euclidean plane with potentially $2$-valent vertices on $\partial\mathcal{P}$ is called a \hdef{patch}. It is called an \hdef{$r$-patch}, if each of its vertices except the ones on the outer face is $r$-valent, while for the valence $\deg(v)$ of a vertex $v$ on the outer face $2 \leq \deg(v) \leq r$ holds. The \hdef{$p$-vector of a patch} is the sequence $(p_3, p_4, \dots)$, where $p_k$ denotes the number of $k$-gonal inner faces of the patch. 
\end{definition}

Note that the union of the set $F_{i}$ of inner faces of a patch is homeomorphic to a $2$-cell (since $\bigcup_{f \in F_{i}} f \cong \mathbb{E}^2 \setminus \operatorname{int}(f_{o}) \cong (\sphere^2 \setminus \set{(0, 0, 1)}) \setminus \operatorname{int}(\psi^{-1}(f_{o}))$, which is a $2$-cell because $\psi^{-1}(f_{o})$ is a $2$-cell, where $f_{o}$ is the outer face and $\psi$ is the stereographic projection). From that we see that $\partial\mathcal{P}$ is a cyclic graph.

\begin{definition} Let $\mathcal{P}$ be an $r$-patch with vertex set $V$. Define $w_{\mathcal{P}}$ to be the weight function $w_{\mathcal{P}} : V \cap \partial\mathcal{P} \to \nats, v \mapsto \deg(v) - 1$, which assigns to every boundary vertex the number of inner faces adjacent to it.
\end{definition}

If one is interested in whether two patches of a construction fit together to assemble a larger patch one has to compare the valences along some path on the boundary of each part. This is captured in the following definition:
\begin{definition}
  A tuple $(w_1, \dots, w_n) \in \nats^n$ is called to \hdef{fit} to another tuple $(w'_1, \dots, w'_n) \in \nats^n$, if $w_i + w'_{n+1-i} = r$ holds for all $i \in \set{1, \dots, n}$. Additionally, two patches $\mathcal{P}$ and $\mathcal{P}'$ are said to fit together along two paths $v_1 - v_2 - \dots - v_n \subseteq \partial\mathcal{P}$, $v'_1 - v'_2 - \dots - v'_n \subseteq \partial\mathcal{P'}$, if $(w_{\mathcal{P}}(v_1), \dots, w_{\mathcal{P}}(v_n))$ fits to $(w_{\mathcal{P'}}(v'_1), \dots, w_{\mathcal{P'}}(v'_n))$. A single tuple $(w_1, \dots, w_n)$ (or a single path $v_1 - v_2 - \dots - v_n \subseteq \partial\mathcal{P}$) is called \hdef{self-fitting}, if $(w_1, \dots, w_n)$ (resp. $v_1 - v_2 - \dots - v_n$) fits with itself.
\end{definition}

\begin{lemma}\label{thm:fitting:tuples}
  Let $\mathcal{P}, \mathcal{P}'$ be two $r$-patches and $v_0 - v_1 - \dots - v_{k + 1} \subseteq \partial\mathcal{P}$, $v'_0 - v'_1 - \dots - v'_{k + 1} \subseteq \partial\mathcal{P}$ be two paths along the boundary of the respective patch. If $w_{\mathcal{P}}(v_0) + w_{\mathcal{P}}(v'_{k + 1}) < r$, $w_{\mathcal{P}}(v_{k+1}) + w_{\mathcal{P}}(v'_0) < r$ and $v_1 - \dots - v_k$ fits to $v'_1 - \dots v'_k$, then there exists an $r$-patch which has $\mathcal{P}$ and $\mathcal{P}'$ as subgraphs and identifies $v_{i} = v'_{k + 1 - i}$, $0 \leq i \leq k + 1$ but not any other two vertices.
  \begin{proof}
    By ``gluing'' $\mathcal{P}$ and $\mathcal{P}'$ along the paths $v_0 - \dots - v_{k+1}$ and $v'_{0} - \dots - v'_{k+1}$, that is, identifying $v_{i}$ with $v'_{k + 1 - i}$, $0 \leq i \leq k + 1$, the resulting map satisfies the requirements, see \autoref{fig:patch:example}. It is a map into the Euclidean plane since both $\mathcal{P}$ and $\mathcal{P}'$ are and both graphs were combined along the outer face, so we topologically attached two $2$-cells along their boundary. Let $\tilde{v}_i$ be the vertex in the new graph which represents $v_{i}$ and $v_{k + 1 - i}$ from the old graphs. By construction $\tilde{v}_i$ is now an inner vertex if $1 \leq i \leq k$ and a vertex on the outer face if $i=0$ or $i = k + 1$. When gluing $v_{i}$ and $v_{i + k - 1}$ ($1 \leq i \leq k $) together, the edges $v_{i}$--$v_{i+1}$ and $v'_{k+1 - i}$--$v'_{k - i}$ as well as the edges $v_{i}$--$v_{i - 1}$ and $v'_{k+1 - i}$--$v'_{k + 2 - i}$ coincide. Therefore the valence of $\tilde{v}_k$ is two fewer than the sum of valences of $v_{i}$ in $\mathcal{P}$ and $v'_{k + 1 - i}$ in $\mathcal{P}'$, thus
    \begin{align*}
      \deg(\tilde{v}_k) = \deg(v_{i}) + \deg(v'_{k + 1 - i}) - 2 &={} (w_{\mathcal{P}}(v_i) + 1) + (w_{\mathcal{P'}}(v'_{k + 1 - i}) + 1) - 2 \\
      &={} w_{\mathcal{P}}(v_i) + w_{\mathcal{P'}}(v'_{k + 1 - i}) = r.
    \end{align*}
    The edges $v_{0}$--$v_{1}$ and $v'_{k+1}$--$v'_{k}$ are now glued together, it follows $\deg(\tilde{v}_0) = \deg(v_0) + \deg(v'_{k + 1}) - 1 = (w_{\mathcal{P}}(v_i) + 1) + (w_{\mathcal{P'}}(v'_{k + 1 - i}) + 1) - 1 = w_{\mathcal{P}}(v_i) + w_{\mathcal{P'}}(v'_{k + 1 - i}) + 1 \leq r$. Analogously $\deg(\tilde{v}_{k + 1}) \leq r$ holds. This proves that the new map is indeed an $r$-patch.

    \begin{tikzfigure}{\label{fig:patch:example}}{Gluing two patches together (for $r = 3$)}
      \draw (-1, -2) -- (1, -2);
      \draw (0, -2) -- (0, -1.5);
      \draw (0, 2) -- (0, 1.5);
      \draw (-1, 2) -- (1, 2);
      \draw (0, -0.75) -- (0.25, -0.5) -- (-0.25, 0) -- (0.25, 0.5) -- (0, 0.75);


      \draw[loosely dotted] (-1, -2) -- (-2, -1.5);
      \draw[loosely dotted] (1, -2) -- (2, -1.5);
      \draw[loosely dotted] (-1, 2) -- (-2, 1.5);
      \draw[loosely dotted] (1, 2) -- (2, 1.5);
      \draw (0.25, -0.5) -- (0.75, -0.5);
      \draw (-0.25, 0) -- (-0.75, 0);
      \draw (0.25, 0.5) -- (0.75, 0.5);
      \draw[loosely dotted] (0, -1.5) -- (0, -0.75);
      \draw[loosely dotted] (0, 1.5) -- (0, 0.75);

      \node at (0, -2) [anchor=north] {$v_0=v'_{k+1}$};
      \fill [black] (0, -2) circle (2pt);
      \node at (0, 2) [anchor=south] {$v_{k+1}=v'_{o}$};
      \fill [black] (0, 2) circle (2pt);
      \node at (0.25, -0.5) [anchor=east, inner sep=7pt] {$v_{i - 1}=v'_{k - i + 2}$};
      \fill [black] (0.25, -0.5) circle (2pt);
      \node at (-0.25, 0) [anchor=west, inner sep=7pt] {$v_{i}=v'_{k - i + 1}$};
      \fill [black] (-0.25, 0) circle (2pt);
      \node at (0.25, 0.5) [anchor=east, inner sep=7pt] {$v_{i + 1}=v'_{k - i}$};
      \fill [black] (0.25, 0.5) circle (2pt);

      \node at (-0.5, -1.5) [anchor=east] {$\mathcal{P}$};
      \node at (0.5, -1.5) [anchor=west] {$\mathcal{P'}$};
    \end{tikzfigure}
  \end{proof}
\end{lemma}

\begin{definition}
  Let $w = (w_1, \dots, w_n) \in \nats^n$ be a tuple. A \hdef{$w$-expansion} of an $r$-patch $\mathcal{P}$ with boundary $\partial\mathcal{P} = v_1 - v_2 - \dots - \dots v_m - v_1$ is another $r$-patch $\mathcal{P'}$ with boundary
\begin{multline*}
  \partial\mathcal{P'} = {v'}_1  - \underparenthesis{{v'}_1^{(1)} - \dots - {v'}_n^{(1)}} - {v'}_2 - \underparenthesis{{v'}_1^{(2)} - \dots - {v'}_n^{(2)}} - \quad\dots \\
  - {v'}_m  -  \underparenthesis{{v'}_1^{(m)} - \dots - {v'}_n^{(m)}}  -  {v'}_1,
\end{multline*}
such that $w_{\mathcal{P}}(v_i) = w_{\mathcal{P'}}(v'_i)$ and $w_{\mathcal{P'}}({v'}_j^{(i)}) = w_j$ for all $i \in \set{1, \dots, m}$, $j \in \set{1, \dots, n}$. We want to call the vertices $v'_i$ \hdef{corner vertices} and the vertices ${v'}_j^{(i)}$ \hdef{side vertices}. Furthermore, a patch is called \hdef{$w$-$k$-gonal}, if it is the $w$-expansion of the patch consisting of only a $k$-gon, i.e. if $w_{\mathcal{P'}}(v'_i) = 1$ for $i \in \set{1, \dots, m}$.
\end{definition}

Using this notation we describe the following construction scheme:

\begin{construction}\label{const:map}
  \begin{cinput}
  \item A map $M$ on a closed orientable $2$-manifold $S$ with $p$-vector $p$, $v$-vector $v$, underlying graph $G = (V, E)$ and faces $F$. 
  \item A self-fitting tuple $w = (w_1, \dots, w_n)$
  \item For each $k$-gonal face $f \in F$ a $w$-$k$-gonal $r$-patch $\mathcal{P}(f)$ with $p$-vector $p^{(f)}$.
  \end{cinput}
  \begin{coutput}
    \item A map $M'$ on $S$ with $v$-vector $v + d \cdot [r]$ for some $d \in \nats$ and $p$-vector $\sum_{f \in F} p^{(f)}$
  \end{coutput} 
  \begin{cdescription} 
    Divide each edge $e \in E$ in the embedding of $G$ into $S$ by $n$ vertices and draw in each face $f$ the dedicated $r$-patch $\mathcal{P}(f)$ such that the corner vertices of $\mathcal{P}(f)$ coincide with the original vertices $V$ and the side vertices are the new vertices added by the subdivision, see \autoref{fig:const:map:overview}. We can do this because both the face we want to replace and the patch we want to insert are homeomorphic to a $2$-cell and the number of vertices of the patch and the subdivision match exactly. Note that each of these patches may allow for many distinct drawings. While the rotation of the drawing is irrelevant for the correctness of the construction, there is a difference based on the orientation of the drawing (drawing the patch “from one side” of $S$ gives a different result than drawing it from the other side). When drawing each patch equally oriented, they allow for \autoref{thm:fitting:tuples} to be applied, so assume hereafter that this was done. These patches form a combined graph $G'$, which is embedded by construction into $S$ (there is a homeomorphism between each subdivided face $f \in F$ and the corresponding patch $\mathcal{P}(f)$). 

For the rest of the construction, the notion ``corner vertices'' will relate to the original vertices $V$, while ``side vertices'' denote those on the subdivisions made before. The remaining vertices of $G'$ will be called ``inner vertices''. $G'$ is simple: $G'$ is nonempty and connected by construction and there is no possibility for it to contain loops or multi-edges since these would also be present in either $G$ or one of the patches $\mathcal{P}(f)$. Lastly, each face of the embedding is homeomorphic to a $2$-cell, since all faces in the new embedding are homeomorphic to an inner face of some patch $\mathcal{P}(f)$, thus the embedding of $G'$ into $S$ is a map $M'$. By \autoref{thm:fitting:tuples} each side vertex is $r$-valent, which is also true for the inner vertices (they come from $r$-valent inner vertices of an $r$-patch). Each corner vertex has the same valence it had previously in the map $M$, therefore the $v$-vector of $M'$ is $v + d \cdot [r]$ for some $d \in \nats$. The $p$-vector of $M'$ is the sum of the $p$-vectors $p^{(f)}$ of each patch $\mathcal{P}(f)$ ($f \in F$). Thus we have constructed another map on $S$ with $v$-vector $v + d \cdot [r]$ and $p$-vector $\sum_{f \in F} p^{(f)}$.
  \end{cdescription}
\end{construction}
\begin{tikzfigure}{\label{fig:const:map:overview}}{}
  \matrix (m) [column sep=1cm] {
    \begin{scope}[scale=0.5]
      \draw (-3,0)--(-3,3)--(-6,5)--(-6,8)--(-3,10)--(0,8)--(6,8)--(6,5)--(7,1)--(3,0)--(-3,0);
      \draw (-3,3)--(0,5)--(0,8);
      \draw (3,0)--(3,3)--(0,5);
      \draw (3,3)--(7,1);
      \draw (3,3)--(6,5);

      % loose ends
      \draw (-3,0)-- (-3.5,-0.666);
      \draw (-3,3)-- (-3.5,2.5);
      \draw (-6,5)--  (-6.5,4.666);
      \draw (-6,8)-- (-6.5,8.333);
      \draw (-3,10)-- (-3,10.5);
      \draw (0,8)-- (0.5,8.333);
      \draw (6,8)-- (6.5,8.5);
      \draw (6,5)-- (6.5,5);
      \draw (7,1)-- (7.5,0.5);
      \draw (3,0)-- (3.5,-0.666);

      \fill[black] (-3,0) circle(3pt);
      \fill[black] (-3,3)circle(3pt);
      \fill[black] (-6,5)circle(3pt);
      \fill[black] (-6,8)circle(3pt);
      \fill[black] (-3,10)circle(3pt);
      \fill[black] (0,8)circle(3pt);
      \fill[black] (6,8)circle(3pt);
      \fill[black] (6,5)circle(3pt);
      \fill[black] (7,1) circle(3pt);
      \fill[black] (3,0)circle(3pt);
      \fill[black] (0,5)circle(3pt);
      \fill[black] (3,3)circle(3pt);

      \node at (-3,6.5) {$f_1$};
      \node at (0,2.5) {$f_2$};
      \node at (3,5.5) {$f_3$};
      \node at (4.5,1.5) {$f_4$};
      \node at (5,3) {$f_5$};
      
    \end{scope}
    &
    \begin{scope}[scale=0.5]
      \draw (-3,0)--(-3,3)--(-6,5)--(-6,8)--(-3,10)--(0,8)--(6,8)--(6,5)--(7,1)--(3,0)--(-3,0);
      \draw (-3,3)--(0,5)--(0,8);
      \draw (3,0)--(3,3)--(0,5);
      \draw (3,3)--(7,1);
      \draw (3,3)--(6,5);

      % loose ends
      \draw (-3,0)-- (-3.5,-0.666);
      \draw (-3,3)-- (-3.5,2.5);
      \draw (-6,5)--  (-6.5,4.666);
      \draw (-6,8)-- (-6.5,8.333);
      \draw (-3,10)-- (-3,10.5);
      \draw (0,8)-- (0.5,8.333);
      \draw (6,8)-- (6.5,8.5);
      \draw (6,5)-- (6.5,5);
      \draw (7,1)-- (7.5,0.5);
      \draw (3,0)-- (3.5,-0.666);

      \fill[black] (-3,0) circle(3pt);
      \fill[black] (-3,3)circle(3pt);
      \fill[black] (-6,5)circle(3pt);
      \fill[black] (-6,8)circle(3pt);
      \fill[black] (-3,10)circle(3pt);
      \fill[black] (0,8)circle(3pt);
      \fill[black] (6,8)circle(3pt);
      \fill[black] (6,5)circle(3pt);
      \fill[black] (7,1) circle(3pt);
      \fill[black] (3,0)circle(3pt);
      \fill[black] (0,5)circle(3pt);
      \fill[black] (3,3)circle(3pt);

      \node at (-3,6.5) {$\mathcal{P}(f_1)$};
      \node at (0,2.5) {$\mathcal{P}(f_2)$};
      \node at (3,5.5) {$\mathcal{P}(f_3)$};
      \node at (4.3,1.2) {$\mathcal{P}(f_4)$};
      \node at (5,3) {$\mathcal{P}(f_5)$};
      
      \foreach \x in {0.2,0.4,0.6,0.8}
      \fill[black] ($(3,0)!\x!(-3,0)$) circle (3pt);    
      \foreach \x in {0.2,0.4,0.6,0.8}
      \fill[black] ($(-3,0)!\x!(-3,3)$) circle (3pt); 
      \foreach \x in {0.2,0.4,0.6,0.8}
      \fill[black] ($(-3,3)!\x!(-6,5)$) circle (3pt);
      \foreach \x in {0.2,0.4,0.6,0.8}
      \fill[black] ($(-6,5)!\x!(-6,8)$) circle (3pt);
      \foreach \x in {0.2,0.4,0.6,0.8}
      \fill[black] ($(-6,8)!\x!(-3,10)$) circle (3pt);
      \foreach \x in {0.2,0.4,0.6,0.8}
      \fill[black] ($(-3,10)!\x!(0,8)$) circle (3pt);
      \foreach \x in {0.2,0.4,0.6,0.8}
      \fill[black] ($(0,8)!\x!(6,8)$) circle (3pt);
      \foreach \x in {0.2,0.4,0.6,0.8}
      \fill[black] ($(6,8)!\x!(6,5)$) circle (3pt);
      \foreach \x in {0.2,0.4,0.6,0.8}
      \fill[black] ($(6,5)!\x!(7,1)$) circle (3pt);     
      \foreach \x in {0.2,0.4,0.6,0.8}
      \fill[black] ($(7,1)!\x!(3,0)$) circle (3pt); 
      \foreach \x in {0.2,0.4,0.6,0.8}
      \fill[black] ($(3,0)!\x!(3,3)$) circle (3pt);
      \foreach \x in {0.2,0.4,0.6,0.8}
      \fill[black] ($(3,3)!\x!(7,1)$) circle (3pt);
      \foreach \x in {0.2,0.4,0.6,0.8}
      \fill[black] ($(3,3)!\x!(6,5)$) circle (3pt);
      \foreach \x in {0.2,0.4,0.6,0.8}
      \fill[black] ($(3,3)!\x!(0,5)$) circle (3pt);
      \foreach \x in {0.2,0.4,0.6,0.8}
      \fill[black] ($(0,5)!\x!(0,8)$) circle (3pt);
      \foreach \x in {0.2,0.4,0.6,0.8}
      \fill[black] ($(0,5)!\x!(-3,3)$) circle (3pt);      
    \end{scope}
    \\
    };
\end{tikzfigure}


As previously stated, these definitions are used to formalize the construction step ``replace each face with a patch''. Up until now, there is no requirement explicitly stated on the interior of the patch. If we expect the result of such a construction to be a polyhedral map, further conditions have to be met. Additionally, when using \autoref{const:map} we have the problem of assigning a patch for each face of the map. While we might need only one type of patch for a $k$-gon for each $k \geq 3$, we could still have to deal with a huge amount of values of $k$. We now want to define a construction scheme for patches for arbitrary $k$ which additionally allow to create polyhedral maps, even from non-polyhedral ones.

\begin{definition}\label{def:expansion:patch} Let $\mathcal{P}$ be an $r$-patch with boundary
\begin{multline*}
  \partial\mathcal{P} = i_0 - i_1 - \dots - i_{m-1} - (i_m = o_0) - \dots - o_s - \dots - o_{n - 1} - (o_n = i'_m) \\
  - i'_{m-1} - \dots - i'_1 - i'_0 - i_0
\end{multline*}
($i_m$ and $o_0$ denote the same vertex, the same goes for $o_n$ and $i'_m$), $1 \leq s < n$, $m > 0$, such that:
  \begin{itemize}
  \item $w_{\mathcal{P}}(i_0) + w_{\mathcal{P}}(i'_0) = r - 1$,
  \item $i_1 - \dots - i_{m-1}$ fits $i'_{m-1} - \dots - i'_1$,
  \item $w_{\mathcal{P}}(o_s) = 1$,
  \item $(w_{\mathcal{P}}(o_{s + 1}), \dots, w_{\mathcal{P}}(o_{n - 1}), w_{\mathcal{P}}(o_n) + w_{\mathcal{P}}(o_0), w_{\mathcal{P}}(o_1), \dots, w_{\mathcal{P}}(o_{s - 1}))$ is a self-fitting tuple
  \end{itemize}
  (see also \autoref{fig:expansion:patch}). Such a patch will be called \hdef{expansion patch with outer tuple $(w_{\mathcal{P}}(o_{s + 1}), \dots, w_{\mathcal{P}}(o_{n - 1}), w_{\mathcal{P}}(o_n) + w_{\mathcal{P}}(o_0), w_{\mathcal{P}}(o_1), \dots, w_{\mathcal{P}}(o_{s - 1}))$}.
\end{definition}

\begin{example}\label{ex:easy:expansion1}
  We want to review the last definition with two examples. A hexagon can be interpreted as an expansion $3$-patch $\mathcal{H}$ with outer tuple $(w_{\mathcal{H}}(o_3) + w_{\mathcal{H}}(o_0), w_{\mathcal{H}}(o_1)) = (2, 1)$, with vertices labeled according to \autoref{def:expansion:patch} in \autoref{fig:easy:expansion1}. Similarly two quadrangles which share a common edge build an expansion $4$-patch $\mathcal{Q}_2$ with outer tuple  $(w_{\mathcal{Q}_2}(o_3) + w_{\mathcal{Q}_2}(o_0), w_{\mathcal{Q}_2}(o_1)) = (2, 2)$, as seen in the same figure.
  \begin{tikzfigure}{\label{fig:easy:expansion1}}{Two expansion patches}
    \matrix (m) [column sep=1cm] {
      \begin{scope}
        \draw (0 : 2) -- (60 : 2) -- (120 : 2) -- (180 : 2) -- (240 : 2) -- (300 : 2) -- (0 : 2);
        \fill [black] (  0 : 2) circle (2pt);
        \node[anchor=180] at (  0 : 2) {$i'_1 = o_3$};
        \fill [black] ( 60 : 2) circle (2pt);
        \node[anchor=240] at ( 60 : 2) {$\bm{o_2 = o_s}$};
        \fill [black] (120 : 2) circle (2pt);
        \node[anchor=300] at (120 : 2) {$o_1$};
        \fill [black] (180 : 2) circle (2pt);
        \node[anchor=  0] at (180 : 2) {$i_1 = o_0$};
        \fill [black] (240 : 2) circle (2pt);
        \node[anchor= 60] at (240 : 2) {$i_0$};
        \fill [black] (300 : 2) circle (2pt);
        \node[anchor=120] at (300 : 2) {$i'_0$};
        \node             at (  0 : 0) {$\mathcal{H}$};
      \end{scope}
      &
      \begin{scope}
        \draw (0, 1) -- (0, -1) -- (-2, -1) -- (-2, 1) -- (0, 1) -- (2, 1) -- (2, -1) -- (0, -1);
        \fill [black] (-2, -1) circle (2pt);
        \node[anchor= 45] at (-2, -1) {$i_0$};
        \fill [black] ( 0, -1) circle (2pt);
        \node[anchor= 90] at ( 0, -1) {$i'_0$};
        \fill [black] ( 2, -1) circle (2pt);
        \node[anchor=135] at ( 2, -1) {$i'_1 = o_3$};
        \fill [black] ( 2,  1) circle (2pt);
        \node[anchor=225] at ( 2,  1) {$\bm{o_2 = o_s}$};
        \fill [black] ( 0,  1) circle (2pt);
        \node[anchor=270] at ( 0,  1) {$o_1$};
        \fill [black] (-2,  1) circle (2pt);
        \node[anchor=315] at (-2,  1) {$i_1 = o_0$};
        \node[anchor=  0] at ( 0,  0) {$\mathcal{Q}_2$};
      \end{scope}
      \\
    };
  \end{tikzfigure}%
\end{example}

\begin{remark}\label{rem:edge:patch}
  We want to dissect this definition a bit to give a geometric intuition. The first thing to note is that by definition we are able to glue two copies of an expansion patch along the paths $i_0 - \dots - i_{m}$ and $i'_{m} - \dots - i'_0$ (by \autoref{thm:fitting:tuples}). When doing this the new patch has a boundary path $o_{s+1} - \dots - o_{n-1} - (o_n = o_0) - o_1 - \dots - o_{s-1}$, which we require to be self-fitting. Therefore we can glue two of those patches along this boundary to get an even larger patch (see \autoref{fig:edge:patch}).
    \begin{tikzfigure}{\label{fig:edge:patch}}{An edge patch}
      \begin{scope}[scale=0.8]
      \draw[shift={(-5,0)}] (9 : 1) -- (81 : 1)  (297 : 1) -- (9 : 1);

      \draw[shift={(-5,0)}] (9 : 1) -- (9 : 2.5);
      \draw[shift={(-5,0)}][loosely dotted] (9 : 2.5) -- (9 : 3.5);
      \draw[shift={(-5,0)}] (9 : 3.5) -- (9 : 5.062);
      \draw[shift={(-5,0)}] (81 : 1) -- (81 : 2.5);
      \draw[shift={(-5,0)}][loosely dotted] (81 : 2.5) -- (81 : 3.5);
      \draw[shift={(-5,0)}] (81 : 3.5) -- (81 : 5.062);
      \draw[shift={(-5,0)}] (297 : 1) -- (297 : 2.5);
      \draw[shift={(-5,0)}][loosely dotted] (297 : 2.5) -- (297 : 3.5);
      \draw[shift={(-5,0)}] (297 : 3.5) -- (297 : 5.062);
      \draw[shift={(-5,0)}][loosely dotted] (4 : 5.012) -- (356 : 5.012);


      \draw[shift={(-5,0)}] (-13 : 5.1315) -- (356 : 5.012) (4 : 5.012) -- (13 : 5.1315);
      \draw[shift={(-5,0)}][loosely dotted] (13 : 5.1315) -- (27 : 5.612);
      \draw[shift={(-5,0)}] (27 : 5.612) -- (36 : 6.180) -- (45 : 5.612);
      \draw[shift={(-5,0)}][loosely dotted] (45 : 5.612) -- (68 : 5.012);
      \draw[shift={(-5,0)}] (68 : 5.012) -- (81 : 5.062);
      \draw[shift={(-5,0)}] (-59 : 5.1315) -- (297 : 5.062);
      \draw[shift={(-5,0)}][loosely dotted] (-45 : 5.612) -- (-59 : 5.1315);
      \draw[shift={(-5,0)}] (-27 : 5.612) -- (-36 : 6.180) -- (-45 : 5.612);
      \draw[shift={(-5,0)}][loosely dotted] (-27 : 5.612) -- (-13 : 5.1315);

      \fill[shift={(-5,0)}] [black] (9 : 1) circle (2pt);
      \node[shift={(-4,0)}][anchor="108"] at (9 : 1) {$i_0$};
      \fill[shift={(-5,0)}] [black] (9 : 2) circle (2pt);
      \node[shift={(-4,0)}][anchor="99"] at (9 : 2) {$i_1$};
      \fill[shift={(-5,0)}] [black] (9 : 4) circle (2pt);
      \node[shift={(-4,0)}][anchor="99"] at (9 : 4) {$i_{m-1}$};
      \fill[shift={(-5,0)}] [black] (9 : 5.062) circle (2pt);
      \node[shift={(-4,0)}][anchor="45"] at (9 : 5.062) {$o_{0}$};
      \fill[shift={(-5,0)}] [black] (-9 : 5.062) circle (2pt);
      \node[shift={(-4,0)}][anchor="0"] at (-9 : 5.062) {$o_{2s \mod n}$};
      \fill[shift={(-5,0)}] [black] (-30 : 5.774) circle (2pt);
      \node[shift={(-4,0)}][anchor="0"] at (-30 : 5.774) {$o_{s - 1}$};
      \fill[shift={(-5,0)}] [black] (-36 : 6.180) circle (2pt);
      \node[shift={(-4,0)}][anchor="-36"] at (-36 : 6.180) {$\bm{o_s}$};
      \fill[shift={(-5,0)}] [black] (-42 : 5.774) circle (2pt);
      \node[shift={(-4,0)}][anchor="-36"] at (-42 : 5.774) {$o_{s + 1}$};
      % \fill[shift={(-5,0)}] [black] (-63 : 5.062) circle (2pt);
      % \node[shift={(-4,0)}][anchor="-63"] at (-63 : 5.062) {$o_{n - 1}$};
      \fill[shift={(-5,0)}] [black] (-63 : 5.062) circle (2pt);
      \node[shift={(-4,0)}][anchor="-117"] at (-63 : 5.062) {$o_{n}$};
      \fill[shift={(-5,0)}] [black] (-63 : 4) circle (2pt);
      \node[shift={(-4,0)}][anchor="198"] at (-63 : 4) {$i'_{m-1}$};
      \fill[shift={(-5,0)}] [black] (-63 : 2) circle (2pt);
      \node[shift={(-4,0)}][anchor="198"] at (-63 : 2) {$i'_{1}$};
      \fill[shift={(-5,0)}] [black] (-63 : 1) circle (2pt);
      \node[shift={(-4,0)}][anchor="180"] at (-63 : 1) {$i'_0$};
      \node[shift={(-4,0)}] at (-27 : 3.5) {$\mathcal{M}$};

      \fill[shift={(-5,0)}] [black] (81 : 1) circle (2pt);
      \node[shift={(-4,0)}][anchor="180"] at (81 : 1) {$i_0$};
      \fill[shift={(-5,0)}] [black] (81 : 2) circle (2pt);
      \node[shift={(-4,0)}][anchor="162"] at (81 : 2) {$i_1$};
      \fill[shift={(-5,0)}] [black] (81 : 4) circle (2pt);
      \node[shift={(-4,0)}][anchor="162"] at (81 : 4) {$i_{m-1}$};
      \fill[shift={(-5,0)}] [black] (81 : 5.062) circle (2pt);
      \node[shift={(-4,0)}][anchor="126"] at (81 : 5.062) {$o_{0}$};
      \fill[shift={(-5,0)}] [black] (72 : 5) circle (2pt);
      \node[shift={(-4,0)}][anchor="82"] at (72 : 5) {$o_{1}$};
      \fill[shift={(-5,0)}] [black] (42 : 5.774) circle (2pt);
      \node[shift={(-4,0)}][anchor="45"] at (42 : 5.774) {$o_{s - 1}$};
      \fill[shift={(-5,0)}] [black] (36 : 6.180) circle (2pt);
      \node[shift={(-4,0)}][anchor="36"] at (36 : 6.180) {$\bm{o_s}$};
      \fill[shift={(-5,0)}] [black] (30 : 5.774) circle (2pt);
      \node[shift={(-4,0)}][anchor="0"] at (30 : 5.774) {$o_{s + 1}$};
      %\fill[shift={(-5,0)}] [black] (9 : 5.062) circle (2pt);
      %\node[shift={(-4,0)}][anchor="0"] at (9 : 5.062) {$o_{n - 1}$};
      \fill[shift={(-5,0)}] [black] (9 : 5.062) circle (2pt);
      \node[shift={(-4,0)}][anchor="-45"] at (9 : 5.062) {$o_{n}$};
      \node[shift={(-4,0)}][anchor="180"] at (9 : 5.062) {$o_{2s \mod n}$};
      
      \fill[shift={(-5,0)}] [black] (9 : 4) circle (2pt);
      \node[shift={(-4,0)}][anchor="270"] at (9 : 4) {$i'_{m-1}$};
      \fill[shift={(-5,0)}] [black] (9 : 2) circle (2pt);
      \node[shift={(-4,0)}][anchor="270"] at (9 : 2) {$i'_{1}$};
      \fill[shift={(-5,0)}] [black] (9 : 1) circle (2pt);
      \node[shift={(-4,0)}][anchor="252"] at (9 : 1) {$i'_0$};
      \node[shift={(-4,0)}] at (45 : 3.5) {$\mathcal{M}$};

      \draw[shift={(5,0)}, rotate around={180:(0,0)}] (9 : 1) -- (81 : 1) (297 : 1) -- (9 : 1);

      \draw[shift={(5,0)}, rotate around={180:(0,0)}] (9 : 1) -- (9 : 2.5);
      \draw[shift={(5,0)}, rotate around={180:(0,0)}][loosely dotted] (9 : 2.5) -- (9 : 3.5);
      \draw[shift={(5,0)}, rotate around={180:(0,0)}] (9 : 3.5) -- (9 : 5.062);
      \draw[shift={(5,0)}, rotate around={180:(0,0)}] (81 : 1) -- (81 : 2.5);
      \draw[shift={(5,0)}, rotate around={180:(0,0)}][loosely dotted] (81 : 2.5) -- (81 : 3.5);
      \draw[shift={(5,0)}, rotate around={180:(0,0)}] (81 : 3.5) -- (81 : 5.062);
      \draw[shift={(5,0)}, rotate around={180:(0,0)}] (297 : 1) -- (297 : 2.5);
      \draw[shift={(5,0)}, rotate around={180:(0,0)}][loosely dotted] (297 : 2.5) -- (297 : 3.5);
      \draw[shift={(5,0)}, rotate around={180:(0,0)}] (297 : 3.5) -- (297 : 5.062);
      
      % \draw[shift={(5,0)}, rotate around={180:(0,0)}] (-13 : 5.1315) -- (13 : 5.1315);
      % \draw[shift={(5,0)}, rotate around={180:(0,0)}][loosely dotted] (13 : 5.1315) -- (27 : 5.612);
      \draw[shift={(5,0)}, rotate around={180:(0,0)}] (36 : 6.180) -- (45 : 5.612);
      \draw[shift={(5,0)}, rotate around={180:(0,0)}][loosely dotted] (45 : 5.612) -- (59 : 5.1315);
      \draw[shift={(5,0)}, rotate around={180:(0,0)}] (59 : 5.1315) -- (81 : 5.062);
      \draw[shift={(5,0)}, rotate around={180:(0,0)}] (-59 : 5.1315) -- (297 : 5.062);
      \draw[shift={(5,0)}, rotate around={180:(0,0)}][loosely dotted] (-45 : 5.612) -- (-59 : 5.1315);
      \draw[shift={(5,0)}, rotate around={180:(0,0)}] (-36 : 6.180) -- (-45 : 5.612);
      %\draw[shift={(5,0)}, rotate around={180:(0,0)}][loosely dotted] (-27 : 5.612) -- (-13 : 5.1315);

      \fill[shift={(5,0)}, rotate around={180:(0,0)}] [black] (9 : 1) circle (2pt);
      \node[shift={(4,0)}][anchor="292"] at (189 : 1) {$i_0$};
      \fill[shift={(5,0)}, rotate around={180:(0,0)}] [black] (9 : 2) circle (2pt);
      \node[shift={(4,0)}][anchor="270"] at (189 : 2) {$i_1$};
      \fill[shift={(5,0)}, rotate around={180:(0,0)}] [black] (9 : 4) circle (2pt);
      \node[shift={(4,0)}][anchor="270"] at (189 : 4) {$i_{m-1}$};
      \fill[shift={(5,0)}, rotate around={180:(0,0)}] [black] (9 : 5.062) circle (2pt);
      \node[shift={(4,0)}][anchor="225"] at (189 : 5) {$o_{0}$};
      \fill[shift={(5,0)}, rotate around={180:(0,0)}] [black] (-30 : 5.774) circle (2pt);
      \node[shift={(4,0)}][anchor="180"] at (150 : 5.774) {$o_{s - 1}$};
      \fill[shift={(5,0)}, rotate around={180:(0,0)}] [black] (-36 : 6.180) circle (2pt);
      \node[shift={(4,0)}][anchor="144"] at (144 : 6.180) {$\bm{o_s}$};
      \fill[shift={(5,0)}, rotate around={180:(0,0)}] [black] (-42 : 5.774) circle (2pt);
      \node[shift={(4,0)}][anchor="144"] at (138 : 5.774) {$o_{s + 1}$};
      \fill[shift={(5,0)}, rotate around={180:(0,0)}] [black] (-63 : 5.062) circle (2pt);
      \node[shift={(4,0)}][anchor="63"] at (117 : 5) {$o_{n}$};
      \fill[shift={(5,0)}, rotate around={180:(0,0)}] [black] (-63 : 4) circle (2pt);
      \node[shift={(4,0)}][anchor="18"] at (117 : 4) {$i'_{m-1}$};
      \fill[shift={(5,0)}, rotate around={180:(0,0)}] [black] (-63 : 2) circle (2pt);
      \node[shift={(4,0)}][anchor="18"] at (117 : 2) {$i'_{1}$};
      \fill[shift={(5,0)}, rotate around={180:(0,0)}] [black] (-63 : 1) circle (2pt);
      \node[shift={(4,0)}][anchor="0"] at (117 : 1) {$i'_0$};
      \node[shift={(4,0)}] at (144 : 3.5) {$\mathcal{M}$};

      \fill[shift={(5,0)}, rotate around={180:(0,0)}] [black] (81 : 1) circle (2pt);
      \node[shift={(4,0)}][anchor="0"] at (261 : 1) {$i_0$};
      \fill[shift={(5,0)}, rotate around={180:(0,0)}] [black] (81 : 2) circle (2pt);
      \node[shift={(4,0)}][anchor="342"] at (261 : 2) {$i_1$};
      \fill[shift={(5,0)}, rotate around={180:(0,0)}] [black] (81 : 4) circle (2pt);
      \node[shift={(4,0)}][anchor="342"] at (261 : 4) {$i_{m-1}$};
      \fill[shift={(5,0)}, rotate around={180:(0,0)}] [black] (81 : 5.062) circle (2pt);
      \node[shift={(4,0)}][anchor="300"] at (261 : 5.062) {$o_{0}$};
      \fill[shift={(5,0)}, rotate around={180:(0,0)}] [black] (42 : 5.774) circle (2pt);
      \node[shift={(4,0)}][anchor="216"] at (222 : 5.774) {$o_{s - 1}$};
      \fill[shift={(5,0)}, rotate around={180:(0,0)}] [black] (36 : 6.180) circle (2pt);
      \node[shift={(4,0)}][anchor="216"] at (216 : 6.180) {$\bm{o_s}$};
      \fill[shift={(5,0)}, rotate around={180:(0,0)}] [black] (30 : 5.774) circle (2pt);
      \node[shift={(4,0)}][anchor="180"] at (210 : 5.774) {$o_{s + 1}$};
      \fill[shift={(5,0)}, rotate around={180:(0,0)}] [black] (9 : 5.062) circle (2pt);
      \node[shift={(4,0)}][anchor="135"] at (189 : 5.062) {$o_{n}$};
      \fill[shift={(5,0)}, rotate around={180:(0,0)}] [black] (9 : 4) circle (2pt);
      \node[shift={(4,0)}][anchor="90"] at (189 : 4) {$i'_{m-1}$};
      \fill[shift={(5,0)}, rotate around={180:(0,0)}] [black] (9 : 2) circle (2pt);
      \node[shift={(4,0)}][anchor="90"] at (189 : 2) {$i'_{1}$};
      \fill[shift={(5,0)}, rotate around={180:(0,0)}] [black] (9 : 1) circle (2pt);
      \node[shift={(4,0)}][anchor="72"] at (189 : 1) {$i'_0$};
      \node[shift={(4,0)}] at (216 : 3.5) {$\mathcal{M}$};
      \end{scope}

    \end{tikzfigure}
\end{remark}

\begin{definition}
  For an expansion patch $\mathcal{M}$, we want to call the patch held by gluing four copies of $\mathcal{M}$ as in \autoref{rem:edge:patch} the \hdef{edge patch} of $\mathcal{M}$. An expansion patch will be said to have the \hdef{polyhedral property} if every two inner faces in the corresponding edge patch meet properly.
\end{definition}

\begin{example}\label{ex:easy:expansion2}
  The examples in \autoref{ex:easy:expansion1} do in fact have the polyhedral property, which we can verify by looking at the edge patch in \autoref{fig:easy:expansion2}.
  \begin{tikzfigure}{\label{fig:easy:expansion2}}{Two edge patches}
    \matrix (m) [column sep=1cm] {
      \begin{scope}[xscale=1.0, yscale=0.866, scale=0.75]
        \draw[very thick] (-1,  0) -- (-2, -2) -- (-4, -2) -- (-5,  0) -- (-4,  2) -- (-2,  2) -- (-1,  0);
        \draw[very thick] (-1,  0) -- ( 1,  0) -- ( 2, -2) -- ( 1, -4) -- (-1, -4) -- (-2, -2);
        \draw[very thick] ( 1,  0) -- ( 2,  2) -- ( 1,  4) -- (-1,  4) -- (-2,  2);
        \draw[very thick] ( 2, -2) -- ( 4, -2) -- ( 5,  0) -- ( 4,  2) -- ( 2,  2);

        \fill[black] (-4,-2) circle(3pt);
        \fill[black] ( 4, 2) circle(3pt);
        \fill[black] (-4, 2) circle(3pt);
        \fill[black] (-2,-2) circle(3pt);
        \fill[black] (-2, 2) circle(3pt);
        \fill[black] ( 2,-2) circle(3pt);
        \fill[black] ( 5, 0) circle(3pt);
        \fill[black] (-5, 0) circle(3pt);
        \fill[black] (-1, 0) circle(3pt);
        \fill[black] ( 1, 0) circle(3pt);
        \fill[black] (-1,-4) circle(3pt);
        \fill[black] ( 1,-4) circle(3pt);
        \fill[black] (-1, 4) circle(3pt);
        \fill[black] ( 2, 2) circle(3pt);
        \fill[black] ( 4,-2) circle(3pt);
        \fill[black] ( 1, 4) circle(3pt);

        \node[anchor=240] at (-4, -2) {$i_0$};
        \node[anchor=180] at (-5,  0) {$o_0$};
        \node[anchor=120] at (-4,  2) {$o_1$};
        \node[anchor= 60] at (-2,  2) {$\bm{o_s}$};
        \node[anchor=  0] at (-1,  0) {$o_3$};
        \node[anchor=300] at (-2, -2) {$i'_0$};

        \node[anchor=240] at (-1, -4) {$i'_0$};
        \node[anchor=180] at (-2, -2) {$i_0$};
        \node[anchor=120] at (-1,  0) {$o_0$};
        \node[anchor= 60] at ( 1,  0) {$o_1$};
        \node[anchor=  0] at ( 2, -2) {$\bm{o_s}$};
        \node[anchor=300] at ( 1, -4) {$o_3$};

        \node[anchor=240] at (-1,  0) {$o_1$};
        \node[anchor=180] at (-2,  2) {$\bm{o_s}$};
        \node[anchor=120] at (-1,  4) {$o_3$};
        \node[anchor= 60] at ( 1,  4) {$i'_0$};
        \node[anchor=  0] at ( 2,  2) {$i_0$};
        \node[anchor=300] at ( 1,  0) {$o_0$};

        \node[anchor=240] at ( 2, -2) {$\bm{o_s}$};
        \node[anchor=180] at ( 1,  0) {$o_3$};
        \node[anchor=120] at ( 2,  2) {$i'_0$};
        \node[anchor= 60] at ( 4,  2) {$i_0$};
        \node[anchor=  0] at ( 5,  0) {$o_0$};
        \node[anchor=300] at ( 4, -2) {$o_1$};

        \node             at (-3,  0) {$\mathcal{H}$};
        \node             at ( 0, -2) {$\mathcal{H}$};
        \node             at ( 0,  2) {$\mathcal{H}$};
        \node             at ( 3,  0) {$\mathcal{H}$};
      \end{scope}
      &
      \begin{scope}[scale=0.75]
        \draw[very thick] (-2, -3) -- (-2,  1) -- (-4,  1) -- (-4,  3) -- (2,  3) -- (2, -1) -- (4, -1) -- (4, -3) -- (-2, -3);
        \draw[very thick] ( 0, -3) -- ( 0,  3);
        \draw[very thick] ( 2, -1) -- ( 0, -1);
        \draw[very thick] ( 0,  1) -- (-2,  1);
        \draw[dotted] (-2, -1) -- ( 0, -1);
        \draw[dotted] ( 2,  1) -- ( 0,  1);
        \draw[dotted] (-2,  3) -- (-2,  1);
        \draw[dotted] ( 2, -3) -- ( 2, -1);

        \foreach \x in {-2,0,2}
        \foreach \y in {-3,-1,1,3}
        \fill[black] (\x, \y) circle(3pt);
        \fill[black] (4,-1) circle(3pt);
        \fill[black] (4,-3) circle(3pt);
        \fill[black] (-4, 1) circle(3pt);
        \fill[black] (-4, 3) circle(3pt);

        \node[anchor=225] at (-4,  1) {$i_0$};
        \node[anchor=135] at (-4,  3) {$o_0$};
        \node[anchor=135] at (-2,  3) {$o_1$};
        \node[anchor= 45] at ( 0,  3) {$\bm{o_s}$};
        \node[anchor=315] at ( 0,  1) {$o_3$};
        \node[anchor=225] at (-2,  1) {$i'_0$};
        \node[anchor=  0] at (-2,  2) {$\mathcal{Q}_2$};

        \node[anchor= 45] at ( 4, -1) {$i_0$};
        \node[anchor=315] at ( 4, -3) {$o_0$};
        \node[anchor=315] at ( 2, -3) {$o_1$};
        \node[anchor=225] at ( 0, -3) {$\bm{o_s}$};
        \node[anchor=135] at ( 0, -1) {$o_3$};
        \node[anchor= 45] at ( 2, -1) {$i'_0$};
        \node[anchor=180] at ( 2, -2) {$\mathcal{Q}_2$};

        \node[anchor=135] at (-2,  1) {$i_0$};
        \node[anchor= 45] at ( 0,  1) {$o_0$};
        \node[anchor= 45] at ( 0, -1) {$o_1$};
        \node[anchor=315] at ( 0, -3) {$\bm{o_s}$};
        \node[anchor=225] at (-2, -3) {$o_3$};
        \node[anchor=135] at (-2, -1) {$i'_0$};
        \node[anchor=270] at (-1, -1) {$\mathcal{Q}_2$};

        \node[anchor=315] at ( 2, -1) {$i_0$};
        \node[anchor=225] at ( 0, -1) {$o_0$};
        \node[anchor=225] at ( 0,  1) {$o_1$};
        \node[anchor=135] at ( 0,  3) {$\bm{o_s}$};
        \node[anchor= 45] at ( 2,  3) {$o_3$};
        \node[anchor=315] at ( 2,  1) {$i'_0$};
        \node[anchor= 90] at ( 1,  1) {$\mathcal{Q}_2$};
      \end{scope}
      \\
    };
  \end{tikzfigure}%
\end{example}

Expansion patches will be our basic building block for all our constructive proofs. We can use them to obtain larger $o$-$k$-gonal patches for any $k\ geq 3$:

\begin{construction}\label{const:expansion:patch}
  \begin{cinput}
  \item An expansion $r$-patch $\mathcal{M}$ with outer tuple $o$ and $p$-vector $p$.
  \end{cinput}
  \begin{coutput}
  \item For every $k \geq 3$ an $o$-$k$-gonal $r$-patch $\mathcal{M}(k)$ with $p$-vector $[k] + k \cdot p$.
  \item If $\mathcal{M}$ has the polyhedral property, then all inner faces of $\mathcal{M}(k)$ meet properly.
  \end{coutput}
  \begin{cdescription} We construct $\mathcal{M}(k)$ from $k$ copies of $\mathcal{M}$ and a single $k$-gon. Let
\begin{multline*}
  \partial\mathcal{P} = i_0 - i_1 - \dots - (i_m = o_0) - o_1 - \dots - o_s - \dots - o_{n - 1} - (o_n = i'_m) - \dots \\ - i'_1 - i'_0 - i_0
\end{multline*}
 be the boundary of $\mathcal{M}$ as in \autoref{def:expansion:patch}. We now form a larger patch using \autoref{thm:fitting:tuples} repeatedly by gluing the edge $\set{i_0, i'_0}$ of each of the $k$ copies of $\mathcal{M}$ to edge of the $k$-gon and also gluing the vertex associated to $i_l$, $1 \leq l \leq m$ from one copy to the vertex associated to $i'_l$ from the adjacent copy. Graphically speaking, we form a ring of $k$ patches of the form $\mathcal{M}$ around the $k$-gon (see \autoref{fig:expansion:patch}). The $p$-vector of $\mathcal{M}(k)$ is therefore $[k] + k \cdot p$.

    \begin{tikzfigure}{\label{fig:expansion:patch}}{Schematic view of \autoref{const:expansion:patch}}

      \node (0, 0) {$k$-gon};

      \draw (0 : 1) -- (72 : 1) -- (144 : 1)  (216 : 1) -- (288 : 1) -- (0 : 1);
      \draw[loosely dotted] (144 : 1) -- (216 : 1);


      \draw (0 : 1) -- (0 : 2.5);
      \draw[loosely dotted] (0 : 2.5) -- (0 : 3.5);
      \draw (0 : 3.5) -- (0 : 5);
      \draw (72 : 1) -- (72 : 2.5);
      \draw[loosely dotted] (72 : 2.5) -- (72 : 3.5);
      \draw (72 : 3.5) -- (72 : 5);
      \draw (144 : 1) -- (144 : 2.5);
      \draw[loosely dotted] (144 : 2.5) -- (144 : 3.5);
      \draw (144 : 3.5) -- (144 : 5);
      \draw (216 : 1) -- (216 : 2.5);
      \draw[loosely dotted] (216 : 2.5) -- (216 : 3.5);
      \draw (216 : 3.5) -- (216 : 5);
      \draw (288 : 1) -- (288 : 2.5);
      \draw[loosely dotted] (288 : 2.5) -- (288 : 3.5);
      \draw (288 : 3.5) -- (288 : 5);


      \draw (-13 : 5.132) -- (13 : 5.132);
      \draw[loosely dotted] (13 : 5.132) -- (27 : 5.612);
      \draw (27 : 5.612) -- (36 : 6.180) -- (45 : 5.612);
      \draw[loosely dotted] (45 : 5.612) -- (59 : 5.132);
      \draw (59 : 5.132) -- (85 : 5.132);
      \draw[loosely dotted] (85 : 5.132) -- (99 : 5.612);
      \draw (99 : 5.612) -- (108 : 6.180) -- (117 : 5.612);
      \draw[loosely dotted] (117 : 5.612) -- (131 : 5.132);
      \draw (131 : 5.132) -- (144 : 5);
      \draw (-131 : 5.132) -- (216 : 5);
      \draw[loosely dotted] (-117 : 5.612) -- (-131 : 5.132);
      \draw (-99 : 5.612) -- (-108 : 6.180) -- (-117 : 5.612);
      \draw[loosely dotted] (-99 : 5.612) -- (-85 : 5.132);
      \draw (-59 : 5.132) -- (-85 : 5.132);
      \draw[loosely dotted] (-45 : 5.612) -- (-85 : 5.132);
      \draw (-27 : 5.612) -- (-36 : 6.180) -- (-45 : 5.612);
      \draw[loosely dotted] (-27 : 5.612) -- (-13 : 5.132);

      \fill [black] (0 : 1) circle (2pt);
      \node[anchor="99"] at (0 : 1) {$i_0$};
      \fill [black] (0 : 2) circle (2pt);
      \node[anchor="90"] at (0 : 2) {$i_1$};
      \fill [black] (0 : 4) circle (2pt);
      \node[anchor="90"] at (0 : 4) {$i_{m-1}$};
      \fill [black] (0 : 5) circle (2pt);
      \node[anchor="45"] at (0 : 5) {$o_{0}$};
      \fill [black] (-9 : 5.062) circle (2pt);
      \node[anchor="0"] at (-9 : 5.062) {$o_{1}$};
      \fill [black] (-30 : 5.774) circle (2pt);
      \node[anchor="0"] at (-30 : 5.774) {$o_{s - 1}$};
      \fill [black] (-36 : 6.180) circle (2pt);
      \node[anchor="-36"] at (-36 : 6.180) {$\bm{o_s}$};
      \fill [black] (-42 : 5.774) circle (2pt);
      \node[anchor="-36"] at (-42 : 5.774) {$o_{s + 1}$};
      \fill [black] (-63 : 5.062) circle (2pt);
      \node[anchor="-63"] at (-63 : 5.062) {$o_{n - 1}$};
      \fill [black] (-72 : 5) circle (2pt);
      \node[anchor="-117"] at (-72 : 5) {$o_{n}$};
      \fill [black] (-72 : 4) circle (2pt);
      \node[anchor="198"] at (-72 : 4) {$i'_{m-1}$};
      \fill [black] (-72 : 2) circle (2pt);
      \node[anchor="198"] at (-72 : 2) {$i'_{1}$};
      \fill [black] (-72 : 1) circle (2pt);
      \node[anchor="180"] at (-72 : 1) {$i'_0$};
      \node at (-36 : 3.5) {$\mathcal{M}$};

      \fill [black] (72 : 1) circle (2pt);
      \node[anchor="180"] at (72 : 1) {$i_0$};
      \fill [black] (72 : 2) circle (2pt);
      \node[anchor="162"] at (72 : 2) {$i_1$};
      \fill [black] (72 : 4) circle (2pt);
      \node[anchor="162"] at (72 : 4) {$i_{m-1}$};
      \fill [black] (72 : 5) circle (2pt);
      \node[anchor="117"] at (72 : 5) {$o_{0}$};
      \fill [black] (63 : 5.062) circle (2pt);
      \node[anchor="72"] at (63 : 5.062) {$o_{1}$};
      \fill [black] (42 : 5.774) circle (2pt);
      \node[anchor="60"] at (42 : 5.774) {$o_{s - 1}$};
      \fill [black] (36 : 6.180) circle (2pt);
      \node[anchor="36"] at (36 : 6.180) {$\bm{o_s}$};
      \fill [black] (30 : 5.774) circle (2pt);
      \node[anchor="0"] at (30 : 5.774) {$o_{s + 1}$};
      \fill [black] (9 : 5.062) circle (2pt);
      \node[anchor="0"] at (9 : 5.062) {$o_{n - 1}$};
      \fill [black] (0 : 5) circle (2pt);
      \node[anchor="-45"] at (0 : 5) {$o_{n}$};
      \fill [black] (0 : 4) circle (2pt);
      \node[anchor="270"] at (0 : 4) {$i'_{m-1}$};
      \fill [black] (0 : 2) circle (2pt);
      \node[anchor="270"] at (0 : 2) {$i'_{1}$};
      \fill [black] (0 : 1) circle (2pt);
      \node[anchor="252"] at (0 : 1) {$i'_0$};
      \node at (36 : 3.5) {$\mathcal{M}$};

      \node at (108 : 3.5) {$\mathcal{M}$};
      \node at (-108 : 3.5) {$\mathcal{M}$};

      \draw[loosely dotted] (171 : 4) -- (189 : 4);
    \end{tikzfigure}

    We do not want to lay out each application of \autoref{thm:fitting:tuples} explicitly but instead note two things: First, each inner vertex of $\mathcal{M}(k)$ has indeed valence $r$. We see this from calculating the valence for each kind of vertices. The vertices from the original $k$-gon are now adjacent additionally to $w_{\mathcal{M}}(i_o) + w_{\mathcal{M}}(i'_o)$ faces, so the valence is $1 + w_{\mathcal{M}}(i_o) + w_{\mathcal{M}}(i'_o) = r$. The vertices $i_l$ and $i'_l$, $1 \leq i < m$ are adjacent to $w_{\mathcal{M}}(i_l) + w_{\mathcal{M}}{i_l'}$ faces and have therefore valence $r$ too, since $i_1 - \dots - i_{m-1}$ and $i'_{m-1} - \dots - i'_1$ were required to be fitting. All other inner vertices come from inner vertices of $\mathcal{M}$, hence they also have valence $r$.
    
    Second, $\mathcal{M}(k)$ is in fact $o$-$k$-gonal since $w_{\mathcal{P}}(o_s) = 1$ and the number of faces adjacent to the vertices between the different copies of $o_s$ are exactly $w_{\mathcal{P}}(o_{s + 1}), \dots, w_{\mathcal{P}}(o_{n - 1}), w_{\mathcal{P}}(o_n) + w_{\mathcal{P}}(o_0), w_{\mathcal{P}}(o_1), \dots w_{\mathcal{P}}(o_{s - 1})$. Finally we want to show that if $\mathcal{M}$ has the polyhedral property every two faces of $\mathcal{M}(k)$ meet properly. We first want to rule out that one of the faces is the single $k$-gon in the middle of the construction. The $k$ copies of $\mathcal{M}$ touch the $k$-gon in different edges, which correspond to the edge $\set{i_0, i'_0}$ in $\mathcal{M}$, therefore the $k$-gon meets any other face properly, if no face of $\mathcal{M}$ contains both vertices $i_0$ and $i'_0$, but not the whole edge $\set{i_0, i'_0}$. Since this edge is a boundary edge, there is exactly one face in $\mathcal{M}$ which contains it, but this face would not meet properly with any other face which contains $i_o$ and $i'_0$ (since it can not contain the edge $\set{i_o, i'_o}$). But if two faces in $\mathcal{M}$ did not meet properly the same would be true for the corresponding edge patch, which contradicts that $\mathcal{M}$ has the polyhedral property. 

    If both $f$ and $f'$ differ from the central $k$-gon they reside inside two of the copies of $\mathcal{M}$. However, they can only have a non-empty intersection if they are in the same copy or in adjacent copies, i.e. two copies which are glued along the vertices $i_0 - \dots - i_m$ from one copy to the vertices $i'_m - \dots - i'_0$ from the other copy. In this case we can think of the subpatch of $\mathcal{M}(k)$ consisting only of those two copies of $\mathcal{M}$ containing $f$ and $f'$ as a subpatch of the edge patch of $\mathcal{M}$ and any two faces meeting non-properly in $\mathcal{M}$ would correspond to two faces in the edge patch meeting non-properly. This can not be by definition, thus we have reached a contradiction.
  \end{cdescription}
\end{construction}

With these constructions at hand we can now finally design a scheme to create a polyhedral map from a non-polyhedral one.

\begin{proposition}\label{thm:const:polymap}
  Given a map $M$ on an orientable closed $2$-manifold $S$ and an expansion $r$-patch $\mathcal{M}$ with outer tuple $o$, \autoref{const:map} returns a polyhedral map on $S$ when we take $\mathcal{P}(f) = \mathcal{M}(k)$ for each $k$-gonal face $f$ of $M$.
\begin{proof}
The only requirement of \autoref{const:map} we did not explicitly state is that $o$ is self-fitting (which it is by definition) and that $\mathcal{M}(k)$ is $o$-$k$-gonal, which was the result of \autoref{const:expansion:patch}. Thus we can apply \autoref{const:map} and get a map $M'$ on $S$ and need to show that $\mathcal{M'}$ is polyhedral; for a schematic overview, see \autoref{fig:const:polymap:overview}. We want to use a proof by contradiction so assume that two faces of $M'$, $f$ and $f'$, do not meet properly. By construction, there are two faces $f_o$ and $f'_o$ in the original map $M$, such that $f$ is contained in $\mathcal{P}(f_o)$ and $f'$ is contained in $\mathcal{P}(f'_o)$. Since $f$ and $f'$ do not meet properly, $f_o \cap f'_o \neq \emptyset$ and $\mathcal{P}(f_o) \cap \mathcal{P}(f'_o) \neq \emptyset$. We consider two cases: either $f_o$ and $f'_o$ denote the same face or they do not. We can quickly rule out the first case since $f_o$ is a $2$-cell and thus no vertices of the boundary of $\mathcal{P}(f_o)$ are identified in $M$ and we proved that the inner faces of $\mathcal{P}(f_o)$ meet properly in \autoref{const:expansion:patch}. Therefore assume that $f_o$ and $f'_o$ are different. The patches $\mathcal{P}(f_o)$ and $\mathcal{P}(f'_o)$ and hence the faces $f$ and $f'$ only have a non-empty intersection if $f_o$ and $f'_o$ are adjacent. Since the central $k$-gons of $\mathcal{M}(k)$ are only adjacent to faces of the corresponding patch (since $m > 0$ in \autoref{def:expansion:patch}), we can also assume that neither $f$ nor $f'$ is one of those central $k$-gons and that $f$ and $f'$ are contained in two copies of $\mathcal{M}$. Let $\mathcal{M}_c$ be the copy of $\mathcal{M}$ which contains $f$ (and is a subpatch of $\mathcal{P}(f_o)$) and let $\mathcal{M}_c'$ be the copy of $\mathcal{M}$ which contains $f'$ (and is a subpatch of $\mathcal{P}(f'_o)$). Thus, we know that $\mathcal{M}_c \neq \mathcal{M}'_c$ and $\mathcal{M}_c \cap \mathcal{M}_c' \neq \emptyset$.
\begin{tikzfigure}{\label{fig:const:polymap:overview}}{\autoref{const:map} for $\mathcal{P}(f) = \mathcal{M}(k)$}
  \matrix (m) [column sep=1cm] {
    \begin{scope}[scale=0.5]
      \draw (-3,0)--(-3,3)--(-6,5)--(-6,8)--(-3,10)--(0,8)--(6,8)--(6,5)--(7,1)--(3,0)--(-3,0);
      \draw (-3,3)--(0,5)--(0,8);
      \draw (3,0)--(3,3)--(0,5);
      \draw (3,3)--(7,1);
      \draw (3,3)--(6,5);

      % loose ends
      \draw (-3,0)-- (-3.5,-0.666);
      \draw (-3,3)-- (-3.5,2.5);
      \draw (-6,5)--  (-6.5,4.666);
      \draw (-6,8)-- (-6.5,8.333);
      \draw (-3,10)-- (-3,10.5);
      \draw (0,8)-- (0.5,8.333);
      \draw (6,8)-- (6.5,8.5);
      \draw (6,5)-- (6.5,5);
      \draw (7,1)-- (7.5,0.5);
      \draw (3,0)-- (3.5,-0.666);

      \fill[black] (-3,0) circle(3pt);
      \fill[black] (-3,3)circle(3pt);
      \fill[black] (-6,5)circle(3pt);
      \fill[black] (-6,8)circle(3pt);
      \fill[black] (-3,10)circle(3pt);
      \fill[black] (0,8)circle(3pt);
      \fill[black] (6,8)circle(3pt);
      \fill[black] (6,5)circle(3pt);
      \fill[black] (7,1) circle(3pt);
      \fill[black] (3,0)circle(3pt);
      \fill[black] (0,5)circle(3pt);
      \fill[black] (3,3)circle(3pt);

      \node at (-3,6.5) {$f_1$};
      \node at (0,2.5) {$f_2$};
      \node at (3,5.5) {$f_3$};
      \node at (4.5,1.5) {$f_4$};
      \node at (5,3) {$f_5$};
      
    \end{scope}
    &
    \begin{scope}[scale=0.5]
      \draw (-3,0)--(-3,3)--(-6,5)--(-6,8)--(-3,10)--(0,8)--(6,8)--(6,5)--(7,1)--(3,0)--(-3,0);
      \draw (-3,3)--(0,5)--(0,8);
      \draw (3,0)--(3,3)--(0,5);
      \draw (3,3)--(7,1);
      \draw (3,3)--(6,5);


      \coordinate[shift={(0,1)}] (penta1) at ( 50:0.5) ;
      \coordinate[shift={(0,1)}] (penta2) at (122:0.5) ;
      \coordinate[shift={(0,1)}] (penta3) at (194:0.5) ;
      \coordinate[shift={(0,1)}] (penta4) at (266:0.5) ;
      \coordinate[shift={(0,1)}] (penta5) at (338:0.5) ;

      \coordinate[shift={(1.5,3)}] (penta9) at  ( 20:0.5) ;
      \coordinate[shift={(1.5,3)}] (penta10) at ( 92:0.5) ;
      \coordinate[shift={(1.5,3)}] (penta11) at (164:0.5) ;
      \coordinate[shift={(1.5,3)}] (penta12) at (236:0.5) ;
      \coordinate[shift={(1.5,3)}] (penta13) at (308:0.5) ;

      \coordinate[shift={(-1.5,3.25)}] (hexa1) at (  0:0.5) ;
      \coordinate[shift={(-1.5,3.25)}] (hexa2) at ( 60:0.5) ;
      \coordinate[shift={(-1.5,3.25)}] (hexa3) at (120:0.5) ;
      \coordinate[shift={(-1.5,3.25)}] (hexa4) at (180:0.5) ;
      \coordinate[shift={(-1.5,3.25)}] (hexa5) at (240:0.5) ;
      \coordinate[shift={(-1.5,3.25)}] (hexa6) at (300:0.5) ;

      \coordinate[shift={(2.2,0.65)}] (tri1) at ( 75:0.4) ;
      \coordinate[shift={(2.2,0.65)}] (tri2) at (195:0.4) ;
      \coordinate[shift={(2.2,0.65)}] (tri3) at (315:0.4) ;
      
      \coordinate[shift={(2.7,1.5)}] (tri4) at ( 30:0.4) ;
      \coordinate[shift={(2.7,1.5)}] (tri5) at (150:0.4) ;
      \coordinate[shift={(2.7,1.5)}] (tri6) at (270:0.4) ;
      
 
      \draw (penta5)--(penta1)--(penta2)--(penta3)--(penta4)--(penta5);
      \draw (penta9)--(penta10)--(penta11)--(penta12)--(penta13)--(penta9);

      \draw (hexa1)--(hexa2)--(hexa3)--(hexa4)--(hexa5)--(hexa6)--(hexa1);

      \draw (tri1)--(tri2)--(tri3)--(tri1);
      \draw (tri4)--(tri5)--(tri6)--(tri4);
           
      % loose ends
      \draw (-3,0)-- (-3.5,-0.666);
      \draw (-3,3)-- (-3.5,2.5);
      \draw (-6,5)--  (-6.5,4.666);
      \draw (-6,8)-- (-6.5,8.333);
      \draw (-3,10)-- (-3,10.5);
      \draw (0,8)-- (0.5,8.333);
      \draw (6,8)-- (6.5,8.5);
      \draw (6,5)-- (6.5,5);
      \draw (7,1)-- (7.5,0.5);
      \draw (3,0)-- (3.5,-0.666);

      \fill[black] (-3,0) circle(3pt);
      \fill[black] (-3,3)circle(3pt);
      \fill[black] (-6,5)circle(3pt);
      \fill[black] (-6,8)circle(3pt);
      \fill[black] (-3,10)circle(3pt);
      \fill[black] (0,8)circle(3pt);
      \fill[black] (6,8)circle(3pt);
      \fill[black] (6,5)circle(3pt);
      \fill[black] (7,1) circle(3pt);
      \fill[black] (3,0)circle(3pt);
      \fill[black] (0,5)circle(3pt);
      \fill[black] (3,3)circle(3pt);

      \draw ($(3,0)!0.4!(3,3)$) --(tri2);
      \draw ($(7,1)!0.6!(3,3)$) --(tri1);
      \draw ($(7,1)!0.4!(3,0)$) --(tri3);


      \draw ($(6,5)!0.6!(3,3)$) --(tri5);
      \draw ($(7,1)!0.4!(3,3)$) --(tri6);
      \draw ($(6,5)!0.4!(7,1)$) --(tri4);

      \draw ($(3,3)!0.6!(0,5)$) --(penta1);
      \draw ($(0,5)!0.6!(-3,3)$) --(penta2);
      \draw ($(-3,3)!0.6!(-3,0)$) --(penta3);
      \draw ($(-3,0)!0.6!(3,0)$) --(penta4);
      \draw ($(3,0)!0.6!(3,3)$) --(penta5);

      \draw ($(6,5)!0.6!(6,8)$) --(penta9);
      \draw ($(6,8)!0.6!(0,8)$) --(penta10);
      \draw ($(0,8)!0.6!(0,5)$) --(penta11);
      \draw ($(0,5)!0.6!(3,3)$) --(penta12);
      \draw ($(3,3)!0.6!(6,5)$) --(penta13);

      \draw ($(0,5)!0.6!(0,8)$) --(hexa1);
      \draw ($(0,8)!0.6!(-3,10)$) --(hexa2);
      \draw ($(-3,10)!0.6!(-6,8)$) --(hexa3);
      \draw ($(-6,8)!0.6!(-6,5)$) --(hexa4);
      \draw ($(-6,5)!0.6!(-3,3)$) --(hexa5);
      \draw ($(-3,3)!0.6!(0,5)$) --(hexa6);

      \node at (-1.5,1) {$\mathcal{M}$};
      \node at (-1.7,2.4) {$\mathcal{M}$};
      \node at (1.5,0.8) {$\mathcal{M}$};
      \node at (1.6,2.6) {$\mathcal{M}$};
      \node at (-0.2,3.8) {$\mathcal{M}$};

      \node at (3.8,0.7) {$\mathcal{M}$};
      \node at (3.8,2) {$\mathcal{M}$};
      \node at (5.45,1.2) {$\mathcal{M}$};

      \node at (6,2.5) {$\mathcal{M}$};
      \node at (4.4,3) {$\mathcal{M}$};
      \node at (5.6,3.8) {$\mathcal{M}$};

      \node at (3,4.5) {$\mathcal{M}$};
      \node at (4.8,5.8) {$\mathcal{M}$};
      \node at (4.3,7.3) {$\mathcal{M}$};
      \node at (1.4,7.3) {$\mathcal{M}$};
      \node at (1.4,5.5) {$\mathcal{M}$};

      % \node at (-2.8,4.5) {$\mathcal{M}$};
      % \node at (-1.5,5.8) {$\mathcal{M}$};
      % \node at (-1.6,7.3) {$\mathcal{M}$};
      % \node at (-3.2,8.2) {$\mathcal{M}$};
      % \node at (-4.8,7.2) {$\mathcal{M}$};
      % \node at (-4.8,5.3) {$\mathcal{M}$};

      \node[shift={(-1.5,3.25)}] at ( 30:1.8) {$\mathcal{M}$};
      \node[shift={(-1.5,3.25)}] at ( 90:1.8) {$\mathcal{M}$} ;
      \node[shift={(-1.5,3.25)}] at (150:1.8) {$\mathcal{M}$} ;
      \node[shift={(-1.5,3.25)}] at (210:1.8) {$\mathcal{M}$} ;
      \node[shift={(-1.5,3.25)}] at (270:1.8) {$\mathcal{M}$} ;
      \node[shift={(-1.5,3.25)}] at (330:1.8) {$\mathcal{M}$} ;

      
      \foreach \x in {0.2,0.4,0.6,0.8}
      \fill[black] ($(3,0)!\x!(-3,0)$) circle (3pt);    
      \foreach \x in {0.2,0.4,0.6,0.8}
      \fill[black] ($(-3,0)!\x!(-3,3)$) circle (3pt); 
      \foreach \x in {0.2,0.4,0.6,0.8}
      \fill[black] ($(-3,3)!\x!(-6,5)$) circle (3pt);
      \foreach \x in {0.2,0.4,0.6,0.8}
      \fill[black] ($(-6,5)!\x!(-6,8)$) circle (3pt);
      \foreach \x in {0.2,0.4,0.6,0.8}
      \fill[black] ($(-6,8)!\x!(-3,10)$) circle (3pt);
      \foreach \x in {0.2,0.4,0.6,0.8}
      \fill[black] ($(-3,10)!\x!(0,8)$) circle (3pt);
      \foreach \x in {0.2,0.4,0.6,0.8}
      \fill[black] ($(0,8)!\x!(6,8)$) circle (3pt);
      \foreach \x in {0.2,0.4,0.6,0.8}
      \fill[black] ($(6,8)!\x!(6,5)$) circle (3pt);
      \foreach \x in {0.2,0.4,0.6,0.8}
      \fill[black] ($(6,5)!\x!(7,1)$) circle (3pt);     
      \foreach \x in {0.2,0.4,0.6,0.8}
      \fill[black] ($(7,1)!\x!(3,0)$) circle (3pt); 
      \foreach \x in {0.2,0.4,0.6,0.8}
      \fill[black] ($(3,0)!\x!(3,3)$) circle (3pt);
      \foreach \x in {0.2,0.4,0.6,0.8}
      \fill[black] ($(3,3)!\x!(7,1)$) circle (3pt);
      \foreach \x in {0.2,0.4,0.6,0.8}
      \fill[black] ($(3,3)!\x!(6,5)$) circle (3pt);
      \foreach \x in {0.2,0.4,0.6,0.8}
      \fill[black] ($(3,3)!\x!(0,5)$) circle (3pt);
      \foreach \x in {0.2,0.4,0.6,0.8}
      \fill[black] ($(0,5)!\x!(0,8)$) circle (3pt);
      \foreach \x in {0.2,0.4,0.6,0.8}
      \fill[black] ($(0,5)!\x!(-3,3)$) circle (3pt);      
    \end{scope}
    \\
  };
  \end{tikzfigure}

Recall that in \autoref{const:map} we constructed $M'$ to be subdividing each edge of $M$ and replacing each face with the corresponding patch. We again want to use the notion of ``corner vertices'' for the original vertices in $M$, ``side vertices'' for the vertices added by the subdivision and ``inner vertices'' for all of the other vertices. Note that both $\mathcal{M}_c$ and $\mathcal{M}_c'$ only contain exactly one corner vertex ($o_s$). If $f$ and $f'$ meet non-properly, then $\mathcal{M}_c$ and $\mathcal{M}_c'$ have to share at least one side vertex (they can not share an inner vertex, as these are completely internal to $\mathcal{P}(f_o)$ and $\mathcal{P}(f'_o)$ and those patches only meet at their boundary). As $M$ does not contain $2$-valent vertices, the intersection of $f_o$ and $f'_o$ does not contain paths of length greater than one. Since $\mathcal{M}_c$ and $\mathcal{M}_c'$ do only contain vertices in the subdivision of adjacent edges in $M$, the intersection $\mathcal{M}_c \cap\mathcal{M}_c'$ contains only vertices in the subdivision of a single edge in $M$, see \autoref{fig:thm:polymap}. However, $\mathcal{M}_c$ and $\mathcal{M}_c'$ are then both contained in an edge patch formed along this edge and the faces in $\mathcal{M}_c$ and $\mathcal{M}'_c$ meet properly by definition of the polyhedral property - we have reached a contradiction.

    \begin{tikzfigure}{\label{fig:thm:polymap}}{Copies of $\mathcal{M}$ adjacent to a subdivided edge form the edge patch of $\mathcal{M}$}
      \begin{scope}[scale=0.8]
      \node[shift={(-4,0)}] (0, 0) {$k$-gon};

      \draw[shift={(-5,0)}] (9 : 1) -- (81 : 1) -- (153 : 1)  (225 : 1) -- (297 : 1) -- (9 : 1);
      \draw[shift={(-5,0)}][loosely dotted] (153 : 1) -- (225 : 1);


      \draw[shift={(-5,0)}] (9 : 1) -- (9 : 2.5);
      \draw[shift={(-5,0)}][loosely dotted] (9 : 2.5) -- (9 : 3.5);
      \draw[shift={(-5,0)}] (9 : 3.5) -- (9 : 5.062);
      \draw[shift={(-5,0)}] (81 : 1) -- (81 : 2.5);
      \draw[shift={(-5,0)}][loosely dotted] (81 : 2.5) -- (81 : 3.5);
      \draw[shift={(-5,0)}] (81 : 3.5) -- (81 : 5.062);
      \draw[shift={(-5,0)}] (153 : 1) -- (153 : 2.5);
      \draw[shift={(-5,0)}][loosely dotted] (153 : 2.5) -- (153 : 3.5);
      \draw[shift={(-5,0)}] (153 : 3.5) -- (153 : 5.062);
      \draw[shift={(-5,0)}] (225 : 1) -- (225 : 2.5);
      \draw[shift={(-5,0)}][loosely dotted] (225 : 2.5) -- (225 : 3.5);
      \draw[shift={(-5,0)}] (225 : 3.5) -- (225 : 5.062);
      \draw[shift={(-5,0)}] (297 : 1) -- (297 : 2.5);
      \draw[shift={(-5,0)}][loosely dotted] (297 : 2.5) -- (297 : 3.5);
      \draw[shift={(-5,0)}] (297 : 3.5) -- (297 : 5.062);
      \draw[shift={(-5,0)}][loosely dotted] (4 : 5.012) -- (356 : 5.012);


      \draw[shift={(-5,0)}] (-13 : 5.1315) -- (356 : 5.012) (4 : 5.012) -- (13 : 5.1315);
      \draw[shift={(-5,0)}][loosely dotted] (13 : 5.1315) -- (27 : 5.612);
      \draw[shift={(-5,0)}] (27 : 5.612) -- (36 : 6.180) -- (45 : 5.612);
      \draw[shift={(-5,0)}][loosely dotted] (45 : 5.612) -- (68 : 5.012);
      \draw[shift={(-5,0)}] (68 : 5.012) -- (85 : 5.1315);
      \draw[shift={(-5,0)}][loosely dotted] (85 : 5.1315) -- (99 : 5.612);
      \draw[shift={(-5,0)}] (99 : 5.612) -- (108 : 6.180) -- (117 : 5.612);
      \draw[shift={(-5,0)}][loosely dotted] (117 : 5.612) -- (131 : 5.1315);
      \draw[shift={(-5,0)}] (131 : 5.1315) -- (153 : 5.062);
      \draw[shift={(-5,0)}] (-131 : 5.1315) -- (225 : 5.062);
      \draw[shift={(-5,0)}][loosely dotted] (-117 : 5.612) -- (-131 : 5.1315);
      \draw[shift={(-5,0)}] (-99 : 5.612) -- (-108 : 6.180) -- (-117 : 5.612);
      \draw[shift={(-5,0)}][loosely dotted] (-99 : 5.612) -- (-85 : 5.1315);
      \draw[shift={(-5,0)}] (-59 : 5.1315) -- (-85 : 5.1315);
      \draw[shift={(-5,0)}][loosely dotted] (-45 : 5.612) -- (-85 : 5.1315);
      \draw[shift={(-5,0)}] (-27 : 5.612) -- (-36 : 6.180) -- (-45 : 5.612);
      \draw[shift={(-5,0)}][loosely dotted] (-27 : 5.612) -- (-13 : 5.1315);

      \fill[shift={(-5,0)}] [black] (9 : 1) circle (2pt);
      \node[shift={(-4,0)}][anchor="108"] at (9 : 1) {$i_0$};
      \fill[shift={(-5,0)}] [black] (9 : 2) circle (2pt);
      \node[shift={(-4,0)}][anchor="99"] at (9 : 2) {$i_1$};
      \fill[shift={(-5,0)}] [black] (9 : 4) circle (2pt);
      \node[shift={(-4,0)}][anchor="99"] at (9 : 4) {$i_{m-1}$};
      \fill[shift={(-5,0)}] [black] (9 : 5.062) circle (2pt);
      \node[shift={(-4,0)}][anchor="45"] at (9 : 5.062) {$o_{0}$};
      \node[shift={(-4,0)}][anchor="180"] at (9 : 5.062) {$o_{2s \mod n}$};
      \fill[shift={(-5,0)}] [black] (-9 : 5.062) circle (2pt);
      \node[shift={(-4,0)}][anchor="0"] at (-9 : 5.062) {$o_{2s \mod n}$};
      \fill[shift={(-5,0)}] [black] (-30 : 5.774) circle (2pt);
      \node[shift={(-4,0)}][anchor="0"] at (-30 : 5.774) {$o_{s - 1}$};
      \fill[shift={(-5,0)}] [black] (-36 : 6.180) circle (2pt);
      \node[shift={(-4,0)}][anchor="-36"] at (-36 : 6.180) {$\bm{o_s}$};
      \fill[shift={(-5,0)}] [black] (-42 : 5.774) circle (2pt);
      \node[shift={(-4,0)}][anchor="-36"] at (-42 : 5.774) {$o_{s + 1}$};
      % \fill[shift={(-5,0)}] [black] (-63 : 5.062) circle (2pt);
      % \node[shift={(-4,0)}][anchor="-63"] at (-63 : 5.062) {$o_{n - 1}$};
      \fill[shift={(-5,0)}] [black] (-63 : 5.062) circle (2pt);
      \node[shift={(-4,0)}][anchor="-117"] at (-63 : 5.062) {$o_{n}$};
      \fill[shift={(-5,0)}] [black] (-63 : 4) circle (2pt);
      \node[shift={(-4,0)}][anchor="198"] at (-63 : 4) {$i'_{m-1}$};
      \fill[shift={(-5,0)}] [black] (-63 : 2) circle (2pt);
      \node[shift={(-4,0)}][anchor="198"] at (-63 : 2) {$i'_{1}$};
      \fill[shift={(-5,0)}] [black] (-63 : 1) circle (2pt);
      \node[shift={(-4,0)}][anchor="180"] at (-63 : 1) {$i'_0$};
      \node[shift={(-4,0)}] at (-27 : 3.5) {$\mathcal{M}$};

      \fill[shift={(-5,0)}] [black] (81 : 1) circle (2pt);
      \node[shift={(-4,0)}][anchor="180"] at (81 : 1) {$i_0$};
      \fill[shift={(-5,0)}] [black] (81 : 2) circle (2pt);
      \node[shift={(-4,0)}][anchor="162"] at (81 : 2) {$i_1$};
      \fill[shift={(-5,0)}] [black] (81 : 4) circle (2pt);
      \node[shift={(-4,0)}][anchor="162"] at (81 : 4) {$i_{m-1}$};
      \fill[shift={(-5,0)}] [black] (81 : 5.062) circle (2pt);
      \node[shift={(-4,0)}][anchor="126"] at (81 : 5.062) {$o_{0}$};
      \fill[shift={(-5,0)}] [black] (72 : 5) circle (2pt);
      \node[shift={(-4,0)}][anchor="83"] at (72 : 5) {$o_{1}$};
      \fill[shift={(-5,0)}] [black] (42 : 5.774) circle (2pt);
      \node[shift={(-4,0)}][anchor="42"] at (42 : 5.774) {$o_{s - 1}$};
      \fill[shift={(-5,0)}] [black] (36 : 6.180) circle (2pt);
      \node[shift={(-4,0)}][anchor="36"] at (36 : 6.180) {$\bm{o_s}$};
      \fill[shift={(-5,0)}] [black] (30 : 5.774) circle (2pt);
      \node[shift={(-4,0)}][anchor="0"] at (30 : 5.774) {$o_{s + 1}$};
      %\fill[shift={(-5,0)}] [black] (9 : 5.062) circle (2pt);
      %\node[shift={(-4,0)}][anchor="0"] at (9 : 5.062) {$o_{n - 1}$};
      \fill[shift={(-5,0)}] [black] (9 : 5.062) circle (2pt);
      \node[shift={(-4,0)}][anchor="-45"] at (9 : 5.062) {$o_{n}$};
      \fill[shift={(-5,0)}] [black] (9 : 4) circle (2pt);
      \node[shift={(-4,0)}][anchor="270"] at (9 : 4) {$i'_{m-1}$};
      \fill[shift={(-5,0)}] [black] (9 : 2) circle (2pt);
      \node[shift={(-4,0)}][anchor="270"] at (9 : 2) {$i'_{1}$};
      \fill[shift={(-5,0)}] [black] (9 : 1) circle (2pt);
      \node[shift={(-4,0)}][anchor="252"] at (9 : 1) {$i'_0$};
      \node[shift={(-4,0)}] at (45 : 3.5) {$\mathcal{M}$};

      \node[shift={(-4,0)}] at (117 : 3.5) {$\mathcal{M}$};
      \node[shift={(-4,0)}] at (-99 : 3.5) {$\mathcal{M}$};

      \draw[shift={(-5,0)}][loosely dotted] (180 : 4) -- (198 : 4);

      \node[shift={(4,0)}] (0, 0) {$k'$-gon};

      \draw[shift={(5,0)}, rotate around={180:(0,0)}] (9 : 1) -- (81 : 1) -- (153 : 1)  (225 : 1) -- (297 : 1) -- (9 : 1);
      \draw[shift={(5,0)}, rotate around={180:(0,0)}][loosely dotted] (153 : 1) -- (225 : 1);


      \draw[shift={(5,0)}, rotate around={180:(0,0)}] (9 : 1) -- (9 : 2.5);
      \draw[shift={(5,0)}, rotate around={180:(0,0)}][loosely dotted] (9 : 2.5) -- (9 : 3.5);
      \draw[shift={(5,0)}, rotate around={180:(0,0)}] (9 : 3.5) -- (9 : 5.062);
      \draw[shift={(5,0)}, rotate around={180:(0,0)}] (81 : 1) -- (81 : 2.5);
      \draw[shift={(5,0)}, rotate around={180:(0,0)}][loosely dotted] (81 : 2.5) -- (81 : 3.5);
      \draw[shift={(5,0)}, rotate around={180:(0,0)}] (81 : 3.5) -- (81 : 5.062);
      \draw[shift={(5,0)}, rotate around={180:(0,0)}] (153 : 1) -- (153 : 2.5);
      \draw[shift={(5,0)}, rotate around={180:(0,0)}][loosely dotted] (153 : 2.5) -- (153 : 3.5);
      \draw[shift={(5,0)}, rotate around={180:(0,0)}] (153 : 3.5) -- (153 : 5.062);
      \draw[shift={(5,0)}, rotate around={180:(0,0)}] (225 : 1) -- (225 : 2.5);
      \draw[shift={(5,0)}, rotate around={180:(0,0)}][loosely dotted] (225 : 2.5) -- (225 : 3.5);
      \draw[shift={(5,0)}, rotate around={180:(0,0)}] (225 : 3.5) -- (225 : 5.062);
      \draw[shift={(5,0)}, rotate around={180:(0,0)}] (297 : 1) -- (297 : 2.5);
      \draw[shift={(5,0)}, rotate around={180:(0,0)}][loosely dotted] (297 : 2.5) -- (297 : 3.5);
      \draw[shift={(5,0)}, rotate around={180:(0,0)}] (297 : 3.5) -- (297 : 5.062);
      
      % \draw[shift={(5,0)}, rotate around={180:(0,0)}] (-13 : 5.1315) -- (13 : 5.1315);
      % \draw[shift={(5,0)}, rotate around={180:(0,0)}][loosely dotted] (13 : 5.1315) -- (27 : 5.612);
      \draw[shift={(5,0)}, rotate around={180:(0,0)}] (36 : 6.180) -- (45 : 5.612);
      \draw[shift={(5,0)}, rotate around={180:(0,0)}][loosely dotted] (45 : 5.612) -- (59 : 5.1315);
      \draw[shift={(5,0)}, rotate around={180:(0,0)}] (59 : 5.1315) -- (85 : 5.1315);
      \draw[shift={(5,0)}, rotate around={180:(0,0)}][loosely dotted] (85 : 5.1315) -- (99 : 5.612);
      \draw[shift={(5,0)}, rotate around={180:(0,0)}] (99 : 5.612) -- (108 : 6.180) -- (117 : 5.612);
      \draw[shift={(5,0)}, rotate around={180:(0,0)}][loosely dotted] (117 : 5.612) -- (131 : 5.1315);
      \draw[shift={(5,0)}, rotate around={180:(0,0)}] (131 : 5.1315) -- (153 : 5.062);
      \draw[shift={(5,0)}, rotate around={180:(0,0)}] (-131 : 5.1315) -- (225 : 5.062);
      \draw[shift={(5,0)}, rotate around={180:(0,0)}][loosely dotted] (-117 : 5.612) -- (-131 : 5.1315);
      \draw[shift={(5,0)}, rotate around={180:(0,0)}] (-99 : 5.612) -- (-108 : 6.180) -- (-117 : 5.612);
      \draw[shift={(5,0)}, rotate around={180:(0,0)}][loosely dotted] (-99 : 5.612) -- (-85 : 5.1315);
      \draw[shift={(5,0)}, rotate around={180:(0,0)}] (-59 : 5.1315) -- (-85 : 5.1315);
      \draw[shift={(5,0)}, rotate around={180:(0,0)}][loosely dotted] (-45 : 5.612) -- (-85 : 5.1315);
      \draw[shift={(5,0)}, rotate around={180:(0,0)}] (-36 : 6.180) -- (-45 : 5.612);
      %\draw[shift={(5,0)}, rotate around={180:(0,0)}][loosely dotted] (-27 : 5.612) -- (-13 : 5.1315);

      \fill[shift={(5,0)}, rotate around={180:(0,0)}] [black] (9 : 1) circle (2pt);
      \node[shift={(4,0)}][anchor="282"] at (189 : 1) {$i_0$};
      \fill[shift={(5,0)}, rotate around={180:(0,0)}] [black] (9 : 2) circle (2pt);
      \node[shift={(4,0)}][anchor="270"] at (189 : 2) {$i_1$};
      \fill[shift={(5,0)}, rotate around={180:(0,0)}] [black] (9 : 4) circle (2pt);
      \node[shift={(4,0)}][anchor="270"] at (189 : 4) {$i_{m-1}$};
      \fill[shift={(5,0)}, rotate around={180:(0,0)}] [black] (9 : 5.062) circle (2pt);
      \node[shift={(4,0)}][anchor="225"] at (189 : 5) {$o_{0}$};
      \fill[shift={(5,0)}, rotate around={180:(0,0)}] [black] (-30 : 5.774) circle (2pt);
      \node[shift={(4,0)}][anchor="180"] at (150 : 5.774) {$o_{s - 1}$};
      \fill[shift={(5,0)}, rotate around={180:(0,0)}] [black] (-36 : 6.180) circle (2pt);
      \node[shift={(4,0)}][anchor="144"] at (144 : 6.180) {$\bm{o_s}$};
      \fill[shift={(5,0)}, rotate around={180:(0,0)}] [black] (-42 : 5.774) circle (2pt);
      \node[shift={(4,0)}][anchor="144"] at (138 : 5.774) {$o_{s + 1}$};
      \fill[shift={(5,0)}, rotate around={180:(0,0)}] [black] (-63 : 5.062) circle (2pt);
      \node[shift={(4,0)}][anchor="63"] at (117 : 5) {$o_{n}$};
      \fill[shift={(5,0)}, rotate around={180:(0,0)}] [black] (-63 : 4) circle (2pt);
      \node[shift={(4,0)}][anchor="18"] at (117 : 4) {$i'_{m-1}$};
      \fill[shift={(5,0)}, rotate around={180:(0,0)}] [black] (-63 : 2) circle (2pt);
      \node[shift={(4,0)}][anchor="18"] at (117 : 2) {$i'_{1}$};
      \fill[shift={(5,0)}, rotate around={180:(0,0)}] [black] (-63 : 1) circle (2pt);
      \node[shift={(4,0)}][anchor="0"] at (117 : 1) {$i'_0$};
      \node[shift={(4,0)}] at (153 : 3.5) {$\mathcal{M}$};

      \fill[shift={(5,0)}, rotate around={180:(0,0)}] [black] (81 : 1) circle (2pt);
      \node[shift={(4,0)}][anchor="0"] at (261 : 1) {$i_0$};
      \fill[shift={(5,0)}, rotate around={180:(0,0)}] [black] (81 : 2) circle (2pt);
      \node[shift={(4,0)}][anchor="342"] at (261 : 2) {$i_1$};
      \fill[shift={(5,0)}, rotate around={180:(0,0)}] [black] (81 : 4) circle (2pt);
      \node[shift={(4,0)}][anchor="342"] at (261 : 4) {$i_{m-1}$};
      \fill[shift={(5,0)}, rotate around={180:(0,0)}] [black] (81 : 5.062) circle (2pt);
      \node[shift={(4,0)}][anchor="300"] at (261 : 5.062) {$o_{0}$};
      \fill[shift={(5,0)}, rotate around={180:(0,0)}] [black] (42 : 5.774) circle (2pt);
      \node[shift={(4,0)}][anchor="216"] at (222 : 5.774) {$o_{s - 1}$};
      \fill[shift={(5,0)}, rotate around={180:(0,0)}] [black] (36 : 6.180) circle (2pt);
      \node[shift={(4,0)}][anchor="216"] at (216 : 6.180) {$\bm{o_s}$};
      \fill[shift={(5,0)}, rotate around={180:(0,0)}] [black] (30 : 5.774) circle (2pt);
      \node[shift={(4,0)}][anchor="180"] at (210 : 5.774) {$o_{s + 1}$};
      \fill[shift={(5,0)}, rotate around={180:(0,0)}] [black] (9 : 5.062) circle (2pt);
      \node[shift={(4,0)}][anchor="135"] at (189 : 5.062) {$o_{n}$};
      \fill[shift={(5,0)}, rotate around={180:(0,0)}] [black] (9 : 4) circle (2pt);
      \node[shift={(4,0)}][anchor="90"] at (189 : 4) {$i'_{m-1}$};
      \fill[shift={(5,0)}, rotate around={180:(0,0)}] [black] (9 : 2) circle (2pt);
      \node[shift={(4,0)}][anchor="90"] at (189 : 2) {$i'_{1}$};
      \fill[shift={(5,0)}, rotate around={180:(0,0)}] [black] (9 : 1) circle (2pt);
      \node[shift={(4,0)}][anchor="72"] at (189 : 1) {$i'_0$};
      \node[shift={(4,0)}] at (225 : 3.5) {$\mathcal{M}$};

      \node[shift={(4,0)}] at (297 : 3.5) {$\mathcal{M}$};
      \node[shift={(4,0)}] at (81 : 3.5) {$\mathcal{M}$};

      \draw[shift={(5,0)}, rotate around={180:(0,0)}][loosely dotted] (180 : 4) -- (198 : 4);

      \end{scope}

    \end{tikzfigure}

\end{proof}
\end{proposition}


\begin{remark} As we have seen in the proof of \autoref{thm:const:polymap} and in \autoref{fig:thm:polymap} the edge patch serves as a replacement figure of an edge in \autoref{const:map}, thus its name.
\end{remark}

\begin{example}\label{ex:easy:expansion3}
  Using the expansion patches $\mathcal{H}$ and $\mathcal{Q}_2$ from \autoref{ex:easy:expansion1} and \autoref{ex:easy:expansion2}, we can finally keep the promise from \autoref{rem:gen:is:no:gen} and deliver a construction which adds to a polyhedral map arbitrary many hexagons (or quadrangles) while inserting only $3$-valent (or $4$-valent) vertices. Given a map $M$ on a closed oriented $2$-manifold, we can simply use \autoref{thm:const:polymap} repeatedly on $M$ with either $\mathcal{H}$ or $\mathcal{Q}_2$ to get the desired result. The theorem inserts at least a single hexagon or quadrangle during each step (which is quite an understatement, the number of polygons added is by far larger), so repeating this step eventually leads to to a map which has more than a specified amount of hexagons or quadrangles.
\end{example}

Putting all these constructions together, we can formulate a proof strategy for \autoref{problem:eberhard}. It has an edge case we have to exclude when we are unable \autoref{thm:eberhard:extended:3}. In \autoref{sec:negative:results} we will see that this case in fact provides no additional $q$-$w$-realizable pairs of sequences.
\begin{proposition}\label{thm:main:const} Let $r = 3$ (resp. $r = 4$) and $p = (p_3, \dots, p_m)$ and $q = (q_3, \dots q_n)$ be an admissible pair of sequences. Let $w = [r]$ and $q = [q_s \times s, q_l \times l]$, where $s \in \set{3, 4, 5}$, $l > 6$, $q_s = \frac{l - 6}{\gcd(6 - s, l - 6)}$ and $q_l = \frac{6 - s}{\gcd(6 - s, l - 6)}$ (resp. $s = 3$, $l > 4$, $q_s = l - 4$ and $q_l = 1$). We want to exclude the single case of $r=3$, $s = 4$, $\chi = 2$, $2 | l$ and
  \begin{align*}
    \sum_{k=3 \atop 2 \nmid k}^{m} p_k = 0 \text{\quad and\quad}\sum_{k=4 \atop 3 \nmid k}^n v_k = 1.
  \end{align*}
Assume there exist
  \begin{itemize}
  \item an expansion $r$-patch $\mathcal{P}_N$ with outer tuple $o$ consisting of $s$-gons and $l$-gons,
  \item an $o$-$6$-gonal (resp. $o$-$4$-gonal) $r$-patch $\mathcal{P}_F$ consisting of $s$-gons and $l$-gons,
  \item an expansion $r$-patch $\mathcal{P}_P$ with the polyhedral property consisting of $s$-gons and $l$-gons.
  \end{itemize}
  Then $(p, v)$ is $q$-$w$-realizable.
  \begin{proof}
    If $r = 3$, we start by using \autoref{thm:eberhard:extended:3} on $p + [q_s \times s, q_l \times l]$ and $v$ to construct a map $M'$ with $p$-vector $p' = p + 2[q_s \times s, q_l \times l] + c'_6 \cdot [6]$ and $v$-vector $v' = v + d' \cdot [3]$ ($c'_6, d' \in \nats$). Note that by adding $2[q_s \times s, q_l \times l]$ we circumvent the exceptional cases of \autoref{thm:eberhard:extended:3} for $\chi = 0$ and $\chi = 2$ with special care taken if $r = 3$ and $s = 4$ in the statement of our theorem. We chose $c'_6$ hexagonal faces of $M'$ and assign to those the patch $\mathcal{P}_F$, every other $k$-gonal face is assigned the patch $\mathcal{P}_N(k)$. Using \autoref{const:map} returns a new map $M''$. Finally, we use \autoref{thm:const:polymap} with $\mathcal{P}_P$ and get a polyhedral map $M$. If we can show that this map has a $p$-vector of the form $p +  [c_s \times s, c_l \times l]$ and a $v$-vector of the form $v + d \cdot [r]$, we can conclude the claim by \autoref{lem:2:valued:eberhard}. 

    Thus we look at the $p$ and $v$-vector of the maps $M''$ and finally $M$: $M''$ has a $p$-vector of $p + [c''_s \times s, c''_l \times s]$, since we removed the additional $c'_6$ hexagons in $M'$ and replaced them with many more $s$-gons and $l$-gons. When doing this we only added $r$-valent vertices, so the $v$-vector of $M''$ is clearly $v'' \defeq v + d'' \cdot [3]$. Finally the $p$-vector and the $v$-vector of $M$ is of the desired form, since we only added more $s$-gons, $l$-gons and $r$-valent vertices in the last construction step, so by \autoref{lem:2:valued:eberhard} we have held a $q$-$w$-realization. 

    For $r = 4$, we just switch out quadrangles for hexagons, $4$-valent vertices for $3$-valent vertices, $c'_4$ for $c'_6$ and use \autoref{thm:eberhard:extended:4} in the beginning. Every step in the proof is completely analogous.
  \end{proof}
\end{proposition}

\begin{remark}
  We will use \autoref{thm:main:const} heavily in the next five sections. Therefore we want to stress what is needed to check if the prerequisites of \autoref{thm:main:const} are fulfilled. As is stated, we need three $r$-patches, $\mathcal{P}_N$, $\mathcal{P}_F$ and $\mathcal{P}_P$, which we will call in this manner for the rest of the thesis. The list of properties is:
  \begin{itemize}
    \item $\mathcal{P}_N$, $\mathcal{P}_F$ and $\mathcal{P}_P$ consist of only $s$-gons and $l$-gons and all inner vertices have valence $r$.
    \item $\mathcal{P}_N$ and $\mathcal{P}_P$ are expansion patches:
      \begin{itemize}
      \item $i_0$ and $i'_0$ are in sum incident to $r-1$ faces,
      \item $i_k$ and $i'_k$, are in sum incident to $r$ faces, $1 \leq k < m$,
      \item starting at the vertex $o_s$ and going in both directions for each pair of vertices $o_{s+k \mod n}$ and $o_{s-k \mod n}$ the identity $w_{\mathcal{P}_X}(o_{s+k \mod n}) + w_{\mathcal{P}_X}(o_{s-k \mod n}) = r$ holds, where we ``identify'' $o_n$ with $o_0$. For ease of comparison we also state the outer tuple $o$ for $\mathcal{P}_N$. 
      \end{itemize}
    \item $\mathcal{P}_F$ is $o$-$6$-gonal or $o$-$4$-gonal, i.e. if starting at some vertex and looking at the number of inner faces incident to this vertex we see the pattern
      \begin{align*}
        \underbrace{1, (\dots o \dots), 1, (\dots o \dots), \dots, 1, (\dots o \dots)}_{\text{six or four times}},
      \end{align*}
      where $o$ is the outer tuple of $\mathcal{P}_N$.
    \item $\mathcal{P}_P$ has the polyhedral property. For this we provide the corresponding edge patch to make the verification easier.
 \end{itemize}
\end{remark}