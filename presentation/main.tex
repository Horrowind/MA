\documentclass[10pt, notheorems]{beamer}
\usepackage[utf8]{inputenc}

\usepackage[ngerman]{babel}
\usepackage{array}
\usepackage{tabularx}
\usepackage{tikz}
\usepackage{amssymb}
\usepackage{ragged2e}
\usepackage{amsthm}
\usepackage{aliascnt}
\usepackage{tikz-3dplot}

\newcommand{\newthm}[3]{%
  \newaliascnt{#2}{#1}%
  \newtheorem{#2}[#2]{#3}%
  \aliascntresetthe{#2}%
  \expandafter\def\csname#2autorefname\endcsname{#3}%
}

\newtheorem{theorem}{Satz}[section]
\newthm{theorem}{example}{Beispiel}
\newthm{theorem}{definition}{Definition}
\newthm{theorem}{proposition}{Proposition}
\newthm{theorem}{lemma}{Lemma}

\usetikzlibrary{
  shapes.geometric,
  matrix,
  intersections,
  positioning,
  arrows,
  arrows.meta,
  decorations.markings,
  calc
}

\tikzset{
     lface/.style={dotted,     decoration={markings, mark=at position 1.0 with {\arrow[>=angle 60]{>}}}, postaction={decorate}},
   lsquare/.style={very thick, decoration={markings, mark=at position {0.5*\pgfdecoratedpathlength+3.5pt} with {\arrow{open square}}}, postaction={decorate}},
  ldiamond/.style={very thick, decoration={markings, mark=at position {0.5*\pgfdecoratedpathlength+5.5pt} with {\arrow{open diamond}}}, postaction={decorate}},
  lvertex/.style={draw, very thick, circle}
}

\setbeamertemplate{background canvas}[vertical shading][bottom=gray!15,top=gray!0]
% stuff from CambridgeUS
\setbeamerfont{block title}{size={}}
\setbeamercolor{titlelike}{parent=structure}
\setbeamertemplate{navigation symbols}{}

\newcommand*\smallsquare{\mathbin{\vcenter{\hbox{\rule{.5ex}{.5ex}}}}}

%% inner theme stuff
%my version of \useinnertheme[shadow]{rounded} mixed with rectangles
\setbeamertemplate{blocks}[rounded][shadow=true]
\setbeamertemplate{items}[square]
\setbeamertemplate{sections/subsections in toc}[square]
\setbeamertemplate{title page}[default][colsep=-4bp,rounded=true,shadow=true]
\setbeamertemplate{part page}[default][colsep=-4bp,rounded=true,shadow=true]
\setbeamertemplate{itemize items}{$\smallsquare$}
\setbeamertemplate{enumerate items}{$\smallsquare$}

% my version of \useoutertheme{infolines} without frame counter
\setbeamercolor*{author in head/foot}{parent=palette tertiary}
\setbeamercolor*{title in head/foot}{parent=palette secondary}
\setbeamercolor*{date in head/foot}{parent=palette primary}
\setbeamercolor*{bibliography entry author}{parent=palette primary}
\setbeamercolor*{bibliography entry location}{parent=palette secondary}
%\setbeamercolor*{section in head/foot}{parent=palette tertiary}
%\setbeamercolor*{subsection in head/foot}{parent=palette primary}
\defbeamertemplate*{footline}{infolines theme}
{
  \leavevmode%
  \hbox{%
  \begin{beamercolorbox}[wd=.35\paperwidth,ht=2.25ex,dp=1ex,center]{author in head/foot}%
    \usebeamerfont{author in head/foot}\insertshorttitle
  \end{beamercolorbox}%
  \begin{beamercolorbox}[wd=.60\paperwidth,ht=2.25ex,dp=1ex,center]{title in head/foot}%
    \usebeamerfont{title in
      head/foot}
    \ifx\insertsubsectionhead\empty
    \relax
    \else~~/~~\insertsubsectionhead{}\fi%
    \insertauthor{}
  \end{beamercolorbox}%
  \begin{beamercolorbox}[wd=.05\paperwidth,ht=2.25ex,dp=1ex,right]{date in head/foot}%
    \usebeamerfont{date in head/foot}%\insertsectiontitle{}\hspace*{2em}
    \insertframenumber{}~~~ %/ \inserttotalframenumber\hspace*{2ex} 
  \end{beamercolorbox}
  }%
  \vskip0pt%
}

%\defbeamertemplate*{headline}{infolines theme}
%{
%  \leavevmode%
%  \hbox{%
%  \begin{beamercolorbox}[wd=.5\paperwidth,ht=4.25ex,dp=1ex,left]{section in head/foot}%
%    \usebeamerfont{section in head/foot}\insertsectionhead\hspace*{2ex}
%  \end{beamercolorbox}%
%  \begin{beamercolorbox}[wd=.5\paperwidth,ht=4.25ex,dp=1ex,left]{subsection in head/foot}%
%    \usebeamerfont{subsection in head/foot}\hspace*{2ex}\insertsubsectionhead
%  \end{beamercolorbox}}%   
%\vskip0pt%
%}

\setbeamersize{text margin left=1em,text margin right=1em}

% colour themes
%\usecolortheme{beaver}
\definecolor{darkred}{rgb}{0.8,0,0}

\setbeamercolor{section in toc}{fg=black,bg=white}
\setbeamercolor{alerted text}{fg=darkred!60!black}
\setbeamercolor*{palette primary}{fg=darkred!60!black,bg=gray!30!white}
\setbeamercolor*{palette secondary}{fg=darkred!60!black,bg=gray!15!white}
\setbeamercolor*{palette tertiary}{bg=darkred!60!black,fg=gray!10!white}
\setbeamercolor*{palette quaternary}{fg=darkred,bg=gray!5!white}

\setbeamercolor*{sidebar}{fg=darkred,bg=gray!15!white}

\setbeamercolor*{palette sidebar primary}{fg=darkred!10!black}
\setbeamercolor*{palette sidebar secondary}{fg=white}
\setbeamercolor*{palette sidebar tertiary}{fg=darkred!50!black}
\setbeamercolor*{palette sidebar quaternary}{fg=gray!10!white}

%\setbeamercolor*{titlelike}{parent=palette primary}
\setbeamercolor{titlelike}{parent=pallette primary,fg=darkred!60!black}
\setbeamercolor{frametitle}{bg=gray!10!white}
\setbeamercolor{frametitle right}{bg=gray!60!white}

\setbeamercolor*{separation line}{}
\setbeamercolor*{fine separation line}{}


%colour of the item box
\setbeamercolor{item}{use={structure,normal text},fg=structure.fg!50!normal
  text.fg, bg=gray!20}
%\setbeamercolor{example text}{use=structure,fg=darkred!60!black,bg=gray!30!white}
\setbeamercolor{block title}{use=structure,fg=darkred!60!black,bg=gray!30!white}
\setbeamercolor{block title alerted}{use=alerted text,bg=darkred!60!black,fg=gray!0!white}
\setbeamercolor{block title example}{use=example text,fg=darkred!60!black,bg=gray!30!white}

\setbeamercolor{block body}{parent=normal text,use=block title,bg=block title.bg!30!bg}
\setbeamercolor{block body alerted}{parent=normal text,use=block title alerted,bg=block title alerted.bg!10!bg}
\setbeamercolor{block body example}{parent=normal text,use=block title,bg=block title.bg!30!bg}

\setbeamercolor{item}{fg=darkred!60!black}

%fontthemes & stuff
%\usefonttheme[onlysmall]{structurebold}
%\usepackage{times}  % times
%\usepackage{helvet}    %helvetica
%\usefonttheme{serif}
%no preview
\setbeamercovered{transparent}

\setlength{\parskip}{\smallskipamount}
\usepackage{lmodern}
\newcommand{\todo}[1]{(TODO)}
\setbeamertemplate{background canvas}[vertical shading][bottom=gray!0,top=gray!0]

\let\emptyset\varnothing
\newcommand{\set}[1]{\{ #1 \}}
\newcommand{\nats}{\mathbb{N}}
\newcommand{\wholes}{\mathbb{Z}}
\newcommand{\reals}{\mathbb{R}}
\newcommand{\sphere}{\mathbb{S}}

\renewcommand{\mod}{\operatorname{mod}}
\renewcommand{\gcd}{\operatorname{ggT}}



\definecolor{darkred}{rgb}{0.8,0,0}
\colorlet{darkred2}{darkred!60!black}
\newcommand{\hdef}[1]{\textcolor{darkred2}{#1}}
\newcommand{\red}[1]{\textcolor{darkred2}{#1}}
\newcommand{\ho}[2]{\only<#1->{\alert<#1>{#2}}}

\AtBeginSection[]{%
  \begin{frame}%
  \vfill%
  \centering%
  \begin{beamercolorbox}[sep=8pt,center,shadow=true,rounded=true]{title}
    \usebeamerfont{title}\insertsectionhead\par%
  \end{beamercolorbox}
  \vfill%
  \end{frame}%
}


\begin{document}
\title{Neue Sätze vom {\sc Eberhard}-Typ}
\author{Sebastian Manecke}
\date{Kolloqium zur Masterarbeit\\Betreut durch Prof. Dr. Ulrich Brehm\\22. November 2016}



\frame{\titlepage}



\begin{frame}
  %TODO: Dreieck, Viereck, Fünfeck
  Gliederung:
\tableofcontents
\end{frame}
\section{$3$-Polytope mit vorgegebenen Flächen- und Eckentypen}
\begin{frame}{$3$-Polytope mit vorgegebenen Flächen und Eckentypen}
  Für ein $3$-Polytop wollen wir die Anzahl der $k$-gonale Flächen mit $p_k$ und die Anzahl der Ecken, in denen $k$ Kanten zusammenstoßen, ($k$-valente Ecken) mit $v_k$ bezeichnen $(k \geq 3)$.
  \pause
  \begin{example}
    Für ein Tetraeder ist $p_3 = 4$ und $v_3 = 4$ und $v_k = p_k = 0$ für alle anderen $k \geq 4$.\\
    \pause
    Für ein Dodekaeder ist $p_5 = 12$ und $v_3 = 20$ und $v_k = p_k = 0$ sonst.\\
  \end{example}
\end{frame}
\begin{frame}{$3$-Polytope mit vorgegebenen Flächen und Eckentypen}
  Eine naheliegende Frage ist, ob wir zu jeder Vorgabe von $p_k$'s und $v_k$'s auch ein $3$-Polytop (eine Realisierung) finden können, welches genau $p_k$ $k$-gonale Flächen besitzt und dessen Anzahl an $k$-valenten Ecken genau $v_k$ ist.
  \pause
  \begin{example}
    Für $p_4 = 500$, $v_3 = 1$ und $p_k = v_k = 0$ sonst, gibt es (offensichtlich) kein $3$-Polytop.\\
    \pause
    Für $p_5 = 12$, $p_6 = 1$, $v_3 = 22$ und $p_k = v_k = 0$ sonst, gibt es ebenfalls kein $3$-Polytop. Beide Beispiele sind also nicht realisierbar.
  \end{example}
\end{frame}

\begin{frame}{$3$-Polytope mit vorgegebenen Flächen und Eckentypen}
  Es ergibt sich (grob vereinfacht) folgendes Bild, Punkte bezeichnen realisierbare Anzahlen $p = (p_3, p_4, \dots)$ und $v = (v_3, v_4, \dots)$:

  \vspace{0.5cm}
  { \centering
    \begin{tikzpicture}
      \coordinate (v) at (0,4);
      \coordinate (p) at (4,0);
      \draw[<->] (v) node[above] {$v$} -- (0,0) --  (p) node[right] {$p$};
      \foreach \point in {1,...,120}{
        \pgfmathparse{mod(\point, 13)}
        \pgfmathsetmacro\xpos{0.5 + 3.5 / 13 * \pgfmathresult}
        \pgfmathparse{mod(\point + 2, 60)}
        \pgfmathsetmacro\ypos{0.5 + 3.5 / 60 * \pgfmathresult}
        \fill[black] (\xpos,\ypos) circle (1pt);
      }
      \only<2->{
        \fill[darkred2] (2, 2.2) circle (1pt) {};
        \node[darkred2, anchor=180] at (2, 2.2) {?}; 
      }
    \end{tikzpicture}\\
  }
  \vspace{0.5cm}
  \pause
  Zu bestimmen, ob ein Punkt im Diagramm realisierbar ist, d.h. ob es ein $3$-Polytop mit vorgegebenen Flächen und Ecken gibt, ist schwierig.
\end{frame}

\begin{frame}{Eine kleine Rechnung}
  Wir betrachten ein $3$-valentes $3$-Polytop (es stoßen in jeder Ecke drei Kanten zusammen), welches $p_3$ Dreiecke, $p_4$ Vierecke, \dots, $p_m$ $m$-gonale Flächen hat.\\
  \medskip
  \pause
  Durch  einfaches und doppeltes Abzählen (jede Kante ist benachbart zu genau zwei Ecken und zu genau zwei Flächen), können wir die Anzahl der Flächen ($f_2$), Kanten ($f_1$) und Ecken ($f_0$) bestimmen:
  \begin{align*}
    f_2 = \sum_{k \geq 3} p_k \qquad f_1 = \tfrac{1}{2}\sum_{k \geq 3} k p_k \qquad f_0 = \tfrac{1}{3}\sum_{k \geq 3} k p_k
  \end{align*}
  \pause
  Von der {\sc Euler}schen Polyederformel wissen wir $f_0 - f_1 + f_2 = 2$ und können dies zur folgenden Gleichung umstellen:
  \begin{align*}
    \sum_{k \geq 3} (6 - k) p_k = 12
  \end{align*}
\end{frame}

\begin{frame}{Interpretation von $\sum_{k \geq 3} (6 - k) p_k = 12$}
  Wir können aus der vorangegangen Formel diskrete Krümmungen für Polygone angeben:

  \centering
    \begin{tikzpicture}[every text node part/.style={align=left}]
      \node (x1) at (0, 0) {$\sum_{k \geq 3} (6 - k) p_k = 12$};
      \pause
      \node[rectangle, text width = 3.5cm, anchor=90] (x2) at (-2, -1) {Die Krümmung einer $k$-gonalen Fläche ist \\ - für $k \leq 5$ positiv\\ - für $k = 6$ null (flach)\\ - für $k \geq 7$ negativ};
      \draw[->] (x1.215) -- (x2);
      \pause
      \node[rectangle, text width = 3.5cm, anchor=90] (x3) at ( 2.5, -1) {Es sind zwölf Krümmungseinheiten notwendig};
      \draw[->] (x1.346) -- (x3);
    \end{tikzpicture}
  
  \pause
  \justifying
  Wir sehen insbesondere, dass die Anzahl der Sechsecke für die Formel irrelevant ist.
\end{frame}

\begin{frame}
  Dieselbe Rechnung für $4$- und $5$-valente $3$-Polytope ergibt:

  \medskip
  \begin{beamercolorbox}[sep=-10pt,center,shadow=true,rounded=true]{block body}
    \begin{align*}
      4\text{-valent:\quad}&\sum_{k \geq 3} (k-4)p_k = 8\\
      5\text{-valent:\quad}&\sum_{k \geq 3} (3k-10)p_k = 20
    \end{align*}
  \end{beamercolorbox}
  \medskip
  \pause
  Man beachte, dass es im $5$-valenten Fall kein ``flaches'' Polygon gibt.
\end{frame}

\begin{frame}{{\sc Eberhard}s Theorem}
  \begin{theorem}[\sc Eberhard]
    Seien natürliche Zahlen $p_3$, $p_4$, $p_5$, $p_7$, \dots, $p_m$ so gegeben, dass
    \begin{align*}
      \sum_{k \geq 3} (6 - k) p_k = 12
    \end{align*}
    gilt. Dann gibt es eine Anzahl $p_6$ und ein $3$-valentes $3$-Polytop, dessen Anzahl an $k$-gonalen Flächen genau $p_k$ entspricht.   
  \end{theorem}
  \pause
  \begin{theorem}[\sc Grünbaum]
    Seien natürliche Zahlen $p_3$, $p_5$, $p_7$, \dots, $p_m$ so gegeben, dass
    \begin{align*}
      \sum_{k \geq 3} (4 - k) p_k = 8
    \end{align*}
    gilt. Dann gibt es eine Anzahl $p_4$ und ein $4$-valentes $3$-Polytop, dessen Anzahl an $k$-gonalen Flächen genau $p_k$ entspricht.   
  \end{theorem}
\end{frame}

\begin{frame}
  In unserem Schaubild:

  \vspace{0.5cm}
  \begin{columns}
    \begin{column}{0.3\textwidth}\\

      { \centering
        \begin{tikzpicture}
          \coordinate (v) at (0,4);
          \coordinate (p) at (4,0);
          \draw[<->] (v) node[above] {$v$} -- (0,0) --  (p) node[right] {$p$};
          \foreach \point in {1,...,120}{
            \pgfmathparse{mod(\point, 13)}
            \pgfmathsetmacro\xpos{0.5 + 3.5 / 13 * \pgfmathresult}
            \pgfmathparse{mod(\point + 2, 60)}
            \pgfmathsetmacro\ypos{0.5 + 3.5 / 60 * \pgfmathresult}
            \fill[black] (\xpos,\ypos) circle (1pt);
          }
          \only<1-7>{
            \node[darkred2, anchor=10] at (2, 2.2) {?}; 
          }
          \only<2-7>{
            \draw[darkred2, ->] (2, 2.2) -- (2.3, 2.4); 
          }
          \only<3-7>{
            \draw[darkred2, ->] (2.3, 2.4) -- (2.6, 2.6); 
          }
          \only<4-7>{
            \draw[darkred2, ->] (2.6, 2.6) -- (2.9, 2.8); 
          }
          \only<5-7>{
            \draw[darkred2, ->] (2.9, 2.8) -- (3.2, 3.0); 
            \node[darkred2, anchor=240] at (3.2, 3.0) {!}; 
          }
          \fill[darkred2] (2, 2.2) circle (1pt) {};

          \only<8>{
            \draw[darkred2, ->] (2.00, 2.2) -- (2.36, 2.2); 
            \draw[darkred2, ->] (2.00, 2.2) -- (2.0, 2.56);
            \draw[darkred2    ] (2.36, 2.2) arc (0:90:0.36);
          }
        \end{tikzpicture}
      }
    \end{column}
    \begin{column}{0.5\textwidth}
      \begin{itemize}
        \uncover<6->{\item Der rote Pfeil entspricht für das klassische $3$-valente {\sc Eberhard}-Theorem einem Hinzufügen von einem Sechseck und zwei $3$-valenten Ecken.}
        \medskip
        \uncover<7->{\item Inhalt der Arbeit: ähnliche Theoreme sowohl für andere Richtungen und auch für Graphen auf Mannigfaltigkeiten zu zeigen\\ (im Vortrag jedoch weiterhin auf $3$-Polytope, der Unterschied tritt nur in Detailfragen auf)}
        \medskip
        \uncover<8->{\item nur bestimmte / ``flache'' Richtungen kommen hierbei infrage}
      \end{itemize}
    \end{column}
  \end{columns}
\end{frame}
\section{Allgemeine {\sc Eberhard}-Typ Probleme}
\begin{frame}{Sequenzen}
  \begin{definition}[Sequenz]
    Eine \hdef{Sequenz} $a$ ist eine Abbildung $a : \nats \setminus \set{0, 1, 2} \to \nats, k \mapsto a_k$ mit endlichem Träger. Auf der Menge der Sequenzen seien Addition von zwei Sequenzen und Skalarmultiplikation mit natürlichen Zahlen komponentenweise definiert.
  \end{definition}
  \pause
  Sequenzen werden hier durch die Angabe ihrer positiven Einträge notiert:
  \begin{example}
    $[2 \times 3, 1 \times 6]$ und $2 \cdot [1 \times 3] + [1 \times 6]$ beschreiben die Sequenz:
    \begin{align*}
      k \mapsto \begin{cases} 2 & \text{für } k = 3\\ 1 & \text{für } k = 6\\ 0 & \text{sonst}\end{cases}
    \end{align*}
  \end{example}
\end{frame}
\begin{frame}{Allgemeine {\sc Eberhard}-Typ Probleme}
  \begin{definition}
    Ein Paar von Sequenzen $(p, v)$ heißt realisierbar, wenn es ein $3$-Polytop (eine Realisierung) gibt, dessen Anzahl an $k$-gonalen Flächen $p_k$ entspricht und dessen Anzahl an $k$-valenten Ecken $v_k$ entspricht.
  \end{definition}
  \pause
  \begin{block}{Allgemeine {\sc Eberhard}-Typ Probleme}
    Seien $p$, $v$ Sequenzen, welche in einer noch zu spezifizierenden Weise ``zulässig'' sind (sie erfüllen ein Analogon der Gleichung $\sum_{k \geq 3} (6 - k) p_k = 12$) und seien weitere Sequenzen $q$ und $w$ gegeben, welche in einer noch zu spezifizierenden Weise ``flach'' sind. Gibt es $c, d \in \nats$ und eine Realisierung deren Anzahl an $k$-gonale Flächen genau $p_k + c \cdot q_k$ entspricht und deren Anzahl an $k$-valenten Ecken genau $v_k + d \cdot w_k$ entspricht?
  \end{block}
  \pause
  Aus $c$ ist $d$ eindeutig bestimmt und umgekehrt, diese Formulierung vereinfacht die Notation an verschiedenen Stellen.
\end{frame}

\begin{frame}{Zulässige Sequenzen}
  Eine ähnliche Rechnung ({\sc Euler}-Formel, doppeltes Abzählen) zeigt, dass zwischen $p$- und $v$-Vektoren und der Anzahl der Kanten $f_1$ eines $3$-Polytopes der folgende Zusammenhang für alle $t \in \reals$ herrscht:
  \begin{beamercolorbox}[sep=-10pt,center,shadow=true,rounded=true]{block body}
    \begin{align*}
      \sum_{k\geq 3} (2 - t \cdot k ) v_k + \sum_{k\geq 3} ( 2 - (1 - t) \cdot k ) p_k &= 4\\
      \sum_{k\geq 3} k \cdot p_k = \sum_{k \geq 3} k \cdot v_k &= 2f_1
    \end{align*}
  \end{beamercolorbox}
  Paare $(p, v)$ heißen \hdef{zulässig}, wenn sie die obigen Gleichungen für bestimmte $f_1 \in \nats$ und alle $t \in \reals$ erfüllen.\\
  Zulässig zu sein ist also eine notwendige Bedingung für Realisierbarkeit.
\end{frame}

\begin{frame}{Flache Sequenzen}
  Eine analoge Rechnung für die Realisierbarkeit von $p+c\cdot q$ und $v+d\cdot w$ liefert:
  \begin{beamercolorbox}[sep=-10pt,center,shadow=true,rounded=true]{block body}
    \begin{align*}
      \sum_{k \geq 3} \left(\frac{2}{t} - k \right) w_k = \sum_{k \geq 3} \left( \frac{2}{1-t} - k \right) q_k = 0.
    \end{align*}
  \end{beamercolorbox}
  \pause
  Falls obige Gleichung für ein $t \in \left[\tfrac{1}{3}, \tfrac{2}{3}\right]$ gilt, dann heißen $q$ und $w$ \hdef{flach mit Parameter $t$}.
  \pause
  \begin{definition}
    Zwei zulässige Sequenzen $p$, $v$ heißen \hdef{$q$-$w$-realisierbar}, falls es $c, d \in \nats$ und eine Realisierung von $(p + c \cdot q, v + d \cdot w)$ gibt.
  \end{definition}
  \pause
  Flach zu sein ist eine notwendige Bedinung für $q$-$w$-Realisierbarkeit.
\end{frame}

\begin{frame}
  \begin{theorem}[{\sc Jendrol', Jucovi{\v{c}}}, 1977, \cite{jendrol1977generalization}] \label{thm:eberhard:extended:3}
    Ein Paar von zulässigen Sequenzen $(p, v)$ ist genau dann $[1 \times 6]$-$[1 \times 3]$-realisierbar, wenn:
    \begin{align*}
      \sum_{k \geq 3,\,2 \nmid k} p_k \neq 0 \text{ oder} \sum_{k \geq 3,\,3 \nmid k} v_k \neq 1.
    \end{align*}
  \end{theorem}
  \pause
  \begin{theorem}[{\sc Jendrol', Jucovi{\v{c}}, Barnette, Grünbaum, Zaks}, 1973, \cite{jucovivc1973theorem}, \cite{barnette1971toroidal}, \cite{grunbaum1969planar}, \cite{zaks1971analogue}] \label{thm:eberhard:extended:4}
    Jedes Paar von zulässigen Sequenzen $(p, v)$ ist $[1 \times 4]$-$[1 \times 4]$-realisierbar.
  \end{theorem}
  \pause
  Beide Theoreme wurden auch für Graphen auf Mannigfaltigkeiten erweitert (in diesen Fällen gibt es jedoch noch Ausnahmen für Graphen auf dem Torus).
\end{frame}

\begin{frame}[fragile]{Genaue Problemstellung}
  Wir wollen {\sc Eberhard}-Typ Probleme für Sequenzen $q$ mit zwei teilerfremden Einträgen und Sequenzen $w$ mit genau einem Eintrag eingehend betrachten:
  \begin{beamercolorbox}[sep=-10pt,center,shadow=true,rounded=true]{block body}
    \begin{align*}
      q &= [q_s \times s, q_l \times l] \qquad s < l,\,\gcd(q_s, q_l) = 1, \\
      w &= [1 \times r]
    \end{align*}
  \end{beamercolorbox}
  \pause
  $\implies$ fünf Fälle für $(s, r)$, jeder nur noch von $l$ abhängig:
  \pause
  \begin{tikzpicture}[line join=round]
    \centering
    \tdplotsetmaincoords{75}{105}
    \matrix (m) [column sep=1cm] {
      \begin{scope}[tdplot_main_coords]
        \node at (0, 0, 1.5) {$(3, 3)$};
        \draw (0.5,0.5,0.5) -- ( 0.5, -0.5, -0.5);
        \draw (0.5,0.5,0.5) -- (  -0.5,0.5, -0.5);
        \draw (0.5,0.5,0.5) -- (  -0.5, -0.5,0.5);
        \draw (0.5, -0.5, -0.5) -- (  -0.5,0.5, -0.5);
        \draw[dotted] ( -0.5,0.5, -0.5) -- (  -0.5, -0.5,0.5);
        \draw ( -0.5, -0.5,0.5) -- ( 0.5, -0.5, -0.5);
      \end{scope}
      &
      \begin{scope}[tdplot_main_coords]
        \node at (0, 0, 1.5) {$(4, 3)$};
        \draw[dotted] (-0.5,-0.5,-0.5) -- ( 0.5,-0.5,-0.5);
        \draw[dotted] (-0.5,-0.5,-0.5) -- (-0.5, 0.5,-0.5);
        \draw[dotted] (-0.5,-0.5,-0.5) -- (-0.5,-0.5, 0.5);
        \draw ( 0.5, 0.5, 0.5) -- (-0.5, 0.5, 0.5);
        \draw ( 0.5, 0.5, 0.5) -- ( 0.5,-0.5, 0.5);
        \draw ( 0.5, 0.5, 0.5) -- ( 0.5, 0.5,-0.5);
        \draw ( 0.5,-0.5,-0.5) -- ( 0.5, 0.5,-0.5);
        \draw ( 0.5, 0.5,-0.5) -- (-0.5, 0.5,-0.5);
        \draw (-0.5, 0.5,-0.5) -- (-0.5, 0.5, 0.5);
        \draw (-0.5, 0.5, 0.5) -- (-0.5,-0.5, 0.5);
        \draw (-0.5,-0.5, 0.5) -- ( 0.5,-0.5, 0.5);
        \draw ( 0.5,-0.5, 0.5) -- ( 0.5,-0.5,-0.5);
      \end{scope}
      &
      \begin{scope}[tdplot_main_coords]
        \def\phi{0.809}
        \def\phiinv{0.309}
        \node at (0, 0, 1.5) {$(5, 3)$};
        
        \draw (\phi, \phiinv, 0) -- (\phi, -\phiinv, 0);
        \draw (0.5, 0.5, -0.5) -- (\phi, \phiinv, 0) --(0.5, 0.5, 0.5);
        \draw (0.5,-0.5, -0.5) -- (\phi,-\phiinv, 0) --(0.5,-0.5, 0.5);
        \draw[dotted] (-\phi, \phiinv, 0) -- (-\phi, -\phiinv, 0);
        \draw[dotted] (-0.5, 0.5, -0.5) -- (-\phi, \phiinv, 0) -- (-0.5, 0.5, 0.5);
        \draw[dotted] (-0.5,-0.5, -0.5) -- (-\phi,-\phiinv, 0) -- (-0.5,-0.5, 0.5);
        \draw (0, \phi, \phiinv) -- (0, \phi, -\phiinv);
        \draw (-0.5, 0.5, 0.5) -- (0, \phi, \phiinv) -- (0.5, 0.5, 0.5);
        \draw[dotted] (-0.5, 0.5,-0.5) -- (0, \phi,-\phiinv);
        \draw (0, \phi,-\phiinv) -- (0.5, 0.5,-0.5);
        \draw (0, -\phi, \phiinv) -- (0, -\phi, -\phiinv);
        \draw (-0.5, -0.5, 0.5) -- (0, -\phi, \phiinv) -- (0.5, -0.5, 0.5);
        \draw[dotted] (-0.5, -0.5,-0.5) -- (0, -\phi,-\phiinv);
        \draw (0, -\phi,-\phiinv) -- (0.5, -0.5,-0.5);
        \draw (\phiinv, 0, \phi) -- (-\phiinv, 0, \phi);
        \draw ( 0.5, -0.5, 0.5) -- ( \phiinv, 0, \phi) -- ( 0.5, 0.5, 0.5);
        \draw (-0.5, -0.5, 0.5) -- (-\phiinv, 0, \phi) -- (-0.5, 0.5, 0.5);
        \draw[dotted] (\phiinv, 0, -\phi) -- (-\phiinv, 0, -\phi);
        \draw (0.5, -0.5, -0.5) -- (\phiinv, 0, -\phi) -- (0.5, 0.5, -0.5);
        \draw[dotted] (-0.5, -0.5, -0.5) -- (-\phiinv, 0, -\phi) -- (-0.5, 0.5, -0.5);
        
      \end{scope}
      &
      \begin{scope}[tdplot_main_coords]
        \node at (0, 0, 1.5) {$(3, 4)$};
        \draw[dotted] (-0.7, 0.0, 0.0) -- ( 0.0,-0.7, 0.0);
        \draw[dotted] (-0.7, 0.0, 0.0) -- ( 0.0, 0.7, 0.0);
        \draw[dotted] (-0.7, 0.0, 0.0) -- ( 0.0, 0.0,-0.7);
        \draw[dotted] (-0.7, 0.0, 0.0) -- ( 0.0, 0.0, 0.7);
        \draw ( 0.7, 0.0, 0.0) -- ( 0.0,-0.7, 0.0);
        \draw ( 0.7, 0.0, 0.0) -- ( 0.0, 0.7, 0.0);
        \draw ( 0.7, 0.0, 0.0) -- ( 0.0, 0.0,-0.7);
        \draw ( 0.7, 0.0, 0.0) -- ( 0.0, 0.0, 0.7);
        \draw ( 0.0,-0.7, 0.0) -- ( 0.0, 0.0,-0.7);
        \draw ( 0.0, 0.0,-0.7) -- ( 0.0, 0.7, 0.0);
        \draw ( 0.0, 0.7, 0.0) -- ( 0.0, 0.0, 0.7);
        \draw ( 0.0, 0.0, 0.7) -- ( 0.0,-0.7, 0.0);
      \end{scope}
      &
      \begin{scope}[tdplot_main_coords]
        \def\phi{0.809}
        \node at (0, 0, 1.5) {$(3, 5)$};
        \draw (0, -0.5, -\phi) -- (0, 0.5, -\phi);
        \draw[dotted] (0, -0.5, -\phi) -- (-0.5, -\phi, 0);
        \draw (0, -0.5, -\phi) -- ( 0.5, -\phi, 0);
        \draw (0, -0.5, -\phi) -- ( \phi, 0, -0.5);
        \draw[dotted] (0, -0.5, -\phi) -- (-\phi, 0, -0.5);
        \draw (0, 0.5, -\phi) -- (-0.5, \phi, 0);
        \draw (0, 0.5, -\phi) -- ( 0.5, \phi, 0);
        \draw (0, 0.5, -\phi) -- ( \phi, 0, -0.5);
        \draw[dotted] (0, 0.5, -\phi) -- (-\phi, 0, -0.5);
        \draw (0, -0.5, \phi) -- (0, 0.5, \phi);
        \draw[dotted] (0, -0.5, \phi) -- (-0.5, -\phi, 0);
        \draw (0, -0.5, \phi) -- ( 0.5, -\phi, 0);
        \draw (0, -0.5, \phi) -- ( \phi, 0, 0.5);
        \draw[dotted] (0, -0.5, \phi) -- (-\phi, 0, 0.5);
        \draw (0, 0.5, \phi) -- (-0.5, \phi, 0);
        \draw (0, 0.5, \phi) -- ( 0.5, \phi, 0);
        \draw (0, 0.5, \phi) -- ( \phi, 0, 0.5);
        \draw[dotted] (0, 0.5, \phi) -- (-\phi, 0, 0.5);
        \draw[dotted] (-0.5, -\phi, 0) -- (0.5, -\phi, 0);
        \draw (-0.5, \phi, 0) -- (0.5, \phi, 0);
        \draw[dotted] (-\phi, 0, -0.5) -- (-\phi, 0, 0.5);
        \draw ( \phi, 0, -0.5) -- ( \phi, 0, 0.5);
        \draw ( \phi, 0, -0.5) -- ( 0.5, -\phi, 0);
        \draw ( \phi, 0,  0.5) -- ( 0.5, -\phi, 0);
        \draw ( \phi, 0, -0.5) -- ( 0.5,  \phi, 0);
        \draw ( \phi, 0,  0.5) -- ( 0.5,  \phi, 0);
        \draw[dotted] (-\phi, 0, -0.5) -- (-0.5, -\phi, 0);
        \draw[dotted] (-\phi, 0,  0.5) -- (-0.5, -\phi, 0);
        \draw[dotted] (-\phi, 0, -0.5) -- (-0.5,  \phi, 0);
        \draw[dotted] (-\phi, 0,  0.5) -- (-0.5,  \phi, 0);
      \end{scope}
      \\
    };
  \end{tikzpicture}
\end{frame}

\begin{frame}
  \begin{itemize}
  \item Ziel: Untersuchung aller Theoremen diesen Typs
    \pause
  \item Sofern das möglich ist, für jeden möglichen Fall von flachen $q$, $w$ eine Konstruktion einer $q$-$w$-Realisierung für alle zulässigen $p$, $v$
    \pause
  \item In allen anderen Fällen: Finden von Gegenbeispielen für $p$, $v$, welche sich nicht $q$-$w$-realisieren lassen
  \end{itemize}
\end{frame}

\section{Beweisschritte der Konstruktion}

\begin{frame}{1. Beweisschritt}
  \begin{theorem}[{\sc Steinitz}]
    Es sind äquivalent:
    \begin{itemize}
    \item Ein Graph $G$ ist der Kantengraph eines $3$-Polytopes
    \item $G$ ist planar und $3$-verbunden (d.h. nach Löschen von $3$ Knoten bleibt der Graph zusammenhängend)
    \item $G$ ist planar, alle Gebiete/Flächen von $G$ sind homöomorph zu einer Kreisscheibe ($\set{x \in \reals^2 : |x| \leq 1}$) und je zwei Flächen \hdef{inzidieren gutartig}, d.h. sie sind identisch, disjunkt, besitzen genau einen gemeinsamen Knoten oder besitzen genau eine gemeinsame Kante.
    \end{itemize}
  \end{theorem}
  \pause
  \begin{example}
    In der folgenden Konfiguration hat das untere Polygon mit dem oberen Polygon zwei gemeinsame Kanten.
    
    { \centering
      \begin{tikzpicture}[scale=0.5]
        \draw (-2, 0) -- (-1, 0) -- (0, 1) -- (1, 0) -- (2, 0);
        \draw (-1, 0) -- (0, -1) -- (1, 0);
        \draw (0, 1) -- (0, -1);
        \draw[loosely dotted] (-2.5, 0) -- (-2, 0);
        \draw[loosely dotted] (2.5, 0) -- (2, 0);

        \fill[black] (-1,0) circle(2pt);
        \fill[black] (1,0) circle(2pt);
        \fill[black] (0,-1) circle(2pt);
        \fill[black] (0,1) circle(2pt);
      \end{tikzpicture}
      \par
    }
  \end{example}
  \pause

  Wir können also die Konstruktionen für planare Graphen durchführen.

\end{frame}

\begin{frame}[fragile]{2. Beweisschritt}{}
  \begin{itemize}
    \item Für den Parameter $l$ gibt es unendlich viele Möglichkeiten in jedem der Fälle.
    \item Durch geschicktes Anordnen der Polygone lassen sich aus zwei $l$-gonalen Flächen zwei $(l + 3)$-gonale konstruieren.
  \end{itemize}
  \pause
  \begin{example}
    Für den Fall $(s, r) = (3, 5)$ können zwei $l$-gonale Flächen, die über eine Kante verbunden sind zu zwei $l + s$-gonalen Flächen vergrößern, wobei wir $18$ Dreiecke einfügen:

    { \centering
      \begin{tikzpicture}
        \matrix (m) [column sep=1cm, row sep=1cm] {
          \begin{scope}[scale=0.6]
            \draw[very thick] (-1, 0) -- (1, 0);
            \draw (-1.2, 0.5) -- (-1, 0) -- (-1.2, -0.5);
            \draw (1.2, 0.5) -- (1, 0) -- (1.2, -0.5);
            \draw (-0.8, 0.5) -- (-1, 0) -- (-0.8, -0.5);
            \draw (0.8, 0.5) -- (1, 0) -- (0.8, -0.5);
            \node (k1) at (-2, 0) {$l$};
            \node (k2) at (2, 0) {$l$};
            
            \fill[black] (-1,0) circle(2pt);
            \fill[black] (1,0) circle(2pt);
          \end{scope}
          &
          \begin{scope}[scale=0.6]
            \draw[very thick] (-1, 1.5) -- (1, 1.5);
            \draw (-1, -1.5) -- (1, -1.5);
            \draw (-1.2, 2) -- (-1, 1.5);
            \draw (-1.2, -2) -- (-1, -1.5);
            \draw (1.2, 2) -- (1, 1.5);
            \draw (1.2, -2) -- (1, -1.5);
            \draw (-0.8, 2) -- (-1, 1.5);
            \draw (-0.8, -2) -- (-1, -1.5);
            \draw (0.8, 2) -- (1, 1.5);
            \draw (0.8, -2) -- (1, -1.5);
            \draw (1, 1.5) -- (0.8, 0.5) -- (0.8, -0.5) -- (1, -1.5);
            \draw (-1, 1.5) -- (-0.8, 0.5) -- (-0.8, -0.5) -- (-1, -1.5);
            \draw (-1, -1.5) -- (0, -1) -- (1, -1.5);
            \draw (-1, 1.5) -- (0, 1) -- (1, 1.5);
            \draw (-0.8, -0.5) -- (0, -1) -- (0.8, -0.5);
            \draw (-0.8, 0.5) -- (0, 1) -- (0.8, 0.5);
            \draw (-0.8, -0.5) -- (0, -0.5) -- (0.8, -0.5);
            \draw (-0.8, 0.5) -- (0, 0.5) -- (0.8, 0.5);
            \draw (-0.8, -0.5) -- (-0.4, 0) -- (0, -0.5) -- (0.4, 0) -- (0.8, -0.5);
            \draw (-0.8, 0.5) -- (-0.4, 0) -- (0, 0.5) -- (0.4, 0) -- (0.8, 0.5);
            \draw (-0.4, 0) -- (0.4, 0);
            \draw (0, -1) -- (0, -0.5);
            \draw (0, 1) -- (0, 0.5);
            
            
            \node (k1) at (-1.8, 0) {$l + 3$};
            \node (k2) at (1.8, 0) {$l + 3$};
            
            \fill[black] (0,-1) circle (2pt);
            \fill[black] (0,1) circle (2pt);
            \fill[black] (0,-0.5) circle (2pt);
            \fill[black] (0,0.5) circle (2pt);
            \fill[black] (-0.8,0.5) circle (2pt);
            \fill[black] (-0.8,-0.5) circle (2pt);          
            \fill[black] (-0.4,0) circle (2pt);
            \fill[black] (0.4,0) circle (2pt);
            \fill[black] (0.8,0.5) circle (2pt);
            \fill[black] (0.8,-0.5) circle (2pt);          
            \fill[black] (-1,1.5) circle (2pt);
            \fill[black] (1,-1.5) circle (2pt);
            \fill[black] (1,1.5) circle (2pt);
            \fill[black] (-1,-1.5) circle (2pt);
          \end{scope}
          \\
        };
      \end{tikzpicture}
      \par }
    Diese Konstruktion lässt sich beliebig oft durchführen.
  \end{example}
  \pause
  Solche Konstruktionen gibt sie für vier der fünf Fälle. Damit ``reduziert'' sich in diesen Fällen der Gesamtaufwand auf $15$ Induktionsanfänge.
\end{frame}

\begin{frame}{2. Beweisschritt}
  \begin{tabularx}{\textwidth}{|c|l|X|}
    \hline
    $(s, r)$ & \multicolumn{1}{c|}{$l$} &\\
    \hline
    $(3, 3)$ & \multicolumn{1}{c|}{?} & \\
    $(4, 3)$ & $l = 7$ &\\
    $(4, 3)$ & $l = 8$ &\\
    $(4, 3)$ & $l = 9$ &\\
    $(4, 3)$ & $l = 10$ &\\
    $(5, 3)$ & $l = 7$ &\\
    $(5, 3)$ & $l = 8$ &\\
    $(5, 3)$ & $l = 9$ &\\
    $(5, 3)$ & $l = 10$ &\\
    $(5, 3)$ & $l = 11$ &\\
    $(3, 4)$ & $l = 5$ &\\
    $(3, 4)$ & $l = 6$ &\\
    $(3, 4)$ & $l = 7$ &\\
    $(3, 5)$ & $l = 4$ &\\
    $(3, 5)$ & $l = 5$ &\\
    $(3, 5)$ & $l = 6$ &\\
    \hline
  \end{tabularx}
\end{frame}

\begin{frame}[fragile]{3. Beweisschritt}
  \begin{itemize}
  \item explizite Konstruktionen von $3$-Polytopen sind sehr aufwendig
    \pause
  \item Ausnutzen der klassischen {\sc Eberhard}-Theoreme führt zu einer ersten Realisierung
    \pause
  \item diese besitzt jedoch zu viele Sechsecke oder Vierecke enthält.
    \pause
  \item Idee: um jede $k$-gonale Fläche einen Ring aus $k$ Kopien eines Patches $\mathcal{M}$ bilden, welcher nur aus $s$- und $l$-gonalen Flächen und $r$-valenten Ecken besteht.
  \end{itemize}
  \medskip
  { \centering\scriptsize
    \begin{tikzpicture}
      \matrix (m) [column sep=1cm] {
        \begin{scope}[scale=0.25]
          \draw (-3,0)--(-3,3)--(-6,5)--(-6,8)--(-3,10)--(0,8)--(6,8)--(6,5)--(7,1)--(3,0)--(-3,0);
          \draw (-3,3)--(0,5)--(0,8);
          \draw (3,0)--(3,3)--(0,5);
          \draw (3,3)--(7,1);
          \draw (3,3)--(6,5);

          % loose ends
          \draw (-3,0)-- (-3.5,-0.666);
          \draw (-3,3)-- (-3.5,2.5);
          \draw (-6,5)--  (-6.5,4.666);
          \draw (-6,8)-- (-6.5,8.333);
          \draw (-3,10)-- (-3,10.5);
          \draw (0,8)-- (0.5,8.333);
          \draw (6,8)-- (6.5,8.5);
          \draw (6,5)-- (6.5,5);
          \draw (7,1)-- (7.5,0.5);
          \draw (3,0)-- (3.5,-0.666);

          \fill[black] (-3,0) circle(3pt);
          \fill[black] (-3,3)circle(3pt);
          \fill[black] (-6,5)circle(3pt);
          \fill[black] (-6,8)circle(3pt);
          \fill[black] (-3,10)circle(3pt);
          \fill[black] (0,8)circle(3pt);
          \fill[black] (6,8)circle(3pt);
          \fill[black] (6,5)circle(3pt);
          \fill[black] (7,1) circle(3pt);
          \fill[black] (3,0)circle(3pt);
          \fill[black] (0,5)circle(3pt);
          \fill[black] (3,3)circle(3pt);

          \node at (-3,6.5) {$f_1$};
          \node at (0,2.5) {$f_2$};
          \node at (3,5.5) {$f_3$};
          \node at (4.5,1.5) {$f_4$};
          \node at (5,3) {$f_5$};
          
        \end{scope}
        &
        \tiny
        \begin{scope}[scale=0.25]
          \draw (-3,0)--(-3,3)--(-6,5)--(-6,8)--(-3,10)--(0,8)--(6,8)--(6,5)--(7,1)--(3,0)--(-3,0);
          \draw (-3,3)--(0,5)--(0,8);
          \draw (3,0)--(3,3)--(0,5);
          \draw (3,3)--(7,1);
          \draw (3,3)--(6,5);


          \coordinate[shift={(0,0.5)}] (penta1) at ( 50:0.5) ;
          \coordinate[shift={(0,0.5)}] (penta2) at (122:0.5) ;
          \coordinate[shift={(0,0.5)}] (penta3) at (194:0.5) ;
          \coordinate[shift={(0,0.5)}] (penta4) at (266:0.5) ;
          \coordinate[shift={(0,0.5)}] (penta5) at (338:0.5) ;

          \coordinate[shift={(0.75,1.5)}] (penta9) at  ( 20:0.5) ;
          \coordinate[shift={(0.75,1.5)}] (penta10) at ( 92:0.5) ;
          \coordinate[shift={(0.75,1.5)}] (penta11) at (164:0.5) ;
          \coordinate[shift={(0.75,1.5)}] (penta12) at (236:0.5) ;
          \coordinate[shift={(0.75,1.5)}] (penta13) at (308:0.5) ;

          \only<1-5>{
            \coordinate[shift={(-0.75,1.625)}] (hexa1) at (  0:0.5) ;
            \coordinate[shift={(-0.75,1.625)}] (hexa2) at ( 60:0.5) ;
            \coordinate[shift={(-0.75,1.625)}] (hexa3) at (120:0.5) ;
            \coordinate[shift={(-0.75,1.625)}] (hexa4) at (180:0.5) ;
            \coordinate[shift={(-0.75,1.625)}] (hexa5) at (240:0.5) ;
            \coordinate[shift={(-0.75,1.625)}] (hexa6) at (300:0.5) ;
            \draw (hexa1)--(hexa2)--(hexa3)--(hexa4)--(hexa5)--(hexa6)--(hexa1);
            \draw ($(0,5)!0.6!(0,8)$) --(hexa1);
            \draw ($(0,8)!0.6!(-3,10)$) --(hexa2);
            \draw ($(-3,10)!0.6!(-6,8)$) --(hexa3);
            \draw ($(-6,8)!0.6!(-6,5)$) --(hexa4);
            \draw ($(-6,5)!0.6!(-3,3)$) --(hexa5);
            \draw ($(-3,3)!0.6!(0,5)$) --(hexa6);
            \node[shift={(-0.75,1.625)}] at ( 30:1.8) {$\mathcal{M}$};
            \node[shift={(-0.75,1.625)}] at ( 90:1.8) {$\mathcal{M}$};
            \node[shift={(-0.75,1.625)}] at (150:1.8) {$\mathcal{M}$};
            \node[shift={(-0.75,1.625)}] at (210:1.8) {$\mathcal{M}$};
            \node[shift={(-0.75,1.625)}] at (270:1.8) {$\mathcal{M}$};
            \node[shift={(-0.75,1.625)}] at (330:1.8) {$\mathcal{M}$};
          }

          \only<6>{
            \node at (-3, 6.5) {$\mathcal{P}$};
          }
          
          \coordinate[shift={(1.1,0.325)}] (tri1) at ( 75:0.4) ;
          \coordinate[shift={(1.1,0.325)}] (tri2) at (195:0.4) ;
          \coordinate[shift={(1.1,0.325)}] (tri3) at (315:0.4) ;
          
          \coordinate[shift={(1.35,0.75)}] (tri4) at ( 30:0.4) ;
          \coordinate[shift={(1.35,0.75)}] (tri5) at (150:0.4) ;
          \coordinate[shift={(1.35,0.75)}] (tri6) at (270:0.4) ;
          
          
          \draw (penta5)--(penta1)--(penta2)--(penta3)--(penta4)--(penta5);
          \draw (penta9)--(penta10)--(penta11)--(penta12)--(penta13)--(penta9);


          \draw (tri1)--(tri2)--(tri3)--(tri1);
          \draw (tri4)--(tri5)--(tri6)--(tri4);
          
          % loose ends
          \draw (-3,0)-- (-3.5,-0.666);
          \draw (-3,3)-- (-3.5,2.5);
          \draw (-6,5)--  (-6.5,4.666);
          \draw (-6,8)-- (-6.5,8.333);
          \draw (-3,10)-- (-3,10.5);
          \draw (0,8)-- (0.5,8.333);
          \draw (6,8)-- (6.5,8.5);
          \draw (6,5)-- (6.5,5);
          \draw (7,1)-- (7.5,0.5);
          \draw (3,0)-- (3.5,-0.666);

          \fill[black] (-3,0) circle(3pt);
          \fill[black] (-3,3)circle(3pt);
          \fill[black] (-6,5)circle(3pt);
          \fill[black] (-6,8)circle(3pt);
          \fill[black] (-3,10)circle(3pt);
          \fill[black] (0,8)circle(3pt);
          \fill[black] (6,8)circle(3pt);
          \fill[black] (6,5)circle(3pt);
          \fill[black] (7,1) circle(3pt);
          \fill[black] (3,0)circle(3pt);
          \fill[black] (0,5)circle(3pt);
          \fill[black] (3,3)circle(3pt);

          \draw ($(3,0)!0.4!(3,3)$) --(tri2);
          \draw ($(7,1)!0.6!(3,3)$) --(tri1);
          \draw ($(7,1)!0.4!(3,0)$) --(tri3);


          \draw ($(6,5)!0.6!(3,3)$) --(tri5);
          \draw ($(7,1)!0.4!(3,3)$) --(tri6);
          \draw ($(6,5)!0.4!(7,1)$) --(tri4);

          \draw ($(3,3)!0.6!(0,5)$) --(penta1);
          \draw ($(0,5)!0.6!(-3,3)$) --(penta2);
          \draw ($(-3,3)!0.6!(-3,0)$) --(penta3);
          \draw ($(-3,0)!0.6!(3,0)$) --(penta4);
          \draw ($(3,0)!0.6!(3,3)$) --(penta5);

          \draw ($(6,5)!0.6!(6,8)$) --(penta9);
          \draw ($(6,8)!0.6!(0,8)$) --(penta10);
          \draw ($(0,8)!0.6!(0,5)$) --(penta11);
          \draw ($(0,5)!0.6!(3,3)$) --(penta12);
          \draw ($(3,3)!0.6!(6,5)$) --(penta13);


          \node at (-1.5,1) {$\mathcal{M}$};
          \node at (-1.7,2.4) {$\mathcal{M}$};
          \node at (1.5,0.8) {$\mathcal{M}$};
          \node at (1.6,2.6) {$\mathcal{M}$};
          \node at (-0.2,3.8) {$\mathcal{M}$};

          \node at (3.8,0.7) {$\mathcal{M}$};
          \node at (3.8,2) {$\mathcal{M}$};
          \node at (5.45,1.2) {$\mathcal{M}$};

          \node at (6,2.5) {$\mathcal{M}$};
          \node at (4.4,3) {$\mathcal{M}$};
          \node at (5.6,3.8) {$\mathcal{M}$};

          \node at (3,4.5) {$\mathcal{M}$};
          \node at (4.8,5.8) {$\mathcal{M}$};
          \node at (4.3,7.3) {$\mathcal{M}$};
          \node at (1.4,7.3) {$\mathcal{M}$};
          \node at (1.4,5.5) {$\mathcal{M}$};

          % \node at (-2.8,4.5) {$\mathcal{M}$};
          % \node at (-1.5,5.8) {$\mathcal{M}$};
          % \node at (-1.6,7.3) {$\mathcal{M}$};
          % \node at (-3.2,8.2) {$\mathcal{M}$};
          % \node at (-4.8,7.2) {$\mathcal{M}$};
          % \node at (-4.8,5.3) {$\mathcal{M}$};



          
          \foreach \x in {0.2,0.4,0.6,0.8}
          \fill[black] ($(3,0)!\x!(-3,0)$) circle (3pt);    
          \foreach \x in {0.2,0.4,0.6,0.8}
          \fill[black] ($(-3,0)!\x!(-3,3)$) circle (3pt); 
          \foreach \x in {0.2,0.4,0.6,0.8}
          \fill[black] ($(-3,3)!\x!(-6,5)$) circle (3pt);
          \foreach \x in {0.2,0.4,0.6,0.8}
          \fill[black] ($(-6,5)!\x!(-6,8)$) circle (3pt);
          \foreach \x in {0.2,0.4,0.6,0.8}
          \fill[black] ($(-6,8)!\x!(-3,10)$) circle (3pt);
          \foreach \x in {0.2,0.4,0.6,0.8}
          \fill[black] ($(-3,10)!\x!(0,8)$) circle (3pt);
          \foreach \x in {0.2,0.4,0.6,0.8}
          \fill[black] ($(0,8)!\x!(6,8)$) circle (3pt);
          \foreach \x in {0.2,0.4,0.6,0.8}
          \fill[black] ($(6,8)!\x!(6,5)$) circle (3pt);
          \foreach \x in {0.2,0.4,0.6,0.8}
          \fill[black] ($(6,5)!\x!(7,1)$) circle (3pt);     
          \foreach \x in {0.2,0.4,0.6,0.8}
          \fill[black] ($(7,1)!\x!(3,0)$) circle (3pt); 
          \foreach \x in {0.2,0.4,0.6,0.8}
          \fill[black] ($(3,0)!\x!(3,3)$) circle (3pt);
          \foreach \x in {0.2,0.4,0.6,0.8}
          \fill[black] ($(3,3)!\x!(7,1)$) circle (3pt);
          \foreach \x in {0.2,0.4,0.6,0.8}
          \fill[black] ($(3,3)!\x!(6,5)$) circle (3pt);
          \foreach \x in {0.2,0.4,0.6,0.8}
          \fill[black] ($(3,3)!\x!(0,5)$) circle (3pt);
          \foreach \x in {0.2,0.4,0.6,0.8}
          \fill[black] ($(0,5)!\x!(0,8)$) circle (3pt);
          \foreach \x in {0.2,0.4,0.6,0.8}
          \fill[black] ($(0,5)!\x!(-3,3)$) circle (3pt);      
        \end{scope}
        \\
      };
      
    \end{tikzpicture}
    \par }

  \pause
  Im nächsten Schritt ersetzen wir alle überschüssigen Sechs- oder Vierecke durch einen eigenen Patch $\mathcal{P}$, welcher ebenfalls nur aus $s$- und $l$-gonalen Flächen und $r$-valenten Ecken besteht, und erhalten eine Karte mit der richtigen Anzahl an Sechs- oder Vierecken.
\end{frame}

\begin{frame}{4. Beweisschritt}
  Die letzte Eigenschaft, die wir für eine $q$-$w$-Realisierung sicherstellen müssen, ist, dass der konstruierte Graph als $3$-Polytop realisierbar ist. 
  \pause
  \begin{lemma}
    Sei $\mathcal{M}$ der Patch aus der vorangegangen Konstruktion, so dass zusätzlich alle Flächen in den vier Kopien entlang jeder Kante gutartig zusammenstoßen. Dann ist der konstruierte Graph als $3$-Polytop realisierbar.
  \end{lemma}

  { \centering
    \begin{tikzpicture}[scale=0.25]

      \begin{scope}[scale=0.8]

        \draw[shift={(-5,0)}] (9 : 1) -- (81 : 1) -- (153 : 1)  (225 : 1) -- (297 : 1) -- (9 : 1);
        \draw[shift={(-5,0)}][dotted] (153 : 1) -- (225 : 1);


        \draw[shift={(-5,0)}] (9 : 1) -- (9 : 2.5);
        \draw[shift={(-5,0)}] (9 : 2.5) -- (9 : 3.5);
        \draw[shift={(-5,0)}] (9 : 3.5) -- (9 : 5.062);
        \draw[shift={(-5,0)}] (81 : 1) -- (81 : 2.5);
        \draw[shift={(-5,0)}] (81 : 2.5) -- (81 : 3.5);
        \draw[shift={(-5,0)}] (81 : 3.5) -- (81 : 5.062);
        \draw[shift={(-5,0)}] (153 : 1) -- (153 : 2.5);
        \draw[shift={(-5,0)}] (153 : 2.5) -- (153 : 3.5);
        \draw[shift={(-5,0)}] (153 : 3.5) -- (153 : 5.062);
        \draw[shift={(-5,0)}] (225 : 1) -- (225 : 2.5);
        \draw[shift={(-5,0)}] (225 : 2.5) -- (225 : 3.5);
        \draw[shift={(-5,0)}] (225 : 3.5) -- (225 : 5.062);
        \draw[shift={(-5,0)}] (297 : 1) -- (297 : 2.5);
        \draw[shift={(-5,0)}] (297 : 2.5) -- (297 : 3.5);
        \draw[shift={(-5,0)}] (297 : 3.5) -- (297 : 5.062);
        \draw[shift={(-5,0)}] (4 : 5.012) -- (356 : 5.012);


        \draw[shift={(-5,0)}] (-13 : 5.1315) -- (356 : 5.012) (4 : 5.012) -- (13 : 5.1315);
        \draw[shift={(-5,0)}] (13 : 5.1315) -- (27 : 5.612);
        \draw[shift={(-5,0)}] (27 : 5.612) -- (36 : 6.180) -- (45 : 5.612);
        \draw[shift={(-5,0)}] (45 : 5.612) -- (68 : 5.012);
        \draw[shift={(-5,0)}] (68 : 5.012) -- (85 : 5.1315);
        \draw[shift={(-5,0)}] (85 : 5.1315) -- (99 : 5.612);
        \draw[shift={(-5,0)}] (99 : 5.612) -- (108 : 6.180) -- (117 : 5.612);
        \draw[shift={(-5,0)}] (117 : 5.612) -- (131 : 5.1315);
        \draw[shift={(-5,0)}] (131 : 5.1315) -- (153 : 5.062);
        \draw[shift={(-5,0)}] (-131 : 5.1315) -- (225 : 5.062);
        \draw[shift={(-5,0)}] (-117 : 5.612) -- (-131 : 5.1315);
        \draw[shift={(-5,0)}] (-99 : 5.612) -- (-108 : 6.180) -- (-117 : 5.612);
        \draw[shift={(-5,0)}] (-99 : 5.612) -- (-85 : 5.1315);
        \draw[shift={(-5,0)}] (-59 : 5.1315) -- (-85 : 5.1315);
        \draw[shift={(-5,0)}] (-45 : 5.612) -- (-85 : 5.1315);
        \draw[shift={(-5,0)}] (-27 : 5.612) -- (-36 : 6.180) -- (-45 : 5.612);
        \draw[shift={(-5,0)}] (-27 : 5.612) -- (-13 : 5.1315);

        \node[shift={(-1,0)}] at (-27 : 3.5) {\hdef{$\mathcal{M}$}};

        \node[shift={(-1,0)}] at (45 : 3.5) {\hdef{$\mathcal{M}$}};

        \node[shift={(-1,0)}] at (117 : 3.5) {$\mathcal{M}$};
        \node[shift={(-1,0)}] at (-99 : 3.5) {$\mathcal{M}$};

        \draw[shift={(-5,0)}][dotted] (180 : 4) -- (198 : 4);

        \draw[shift={(5,0)}, rotate around={180:(0,0)}] (9 : 1) -- (81 : 1) -- (153 : 1)  (225 : 1) -- (297 : 1) -- (9 : 1);
        \draw[shift={(5,0)}, rotate around={180:(0,0)}][dotted] (153 : 1) -- (225 : 1);


        \draw[shift={(5,0)}, rotate around={180:(0,0)}] (9 : 1) -- (9 : 2.5);
        \draw[shift={(5,0)}, rotate around={180:(0,0)}] (9 : 2.5) -- (9 : 3.5);
        \draw[shift={(5,0)}, rotate around={180:(0,0)}] (9 : 3.5) -- (9 : 5.062);
        \draw[shift={(5,0)}, rotate around={180:(0,0)}] (81 : 1) -- (81 : 2.5);
        \draw[shift={(5,0)}, rotate around={180:(0,0)}] (81 : 2.5) -- (81 : 3.5);
        \draw[shift={(5,0)}, rotate around={180:(0,0)}] (81 : 3.5) -- (81 : 5.062);
        \draw[shift={(5,0)}, rotate around={180:(0,0)}] (153 : 1) -- (153 : 2.5);
        \draw[shift={(5,0)}, rotate around={180:(0,0)}] (153 : 2.5) -- (153 : 3.5);
        \draw[shift={(5,0)}, rotate around={180:(0,0)}] (153 : 3.5) -- (153 : 5.062);
        \draw[shift={(5,0)}, rotate around={180:(0,0)}] (225 : 1) -- (225 : 2.5);
        \draw[shift={(5,0)}, rotate around={180:(0,0)}] (225 : 2.5) -- (225 : 3.5);
        \draw[shift={(5,0)}, rotate around={180:(0,0)}] (225 : 3.5) -- (225 : 5.062);
        \draw[shift={(5,0)}, rotate around={180:(0,0)}] (297 : 1) -- (297 : 2.5);
        \draw[shift={(5,0)}, rotate around={180:(0,0)}] (297 : 2.5) -- (297 : 3.5);
        \draw[shift={(5,0)}, rotate around={180:(0,0)}] (297 : 3.5) -- (297 : 5.062);
        
        % \draw[shift={(5,0)}, rotate around={180:(0,0)}] (-13 : 5.1315) -- (13 : 5.1315);
        % \draw[shift={(5,0)}, rotate around={180:(0,0)}][dotted] (13 : 5.1315) -- (27 : 5.612);
        \draw[shift={(5,0)}, rotate around={180:(0,0)}] (36 : 6.180) -- (45 : 5.612);
        \draw[shift={(5,0)}, rotate around={180:(0,0)}] (45 : 5.612) -- (59 : 5.1315);
        \draw[shift={(5,0)}, rotate around={180:(0,0)}] (59 : 5.1315) -- (85 : 5.1315);
        \draw[shift={(5,0)}, rotate around={180:(0,0)}] (85 : 5.1315) -- (99 : 5.612);
        \draw[shift={(5,0)}, rotate around={180:(0,0)}] (99 : 5.612) -- (108 : 6.180) -- (117 : 5.612);
        \draw[shift={(5,0)}, rotate around={180:(0,0)}] (117 : 5.612) -- (131 : 5.1315);
        \draw[shift={(5,0)}, rotate around={180:(0,0)}] (131 : 5.1315) -- (153 : 5.062);
        \draw[shift={(5,0)}, rotate around={180:(0,0)}] (-131 : 5.1315) -- (225 : 5.062);
        \draw[shift={(5,0)}, rotate around={180:(0,0)}] (-117 : 5.612) -- (-131 : 5.1315);
        \draw[shift={(5,0)}, rotate around={180:(0,0)}] (-99 : 5.612) -- (-108 : 6.180) -- (-117 : 5.612);
        \draw[shift={(5,0)}, rotate around={180:(0,0)}] (-99 : 5.612) -- (-85 : 5.1315);
        \draw[shift={(5,0)}, rotate around={180:(0,0)}] (-59 : 5.1315) -- (-85 : 5.1315);
        \draw[shift={(5,0)}, rotate around={180:(0,0)}] (-45 : 5.612) -- (-85 : 5.1315);
        \draw[shift={(5,0)}, rotate around={180:(0,0)}] (-36 : 6.180) -- (-45 : 5.612);
        % \draw[shift={(5,0)}, rotate around={180:(0,0)}][dotted] (-27 : 5.612) -- (-13 : 5.1315);

        \node[shift={(1,0)}] at (153 : 3.5) {\hdef{$\mathcal{M}$}};
        \node[shift={(1,0)}] at (225 : 3.5) {\hdef{$\mathcal{M}$}};

        \node[shift={(1,0)}] at (297 : 3.5) {$\mathcal{M}$};
        \node[shift={(1,0)}] at (81 : 3.5) {$\mathcal{M}$};

        \draw[shift={(5,0)}, rotate around={180:(0,0)}][dotted] (180 : 4) -- (198 : 4);

      \end{scope}

    \end{tikzpicture}
    \par
  }
\end{frame}
\begin{frame}
  Diese Schritte lassen sich in $11$ Fällen durchführen.\\
  \medskip
  \begin{tabularx}{\textwidth}{|c|m{2cm}|X|}
    \hline
    $(s, r)$ & \multicolumn{1}{c|}{$l$} &\\
    \hline
    $(3, 3)$ & \multicolumn{1}{c|}{?} & \\
    $(4, 3)$ & $l = 7, 9$ & \only<2,3>{Für alle $p, v$ möglich}\\
    $(4, 3)$ & $l = 8$ &\\
    $(4, 3)$ & $l = 10$ &\only<3>{Für fast alle $p, v$ möglich}\\
    $(5, 3)$ & $l = 7, 8, 9, 11$ & \only<2,3>{Für alle $p, v$ möglich}\\
    $(5, 3)$ & $l = 10$ &\\
    $(3, 4)$ & $l = 5, 7$ & \only<2,3>{Für alle $p, v$ möglich}\\
    $(3, 4)$ & $l = 6$ &\\
    $(3, 5)$ & $l = 4, 5$ & \only<2,3>{Für alle $p, v$ möglich}\\
    $(3, 5)$ & $l = 5$ &\\
    \hline
  \end{tabularx}
\end{frame}

\begin{frame}
  Diese Schritte lassen sich in $11$ Fällen durchführen.\\
  \medskip
  \begin{tabularx}{\textwidth}{|c|m{2cm}|X|}
    \hline
    $(s, r)$ & \multicolumn{1}{c|}{$l$} &\\
    \hline
    $(3, 3)$ & \multicolumn{1}{c|}{?} & \\
    $(4, 3)$ & $l \equiv 1~(\mod 2)$ & Für alle $p, v$ möglich\\
    $(4, 3)$ & $l \equiv 0~(\mod 4)$ &\\
    $(4, 3)$ & $l \equiv 2~(\mod 4)$ & Für fast alle $p, v$ möglich\\
    $(5, 3)$ & $l \not\equiv 0~(\mod 5)$ & Für alle $p, v$ möglich\\
    $(5, 3)$ & $l \equiv 0~(\mod 5)$ &\\
    $(3, 4)$ & $l \not\equiv 0~(\mod 3)$ & Für alle $p, v$ möglich\\
    $(3, 4)$ & $l \equiv 0~(\mod 3)$ &\\
    $(3, 5)$ & $l \not\equiv 0~(\mod 3)$ & Für alle $p, v$ möglich\\
    $(3, 5)$ & $l \equiv 0~(\mod 3)$ &\\
    \hline
  \end{tabularx}
\end{frame}


\section{Negative Resultate}

\begin{frame}{Negative Resultate}
  Viele der Unterfälle von $(s, r) = (3, 3)$ lassen sich ausschließen:
  \begin{theorem}
    Seien $p$ und $v$ zulässige Sequenzen und $q = [q_3 \times 3, q_l \times l]$, $w = [1 \times 3]$, $l \geq 11$ und $\sum_{k \geq 4} 2k  v_k + \sum_{k \geq 4} \lfloor \tfrac{k}{2} \rfloor p_k < 3p_3$. Dann ist $(p, v)$ nicht $q$-$w$-realisierbar, außer für $p = [4 \times 3]$, $v = [4 \times 3]$.
    \medskip\\
    \pause
    { \normalfont
      Beweisidee: Nur im Fall des Tetraeders können zwei Dreiecke eine gemeinsame Kante mit $3$-valenten Endpunkten besitzen.

      { \centering
        \begin{tikzpicture}[scale=0.5]
          \draw (-2, 0) -- (-1, 0) -- (0, 1) -- (1, 0) -- (2, 0);
          \draw (-1, 0) -- (0, -1) -- (1, 0);
          \draw (0, 1) -- (0, -1);
          \draw[loosely dotted] (-2.5, 0) -- (-2, 0);
          \draw[loosely dotted] (2.5, 0) -- (2, 0);

          \fill[black] (-1,0) circle(2pt);
          \fill[black] (1,0) circle(2pt);
          \fill[black] (0,-1) circle(2pt);
          \fill[black] (0,1) circle(2pt);
        \end{tikzpicture}
        \par
      }
      Eine Abschätzung der Anzahl an Kanten die an Dreiecken anliegen zeigt dann das Resultat.
    }
  \end{theorem}
\end{frame}

\begin{frame}
  \begin{tabularx}{\textwidth}{|c|c|X|}
    \hline
    $(s, r)$ & $l$ &\\
    \hline
    $(3, 3)$ & $l = 7$               & \only<3-6>{Für alle $p, v$ möglich}\\
    $(3, 3)$ & $l = 8$               & \only<3-6>{Für alle $p, v$ möglich}\\
    $(3, 3)$ & $l = 9$               & \only<5-6>{Es gibt unrealisierbare Beispiele}\\
    $(3, 3)$ & $l = 10$              & \only<3-6>{Für alle $p, v$ möglich}\\
    $(3, 3)$ & $l \geq 11$           & \only<2-6>{Es gibt unrealisierbare Beispiele}\\
    $(4, 3)$ & $l \equiv 0~(\mod 4)$ & \only<5-6>{Es gibt unrealisierbare Beispiele}\\
    $(4, 3)$ & $l \equiv 1~(\mod 2)$ & Für alle $p, v$ möglich\\
    $(4, 3)$ & $l \equiv 2~(\mod 4)$ & \only<1-5>{Für fast alle $p, v$ möglich} \only<6>{Alle unrealisierbaren Beispiele bekannt}\\
    $(5, 3)$ & $l \equiv 0~(\mod 5)$ & \only<5-6>{Es gibt unrealisierbare Beispiele}\\
    $(5, 3)$ & $l \not\equiv 0~(\mod 5)$ & Für alle $p, v$ möglich\\
    $(3, 4)$ & $l \equiv 0~(\mod 3)$ & \only<5-6>{Es gibt unrealisierbare Beispiele}\\
    $(3, 4)$ & $l \not\equiv 0~(\mod 3)$ & Für alle $p, v$ möglich\\
    $(3, 5)$ & $l \equiv 0~(\mod 3)$ & \only<5-6>{Es gibt unrealisierbare Beispiele}\\
    $(3, 5)$ & $l \not\equiv 0~(\mod 3)$ & Für alle $p, v$ möglich\\
    \hline
  \end{tabularx}
  \uncover<4-6>{Durch graphentheoretische Überlegungen lassen sich für alle Fälle mit $\gcd(s, l) \neq 1$ Beispiele für $(p, v)$ finden, die nicht $q$-$w$-realisierbar sind.}
\end{frame}

\begin{frame}{Offene Fragen}
  \begin{itemize}
    \item Welche $(p, v)$ sind nicht $q$-$w$-realisierbar für $s | l$?\\
      Vermutung: Wir haben bereits alle Gegenbeispiele gefunden (wie im Fall $s = 4$, $r = 3$ und $l \equiv 2~(\mod 4)$)
      \pause
    \item Welche $(p, v)$ lassen sich $[q_3 \times 3, q_l \times l]$-$[1 \times 3]$-realisieren? Diese Frage ist sehr schwierig für $l \geq 13$, wie sieht die Antwort für $l = 11$ oder $l = 12$ aus?
      \pause
    \item Wie sieht die Menge der $(c, d)$ aus, für die $(p + c \cdot q, v + d \cdot w)$ realisierbar sind?
  \end{itemize}
\end{frame}

\begin{frame}
  \centering
  Vielen Dank für Ihre Aufmerksamkeit.
\end{frame}
\begin{frame}
  \footnotesize{
    \bibliographystyle{alpha}
    \bibliography{bibliography}{}
  }
\end{frame}

\begin{frame}
  \begin{tikzpicture}[remember picture, overlay]
  \foreach \x in {{(-67.500000, -70.148056)}, {(-54.000000, -77.942284)}, {(-40.500000, -70.148056)}, {(-27.000000, -77.942284)}, {(-13.500000, -70.148056)}, {(0.000000, -77.942284)}, {(13.500000, -70.148056)}, {(27.000000, -77.942284)}, {(40.500000, -70.148056)}, {(54.000000, -77.942284)}, {(67.500000, -70.148056)}, {(-67.500000, -54.559601)}, {(-54.000000, -62.353828)}, {(-40.500000, -54.559601)}, {(-27.000000, -62.353828)}, {(-13.500000, -54.559601)}, {(0.000000, -62.353828)}, {(13.500000, -54.559601)}, {(27.000000, -62.353828)}, {(40.500000, -54.559601)}, {(54.000000, -62.353828)}, {(67.500000, -54.559601)}, {(-67.500000, -38.971142)}, {(-54.000000, -46.765373)}, {(-40.500000, -38.971142)}, {(-27.000000, -46.765373)}, {(-13.500000, -38.971142)}, {(0.000000, -46.765373)}, {(13.500000, -38.971142)}, {(27.000000, -46.765373)}, {(40.500000, -38.971142)}, {(54.000000, -46.765373)}, {(67.500000, -38.971142)}, {(-67.500000, -23.382687)}, {(-54.000000, -31.176914)}, {(-40.500000, -23.382687)}, {(-27.000000, -31.176914)}, {(-13.500000, -23.382687)}, {(0.000000, -31.176914)}, {(13.500000, -23.382687)}, {(27.000000, -31.176914)}, {(40.500000, -23.382687)}, {(54.000000, -31.176914)}, {(67.500000, -23.382687)}, {(-67.500000, -7.794229)}, {(-54.000000, -15.588457)}, {(-40.500000, -7.794229)}, {(-27.000000, -15.588457)}, {(-13.500000, -7.794229)}, {(0.000000, -15.588457)}, {(13.500000, -7.794229)}, {(27.000000, -15.588457)}, {(40.500000, -7.794229)}, {(54.000000, -15.588457)}, {(67.500000, -7.794229)}, {(-67.500000, 7.794229)}, {(-54.000000, 0.000000)}, {(-40.500000, 7.794229)}, {(-27.000000, 0.000000)}, {(-13.500000, 7.794229)}, {(0.000000, 0.000000)}, {(13.500000, 7.794229)}, {(27.000000, 0.000000)}, {(40.500000, 7.794229)}, {(54.000000, 0.000000)}, {(67.500000, 7.794229)}, {(-67.500000, 23.382687)}, {(-54.000000, 15.588457)}, {(-40.500000, 23.382687)}, {(-27.000000, 15.588457)}, {(-13.500000, 23.382687)}, {(0.000000, 15.588457)}, {(13.500000, 23.382687)}, {(27.000000, 15.588457)}, {(40.500000, 23.382687)}, {(54.000000, 15.588457)}, {(67.500000, 23.382687)}, {(-67.500000, 38.971142)}, {(-54.000000, 31.176914)}, {(-40.500000, 38.971142)}, {(-27.000000, 31.176914)}, {(-13.500000, 38.971142)}, {(0.000000, 31.176914)}, {(13.500000, 38.971142)}, {(27.000000, 31.176914)}, {(40.500000, 38.971142)}, {(54.000000, 31.176914)}, {(67.500000, 38.971142)}, {(-67.500000, 54.559601)}, {(-54.000000, 46.765373)}, {(-40.500000, 54.559601)}, {(-27.000000, 46.765373)}, {(-13.500000, 54.559601)}, {(0.000000, 46.765373)}, {(13.500000, 54.559601)}, {(27.000000, 46.765373)}, {(40.500000, 54.559601)}, {(54.000000, 46.765373)}, {(67.500000, 54.559601)}, {(-67.500000, 70.148056)}, {(-54.000000, 62.353828)}, {(-40.500000, 70.148056)}, {(-27.000000, 62.353828)}, {(-13.500000, 70.148056)}, {(0.000000, 62.353828)}, {(13.500000, 70.148056)}, {(27.000000, 62.353828)}, {(40.500000, 70.148056)}, {(54.000000, 62.353828)}, {(67.500000, 70.148056)}, {(-67.500000, 85.736511)}, {(-54.000000, 77.942284)}, {(-40.500000, 85.736511)}, {(-27.000000, 77.942284)}, {(-13.500000, 85.736511)}, {(0.000000, 77.942284)}, {(13.500000, 85.736511)}, {(27.000000, 77.942284)}, {(40.500000, 85.736511)}, {(54.000000, 77.942284)}, {(67.500000, 85.736511)}}{
    \begin{scope}[scale=0.35,shift=\x]
      \begin{scope}[rotate=0, yscale=0.866]
        \draw (-0.5, 1) -- (-2.5, 5) -- (-4.5, 6.333) -- (-4, 8) -- (-4.5, 9) -- (-4, 10) -- (-2.5, 10.333) -- (-1, 10) -- (-0.5, 9) -- (0.5, 9) -- (1, 8) -- (2.5, 7.666) -- (2.5, 5) -- (0.5, 1) -- (-0.5, 1);
        \draw (-2.5, 5) -- (-2, 6) -- (1, 8);
        \draw (-2, 6) -- (-2, 8) -- (-4, 8);
        \draw (-2, 8) -- (-0.5, 9);
 
      \end{scope}
 \begin{scope}[rotate=60, yscale=0.866]
      \draw (-0.5, 1) -- (-2.5, 5) -- (-4.5, 6.333) -- (-4, 8) -- (-4.5, 9) -- (-4, 10) -- (-2.5, 10.333) -- (-1, 10) -- (-0.5, 9) -- (0.5, 9) -- (1, 8) -- (2.5, 7.666) -- (2.5, 5) -- (0.5, 1) -- (-0.5, 1);
        \draw (-2.5, 5) -- (-2, 6) -- (1, 8);
        \draw (-2, 6) -- (-2, 8) -- (-4, 8);
        \draw (-2, 8) -- (-0.5, 9);
      \end{scope}
       \begin{scope}[rotate=120, yscale=0.866]
        \draw (-0.5, 1) -- (-2.5, 5) -- (-4.5, 6.333) -- (-4, 8) -- (-4.5, 9) -- (-4, 10) -- (-2.5, 10.333) -- (-1, 10) -- (-0.5, 9) -- (0.5, 9) -- (1, 8) -- (2.5, 7.666) -- (2.5, 5) -- (0.5, 1) -- (-0.5, 1);
        \draw (-2.5, 5) -- (-2, 6) -- (1, 8);
        \draw (-2, 6) -- (-2, 8) -- (-4, 8);
        \draw (-2, 8) -- (-0.5, 9);
      \end{scope}
       \begin{scope}[rotate=180, yscale=0.866]
        \draw (-0.5, 1) -- (-2.5, 5) -- (-4.5, 6.333) -- (-4, 8) -- (-4.5, 9) -- (-4, 10) -- (-2.5, 10.333) -- (-1, 10) -- (-0.5, 9) -- (0.5, 9) -- (1, 8) -- (2.5, 7.666) -- (2.5, 5) -- (0.5, 1) -- (-0.5, 1);
        \draw (-2.5, 5) -- (-2, 6) -- (1, 8);
        \draw (-2, 6) -- (-2, 8) -- (-4, 8);
        \draw (-2, 8) -- (-0.5, 9);
      \end{scope}
       \begin{scope}[rotate=240, yscale=0.866]
        \draw (-0.5, 1) -- (-2.5, 5) -- (-4.5, 6.333) -- (-4, 8) -- (-4.5, 9) -- (-4, 10) -- (-2.5, 10.333) -- (-1, 10) -- (-0.5, 9) -- (0.5, 9) -- (1, 8) -- (2.5, 7.666) -- (2.5, 5) -- (0.5, 1) -- (-0.5, 1);
        \draw (-2.5, 5) -- (-2, 6) -- (1, 8);
        \draw (-2, 6) -- (-2, 8) -- (-4, 8);
        \draw (-2, 8) -- (-0.5, 9);
      \end{scope}
       \begin{scope}[rotate=300, yscale=0.866]
        \draw (-0.5, 1) -- (-2.5, 5) -- (-4.5, 6.333) -- (-4, 8) -- (-4.5, 9) -- (-4, 10) -- (-2.5, 10.333) -- (-1, 10) -- (-0.5, 9) -- (0.5, 9) -- (1, 8) -- (2.5, 7.666) -- (2.5, 5) -- (0.5, 1) -- (-0.5, 1);
        \draw (-2.5, 5) -- (-2, 6) -- (1, 8);
        \draw (-2, 6) -- (-2, 8) -- (-4, 8);
        \draw (-2, 8) -- (-0.5, 9);
      \end{scope}
    \end{scope}
  };
\end{tikzpicture}

\end{frame}

% Extra
\begin{frame}
  \vfill
  \centering
  \begin{beamercolorbox}[sep=8pt,center,shadow=true,rounded=true]{title}
    \usebeamerfont{title}Extra\par
  \end{beamercolorbox}
  \vfill
\end{frame}
\begin{frame}{Graphen und Einbettungen}
  \begin{definition}
    Ein \hdef{schlichter Graph} $G = (V, E)$ besteht aus einer endlichen Menge von Knoten $V$ und einer Kantenmenge $E \subseteq \set{\set{u, v} : u, v \in E, u \neq v}$.
  \end{definition}
  \pause
  \begin{definition}
    Eine \hdef{Einbettung} $\phi$ eines Graphen $G = (V, E)$ (nicht notwendigerweise schlicht) in eine (topologische) $2$-Mannigfaltigkeit $S$ ist eine Abbildung welche Knoten Punkte aus $S$ zuordnet und Kanten homöomorphe Bilder von $[0, 1]$ in $S$ zuordnet sodass die natürliche Grapheninzidenz erhalten bleibt.
  \end{definition}
\end{frame}
\begin{frame}{}
  \begin{definition}[Flächen von Einbettungen]
    Die Abschlüsse der Zusammenhangskomponenten des Komplements des Bildes einer Einbettung heißen Flächen.
  \end{definition}
  \pause
  \vspace{1cm}
  \centering
  \begin{tikzpicture}
    \only<2>{
      \fill[gray!30!white] (-1.5, -0.6) .. controls (-1.2,-0.8) and (-0.75, -0.8) .. (-0.5, -0.8) .. controls (-0.6,-0.9) and (-0.6, -1.0) .. (-0.6, -1.2) .. controls (-0.7, -1.2) and (-1.2,-0.9) .. cycle;
    }
    \only<3>{
      \fill[gray!30!white] (-3.5,0) .. controls (-3.5, 1) and (-1.5, 1.5) .. (0, 1.5) 
      .. controls ( 1.5, 1.5) and ( 3.5, 1) .. (3.5, 0)
      .. controls ( 3.5,-1) and ( 1.5,-1.5) .. (0,-1.5)
      .. controls (-1.5,-1.5) and (-3.5,-1) .. (-3.5, 0);
      \fill[white] (-1.75,0) .. controls (-1.5,-0.3) and (-1,-0.35) .. (0,-.35) .. controls (1,-0.35) and (1.5,-0.3) .. (1.75,0).. controls (1.5,0.3) and (1,0.5) .. (0,.5) .. controls (-1,0.5) and (-1.5,0.3) .. (-1.75,0);
      \fill[white] (-1.5, -0.6) .. controls (-1.2,-0.8) and (-0.75, -0.8) .. (-0.5, -0.8) .. controls (-0.6,-0.9) and (-0.6, -1.0) .. (-0.6, -1.2) .. controls (-0.7, -1.2) and (-1.2,-0.9) .. cycle;

    }
    \draw (-3.5,0) .. controls (-3.5, 1) and (-1.5, 1.5) .. (0, 1.5);
    \draw ( 3.5,0) .. controls ( 3.5, 1) and ( 1.5, 1.5) .. (0, 1.5);
    \draw (-3.5,0) .. controls (-3.5,-1) and (-1.5,-1.5) .. (0,-1.5);
    \draw ( 3.5,0) .. controls ( 3.5,-1) and ( 1.5,-1.5) .. (0,-1.5);
    \draw (-1.875, 0.15) -- (-1.75,0);
    \draw ( 1.875, 0.15) -- ( 1.75,0);
    \draw (-1.75,0) .. controls (-1.5,-0.3) and (-1,-0.35) .. (0,-.35) .. controls (1,-0.35) and (1.5,-0.3) .. (1.75,0);
    \draw (-1.75,0) .. controls (-1.5,0.3) and (-1,0.5) .. (0,.5) .. controls (1,0.5) and (1.5,0.3) .. (1.75,0);
    \fill[black, shift={(-1.5, -0.6)}, rotate=-30, xscale=0.9] (0, 0) circle (0.8pt);
    \fill[black, shift={(-0.5, -0.8)}, xscale=0.9] (0, 0) circle (0.8pt);
    \fill[black, shift={(-0.6, -1.2)}, xscale=0.9] (0, 0) circle (0.8pt);
    \draw (-1.5, -0.6) .. controls (-1.2,-0.8) and (-0.75, -0.8) .. (-0.5, -0.8);
    \draw (-0.5, -0.8) .. controls (-0.6,-0.9) and (-0.6, -1.0) .. (-0.6, -1.2);
    \draw (-1.5, -0.6) .. controls (-1.2,-0.9) and (-0.7, -1.2) .. (-0.6, -1.2);
  \end{tikzpicture}
\end{frame}
\begin{frame}{Polyedrische Karten}
  \begin{definition}
    Eine \hdef{Karte} ist eine Einbettung eines schlichten Graphen mit mindestens $3$-valenten Ecken bei der jede Fläche eine abgeschlossene $2$-Zelle ist (homöomorph zu $\set{x \in \reals^2 : |x| \leq 1}$).
  \end{definition}
  \pause
  \begin{definition}
    Eine Karte ist \hdef{polyedrisch}, wenn je zwei verschiedene Flächen gutartig inzidieren, d.h. je zwei verschiedene Flächen sind disjunkt, inzident zu genau einer gemeinsamen Ecke oder inzident zu genau einer gemeinsamen Kante.
  \end{definition}
  \pause
  \begin{definition}[$p$- und $v$-Vektoren]
    Der \hdef{$p$-Vektor} einer Karte $M$ ist die Sequenz~$(p_k)$, wobei $p_k$ die Anzahl der $k$-gonalen Flächen von $M$ angibt. Der \hdef{$v$-Vektor} einer Karte $M$ ist die Sequenz~$(v_k)$, wobei $v_k$ die Anzahl der $k$-valenten Ecken von $M$ angibt.
  \end{definition}
\end{frame}
\begin{frame}{Realisierbarkeit}
  \begin{definition}
    Ein Paar von Sequenzen $p = (p_3, \dots, p_m)$ und $v = (v_3, \dots, v_n)$ heißt \hdef{realisierbar} auf einer $2$-Mannigfaltigkeit, wenn es eine polyedrische Karte gibt, welche $p$ als $p$-Vektor und $v$ als $v$-Vektor hat. Diese Karte heißt \hdef{Realisierung}.
  \end{definition}
  \pause
  Wir haben in den vorangegangenen {\sc Eberhard}-Theoremen gesehen, dass es grundlegende Gleichungen gibt, welche für die Realisierbarkeit erfüllt sein müssen. Diese leiten sich aus der {\sc Euler}-{\sc Poincaré}-Formel ab:
  \begin{block}{{\sc Euler}-{\sc Poincaré}-Formel}
    Für Einbettungen mit $f_0$ Ecken, $f_1$ Kanten und $f_2$ Flächen in eine Mannigfaltigkeit mit {\sc Euler}-Charakteristik gilt:
    \begin{align*}
      f_0 - f_1 + f_2 = \chi.
    \end{align*}
  \end{block}
\end{frame}

\begin{frame}{Zulässige Sequenzen}
  Mit einer ähnlichen Rechnung wie zuvor lässt sich aus der {\sc Euler}-{\sc Poincaré}-Formel zeigen, dass zwischen $p$- und $v$-Vektoren und der Anzahl der Kanten $f_1$ von Karten auf $2$-Mannigfaltigkeiten $S$ der Charakteristik $\chi$ der folgende Zusammenhang für alle $t \in \reals$ herrscht:
  \begin{beamercolorbox}[sep=-10pt,center,shadow=true,rounded=true]{block body}
    \begin{align*}
      \sum_{k \geq 3} (2 - t \cdot k ) v_k + \sum_{k \geq 3} ( 2 - (1 - t) \cdot k ) p_k &= 2 \chi\\
      \sum_{k \geq 3} k \cdot p_k = \sum_{k \geq 3} k \cdot v_k &= 2f_1
    \end{align*}
  \end{beamercolorbox}
  Wir möchten Paare $(p, v)$ als \hdef{zulässig für $S$} bezeichnen, wenn sie die obigen Gleichungen für bestimmte $f_1$ und alle $t \in \reals$ erfüllen. Zulässig zu sein ist also eine notwendige Bedingung für Realisierbarkeit.
\end{frame}

\begin{frame}{Flache Sequenzen}
  Völlig analog können wir auch Bedingungen an Sequenzen $q$ und $w$ finden, wenn $p+c\cdot q$ und $v+d\cdot w$ realisierbar sein sollen:
  \begin{beamercolorbox}[sep=-10pt,center,shadow=true,rounded=true]{block body}
    \begin{align*}
      \sum_{k \geq 3} \left(\frac{2}{t} - k \right) w_k = \sum_{k \geq 3} \left( \frac{2}{1-t} - k \right) q_k = 0.
    \end{align*}
  \end{beamercolorbox}
  \pause
  Falls obige Gleichung für ein $t \in \left[\tfrac{1}{3}, \tfrac{2}{3}\right]$ gilt, dann heißen $q$ und $w$ \hdef{flach mit Parameter $t$}.
  \pause
  \begin{definition}
    Zwei zulässige Sequenzen $p$, $v$ für eine $2$-Mannigfaltigkeit $S$ mit {\sc Euler}-Charakteristic $\chi$ heißen \hdef{$q$-$w$-realisierbar auf $S$}, falls es $c, d \in \nats$ und eine Realisierung von $(p + c \cdot q, v + d \cdot w)$ auf $S$ gibt.
  \end{definition}
\end{frame}

\begin{frame}
  \begin{theorem}[{\sc Jendrol', Jucovi{\v{c}}}, 1977, \cite{jendrol1977generalization}] \label{thm:eberhard:extended:3}
    Ein Paar von zulässigen Sequenzen $p$, $v$ für eine geschlossenen orientierbaren $2$-Mannigfaltigkeit $S$ ist $[6]$-$[3]$-realisierbar genau dann, wenn:
    \begin{align*}
      \sum_{k \geq 3,\,2 \nmid k} p_k \neq 0 \text{ oder} \sum_{k \geq 4,\,3 \nmid k} v_k \neq 1& &\text{ wenn } \chi = 2, \\
      p \neq [1 \times 5, 1 \times 7] \text{ oder } v \neq [4 \times 3]& &\text{ wenn } \chi = 0.
    \end{align*}
  \end{theorem}
  \pause
  \begin{theorem}[{\sc Jendrol', Jucovi{\v{c}}, Barnette, Grünbaum, Zaks}, 1973, \cite{jucovivc1973theorem}, \cite{barnette1971toroidal}, \cite{grunbaum1969planar}, \cite{zaks1971analogue}] \label{thm:eberhard:extended:4}
    Ein Paar von zulässigen Sequenzen $p$, $v$ für auf einer geschlossenen orientierbaren $2$-Mannigfaltigkeit $S$ ist $[4]$-$[4]$-realisierbar genau dann, wenn
    \begin{align*}
      p \neq [1 \times 3, 1 \times 5] \text{ oder } v \neq [4 \times 4]& &\text{ wenn } \chi = 0,\\
      p \neq [4 \times 4] \text{ oder } v \neq [1 \times 3, 1 \times 5]& &\text{ wenn } \chi = 0.
    \end{align*}
  \end{theorem}
\end{frame}


\begin{frame}{Münchhausentrick}{}
  {\sc Eberhard}-Typ Probleme für Sequenzen $q = [q_s \times s, q_l \times l]$ mit zwei teilerfremden Einträgen und Sequenzen $w = [1 \times r]$ mit genau einem Eintrag, $s, l, r \in \nats$ haben eine nützliche Eigenschaft, die die Konstruktion vereinfacht:
  \begin{lemma}
    Seien $p, v$ zulässige Sequenzen für eine $2$-Mannigfaltigkeit $S$ und $q$, $w$ wie oben und flach mit Parameter $t \in \left[\tfrac{1}{3}, \tfrac{2}{3}\right]$. Sei $M$ eine polyedrische Karte mit $p$-Vektor $p + [c_s \cdot q_s \times s, c_l \cdot q_l \times l]$ und $v$-Vektor $v + [d \times r]$ für $c_s, c_l, d \in \nats$. Dann ist $M$ eine $q$-$w$-Realisierung von $(p, v)$.
  \end{lemma}
  \pause
  Das heißt, sobald wir eine Realisierung finden, die neben $p$ und $v$ nur $s$- und $l$-gonale Flächen und $r$-valente Ecken besitzt, dann ist dies auch automatisch eine gewünschte $q$-$w$-Realisierung von $(p, v)$ und es ist kein genaues Nachzählen der \newline $s$- und $l$-gonale Flächen notwendig (hierbei ist $\gcd(q_s, q_l) = 1$ wichtig!).
\end{frame}

\begin{frame}{Negative Resultate}
  Viele der Unterfälle von $(s, r) = (3, 3)$ lassen sich ausschließen:
  \begin{definition}
    \hdef{$C(p, v, q, w, S)$} ist die Menge aller Paare $(c, d) \in \nats^2$ für die $(p + c \cdot q, v + d \cdot w)$ auf $S$ realisierbar ist.
  \end{definition}
  \pause
  \begin{theorem}
    Seien  $p$ und $v$ zulässige Sequenzen, $S$ eine kompakten $2$-Mannigfaltigkeit und $q = [q_3 \times 3, q_l \times l]$ und $w = [3]$, $l \in \nats, l \geq 13$. Dann ist $|C(p, v, q, w, S)| < \infty$. Wenn $l \geq 11$ und $\sum_{k \geq 4} 2k \cdot v_k + \sum_{k \geq 4} \lfloor \tfrac{k}{2} \rfloor p_k < 3p_3$, dann ist $C(p, v, q, w, S) = \emptyset$, außer wenn $S \cong \sphere^2$, $p = [4 \times 3]$, $v = [4 \times 3]$. In diesem Fall ist $C(p, v, q, w, \sphere^2) = \set{(0, 0)}$.

  \pause
    Beweisidee: In einer polyedrischen Karte kann die folgende Figur nicht auftreten, außer im Fall des Tetraeders.

    { \centering
      \begin{tikzpicture}[scale=0.5]
        \draw (-2, 0) -- (-1, 0) -- (0, 1) -- (1, 0) -- (2, 0);
        \draw (-1, 0) -- (0, -1) -- (1, 0);
        \draw (0, 1) -- (0, -1);
        \draw[loosely dotted] (-2.5, 0) -- (-2, 0);
        \draw[loosely dotted] (2.5, 0) -- (2, 0);

        \fill[black] (-1,0) circle(2pt);
        \fill[black] (1,0) circle(2pt);
        \fill[black] (0,-1) circle(2pt);
        \fill[black] (0,1) circle(2pt);
      \end{tikzpicture}
      \par
    }
    Eine Abschätzung der Anzahl an Kanten die an Dreiecken anliegen zeigt das Resultat.
  \end{theorem}
\end{frame}

\begin{frame}{Negative Resultate}
  \begin{definition} Ein Knoten in einem Graphen heißt \hdef{$(a \mod r)$-valent}, wenn seine Valenz kongruent $a$ modulo $r$ ist, $r \in \nats_{>0}$, $a \in \set{0, \dots, r - 1}$. Ein Graph heißt (fast) $(a \mod r)$-valent, wenn alle (alle bis auf genau einem) Knoten $(a \mod r)$-valent sind. Dual heißen Flächen \hdef{$(a \mod s)$-gonal}, wenn ihre Knotenanzahl kongruent $a$ modulo $s$ ist, $s \in \nats_{>0}$, $a \in \set{0, \dots, s - 1}$ und der Graph heißt (fast) $(a \mod s)$-gonal wenn alle (alle bis auf genau eine) Fläche  $(a \mod s)$-gonal ist.
  \end{definition}
  
  \begin{theorem}[Malkevitch '70, \cite{malkevitch1970properties}] Sei $r, s \in \set{2, 3, 4, 5}$. Kein verbundener planarer Graph ist sowohl $(0 \mod r)$-valent als auch fast $(0 \mod s)$-gonal. Dual: Kein verbundener planarer Graph ist sowohl $(0 \mod s)$-gonal als auch fast $(0 \mod r)$-valent.
  \end{theorem}
\end{frame}

\begin{frame}
  \begin{example}
    Es lässt sich damit zeigen, dass die folgenden für die Sphäre $\sphere^2$ zulässigen Sequenzen $p$, $v$ keine $q$-$w$-Realisierung besitzen (für alle $k \in \nats$):
    \begin{alignat*}{10}
      &p = [& 6 \times 4, 6&], \quad v = [& 8 \times 3   &], \quad q = [&(2k + 1) \times 4&,{} & 8 + 4k&], \quad w = [3]\\
      &p = [&12 \times 5, 6&], \quad v = [&22 \times 3   &], \quad q = [&(5k + 4) \times 5&,{} &10 + 5k&], \quad w = [3]\\
      &p = [& 8 \times 3, 4&], \quad v = [& 8 \times 4   &], \quad q = [&(3k + 2) \times 3&,{} & 6 + 3k&], \quad w = [4]\\
      &p = [&22 \times 3, 4&], \quad v = [&14 \times 5   &], \quad q = [&(9k + 8) \times 3&,{} & 6 + 3k&], \quad w = [5]\\
      &p = [& 7 \times 4   &], \quad v = [& 8 \times 3, 4&], \quad q = [&(2k + 2) \times 4&,{} &10 + 4k&], \quad w = [3]
    \end{alignat*}
    
  \end{example}
\end{frame}

\begin{frame}
  Dieses Resultat lässt sich ergänzen mit:
  \begin{theorem}
    Sei $(s, r) \in \set{(3, 3), (3, 4), (3, 5), (4, 3), (5, 3)}$. Es gibt keinen verbundenen planaren Graphen welcher sowohl $(r \mod 2r)$-valent als auch fast $(s \mod 2s)$-gonal ist. Dual: Es gibt keinen verbundenen planaren Graphen welcher sowohl $(s \mod 2s)$-gonal als auch fast $(r \mod 2r)$-valent ist. 
  \end{theorem}

  \begin{example}
    $p = [4 \times 3, 6]$, $v = [6 \times 3]$ sind zulässig auf der Sphäre $\sphere^2$, aber nicht $[3, 9]$-$[3]$-realisierbar.
  \end{example}
\end{frame}

\begin{frame}
  \begin{columns}
    \begin{column}{0.3\textwidth}  
      \begin{tikzpicture}
        \begin{scope}[yscale=0.866, scale=0.5]
          \draw (-1,0)--(-0.5,1)--(-1,4)--(0.5,10.5)--(-0.5,11.5)--(-2.5,11)--(-3.5,10.5)--(-4.5,10.5)--(-5.5,11)--(-6.375,10.25)--(-6.875,9.25)--(-7,8)--(-7.625,5.75)--(-3.5,1)--(-1,0);

          \draw (-1,4)-- (-3,6)--(-4.5,7)--(-7,8);
          \draw (-3,6)--(-4,8)--(-4,9.5)--(-3.5,10.5);
          \draw (-4,9.5)--(-4.5,10.5);
          \draw (-4,8)--(-4.5,7);

          \fill[black] (-1,0)          circle (2pt);
          \fill[black] (-0.5,1)        circle (2pt);
          \fill[black] (-1,4)          circle (2pt);
          \fill[black] (0.5,10.5)      circle (2pt);
          \fill[black] (-0.5,11.5)     circle (2pt);
          \fill[black] (-2.5,11)       circle (2pt);
          \fill[black] (-3.5,10.5)     circle (2pt);
          \fill[black] (-4.5,10.5)     circle (2pt);
          \fill[black] (-5.5,11)       circle (2pt);
          \fill[black] (-6.375,10.25)  circle (2pt);
          \fill[black] (-6.875,9.25)   circle (2pt);
          \fill[black] (-7,8)          circle (2pt);
          \fill[black] (-7.625,5.75)   circle (2pt);
          \fill[black] (-3.5,1)        circle (2pt);
          \fill[black] (-4,8)          circle (2pt);
          \fill[black] (-4.5,7)        circle (2pt);
          \fill[black] (-4,9.5)        circle (2pt);
          \fill[black] (-3,6)          circle (2pt);
        \end{scope}
      \end{tikzpicture}
    \end{column}
    \begin{column}{0.5\textwidth} 
      \begin{tikzpicture}
        \begin{scope}[scale=3]
          \fill[black] (-0.151641, 0.276786) circle (0.3pt);
          \fill[black] (0.110346, 0.305984) circle (0.3pt);
          \fill[black] (0.201969, 0.534986) circle (0.3pt);
          \fill[black] (-0.550102, 0.038030) circle (0.3pt);
          \fill[black] (-0.469252, -0.270235) circle (0.3pt);
          \fill[black] (-0.242657, -0.483949) circle (0.3pt);
          \fill[black] (0.069886, -0.546812) circle (0.3pt);
          \fill[black] (0.285052, -0.135070) circle (0.3pt);
          \fill[black] (0.298799, 0.128253) circle (0.3pt);
          \fill[black] (0.522273, 0.232891) circle (0.3pt);
          \fill[black] (0.802123, 0.597159) circle (0.3pt);
          \fill[black] (0.727374, 0.686242) circle (0.3pt);
          \fill[black] (0.642788, 0.766044) circle (0.3pt);
          \fill[black] (0.549509, 0.835488) circle (0.3pt);
          \fill[black] (0.448799, 0.893633) circle (0.3pt);
          \fill[black] (-0.276845, 0.480678) circle (0.3pt);
          \fill[black] (-0.998308, 0.058145) circle (0.3pt);
          \fill[black] (-0.998308, -0.058145) circle (0.3pt);
          \fill[black] (-0.316770, 0.673568) circle (0.3pt);
          \fill[black] (-0.529803, 0.619619) circle (0.3pt);
          \fill[black] (-0.734747, 0.512559) circle (0.3pt);
          \fill[black] (-0.918216, 0.396080) circle (0.3pt);
          \fill[black] (-0.957990, 0.286803) circle (0.3pt);
          \fill[black] (-0.984808, 0.173648) circle (0.3pt);
          \fill[black] (-0.866025, 0.500000) circle (0.3pt);
          \fill[black] (0.342020, 0.939693) circle (0.3pt);
          \fill[black] (0.230616, 0.973045) circle (0.3pt);
          \fill[black] (0.116093, 0.993238) circle (0.3pt);
          \fill[black] (0.000000, 1.000000) circle (0.3pt);
          \fill[black] (-0.161697, 0.852892) circle (0.3pt);
          \fill[black] (-0.116093, 0.993238) circle (0.3pt);
          \fill[black] (-0.230616, 0.973045) circle (0.3pt);
          \fill[black] (-0.342020, 0.939693) circle (0.3pt);
          \fill[black] (-0.448799, 0.893633) circle (0.3pt);
          \fill[black] (-0.499822, 0.763516) circle (0.3pt);
          \fill[black] (-0.549509, 0.835488) circle (0.3pt);
          \fill[black] (-0.642788, 0.766044) circle (0.3pt);
          \fill[black] (-0.727374, 0.686242) circle (0.3pt);
          \fill[black] (-0.802123, 0.597159) circle (0.3pt);
          \fill[black] (-0.984808, -0.173648) circle (0.3pt);
          \fill[black] (-0.957990, -0.286803) circle (0.3pt);
          \fill[black] (-0.918216, -0.396080) circle (0.3pt);
          \fill[black] (-0.866025, -0.500000) circle (0.3pt);
          \fill[black] (-0.674368, -0.428128) circle (0.3pt);
          \fill[black] (-0.802123, -0.597159) circle (0.3pt);
          \fill[black] (-0.727374, -0.686242) circle (0.3pt);
          \fill[black] (-0.642788, -0.766044) circle (0.3pt);
          \fill[black] (-0.549509, -0.835488) circle (0.3pt);
          \fill[black] (-0.388407, -0.698006) circle (0.3pt);
          \fill[black] (-0.448799, -0.893633) circle (0.3pt);
          \fill[black] (-0.342020, -0.939693) circle (0.3pt);
          \fill[black] (-0.230616, -0.973045) circle (0.3pt);
          \fill[black] (-0.116093, -0.993238) circle (0.3pt);
          \fill[black] (-0.000000, -1.000000) circle (0.3pt);
          \fill[black] (0.116093, -0.993238) circle (0.3pt);
          \fill[black] (0.495962, -0.248431) circle (0.3pt);
          \fill[black] (0.866025, 0.500000) circle (0.3pt);
          \fill[black] (0.230616, -0.973045) circle (0.3pt);
          \fill[black] (0.342020, -0.939693) circle (0.3pt);
          \fill[black] (0.448799, -0.893633) circle (0.3pt);
          \fill[black] (0.554387, -0.703802) circle (0.3pt);
          \fill[black] (0.649326, -0.493105) circle (0.3pt);
          \fill[black] (0.690734, -0.277388) circle (0.3pt);
          \fill[black] (0.549509, -0.835488) circle (0.3pt);
          \fill[black] (0.642788, -0.766044) circle (0.3pt);
          \fill[black] (0.727374, -0.686242) circle (0.3pt);
          \fill[black] (0.802123, -0.597159) circle (0.3pt);
          \fill[black] (0.866025, -0.500000) circle (0.3pt);
          \fill[black] (0.791252, -0.454994) circle (0.3pt);
          \fill[black] (0.918216, -0.396080) circle (0.3pt);
          \fill[black] (0.957990, -0.286803) circle (0.3pt);
          \fill[black] (0.984808, -0.173648) circle (0.3pt);
          \fill[black] (0.998308, -0.058145) circle (0.3pt);
          \fill[black] (0.860802, -0.111982) circle (0.3pt);
          \fill[black] (0.998308, 0.058145) circle (0.3pt);
          \fill[black] (0.984808, 0.173648) circle (0.3pt);
          \fill[black] (0.957990, 0.286803) circle (0.3pt);
          \fill[black] (0.918216, 0.396080) circle (0.3pt);


          \coordinate (x0) at (-0.151641, 0.276786);
          \coordinate (x1) at (0.110346, 0.305984);
          \coordinate (x2) at (0.201969, 0.534986);
          \coordinate (x3) at (-0.550102, 0.038030);
          \coordinate (x4) at (-0.469252, -0.270235);
          \coordinate (x5) at (-0.242657, -0.483949);
          \coordinate (x6) at (0.069886, -0.546812);
          \coordinate (x7) at (0.285052, -0.135070);
          \coordinate (x8) at (0.298799, 0.128253);
          \coordinate (x9) at (0.522273, 0.232891);
          \coordinate (x10) at (0.802123, 0.597159);
          \coordinate (x11) at (0.727374, 0.686242);
          \coordinate (x12) at (0.642788, 0.766044);
          \coordinate (x13) at (0.549509, 0.835488);
          \coordinate (x14) at (0.448799, 0.893633);
          \coordinate (x15) at (-0.276845, 0.480678);
          \coordinate (x16) at (-0.998308, 0.058145);
          \coordinate (x17) at (-0.998308, -0.058145);
          \coordinate (x18) at (-0.316770, 0.673568);
          \coordinate (x19) at (-0.529803, 0.619619);
          \coordinate (x20) at (-0.734747, 0.512559);
          \coordinate (x21) at (-0.918216, 0.396080);
          \coordinate (x22) at (-0.957990, 0.286803);
          \coordinate (x23) at (-0.984808, 0.173648);
          \coordinate (x24) at (-0.866025, 0.500000);
          \coordinate (x25) at (0.342020, 0.939693);
          \coordinate (x26) at (0.230616, 0.973045);
          \coordinate (x27) at (0.116093, 0.993238);
          \coordinate (x28) at (0.000000, 1.000000);
          \coordinate (x29) at (-0.161697, 0.852892);
          \coordinate (x30) at (-0.116093, 0.993238);
          \coordinate (x31) at (-0.230616, 0.973045);
          \coordinate (x32) at (-0.342020, 0.939693);
          \coordinate (x33) at (-0.448799, 0.893633);
          \coordinate (x34) at (-0.499822, 0.763516);
          \coordinate (x35) at (-0.549509, 0.835488);
          \coordinate (x36) at (-0.642788, 0.766044);
          \coordinate (x37) at (-0.727374, 0.686242);
          \coordinate (x38) at (-0.802123, 0.597159);
          \coordinate (x39) at (-0.984808, -0.173648);
          \coordinate (x40) at (-0.957990, -0.286803);
          \coordinate (x41) at (-0.918216, -0.396080);
          \coordinate (x42) at (-0.866025, -0.500000);
          \coordinate (x43) at (-0.674368, -0.428128);
          \coordinate (x44) at (-0.802123, -0.597159);
          \coordinate (x45) at (-0.727374, -0.686242);
          \coordinate (x46) at (-0.642788, -0.766044);
          \coordinate (x47) at (-0.549509, -0.835488);
          \coordinate (x48) at (-0.388407, -0.698006);
          \coordinate (x49) at (-0.448799, -0.893633);
          \coordinate (x50) at (-0.342020, -0.939693);
          \coordinate (x51) at (-0.230616, -0.973045);
          \coordinate (x52) at (-0.116093, -0.993238);
          \coordinate (x53) at (-0.000000, -1.000000);
          \coordinate (x54) at (0.116093, -0.993238);
          \coordinate (x55) at (0.495962, -0.248431);
          \coordinate (x56) at (0.866025, 0.500000);
          \coordinate (x57) at (0.230616, -0.973045);
          \coordinate (x58) at (0.342020, -0.939693);
          \coordinate (x59) at (0.448799, -0.893633);
          \coordinate (x60) at (0.554387, -0.703802);
          \coordinate (x61) at (0.649326, -0.493105);
          \coordinate (x62) at (0.690734, -0.277388);
          \coordinate (x63) at (0.549509, -0.835488);
          \coordinate (x64) at (0.642788, -0.766044);
          \coordinate (x65) at (0.727374, -0.686242);
          \coordinate (x66) at (0.802123, -0.597159);
          \coordinate (x67) at (0.866025, -0.500000);
          \coordinate (x68) at (0.791252, -0.454994);
          \coordinate (x69) at (0.918216, -0.396080);
          \coordinate (x70) at (0.957990, -0.286803);
          \coordinate (x71) at (0.984808, -0.173648);
          \coordinate (x72) at (0.998308, -0.058145);
          \coordinate (x73) at (0.860802, -0.111982);
          \coordinate (x74) at (0.998308, 0.058145);
          \coordinate (x75) at (0.984808, 0.173648);
          \coordinate (x76) at (0.957990, 0.286803);
          \coordinate (x77) at (0.918216, 0.396080);

          \draw (-0.151641, 0.276786) -- (0.110346, 0.305984);
          \draw (0.110346, 0.305984) -- (0.201969, 0.534986);
          \draw (0.201969, 0.534986) -- (-0.151641, 0.276786);
          \draw (-0.151641, 0.276786) -- (-0.550102, 0.038030);
          \draw (-0.550102, 0.038030) -- (-0.469252, -0.270235);
          \draw (-0.469252, -0.270235) -- (-0.242657, -0.483949);
          \draw (-0.242657, -0.483949) -- (0.069886, -0.546812);
          \draw (0.069886, -0.546812) -- (0.285052, -0.135070);
          \draw (0.285052, -0.135070) -- (0.298799, 0.128253);
          \draw (0.298799, 0.128253) -- (0.110346, 0.305984);
          \draw (0.285052, -0.135070) -- (0.522273, 0.232891);
          \draw (0.522273, 0.232891) -- (0.298799, 0.128253);
          \draw (0.522273, 0.232891) -- (0.802123, 0.597159);
          \draw (0.802123, 0.597159) -- (0.727374, 0.686242);
          \draw (0.727374, 0.686242) -- (0.642788, 0.766044);
          \draw (0.642788, 0.766044) -- (0.549509, 0.835488);
          \draw (0.549509, 0.835488) -- (0.201969, 0.534986);
          \draw (0.549509, 0.835488) -- (0.448799, 0.893633);
          \draw (0.448799, 0.893633) -- (-0.276845, 0.480678);
          \draw (-0.276845, 0.480678) -- (-0.998308, 0.058145);
          \draw (-0.998308, 0.058145) -- (-0.998308, -0.058145);
          \draw (-0.998308, -0.058145) -- (-0.550102, 0.038030);
          \draw (-0.276845, 0.480678) -- (-0.316770, 0.673568);
          \draw (-0.316770, 0.673568) -- (-0.529803, 0.619619);
          \draw (-0.529803, 0.619619) -- (-0.734747, 0.512559);
          \draw (-0.734747, 0.512559) -- (-0.918216, 0.396080);
          \draw (-0.918216, 0.396080) -- (-0.957990, 0.286803);
          \draw (-0.957990, 0.286803) -- (-0.984808, 0.173648);
          \draw (-0.984808, 0.173648) -- (-0.998308, 0.058145);
          \draw (-0.734747, 0.512559) -- (-0.866025, 0.500000);
          \draw (-0.866025, 0.500000) -- (-0.918216, 0.396080);
          \draw (0.448799, 0.893633) -- (0.342020, 0.939693);
          \draw (0.342020, 0.939693) -- (0.230616, 0.973045);
          \draw (0.230616, 0.973045) -- (0.116093, 0.993238);
          \draw (0.116093, 0.993238) -- (0.000000, 1.000000);
          \draw (0.000000, 1.000000) -- (-0.161697, 0.852892);
          \draw (-0.161697, 0.852892) -- (-0.316770, 0.673568);
          \draw (0.000000, 1.000000) -- (-0.116093, 0.993238);
          \draw (-0.116093, 0.993238) -- (-0.161697, 0.852892);
          \draw (-0.116093, 0.993238) -- (-0.230616, 0.973045);
          \draw (-0.230616, 0.973045) -- (-0.342020, 0.939693);
          \draw (-0.342020, 0.939693) -- (-0.448799, 0.893633);
          \draw (-0.448799, 0.893633) -- (-0.499822, 0.763516);
          \draw (-0.499822, 0.763516) -- (-0.529803, 0.619619);
          \draw (-0.448799, 0.893633) -- (-0.549509, 0.835488);
          \draw (-0.549509, 0.835488) -- (-0.499822, 0.763516);
          \draw (-0.549509, 0.835488) -- (-0.642788, 0.766044);
          \draw (-0.642788, 0.766044) -- (-0.727374, 0.686242);
          \draw (-0.727374, 0.686242) -- (-0.802123, 0.597159);
          \draw (-0.802123, 0.597159) -- (-0.866025, 0.500000);
          \draw (-0.998308, -0.058145) -- (-0.984808, -0.173648);
          \draw (-0.984808, -0.173648) -- (-0.957990, -0.286803);
          \draw (-0.957990, -0.286803) -- (-0.918216, -0.396080);
          \draw (-0.918216, -0.396080) -- (-0.866025, -0.500000);
          \draw (-0.866025, -0.500000) -- (-0.674368, -0.428128);
          \draw (-0.674368, -0.428128) -- (-0.469252, -0.270235);
          \draw (-0.866025, -0.500000) -- (-0.802123, -0.597159);
          \draw (-0.802123, -0.597159) -- (-0.674368, -0.428128);
          \draw (-0.802123, -0.597159) -- (-0.727374, -0.686242);
          \draw (-0.727374, -0.686242) -- (-0.642788, -0.766044);
          \draw (-0.642788, -0.766044) -- (-0.549509, -0.835488);
          \draw (-0.549509, -0.835488) -- (-0.388407, -0.698006);
          \draw (-0.388407, -0.698006) -- (-0.242657, -0.483949);
          \draw (-0.549509, -0.835488) -- (-0.448799, -0.893633);
          \draw (-0.448799, -0.893633) -- (-0.388407, -0.698006);
          \draw (-0.448799, -0.893633) -- (-0.342020, -0.939693);
          \draw (-0.342020, -0.939693) -- (-0.230616, -0.973045);
          \draw (-0.230616, -0.973045) -- (-0.116093, -0.993238);
          \draw (-0.116093, -0.993238) -- (-0.000000, -1.000000);
          \draw (-0.000000, -1.000000) -- (0.069886, -0.546812);
          \draw (-0.000000, -1.000000) -- (0.116093, -0.993238);
          \draw (0.116093, -0.993238) -- (0.495962, -0.248431);
          \draw (0.495962, -0.248431) -- (0.866025, 0.500000);
          \draw (0.866025, 0.500000) -- (0.802123, 0.597159);
          \draw (0.116093, -0.993238) -- (0.230616, -0.973045);
          \draw (0.230616, -0.973045) -- (0.342020, -0.939693);
          \draw (0.342020, -0.939693) -- (0.448799, -0.893633);
          \draw (0.448799, -0.893633) -- (0.554387, -0.703802);
          \draw (0.554387, -0.703802) -- (0.649326, -0.493105);
          \draw (0.649326, -0.493105) -- (0.690734, -0.277388);
          \draw (0.690734, -0.277388) -- (0.495962, -0.248431);
          \draw (0.448799, -0.893633) -- (0.549509, -0.835488);
          \draw (0.549509, -0.835488) -- (0.554387, -0.703802);
          \draw (0.549509, -0.835488) -- (0.642788, -0.766044);
          \draw (0.642788, -0.766044) -- (0.727374, -0.686242);
          \draw (0.727374, -0.686242) -- (0.802123, -0.597159);
          \draw (0.802123, -0.597159) -- (0.866025, -0.500000);
          \draw (0.866025, -0.500000) -- (0.791252, -0.454994);
          \draw (0.791252, -0.454994) -- (0.649326, -0.493105);
          \draw (0.866025, -0.500000) -- (0.918216, -0.396080);
          \draw (0.918216, -0.396080) -- (0.791252, -0.454994);
          \draw (0.918216, -0.396080) -- (0.957990, -0.286803);
          \draw (0.957990, -0.286803) -- (0.984808, -0.173648);
          \draw (0.984808, -0.173648) -- (0.998308, -0.058145);
          \draw (0.998308, -0.058145) -- (0.860802, -0.111982);
          \draw (0.860802, -0.111982) -- (0.690734, -0.277388);
          \draw (0.998308, -0.058145) -- (0.998308, 0.058145);
          \draw (0.998308, 0.058145) -- (0.860802, -0.111982);
          \draw (0.998308, 0.058145) -- (0.984808, 0.173648);
          \draw (0.984808, 0.173648) -- (0.957990, 0.286803);
          \draw (0.957990, 0.286803) -- (0.918216, 0.396080);
          \draw (0.918216, 0.396080) -- (0.866025, 0.500000);
        \end{scope}
      \end{tikzpicture}
    \end{column}
  \end{columns}
\end{frame}

\begin{frame}
  \begin{tabularx}{\textwidth}{|c|c|X|}
    \hline
    $(s, r)$ & $l$ &\\
    \hline
    $(3, 3)$ & $l = 7, 8, 10$        & Für alle $p, v, S$ möglich\\
    $(3, 3)$ & $l = 9$               & Es gibt unrealisierbare Beispiele für $S \cong \sphere^2$\\
    $(3, 3)$ & $l \geq 11$           & Es gibt unrealisierbare Beispiele für alle $S$\\
    $(4, 3)$ & $l \equiv 0~(\mod 4)$ & Es gibt unrealisierbare Beispiele für $S \cong \sphere^2$\\
    $(4, 3)$ & $l \equiv 1~(\mod 2)$ & Für alle $p, v, S$ möglich\\
    $(4, 3)$ & $l \equiv 2~(\mod 4)$ & Alle unrealisierbaren Beispiele bekannt\\
    $(5, 3)$ & $l \equiv 0~(\mod 5)$ & Es gibt unrealisierbare Beispiele für $S \cong \sphere^2$\\
    $(5, 3)$ & $l \not\equiv 0~(\mod 5)$ & Für alle $p, v, S$ möglich\\
    $(3, 4)$ & $l \equiv 0~(\mod 3)$ & Es gibt unrealisierbare Beispiele für $S \cong \sphere^2$\\
    $(3, 4)$ & $l \not\equiv 0~(\mod 3)$ & Für alle $p, v, S$ möglich\\
    $(3, 5)$ & $l \equiv 0~(\mod 3)$ & Es gibt unrealisierbare Beispiele für $S \cong \sphere^2$\\
    $(3, 5)$ & $l \not\equiv 0~(\mod 3)$ & Für alle $p, v, S$ möglich\\
    \hline
  \end{tabularx}
\end{frame}


\end{document}
