\checkoddpage\ifoddpage\hbox{}\clearpage\else\fi%
\section{$5$-valent {\sc Eberhard}-like theorems with triangles}\label{sec:3:5}

In the previous two section, we could rely on \autoref{thm:eberhard:extended:3} and \autoref{thm:eberhard:extended:4} to do the main work for us and create an initial map we can begin to work with. Since the previous work on $5$-valent {\sc Eberhard} theorems is not as advanced as in the $3$-valent and $4$-valent cases and does not consider the additional $v$-vector, we need a new $5$-valent {\sc Eberhard} theorem which we can use as first construction step and which includes both a $p$-vector and a $v$-vector. But the situation is not as difficult as doing a whole construction of a polyhedral map by ourselves, as in fact, we can use \autoref{thm:eberhard:extended:4} nonetheless and just need to convert the $4$-valent vertices into $5$-valent ones.

\begin{proposition}
  Let $p = (p_3, p_4, \dots, p_n)$, $v = (v_3, v_4, \dots, v_m)$ be a pair of admissible sequences, then $p$ is $[2\times3, 4]$-$[5]$-realizable on a closed orientable $2$-manifold $S$ with {\sc Euler} characteristic $\chi$.
  \begin{proof}
    Let $p' \defeq p + v$ and $v' = [v'_4 \times 4]$ for some $v'_4$ we want to specify shortly. Since $p$ and $v$ are admissible, \eqref{eq:vp:4} holds for $p$ and $v$, and it also holds for $p'$ and $v'$:
    \begin{align*}
      &\sum_{k=3}^{\max(n, m)} (4 - k) p'_k + \sum_{k=3}^{\max(n, m)} (4 - k) v'_k \\
      ={}& \sum_{k=3}^n (4 - k) p_k + \sum_{k=3}^m (4 - k) v_k = 4 \chi
    \end{align*}
    as well as \eqref{eq:handshake}
    \begin{align*}
      \sum_{k = 3}^n k (p_k + v_k) = \sum_{k = 3}^n k p_k + \sum_{k = 3}^n k v_k = 4e = \sum_{k = 3} k v'_k
    \end{align*}
    if we set $v'_4 \defeq e$. Hence the sequences $p'$, $v'$ are $[4]$-$[4]$-realizable by \autoref{thm:eberhard:extended:4} as a polyhedral map on $S$. Let $M'$ be this realization with $e'$ edges and $v'$ vertices. We use $M'$ to create a realization with valence $5$ by inserting additional triangles and quadrangles. This is done by replacing each edge of the realization by four triangles and a single quadrangle while replacing each vertex by eight triangles and five squares as shown in \autoref{fig:case34:img1}. Since $S$ is orientable we can do this such that the resulting map is $5$-valent.

    \begin{tikzfigure}{\label{fig:case34:img1}}{Building a $5$-valent polyhedral map out of a $4$-valent one by adding triangles and quadrangles \todo{Finish this!}}
      \matrix (m) [column sep=1cm] {
        \begin{scope}[scale=0.5]
          \filldraw[fill=gray!50!white] (-3.5, -0.5) -- (-3, 0) -- (-1, 0) -- (-1, -2) -- (-1.5, -2.5);
          \filldraw[fill=gray!50!white] (3.5, -0.5) -- (3, 0) -- (1, 0) -- (1, -2) -- (1.5, -2.5);
          \filldraw[fill=gray!50!white] (3.5, 0.5) -- (3, 0) -- (1, 0) -- (1, 2) -- (1.5, 2.5);
          \filldraw[fill=gray!50!white] (-3.5, 0.5) -- (-3, 0) -- (-1, 0) -- (-1, 2) -- (-1.5, 2.5);
          \filldraw[fill=gray!50!white] (-0.5, -2.5) -- (-1, -2) -- (-1, 0) -- (1, 0) -- (1, -2) -- (0.5, -2.5);
          \filldraw[fill=gray!50!white] (-0.5, 2.5) -- (-1, 2) -- (-1, 0) -- (1, 0) -- (1, 2) -- (0.5, 2.5);

          \draw (-3.5, 0) -- (-3, 0);
          \draw (3.5, 0) -- (3, 0);
          \draw (-1, 2.5) -- (-1, 2);
          \draw (-1, -2.5) -- (-1, -2);
          \draw (1, 2.5) -- (1, 2);
          \draw (1, -2.5) -- (1, -2);

          \draw[very thick] (-3, 0) -- (3, 0);
          \draw[very thick] (-1, -2) -- (-1, 2);
          \draw[very thick] (1, -2) -- (1, 2);

          \fill[black] (-3,0)  circle(3pt);
          \fill[black] (-1,0)  circle(3pt);
          \fill[black] (1,0)   circle(3pt);
          \fill[black] (3,0)   circle(3pt);
          \fill[black] (1,2)   circle(3pt);
          \fill[black] (1,-2)  circle(3pt);
          \fill[black] (-1,2)  circle(3pt);
          \fill[black] (-1,-2) circle(3pt);

        \end{scope}
        &
        \begin{scope}[scale=0.5]
          \filldraw[fill=gray!50!white] (-9.5, -3.5) -- (-9, -3) -- (-7, -3) -- (-7, -5) -- (-7.5, -5.5);
          \filldraw[fill=gray!50!white] (9.5, -3.5) -- (9, -3) -- (7, -3) -- (7, -5) -- (7.5, -5.5);
          \filldraw[fill=gray!50!white] (9.5, 3.5) -- (9, 3) -- (7, 3) -- (7, 5) -- (7.5, 5.5);
          \filldraw[fill=gray!50!white] (-9.5, 3.5) -- (-9, 3) -- (-7, 3) -- (-7, 5) -- (-7.5, 5.5);
          \filldraw[fill=gray!50!white] (-0.5, -5.5) -- (-1, -5) -- (-1, -3) -- (1, -3) -- (1, -5) -- (0.5, -5.5);
          \filldraw[fill=gray!50!white] (-0.5, 5.5) -- (-1, 5) -- (-1, 3) -- (1, 3) -- (1, 5) -- (0.5, 5.5);

          %\filldraw[fill=gray!75!white] (-5, -1) -- (-3, 1) -- (-3, 3) -- (-1, 1) -- (1, 1) -- (-1, -1) -- (-1, -3) -- (-3, -1) -- cycle;

          \draw (-9, -3) -- (9, -3);
          \draw (-9, -1) -- (9, -1);
          \draw (-9, 1) -- (9, 1);
          \draw (-9, 3) -- (9, 3);

          \draw (-7, -5) -- (-7, 5);
          \draw (-5, -5) -- (-5, 5);
          \draw (-3, -5) -- (-3, 5);
          \draw (-1, -5) -- (-1, 5);
          \draw ( 1, -5) -- ( 1, 5);
          \draw ( 3, -5) -- ( 3, 5);
          \draw ( 5, -5) -- ( 5, 5);
          \draw ( 7, -5) -- ( 7, 5);

          \draw (-9, -3) -- (-7, -1);
          \draw (-9,  1) -- (-7,  3);
          \draw (-7, -3) -- (-5, -5);
          \draw (-7,  1) -- (-5, -1);
          \draw (-7,  5) -- (-5,  3);
          \draw (-5, -3) -- (-3, -1);
          \draw (-5,  1) -- (-3,  3);
          \draw (-3, -3) -- (-1, -5);
          \draw (-3,  1) -- (-1, -1);
          \draw (-3,  5) -- (-1,  3);
          \draw (-1, -3) -- ( 1, -1);
          \draw (-1,  1) -- ( 1,  3);
          \draw ( 1, -3) -- ( 3, -5);
          \draw ( 1,  1) -- ( 3, -1);
          \draw ( 1,  5) -- ( 3,  3);
          \draw ( 3, -3) -- ( 5, -1);
          \draw ( 3,  1) -- ( 5,  3);
          \draw ( 5, -3) -- ( 7, -5);
          \draw ( 5,  1) -- ( 7, -1);
          \draw ( 5,  5) -- ( 7,  3);
          \draw ( 7, -3) -- ( 9, -1);
          \draw ( 7,  1) -- ( 9,  3);

          \draw[very thick] (-9, -3) -- (-9,  3);
          \draw[very thick] ( 9, -3) -- ( 9,  3);
          \draw[very thick] (-7, -5) -- (-1, -5);
          \draw[very thick] ( 7, -5) -- ( 1, -5);
          \draw[very thick] (-7,  5) -- (-1,  5);
          \draw[very thick] ( 7,  5) -- ( 1,  5);

          \draw[very thick] ( 7,  3) -- ( 1,  3);
          \draw[very thick] ( 7, -3) -- ( 1, -3);
          \draw[very thick] ( 7,  3) -- ( 7, -3);
          \draw[very thick] ( 1,  3) -- ( 1, -3);
          \draw[very thick] (-7,  3) -- (-1,  3);
          \draw[very thick] (-7, -3) -- (-1, -3);
          \draw[very thick] (-7,  3) -- (-7, -3);
          \draw[very thick] (-1,  3) -- (-1, -3);

          % \draw[very thick] (-5, -1) -- (-5, 1); \draw[very thick] (5, -1) -- (5, 1);
          % \draw[very thick] (-3, -3) -- (-1, -3); \draw[very thick] (1, -3) -- (3, -3);
          % \draw[very thick] (-3, 3) -- (-1, 3); \draw[very thick] (1, 3) -- (3, 3);
          \draw (-7, -5.5) -- (-7, -5);
          \draw[very thick] (-7, -5) -- (-6.5, -5.5);
          \draw (-5, -5.5) -- (-5, -5);
          \draw (-3.5, -5.5) -- (-3, -5) -- (-3, -5.5);
          \draw[very thick] (-1, -5.5) -- (-1, -5);
          \draw ( 1, -5.5) -- ( 1, -5);
          \draw[very thick] (1, -5) -- (1.5, -5.5);
          \draw ( 3, -5.5) -- ( 3, -5);
          \draw ( 4.5, -5.5) -- (5, -5) -- (5, -5.5);
          \draw[very thick] ( 7, -5.5) -- ( 7, -5);

          \foreach \x in {-7,-5,-3,-1,1,3,5,7}
          \foreach \y in {-5,5}  
          \fill[black] (\x,\y) circle(3pt);

          \foreach \x in {-9,-7,-5,-3,-1,1,3,5,7,9}
          \foreach \y in {-3,-1,1,3}  
          \fill[black] (\x,\y) circle(3pt);

        \end{scope}
        \\
        };
    \end{tikzfigure}
    The resulting map $M''$ is $5$-valent and has a $p$-vector of $p' + [c_1 \times 3, c_2 \times 4]$ (for some $c_1, c_2 \in \nats$ we do not want to state explicitly), but lacks lacks the specified vertices of the original $v$-vector $v$. To fix this we take for $k$-valent entry in $v$ a different $k$-gon of $M''$ and replace the $k$-gon together with the ring of triangles and quadrangles as in \autoref{fig:case34:img2}.
    \begin{tikzfigure}{\label{fig:case34:img2}}{Replacing a $k$-gon to get a $k$-valent vertex}
      \matrix (m) [ column sep=1cm] {
        \begin{scope}[scale=0.9, xscale=-1]
          \fill[fill=gray!50!white](  0 :   1) -- ( 72 :   1) -- (144 :   1) -- (216 :   1) -- (288 :   1) -- (  0 :   1);
          \draw (  0.0 : 1.000) -- ( 72.0 : 1.000) -- (144.0 : 1.000) -- (216.0 : 1.000) -- (288 :   1) -- (  0 :   1);
          \draw (  0.0 : 3.000) -- ( 72.0 : 3.000) -- (144.0 : 3.000) -- (216.0 : 3.000) -- (288 :   3) -- (  0 :   3);
          \draw (  0.0 : 1.000) -- (337.6 : 2.497) -- (288.0 : 1.000) -- (310.4 : 2.497);
          \draw ( 72.0 : 1.000) -- ( 49.6 : 2.497) -- (  0.0 : 1.000) -- ( 22.4 : 2.497);
          \draw (144.0 : 1.000) -- (121.6 : 2.497) -- ( 72.0 : 1.000) -- ( 94.4 : 2.497);
          \draw (216.0 : 1.000) -- (193.6 : 2.497) -- (144.0 : 1.000) -- (166.4 : 2.497);
          \draw (288.0 : 1.000) -- (265.6 : 2.497) -- (216.0 : 1.000) -- (238.4 : 2.497);

          \foreach \x in {0,72,144,216,288}
          \foreach \y in {1,3}  
          \fill[black] (\x : \y) circle(2pt);

          \foreach \x in {22.4, 94.4,166.4,238.4,310.4,337.6,49.6,121.6,193.6,265.6}
          \fill[black] (\x : 2.497) circle(2pt);

        \end{scope}
        &
        \begin{scope}[scale=0.9, xscale=-1]
          \draw (  0.0 : 3.000) -- ( 72.0 : 3.000) -- (144.0 : 3.000) -- (216.0 : 3.000) -- (288 :   3) -- (  0 :   3);
          \draw (  0.0 : 0.000) -- (337.6 : 2.497) -- ( 22.4 : 2.497);
          \draw ( 72.0 : 0.000) -- ( 49.6 : 2.497) -- ( 94.4 : 2.497);
          \draw (144.0 : 0.000) -- (121.6 : 2.497) -- (166.4 : 2.497);
          \draw (216.0 : 0.000) -- (193.6 : 2.497) -- (238.4 : 2.497);
          \draw (288.0 : 0.000) -- (265.6 : 2.497) -- (310.4 : 2.497);

          \fill[black] (0,0) circle(2pt);
          \foreach \x in {0,72,144,216,288}
          \fill[black] (\x : 3) circle(2pt);

          \foreach \x in {22.4, 94.4, 166.4, 238.4, 310.4, 337.6, 49.6, 121.6, 193.6, 265.6}
          \fill[black] (\x : 2.497) circle(2pt);

        \end{scope}
        \\
      };
    \end{tikzfigure}
    After this construction we have held a map with $p$-vector $p'' \defeq p' + [c'_3 \times 3, c'_4 \times 4] - v = p + [c'_3 \times 3, c'_4 \times 4]$ and with $v$-vector $v'' \defeq v + d \cdot [5]$ (for some $c'_3, c'_4, d \in \nats$) since we only inserted additional triangles, quadrangles and $5$-valent vertices while removing $k$-gons and inserting $k$-valent vertices as specified by $v$. By \autoref{lem:2:valued:eberhard}, this suffices to show the claim.
  \end{proof}
\end{proposition}

Our next corner stone is to revisit \autoref{thm:main:const} for the $5$-valent case:
\begin{proposition}\label{thm:main:const:5} Let $r = 5$, $p$ and $v$ be an admissible pair of sequences for which we can apply \autoref{thm:eberhard:extended:3} (or \autoref{thm:eberhard:extended:4}). Let $w = [r]$ and $q = [q_3 \times 3, q_l \times l]$, where $l \geq 4$, $q_s = 3l - 10$ and $q_l = 1$. Assume there exist
  \begin{itemize}
  \item an expansion $r$-patch $\mathcal{P}_N$ with outer arc $o$ consisting of triangles and $l$-gons,
  \item an $o$-$4$-gonal $r$-patch $\mathcal{P}_F$ consisting of triangles and $l$-gons,
  \item an expansion $r$-patch $\mathcal{P}_P$ with the polyhedral property consisting of triangles and $l$-gons.
  \end{itemize}
  Then $(p, v)$ is $q$-$[5]$-realizable.
  \begin{proof}
    Analogous to \autoref{thm:main:const}.
  \end{proof}
\end{proposition}

\begin{construction}\label{const:edge:replacement:3:5:1}
  \begin{cinput}
  \item A $3$-patch with $p$-vector $p$ and a specified edge with exactly one vertex incident to some $k_1$-gon and the other vertex incident to some $k_2$-gon.
  \end{cinput}
  \begin{coutput}
  \item A new $3$-patch with $p$-vector $p - [k_1, k_2] + [18k \times 3] + [k_1 + 3k, k_2 + 3k]$ for all $k \in \nats$.
  \item If every two faces of the $3$-patch meet proper, then this is carried over to the new patch.
  \end{coutput}
  \begin{cdescription}
    Using the replacement of the single edge as seen in \autoref{fig:const:edge:replacement:3:5:1} results in a new map with $p$-vector $p - [k_1, k_2] + [18 \times 3] + [k_1 + 3, k_2 + 3]$. The thick line on the left is the specified edge and the thick line on the right is a new edge which we can use to repeat the construction. Every time we use this construction we add 18 new triangles while increasing the number of vertices of the left and the right polygon by three; doing this $k$ times has the desired result, since we only inserted $5$-valent vertices. That all faces meet proper follows by induction, as this property is preserved in each step.
    \begin{tikzfigure}{\label{fig:const:edge:replacement:3:5:1}}{}
      \matrix (m) [column sep=1cm, row sep=1cm] {
        \begin{scope}
          \draw[lsquare] (-1, 0) -- (1, 0);
          \draw (-1.2, 0.5) -- (-1, 0) -- (-1.2, -0.5);
          \draw (1.2, 0.5) -- (1, 0) -- (1.2, -0.5);
          \draw (-0.8, 0.5) -- (-1, 0) -- (-0.8, -0.5);
          \draw (0.8, 0.5) -- (1, 0) -- (0.8, -0.5);
          \node (k1) at (-2, 0) {$k_1$};
          \node (k2) at (2, 0) {$k_2$};
          \draw[lface] (-1, 0) -- (k1);
          \draw[lface] ( 1, 0) -- (k2);
          
          \fill[black] (-1,0) circle(2pt);
          \fill[black] (1,0) circle(2pt);
        \end{scope}
        &
        \begin{scope}
          \draw[lsquare] (-1, 1.5) -- (1, 1.5);
          \draw (-1, -1.5) -- (1, -1.5);
          \draw (-1.2, 2) -- (-1, 1.5);
          \draw (-1.2, -2) -- (-1, -1.5);
          \draw (1.2, 2) -- (1, 1.5);
          \draw (1.2, -2) -- (1, -1.5);
          \draw (-0.8, 2) -- (-1, 1.5);
          \draw (-0.8, -2) -- (-1, -1.5);
          \draw (0.8, 2) -- (1, 1.5);
          \draw (0.8, -2) -- (1, -1.5);
          \draw (1, 1.5) -- (0.8, 0.5) -- (0.8, -0.5) -- (1, -1.5);
          \draw (-1, 1.5) -- (-0.8, 0.5) -- (-0.8, -0.5) -- (-1, -1.5);
          \draw (-1, -1.5) -- (0, -1) -- (1, -1.5);
          \draw (-1, 1.5) -- (0, 1) -- (1, 1.5);
          \draw (-0.8, -0.5) -- (0, -1) -- (0.8, -0.5);
          \draw (-0.8, 0.5) -- (0, 1) -- (0.8, 0.5);
          \draw (-0.8, -0.5) -- (0, -0.5) -- (0.8, -0.5);
          \draw (-0.8, 0.5) -- (0, 0.5) -- (0.8, 0.5);
          \draw (-0.8, -0.5) -- (-0.4, 0) -- (0, -0.5) -- (0.4, 0) -- (0.8, -0.5);
          \draw (-0.8, 0.5) -- (-0.4, 0) -- (0, 0.5) -- (0.4, 0) -- (0.8, 0.5);
          \draw (-0.4, 0) -- (0.4, 0);
          \draw (0, -1) -- (0, -0.5);
          \draw (0, 1) -- (0, 0.5);
          
          
          \node (k1) at (-1.8, 0) {$k_1 + 3$};
          \node (k2) at (1.8, 0) {$k_2 + 3$};
          \draw[lface] (-1, 1.5) -- (k1);
          \draw[lface] ( 1, 1.5) -- (k2);
          
          \fill[black] (0,-1) circle (2pt);
          \fill[black] (0,1) circle (2pt);
          \fill[black] (0,-0.5) circle (2pt);
          \fill[black] (0,0.5) circle (2pt);
          \fill[black] (-0.8,0.5) circle (2pt);
          \fill[black] (-0.8,-0.5) circle (2pt);          
          \fill[black] (-0.4,0) circle (2pt);
          \fill[black] (0.4,0) circle (2pt);
          \fill[black] (0.8,0.5) circle (2pt);
          \fill[black] (0.8,-0.5) circle (2pt);          
          \fill[black] (-1,1.5) circle (2pt);
          \fill[black] (1,-1.5) circle (2pt);
          \fill[black] (1,1.5) circle (2pt);
          \fill[black] (-1,-1.5) circle (2pt);          

        \end{scope}
        \\
        \begin{scope}
          \draw[lsquare] (-1, 0) -- (1, 0);
          \draw (-1.2, 0.5) -- (-1, 0) -- (-1.2, -0.5);
          \draw (1.2, 0.5) -- (1, 0) -- (1.2, -0.5);
          \draw (-0.8, 0.5) -- (-1, 0) -- (-1, 0.5);
          \draw (1, -0.5) -- (1, 0) -- (0.8, -0.5);
          \node (k1) at (-2, 0) {$k_1$};
          \node (k2) at (2, 0) {$k_2$};
          \draw[lface] (-1, 0) -- (k1);
          \draw[lface] ( 1, 0) -- (k2);
          
          \fill[black] (-1,0) circle(2pt);
          \fill[black] (1,0) circle(2pt);
        \end{scope}
        &
        \begin{scope}
          \draw[lsquare] (-1, 1.5) -- (1, 1.5);
          \draw (-1, -1.5) -- (1, -1.5);
          \draw (-1.2, 2) -- (-1, 1.5);
          \draw (-1.2, -2) -- (-1, -1.5);
          \draw (1.2, 2) -- (1, 1.5);
          \draw (1.2, -2) -- (1, -1.5);
          \draw (1.0, -2) -- (1, -1.5);
          \draw (-1, 2) -- (-1, 1.5);
          \draw (-0.8, 2) -- (-1, 1.5);
          \draw (0.8, -2) -- (1, -1.5);
          \draw (1, 1.5) -- (0.8, 0.5) -- (0.8, -0.5) -- (1, -1.5);
          \draw (-1, 1.5) -- (-0.8, 0.5) -- (-0.8, -0.5) -- (-1, -1.5);

          \draw (-1, -1.5) -- (-0.3, -0.75) -- (0.8,  -0.5) -- (0.3, -0.25) -- (-0.3, -0.25) -- (-0.8, -0.5) -- (-0.3, 0.25) -- (0.3, 0.25) -- (0.8, 0.5) -- (0.3, 0.75) -- (-0.8, 0.5) -- (1, 1.5);
          \draw (-1, -1.5) -- (0.8, -0.5);
          \draw (-0.3, -0.75) -- (-0.3, -0.25) -- (0.3, 0.25) -- (0.3, -0.25) -- (-0.3, -0.75) -- (-0.8, -0.5);
          \draw (-0.3, -0.25) -- (-0.3, 0.25) -- (0.3, 0.75) -- (0.3, 0.25);
          \draw (-0.3,  0.25) -- (-0.8,  0.5);
          \draw ( 0.3, -0.25) -- ( 0.8,  0.5);
          \draw ( 0.3,  0.75) -- ( 1.0,  1.5);
          
          \node (k1) at (-1.8, 0) {$k_1 + 3$};
          \node (k2) at (1.8, 0) {$k_2 + 3$};
          \draw[lface] (-1, 1.5) -- (k1);
          \draw[lface] ( 1, 1.5) -- (k2);
          
          \fill[black] (-0.8,0.5) circle (2pt);
          \fill[black] (-0.8,-0.5) circle (2pt);          
          \fill[black] (0.8,0.5) circle (2pt);
          \fill[black] (0.8,-0.5) circle (2pt);          
          \fill[black] (-1,1.5) circle (2pt);
          \fill[black] (1,-1.5) circle (2pt);
          \fill[black] (1,1.5) circle (2pt);
          \fill[black] (-1,-1.5) circle (2pt);          

          \fill[black] (-0.3,-0.75) circle (2pt);
          \fill[black] (-0.3,-0.25) circle (2pt);
          \fill[black] (-0.3, 0.25) circle (2pt);
          \fill[black] ( 0.3, 0.75) circle (2pt);
          \fill[black] ( 0.3, 0.25) circle (2pt);
          \fill[black] ( 0.3,-0.25) circle (2pt);

        \end{scope}
        \\
      };
    \end{tikzfigure}
  \end{cdescription}
\end{construction}

\begin{construction}\label{const:edge:replacement:3:5:2}
  \begin{cinput}
  \item A $3$-patch with $p$-vector $p$ and a specified edge adjacent to both some $k_1$-gon and some $k_2$-gon.
  \end{cinput}
  \begin{coutput}
  \item A new $3$-patch with $p$-vector $p - [k_1, k_2] + [18k \times 3] + [k_1 + 3k, k_2 + 3k]$ for all $k \in \nats$.
  \end{coutput}
  \begin{cdescription}
    Using the replacement of the single edge as seen in \autoref{fig:const:edge:replacement:3:5:1} results in a new map with $p$-vector $p - [k_1, k_2] + [18 \times 3] + [k_1 + 3, k_2 + 3]$. The thick line on the left is the specified edge and the thick line on the right is a new edge which we can use to repeat the construction. Every time we use this construction we add 18 new triangles while increasing the number of vertices of the left and the right polygon by three; doing this $k$ times has the desired result, since we only inserted $5$-valent vertices.
    \begin{tikzfigure}{\label{fig:const:edge:replacement:3:5:2}}{}
      \matrix (m) [column sep=1cm] {
        \begin{scope}
          \draw[ldiamond] (0, -1) -- (0, 1);
          \draw (0.5, -1.2) -- (0, -1) -- (-0.5, -1.2);
          \draw (0.5,  1.2) -- (0,  1) -- (-0.5,  1.2);
          \draw (0.5, -0.8) -- (0, -1) -- (-0.5, -0.8);
          \draw (0.5,  0.8) -- (0,  1) -- (-0.5,  0.8);
          \node (n1) at (-1.5, 0) {$k_1$};
          \node (n2) at (1.5, 0) {$k_2$};

          \draw[lface] (0,0)--(n1);
          \draw[lface] (0,0)--(n2);
          \fill[black] (0, 1) circle(2pt);
          \fill[black] (0,-1) circle(2pt);

        \end{scope}
        &
        \begin{scope}[scale=1]
          \draw[ldiamond] (0, 1.75) -- (0, 2.5);
          \draw (0.5, -2.2) -- (0, -2) -- (-0.5, -2.2);
          \draw (0.5,  2.7) -- (0,  2.5) -- (-0.5,  2.7);
          \draw (0.5, -1.8) -- (0, -2) -- (-0.5, -1.8);
          \draw (0.5,  2.3) -- (0,  2.5) -- (-0.5,  2.3);

          \draw (0, -1.75) -- (-1, 0) -- (0, 1.75) -- (1, 0) -- (0, -1.75) -- (-0.2, -0.75) -- (-0.5, 0) -- (-0.2, 0.75) -- (0, 1.75) -- (0.2, 0.75) -- (0.5, 0) -- (0.2, -0.75) -- (0, -1.75) -- (0, -2);
          \draw (-0.2, -0.75) -- (0, -0.5) -- (0.2, -0.75) -- (-0.2, -0.75);
          \draw (-0.2,  0.75) -- (0,  0.5) -- (0.2,  0.75) -- (-0.2,  0.75);
          \draw (-1, 0) -- (-0.5, 0) -- (0, -0.5) -- (0.5, 0) -- (1, 0);
          \draw (-0.5, 0) -- (0, 0.5) -- (0.5, 0);
          \draw (0, -0.5) -- (0, 0.5);
          \draw (-0.2, -0.75) -- (-1, 0) -- (-0.2, 0.75);
          \draw ( 0.2, -0.75) -- ( 1, 0) -- ( 0.2, 0.75);
          \node (n1) at (-1.8, 0) {$k_1 + 3$};
          \node (n2) at (1.8, 0) {$k_2 + 3$};
          %\node at (0, 0) {$4$};

          \draw[lface] (0,2.125)--(n1);
          \draw[lface] (0,2.125)--(n2);

          \fill[black] (-1,0) circle(2pt);
          \fill[black] ( 1,0) circle(2pt);
          \fill[black] ( 0,-1.75) circle(2pt);
          \fill[black] (0,1.75) circle(2pt);
          \fill[black] (0.2,0.75) circle(2pt);
          \fill[black] (-0.2,0.75) circle(2pt);
          \fill[black] (-0.2,-0.75) circle(2pt);
          \fill[black] (0.2,-0.75) circle(2pt);
          \fill[black] (0,2.5) circle(2pt);
          \fill[black] (0,-2) circle(2pt);
          \fill[black] (0.5,0) circle(2pt);
          \fill[black] (-0.5,0) circle(2pt);
          \fill[black] (0,-0.5) circle(2pt);
          \fill[black] (0,0.5) circle(2pt);

        \end{scope}
        \\
      };
    \end{tikzfigure}
  \end{cdescription}
\end{construction}

\begin{construction}\label{const:edge:replacement:3:5:3}
  \begin{cinput}
  \item A $3$-patch with $p$-vector $p$ and a specified edge adjacent to both some $k_1$-gon and some $k_2$-gon.
  \end{cinput}
  \begin{coutput}
  \item A new $3$-patch with $p$-vector $p - [k_1, k_2] + [18k \times 3] + [k_1 + 3k, k_2 + 3k]$ for all $k \in \nats$.
  \end{coutput}
  \begin{cdescription}
    Using the replacement of the single edge as seen in \autoref{fig:const:edge:replacement:3:5:1} results in a new map with $p$-vector $p - [k_1, k_2] + [18 \times 3] + [k_1 + 3, k_2 + 3]$. The thick line on the left is the specified edge and the thick line on the right is a new edge which we can use to repeat the construction. Every time we use this construction we add 18 new triangles while increasing the number of vertices of the left and the right polygon by three; doing this $k$ times has the desired result, since we only inserted $5$-valent vertices.
    \begin{tikzfigure}{\label{fig:const:edge:replacement:3:5:3}}{}
      \matrix (m) [column sep=1cm] {
        \begin{scope}

          \draw (-0.5, 0.7) -- (0, 0) -- (0.5, 0.7);
          \draw (-0.5, -0.7) -- (0, 0) -- (0.5, -0.7);
          \draw (0,0) -- (0, -0.7);

          \fill[black] (0,0) circle(2pt);
          \node (n1) at (-1.5, 0) {$k_1$};
          \node (n2) at (1.5, 0) {$k_2$};
          \node[lvertex](m) at (0,0){};

          \draw[lface] (m)--(n1);
          \draw[lface] (m)--(n2);
        \end{scope}
        &
        \begin{scope}[scale=0.8]
          
          \fill[black] (0,3.5) circle(2pt);
          \fill[black] (-1.5,1.5) circle(2pt);
          \fill[black] (1.5,1.5) circle(2pt);
          \fill[black] (1.5,-1.5) circle(2pt);
          \fill[black] (-1.5,-1.5) circle(2pt);
          \fill[black] (-1,0) circle(2pt);
          \fill[black] (1,0) circle(2pt);
          \fill[black] (0,2) circle(2pt);
          \fill[black] (-0.5,1) circle(2pt);
          \fill[black] (0.5,1) circle(2pt);      
          \fill[black] (0,0) circle(2pt);
          \fill[black] (0,-1) circle(2pt);
          \fill[black] (0,-3) circle(2pt);

          \draw (-1,-3.5) -- (0, -3) -- (-1.5, -1.5)--(-1.5, 1.5)--(0,3.5)--(-1,4);
          \draw (1,-3.5) -- (0, -3) -- (1.5, -1.5)--(1.5, 1.5)--(0,3.5);
          \draw (0,3.5)--(1,4);
          \draw (0,-3)--(0,-3.5);

          \draw (0, -1) -- (-1.5,-1.5) -- (-1,0)--(-1.5,1.5)--(0,2)--(-0.5,1)--(-1,0)--(0,-1);
          \draw (0, -1) -- (1.5,-1.5) -- (1,0)--(1.5,1.5)--(0,2)--(0.5,1)--(1,0)--(0,-1);
          
          \draw (-1.0,0) -- (0, 0) -- (1,0);
          \draw (-1.5,1.5)--(-0.5, 1) -- (0,0) -- (0.5,1)--(1.5,1.5);
          \draw (0,0) -- (0,-1);
          \draw (-0.5,1) -- (0.5,1);
          \draw (0,2) -- (0,3.5);
          \draw (-1.5,-1.5) -- (1.5,-1.5);

          \node (n1) at (-2.5, 1) {$k_1$};
          \node (n2) at (2.5, 1) {$k_2$};
          \node[lvertex] (m)  at (0,3.5){}; 
          
          \draw[lface](m)--(n1);
          \draw[lface](m)--(n2);

          \node at (-0.5,2.4) {$3$};
          \node at (0.5, 2.4) {$3$};
          \node at (-0.75, 1.5) {$3$};
          \node at (0.75, 1.5) {$3$};
          \node at (0, 1.25) {$3$};
          \node at (0, 0.66) {$3$};
          \node at (-0.5, 0.5) {$3$};
          \node at (0.5, 0.5) {$3$};
          \node at (-0.25,-0.4) {$3$};
          \node at (0.25, -0.4) {$3$};
          \node at (-0.75, -1) {$3$};
          \node at (0.75, -1) {$3$};
          \node at (0, -1.25) {$3$};
          \node at (0, -2.25) {$3$};
          \node at (-1.25,0) {$3$};
          \node at (1.25, 0) {$3$};
          \node at (-1,0.75) {$3$};
          \node at (1, 0.75) {$3$};

        \end{scope}
        \\
      };
    \end{tikzfigure}
  \end{cdescription}
\end{construction}
\clearpage
\begin{theorem}
  Let $p = (p_3, p_4, \dots, p_n)$, $v = (v_3, v_4, \dots, v_m)$ be a pair of admissible sequences for a closed orientable $2$-manifold $S$ with {\sc Euler} characteristic $\chi$, i.e. \eqref{eq:vp:5} and \eqref{eq:handshake} hold for some $e \in \nats$. Then $(p, v)$ is $[(9k + 2) \times 3, 3k + 4]$-$[5]$-realizable for all $k \in \nats$.
  \begin{proof}
    Using \autoref{const:edge:replacement:3:5:1} on the edges in both $5$-patches in \autoref{fig:expansion:patch:3:4:5} as indicated results in $5$-patches consisting only of triangles and $n$-gons with the specified amount of vertices. The left patch is an expansion patch with outer tuple $o$, while the right patch is $o$-$6$-gonal, these properties do not change when using \autoref{const:edge:replacement:5:1} and \autoref{const:edge:replacement:5:2} on these patches. In fact, the resulting patch on the left is even polyhedral, so we can apply \autoref{thm:main:const:5}.
  \end{proof}
\end{theorem}
\begin{tikzfigure2}{}
  \begin{tikzsubfigure}{\label{fig:expansion:patch:3:4:5}}{\todo{better picture}}{0.5}
    \begin{scope}[scale=0.9, yscale=0.866]
      \draw (-0.5, 1) -- (-4.5, 5) -- (-3.5, 5.666) -- (-3.5, 7) -- (-2.5, 7.666) -- (-1.5, 7) -- (-0.5, 6.666) -- (0.5, 7.333) -- (1.5, 7) -- (0.5, 1) -- (-0.5, 1);
      \draw[lsquare] (-0.5, 1) -- (-3.5, 5.666) -- (-0.5, 6.666) -- (-0.5, 1);
      \draw (-3.5, 5.666) -- (-1.5, 7);
      \draw (-0.5, 6.666) -- (1.5, 7);

      \node[anchor= 90] at (-0.5, 1)     {$i_{0}$};
      \node[anchor=  0] at (-4.5, 5)     {$i_{1}=o_{0}$};
      \node[anchor=330] at (-3.5, 5.666) {$o_{1}$};
      \node[anchor=  0] at (-3.5, 7)     {$o_{2}$};
      \node[anchor=270] at (-2.5, 7.666) {$o_{3}$};
      \node[anchor=270] at (-1.5, 7)     {$o_{4}$};
      \node[anchor=270] at (-0.5, 6.666) {$o_{5}$};
      \node[anchor=270] at (0.5, 7.333)  {$o_{6}$};
      \node[anchor=220] at (1.5, 7)      {$i_{1}'=o_{7}$};
      \node[anchor= 90] at (0.5, 1)      {$i_{0}'$};

      \fill[black] (-0.5, 1)     circle(2pt);
      \fill[black] (-4.5, 5)     circle(2pt);
      \fill[black] (-3.5, 5.666) circle(2pt);
      \fill[black] (-3.5, 7)     circle(2pt);
      \fill[black] (-2.5, 7.666) circle(2pt);
      \fill[black] (-1.5, 7)     circle(2pt);
      \fill[black] (-0.5, 6.666) circle(2pt);
      \fill[black] (0.5, 7.333)  circle(2pt);
      \fill[black] (1.5, 7)      circle(2pt);
      \fill[black] (0.5, 1)      circle(2pt);

      \node at (-3.5,5) {$3$};
      \node at (0.5,7.05) {$3$};
      \node at (-1.5,6.6) {$3$};
      \node at (-1.5,5) {$3$};
      \node at (0.5,4) {$4$};
      \node at (-2.5,7) {$4$};

    \end{scope}
  \end{tikzsubfigure}~
  \begin{tikzsubfigure}{}{}{0.5}
    \begin{scope}[scale=0.5]
      \begin{scope}[yscale=0.866]
        \draw[very thick] (-0.5, 1) -- (-4.5, 5) -- (-3.5, 5.666) -- (-3.5, 7) -- (-2.5, 7.666) -- (-1.5, 7) -- (-0.5, 6.666) -- (0.5, 7.333) -- (1.5, 7) -- (0.5, 1) -- (-0.5, 1);
        \draw (-0.5, 1) -- (-3.5, 5.666) -- (-0.5, 6.666) -- (-0.5, 1);
        \draw (-3.5, 5.666) -- (-1.5, 7);
        \draw (-0.5, 6.666) -- (1.5, 7);

        \fill[black] (-0.5, 1)     circle(3pt);
        \fill[black] (-4.5, 5)     circle(3pt);
        \fill[black] (-3.5, 5.666) circle(3pt);
        \fill[black] (-3.5, 7)     circle(3pt);
        \fill[black] (-2.5, 7.666) circle(3pt);
        \fill[black] (-1.5, 7)     circle(3pt);
        \fill[black] (-0.5, 6.666) circle(3pt);
        \fill[black] (0.5, 7.333)  circle(3pt);
        \fill[black] (1.5, 7)      circle(3pt);
        \fill[black] (0.5, 1)      circle(3pt);

      \end{scope}
      \begin{scope}[rotate=-60, yscale=0.866]
        \draw[very thick] (-0.5, 1) -- (-4.5, 5) -- (-3.5, 5.666) -- (-3.5, 7) -- (-2.5, 7.666) -- (-1.5, 7) -- (-0.5, 6.666) -- (0.5, 7.333) -- (1.5, 7) -- (0.5, 1) -- (-0.5, 1);
        \draw (-0.5, 1) -- (-3.5, 5.666) -- (-0.5, 6.666) -- (-0.5, 1);
        \draw (-3.5, 5.666) -- (-1.5, 7);
        \draw (-0.5, 6.666) -- (1.5, 7);

        \fill[black] (-0.5, 1)     circle(3pt);
        \fill[black] (-4.5, 5)     circle(3pt);
        \fill[black] (-3.5, 5.666) circle(3pt);
        \fill[black] (-3.5, 7)     circle(3pt);
        \fill[black] (-2.5, 7.666) circle(3pt);
        \fill[black] (-1.5, 7)     circle(3pt);
        \fill[black] (-0.5, 6.666) circle(3pt);
        \fill[black] (0.5, 7.333)  circle(3pt);
        \fill[black] (1.5, 7)      circle(3pt);
        \fill[black] (0.5, 1)      circle(3pt);

      \end{scope}
      \begin{scope}[yscale=0.866, shift={(0 cm,14 cm)}, rotate=180]
        \draw[very thick] (-0.5, 1) -- (-4.5, 5) -- (-3.5, 5.666) -- (-3.5, 7) -- (-2.5, 7.666) -- (-1.5, 7) -- (-0.5, 6.666) -- (0.5, 7.333) -- (1.5, 7) -- (0.5, 1) -- (-0.5, 1);
        \draw (-0.5, 1) -- (-3.5, 5.666) -- (-0.5, 6.666) -- (-0.5, 1);
        \draw (-3.5, 5.666) -- (-1.5, 7);
        \draw (-0.5, 6.666) -- (1.5, 7);

        \fill[black] (-0.5, 1)     circle(3pt);
        \fill[black] (-4.5, 5)     circle(3pt);
        \fill[black] (-3.5, 5.666) circle(3pt);
        \fill[black] (-3.5, 7)     circle(3pt);
        \fill[black] (-2.5, 7.666) circle(3pt);
        \fill[black] (-1.5, 7)     circle(3pt);
        \fill[black] (-0.5, 6.666) circle(3pt);
        \fill[black] (0.5, 7.333)  circle(3pt);
        \fill[black] (1.5, 7)      circle(3pt);
        \fill[black] (0.5, 1)      circle(3pt);

      \end{scope}
      \begin{scope}[shift={(0 cm,12.124 cm)},rotate=120,yscale=0.866]
        \draw[very thick] (-0.5, 1) -- (-4.5, 5) -- (-3.5, 5.666) -- (-3.5, 7) -- (-2.5, 7.666) -- (-1.5, 7) -- (-0.5, 6.666) -- (0.5, 7.333) -- (1.5, 7) -- (0.5, 1) -- (-0.5, 1);
        \draw (-0.5, 1) -- (-3.5, 5.666) -- (-0.5, 6.666) -- (-0.5, 1);
        \draw (-3.5, 5.666) -- (-1.5, 7);
        \draw (-0.5, 6.666) -- (1.5, 7);

        \fill[black] (-0.5, 1)     circle(3pt);
        \fill[black] (-4.5, 5)     circle(3pt);
        \fill[black] (-3.5, 5.666) circle(3pt);
        \fill[black] (-3.5, 7)     circle(3pt);
        \fill[black] (-2.5, 7.666) circle(3pt);
        \fill[black] (-1.5, 7)     circle(3pt);
        \fill[black] (-0.5, 6.666) circle(3pt);
        \fill[black] (0.5, 7.333)  circle(3pt);
        \fill[black] (1.5, 7)      circle(3pt);
        \fill[black] (0.5, 1)      circle(3pt);

      \end{scope}
    \end{scope}
  \end{tikzsubfigure}
  \begin{tikzsubfigure}{}{}{1.0}
    \begin{scope}[scale=5]
      % \node (n0) at (-0.239864, 0.377966) {0};
% \node (n1) at (-0.003456, 0.583197) {1};
% \node (n2) at (0.108875, 0.831293) {2};
% \node (n3) at (-0.325154, 0.690268) {3};
% \node (n4) at (-0.710475, 0.445736) {4};
% \node (n5) at (-0.601183, 0.273364) {5};
% \node (n6) at (-0.748626, -0.123111) {6};
% \node (n7) at (0.178219, 0.178039) {7};
% \node (n8) at (0.430652, 0.625281) {8};
% \node (n9) at (-0.370171, 0.786654) {9};
% \node (n10) at (-0.366925, 0.877877) {10};
% \node (n11) at (-0.524604, 0.826644) {11};
% \node (n12) at (0.079154, -0.168211) {12};
% \node (n13) at (0.229642, -0.488014) {13};
% \node (n14) at (-0.162909, -0.853997) {14};
% \node (n15) at (-0.219154, -0.925891) {15};
% \node (n16) at (-0.061475, -0.977123) {16};
% \node (n17) at (0.761858, -0.418835) {17};
% \node (n18) at (0.812852, -0.494544) {18};
% \node (n19) at (0.910303, -0.360414) {19};

\node (n1) at (-0.003456, 0.583197) {3};
\node (n2) at (0.108875, 0.831293) {3};
\node (n3) at (-0.325154, 0.690268) {3};
\node (n4) at (-0.710475, 0.445736) {3};
\node (n5) at (-0.601183, 0.273364) {4};
\node (n6) at (-0.748626, -0.123111) {4};
\node (n7) at (0.178219, 0.178039) {4};
\node (n8) at (0.430652, 0.625281) {4};
\node (n9) at (-0.370171, 0.786654) {4};
\node (n10) at (-0.366925, 0.877877) {3};
\node (n11) at (-0.524604, 0.826644) {3};
\node (n12) at (0.079154, -0.168211) {4};
\node (n13) at (0.229642, -0.488014) {3};
\node (n14) at (-0.162909, -0.853997) {4};
\node (n15) at (-0.219154, -0.925891) {3};
\node (n16) at (-0.061475, -0.977123) {3};
\node (n17) at (0.761858, -0.418835) {4};
\node (n18) at (0.812852, -0.494544) {3};
\node (n19) at (0.910303, -0.360414) {3};

\fill[black] (0.425779, 0.904827) circle (0.3pt);
\fill[black] (-0.286535, 0.606764) circle (0.3pt);
\fill[black] (-0.149612, 0.238000) circle (0.3pt);
\fill[black] (0.187381, 0.982287) circle (0.3pt);
\fill[black] (-0.876307, 0.481754) circle (0.3pt);
\fill[black] (-0.968583, 0.248690) circle (0.3pt);
\fill[black] (-1.000000, 0.000000) circle (0.3pt);
\fill[black] (-0.968583, -0.248690) circle (0.3pt);
\fill[black] (-0.876307, -0.481754) circle (0.3pt);
\fill[black] (0.929776, 0.368125) circle (0.3pt);
\fill[black] (0.809017, 0.587785) circle (0.3pt);
\fill[black] (0.637424, 0.770513) circle (0.3pt);
\fill[black] (-0.062791, 0.998027) circle (0.3pt);
\fill[black] (-0.728969, 0.684547) circle (0.3pt);
\fill[black] (-0.309017, 0.951057) circle (0.3pt);
\fill[black] (-0.535827, 0.844328) circle (0.3pt);
\fill[black] (-0.728969, -0.684547) circle (0.3pt);
\fill[black] (0.992115, 0.125333) circle (0.3pt);
\fill[black] (0.425779, -0.904827) circle (0.3pt);
\fill[black] (-0.535827, -0.844328) circle (0.3pt);
\fill[black] (0.187381, -0.982287) circle (0.3pt);
\fill[black] (-0.309017, -0.951057) circle (0.3pt);
\fill[black] (-0.062791, -0.998027) circle (0.3pt);
\fill[black] (0.637424, -0.770513) circle (0.3pt);
\fill[black] (0.992115, -0.125333) circle (0.3pt);
\fill[black] (0.809017, -0.587785) circle (0.3pt);
\fill[black] (0.929776, -0.368125) circle (0.3pt);

\coordinate (x0) at (0.425779, 0.904827);
\coordinate (x1) at (-0.286535, 0.606764);
\coordinate (x2) at (-0.149612, 0.238000);
\coordinate (x3) at (0.187381, 0.982287);
\coordinate (x4) at (-0.876307, 0.481754);
\coordinate (x5) at (-0.968583, 0.248690);
\coordinate (x6) at (-1.000000, 0.000000);
\coordinate (x7) at (-0.968583, -0.248690);
\coordinate (x8) at (-0.876307, -0.481754);
\coordinate (x9) at (0.929776, 0.368125);
\coordinate (x10) at (0.809017, 0.587785);
\coordinate (x11) at (0.637424, 0.770513);
\coordinate (x12) at (-0.062791, 0.998027);
\coordinate (x13) at (-0.728969, 0.684547);
\coordinate (x14) at (-0.309017, 0.951057);
\coordinate (x15) at (-0.535827, 0.844328);
\coordinate (x16) at (-0.728969, -0.684547);
\coordinate (x17) at (0.992115, 0.125333);
\coordinate (x18) at (0.425779, -0.904827);
\coordinate (x19) at (-0.535827, -0.844328);
\coordinate (x20) at (0.187381, -0.982287);
\coordinate (x21) at (-0.309017, -0.951057);
\coordinate (x22) at (-0.062791, -0.998027);
\coordinate (x23) at (0.637424, -0.770513);
\coordinate (x24) at (0.992115, -0.125333);
\coordinate (x25) at (0.809017, -0.587785);
\coordinate (x26) at (0.929776, -0.368125);

\node[lvertex] at (x2) {};
\draw[lface] (x2) -- (n6);
\draw[lface] (x2) -- (n8);

\node[lvertex] at (x18) {};
\draw[lface] (x18) -- (n14);
\draw[lface] (x18) -- (n17);

\draw (x0) -- (x1);
\draw (x1) -- (x2);
\draw (x2) -- (x0);
\draw (x0) -- (x3);
\draw (x3) -- (x1);
\draw (x3) -- (x4);
\draw[lsquare] (x4) -- (x1);
\draw[lface] (x1) -- (n5);
\draw[lface] (x4) -- (n9);
\draw (x4) -- (x5);
\draw (x5) -- (x1);
\draw (x5) -- (x6);
\draw (x6) -- (x2);
\draw (x6) -- (x7);
\draw (x7) -- (x8);
\draw (x8) -- (x2);
\draw[ldiamond] (x8) -- (x9) node[midway] (m1) {};
\draw[lface] (m1) -- (n7);
\draw[lface] (m1) -- (n12);
\draw (x9) -- (x10);
\draw (x10) -- (x2);
\draw (x10) -- (x11);
\draw (x11) -- (x0);
\draw (x3) -- (x12);
\draw (x12) -- (x13);
\draw (x13) -- (x4);
\draw (x12) -- (x14);
\draw (x14) -- (x13);
\draw (x14) -- (x15);
\draw (x15) -- (x13);
\draw (x8) -- (x16);
\draw (x16) -- (x17);
\draw (x17) -- (x9);
\draw (x16) -- (x18);
\draw (x18) -- (x17);
\draw (x16) -- (x19);
\draw (x19) -- (x20);
\draw (x20) -- (x18);
\draw (x19) -- (x21);
\draw (x21) -- (x20);
\draw (x21) -- (x22);
\draw (x22) -- (x20);
\draw (x18) -- (x23);
\draw (x23) -- (x24);
\draw (x24) -- (x17);
\draw (x23) -- (x25);
\draw (x25) -- (x24);
\draw (x25) -- (x26);
\draw (x26) -- (x24);

    \end{scope}
  \end{tikzsubfigure}
\end{tikzfigure2}
\clearpage
\begin{theorem}
  Let $p = (p_3, p_4, \dots, p_n)$, $v = (v_3, v_4, \dots, v_m)$ be a pair of admissible sequences for a closed orientable $2$-manifold $S$ with {\sc Euler} characteristic $\chi$, i.e. \eqref{eq:vp:5} and \eqref{eq:handshake} hold for some $e \in \nats$. Then $(p, v)$ is $[(9k + 2) \times 3, 3k + 4]$-$[5]$-realizable for all $k \in \nats$.
  \begin{proof}
    Using \autoref{const:edge:replacement:3:5:1} on the edges drawn thick in both $5$-patches in \autoref{fig:expansion:patch:3:5:5} results in $5$-patches consisting only of triangles and $n$-gons with the specified amount of vertices. The left patch is an expansion patch with outer tuple $o$, while the right patch is $o$-$6$-gonal, these properties do not change when using \autoref{const:edge:replacement:5:1} and \autoref{const:edge:replacement:5:2} on the thick lines. In fact, the resulting patch on the left is even polyhedral, so we can apply \autoref{thm:main:const:5}.
  \end{proof}
\end{theorem}

\begin{tikzfigure2}{}
  \begin{tikzsubfigure}{}{}{0.5}
    \begin{scope}[scale=0.6, yscale=0.866]
      \draw (-0.5, 1) -- (-3.5, 13) -- (-2.5, 14.5) -- (-1.5, 14.75) -- (-0.5, 14) -- (0.5, 12) -- (1.5, 11.25)  (2.5, 11.5) -- (3.5, 13) -- (4.5, 13.666) -- (5.5, 13.333) -- (6.5, 13) -- (6.75, 11.833) -- (7, 10.666) -- (8, 10) -- (0.5, 1) -- (-0.5, 1);
      \draw (-1.5, 13) -- (-3.5, 13);
      \draw (-1.5, 13) -- (-2.5, 14.5);
      \draw (-1.5, 13) -- (-1.5, 14.75);
      \draw (-1.5, 13) -- (-0.5, 14);
      \draw (-1.5, 13) -- ( 0.5, 12);
      \draw ( 0.5, 12) -- (-3.5, 13);
      \draw ( 1.5, 11.25) -- (0.5, 1);
      \draw ( 1.5, 11.25) -- (7, 10.666);
      \draw ( 2.5, 11.5) -- (7, 10.666);
      \draw ( 2.5, 11.5) -- (6.75, 11.833);
      \draw (5.5, 13.333) -- (6.75, 11.833);
      \draw ( 7  , 10.666) -- (0.5, 1);
      \draw[lsquare] (1.5, 11.25) -- (2.5, 11.5);

      \node (k1) at (-0.5,6) {$5$};
      \node (k2) at (4.5,12.5) {$5$};
      \node at (-2.5,13.5) {$3$};
      \node at (-1.8,14) {$3$};
      \node at (-1.1,14) {$3$};
      \node at (-0.5,13) {$3$};
      \node (h1) at (-1.5,11.5) {$3$};
      \draw[dashed] (h1)--(-1.5,12.7);
      \node (h2) at (2.6,10) {$3$};
      \draw[dashed] (h2)--(2.6,11.35);
      \node at (6.35,12.7) {$3$};
      \node at (6,11.2) {$3$};
      \node at (7,9.7) {$3$};
      \node at (2.6,6) {$3$};
      \draw[lface] (1.5, 11.25) -- (k1);
      \draw[lface] (2.5, 11.5) -- (k2);
      


      \node[anchor= 90] at (-0.5, 1)     {$i_{0}$};
      \node[anchor=330] at (-3.5, 13)    {$i_{1}=o_{0}$};
      \node[anchor=330] at (-2.5, 14.5)  {$o_{1}$};
      \node[anchor=270] at (-1.5, 14.75) {$o_{2}$};
      \node[anchor=240] at (-0.5, 14)    {$o_{3}$};
      \node[anchor=220] at (0.5, 12)     {$o_{4}$};
      \node[anchor=270] at (1.5, 11.25)  {$o_{5}$};
      \node[anchor=300] at (2.5, 11.5)   {$o_{6}$};
      \node[anchor=300] at (3.5, 13)     {$o_{7}$};
      \node[anchor=300] at (4.5, 13.666) {$o_{8}$};
      \node[anchor=270] at (5.5, 13.333) {$o_{9}$};
      \node[anchor=180] at (6.5, 13)     {$o_{10}$};
      \node[anchor=180] at (6.75, 11.833){$o_{11}$};  
      \node[anchor=180] at (7, 10.666)   {$o_{12}$};
      \node[anchor=150] at (8, 10)       {$i_{1}'=o_{13}$};
      \node[anchor= 90] at (0.5, 1)      {$i_{0}'$};

      \fill[black] (-0.5, 1)     circle(2pt);
      \fill[black] (-3.5, 13)    circle(2pt);
      \fill[black] (-2.5, 14.5)  circle(2pt);
      \fill[black] (-1.5, 14.75) circle(2pt);
      \fill[black] (-0.5, 14)    circle(2pt);
      \fill[black] (0.5, 12)     circle(2pt);
      \fill[black] (1.5, 11.25)  circle(2pt);
      \fill[black] (2.5, 11.5)   circle(2pt);
      \fill[black] (3.5, 13)     circle(2pt);
      \fill[black] (4.5, 13.666) circle(2pt);
      \fill[black] (5.5, 13.333) circle(2pt);
      \fill[black] (6.5, 13)     circle(2pt);
      \fill[black] (6.75, 11.833)circle(2pt);
      \fill[black] (7, 10.666)   circle(2pt);
      \fill[black] (8, 10)       circle(2pt);
      \fill[black] (0.5, 1)      circle(2pt);
      \fill[black] (-1.5, 13)    circle(2pt);
      
    \end{scope}
  \end{tikzsubfigure}
  \begin{tikzsubfigure}{}{}{0.5}
    \begin{scope}[scale=0.35]
      \begin{scope}[yscale=0.866]
        \draw[very thick] (-0.5, 1) -- (-3.5, 13) -- (-2.5, 14.5) -- (-1.5, 14.75) -- (-0.5, 14) -- (0.5, 12) -- (1.5, 11.25) -- (2.5, 11.5) -- (3.5, 13) -- (4.5, 13.666) -- (5.5, 13.333) -- (6.5, 13) -- (6.75, 11.833) -- (7, 10.666) -- (8, 10) -- (0.5, 1) -- (-0.5, 1);
        \draw (-1.5, 13) -- (-3.5, 13);
        \draw (-1.5, 13) -- (-2.5, 14.5);
        \draw (-1.5, 13) -- (-1.5, 14.75);
        \draw (-1.5, 13) -- (-0.5, 14);
        \draw (-1.5, 13) -- ( 0.5, 12);
        \draw ( 0.5, 12) -- (-3.5, 13);
        \draw ( 1.5, 11.25) -- (0.5, 1);
        \draw ( 1.5, 11.25) -- (7, 10.666);
        \draw ( 2.5, 11.5) -- (7, 10.666);
        \draw ( 2.5, 11.5) -- (6.75, 11.833);
        \draw (5.5, 13.333) -- (6.75, 11.833);
        \draw ( 7  , 10.666) -- (0.5, 1);

        \fill[black] (-0.5, 1)     circle(4pt);
        \fill[black] (-3.5, 13)    circle(4pt);
        \fill[black] (-2.5, 14.5)  circle(4pt);
        \fill[black] (-1.5, 14.75) circle(4pt);
        \fill[black] (-0.5, 14)    circle(4pt);
        \fill[black] (0.5, 12)     circle(4pt);
        \fill[black] (1.5, 11.25)  circle(4pt);
        \fill[black] (2.5, 11.5)   circle(4pt);
        \fill[black] (3.5, 13)     circle(4pt);
        \fill[black] (4.5, 13.666) circle(4pt);
        \fill[black] (5.5, 13.333) circle(4pt);
        \fill[black] (6.5, 13)     circle(4pt);
        \fill[black] (6.75, 11.833)circle(4pt);
        \fill[black] (7, 10.666)   circle(4pt);
        \fill[black] (8, 10)       circle(4pt);
        \fill[black] (0.5, 1)      circle(4pt);
        \fill[black] (-1.5, 13)    circle(4pt);

      \end{scope}
      \begin{scope}[rotate=60, yscale=0.866]
        \draw[very thick] (-0.5, 1) -- (-3.5, 13) -- (-2.5, 14.5) -- (-1.5, 14.75) -- (-0.5, 14) -- (0.5, 12) -- (1.5, 11.25) -- (2.5, 11.5) -- (3.5, 13) -- (4.5, 13.666) -- (5.5, 13.333) -- (6.5, 13) -- (6.75, 11.833) -- (7, 10.666) -- (8, 10) -- (0.5, 1) -- (-0.5, 1);
        \draw (-1.5, 13) -- (-3.5, 13);
        \draw (-1.5, 13) -- (-2.5, 14.5);
        \draw (-1.5, 13) -- (-1.5, 14.75);
        \draw (-1.5, 13) -- (-0.5, 14);
        \draw (-1.5, 13) -- ( 0.5, 12);
        \draw ( 0.5, 12) -- (-3.5, 13);
        \draw ( 1.5, 11.25) -- (0.5, 1);
        \draw ( 1.5, 11.25) -- (7, 10.666);
        \draw ( 2.5, 11.5) -- (7, 10.666);
        \draw ( 2.5, 11.5) -- (6.75, 11.833);
        \draw (5.5, 13.333) -- (6.75, 11.833);
        \draw ( 7  , 10.666) -- (0.5, 1);

        \fill[black] (-0.5, 1)     circle(4pt);
        \fill[black] (-3.5, 13)    circle(4pt);
        \fill[black] (-2.5, 14.5)  circle(4pt);
        \fill[black] (-1.5, 14.75) circle(4pt);
        \fill[black] (-0.5, 14)    circle(4pt);
        \fill[black] (0.5, 12)     circle(4pt);
        \fill[black] (1.5, 11.25)  circle(4pt);
        \fill[black] (2.5, 11.5)   circle(4pt);
        \fill[black] (3.5, 13)     circle(4pt);
        \fill[black] (4.5, 13.666) circle(4pt);
        \fill[black] (5.5, 13.333) circle(4pt);
        \fill[black] (6.5, 13)     circle(4pt);
        \fill[black] (6.75, 11.833)circle(4pt);
        \fill[black] (7, 10.666)   circle(4pt);
        \fill[black] (8, 10)       circle(4pt);
        \fill[black] (0.5, 1)      circle(4pt);
        \fill[black] (-1.5, 13)    circle(4pt);

      \end{scope}
      \begin{scope}[yscale=0.866, shift={(0 cm,26 cm)}, rotate=180]
        \draw[very thick] (-0.5, 1) -- (-3.5, 13) -- (-2.5, 14.5) -- (-1.5, 14.75) -- (-0.5, 14) -- (0.5, 12) -- (1.5, 11.25) -- (2.5, 11.5) -- (3.5, 13) -- (4.5, 13.666) -- (5.5, 13.333) -- (6.5, 13) -- (6.75, 11.833) -- (7, 10.666) -- (8, 10) -- (0.5, 1) -- (-0.5, 1);
        \draw (-1.5, 13) -- (-3.5, 13);
        \draw (-1.5, 13) -- (-2.5, 14.5);
        \draw (-1.5, 13) -- (-1.5, 14.75);
        \draw (-1.5, 13) -- (-0.5, 14);
        \draw (-1.5, 13) -- ( 0.5, 12);
        \draw ( 0.5, 12) -- (-3.5, 13);
        \draw ( 1.5, 11.25) -- (0.5, 1);
        \draw ( 1.5, 11.25) -- (7, 10.666);
        \draw ( 2.5, 11.5) -- (7, 10.666);
        \draw ( 2.5, 11.5) -- (6.75, 11.833);
        \draw (5.5, 13.333) -- (6.75, 11.833);
        \draw ( 7  , 10.666) -- (0.5, 1);

        \fill[black] (-0.5, 1)     circle(4pt);
        \fill[black] (-3.5, 13)    circle(4pt);
        \fill[black] (-2.5, 14.5)  circle(4pt);
        \fill[black] (-1.5, 14.75) circle(4pt);
        \fill[black] (-0.5, 14)    circle(4pt);
        \fill[black] (0.5, 12)     circle(4pt);
        \fill[black] (1.5, 11.25)  circle(4pt);
        \fill[black] (2.5, 11.5)   circle(4pt);
        \fill[black] (3.5, 13)     circle(4pt);
        \fill[black] (4.5, 13.666) circle(4pt);
        \fill[black] (5.5, 13.333) circle(4pt);
        \fill[black] (6.5, 13)     circle(4pt);
        \fill[black] (6.75, 11.833)circle(4pt);
        \fill[black] (7, 10.666)   circle(4pt);
        \fill[black] (8, 10)       circle(4pt);
        \fill[black] (0.5, 1)      circle(4pt);
        \fill[black] (-1.5, 13)    circle(4pt);

      \end{scope}
      \begin{scope}[shift={(0 cm,22.517 cm)},rotate=240,yscale=0.866]
        \draw[very thick] (-0.5, 1) -- (-3.5, 13) -- (-2.5, 14.5) -- (-1.5, 14.75) -- (-0.5, 14) -- (0.5, 12) -- (1.5, 11.25) -- (2.5, 11.5) -- (3.5, 13) -- (4.5, 13.666) -- (5.5, 13.333) -- (6.5, 13) -- (6.75, 11.833) -- (7, 10.666) -- (8, 10) -- (0.5, 1) -- (-0.5, 1);
        \draw (-1.5, 13) -- (-3.5, 13);
        \draw (-1.5, 13) -- (-2.5, 14.5);
        \draw (-1.5, 13) -- (-1.5, 14.75);
        \draw (-1.5, 13) -- (-0.5, 14);
        \draw (-1.5, 13) -- ( 0.5, 12);
        \draw ( 0.5, 12) -- (-3.5, 13);
        \draw ( 1.5, 11.25) -- (0.5, 1);
        \draw ( 1.5, 11.25) -- (7, 10.666);
        \draw ( 2.5, 11.5) -- (7, 10.666);
        \draw ( 2.5, 11.5) -- (6.75, 11.833);
        \draw (5.5, 13.333) -- (6.75, 11.833);
        \draw ( 7  , 10.666) -- (0.5, 1);

        \fill[black] (-0.5, 1)     circle(4pt);
        \fill[black] (-3.5, 13)    circle(4pt);
        \fill[black] (-2.5, 14.5)  circle(4pt);
        \fill[black] (-1.5, 14.75) circle(4pt);
        \fill[black] (-0.5, 14)    circle(4pt);
        \fill[black] (0.5, 12)     circle(4pt);
        \fill[black] (1.5, 11.25)  circle(4pt);
        \fill[black] (2.5, 11.5)   circle(4pt);
        \fill[black] (3.5, 13)     circle(4pt);
        \fill[black] (4.5, 13.666) circle(4pt);
        \fill[black] (5.5, 13.333) circle(4pt);
        \fill[black] (6.5, 13)     circle(4pt);
        \fill[black] (6.75, 11.833)circle(4pt);
        \fill[black] (7, 10.666)   circle(4pt);
        \fill[black] (8, 10)       circle(4pt);
        \fill[black] (0.5, 1)      circle(4pt);
        \fill[black] (-1.5, 13)    circle(4pt);

      \end{scope}
    \end{scope}
  \end{tikzsubfigure}
\end{tikzfigure2}
\begin{figure}
  \ContinuedFloat
  \begin{tikzsubfigure}{}{}{1.0}
    \begin{scope}[scale=8]
      \coordinate (x0) at (0.258819, -0.965926);
\coordinate (x1) at (0.463814, -0.244151);
\coordinate (x2) at (0.067105, 0.192878);
\coordinate (x3) at (-0.965926, -0.258819);
\coordinate (x4) at (0.965926, 0.258819);
\coordinate (x5) at (-0.258819, 0.965926);
\coordinate (x6) at (0.933580, 0.358368);
\coordinate (x7) at (0.453990, 0.891007);
\coordinate (x8) at (-0.052336, 0.998630);
\coordinate (x9) at (0.358368, 0.933580);
\coordinate (x10) at (0.052336, 0.998630);
\coordinate (x11) at (0.258819, 0.965926);
\coordinate (x12) at (0.156434, 0.987688);
\coordinate (x13) at (-0.156434, 0.987688);
\coordinate (x14) at (-0.358368, 0.933580);
\coordinate (x15) at (-0.998630, -0.052336);
\coordinate (x16) at (-0.987688, -0.156434);
\coordinate (x17) at (-0.891007, 0.453990);
\coordinate (x18) at (-0.933580, 0.358368);
\coordinate (x19) at (-0.998630, 0.052336);
\coordinate (x20) at (-0.965926, 0.258819);
\coordinate (x21) at (-0.987688, 0.156434);
\coordinate (x22) at (-0.933580, -0.358368);
\coordinate (x23) at (0.052336, -0.998630);
\coordinate (x24) at (0.156434, -0.987688);
\coordinate (x25) at (-0.453990, -0.891007);
\coordinate (x26) at (-0.358368, -0.933580);
\coordinate (x27) at (-0.052336, -0.998630);
\coordinate (x28) at (-0.258819, -0.965926);
\coordinate (x29) at (-0.156434, -0.987688);
\coordinate (x30) at (0.358368, -0.933580);
\coordinate (x31) at (0.998630, 0.052336);
\coordinate (x32) at (0.987688, 0.156434);
\coordinate (x33) at (0.891007, -0.453990);
\coordinate (x34) at (0.933580, -0.358368);
\coordinate (x35) at (0.998630, -0.052336);
\coordinate (x36) at (0.965926, -0.258819);
\coordinate (x37) at (0.987688, -0.156434);
\coordinate (x38) at (0.891007, 0.453990);
\coordinate (x39) at (0.838671, 0.544639);
\coordinate (x40) at (0.707107, 0.707107);
\coordinate (x41) at (0.777146, 0.629320);
\coordinate (x42) at (0.544639, 0.838671);
\coordinate (x43) at (0.629320, 0.777146);
\coordinate (x44) at (-0.453990, 0.891007);
\coordinate (x45) at (-0.544639, 0.838671);
\coordinate (x46) at (-0.707107, 0.707107);
\coordinate (x47) at (-0.629320, 0.777146);
\coordinate (x48) at (-0.838671, 0.544639);
\coordinate (x49) at (-0.777146, 0.629320);
\coordinate (x50) at (-0.891007, -0.453990);
\coordinate (x51) at (-0.838671, -0.544639);
\coordinate (x52) at (-0.707107, -0.707107);
\coordinate (x53) at (-0.777146, -0.629320);
\coordinate (x54) at (-0.544639, -0.838671);
\coordinate (x55) at (-0.629320, -0.777146);
\coordinate (x56) at (0.453990, -0.891007);
\coordinate (x57) at (0.544639, -0.838671);
\coordinate (x58) at (0.707107, -0.707107);
\coordinate (x59) at (0.629320, -0.777146);
\coordinate (x60) at (0.838671, -0.544639);
\coordinate (x61) at (0.777146, -0.629320);
\draw (0.258819, -0.965926) -- (0.463814, -0.244151);
\draw (0.463814, -0.244151) -- (0.067105, 0.192878);
\draw (0.067105, 0.192878) -- (-0.965926, -0.258819);
\draw (-0.965926, -0.258819) -- (0.258819, -0.965926);
\draw (0.258819, -0.965926) -- (0.965926, 0.258819);
\draw (0.965926, 0.258819) -- (0.463814, -0.244151);
\draw (0.965926, 0.258819) -- (0.067105, 0.192878);
\draw (0.067105, 0.192878) -- (-0.258819, 0.965926);
\draw (-0.258819, 0.965926) -- (-0.965926, -0.258819);
\draw (0.965926, 0.258819) -- (0.933580, 0.358368);
\draw (0.933580, 0.358368) -- (0.067105, 0.192878);
\draw (0.933580, 0.358368) -- (0.453990, 0.891007);
\draw (0.453990, 0.891007) -- (-0.052336, 0.998630);
\draw (-0.052336, 0.998630) -- (-0.258819, 0.965926);
\draw (0.453990, 0.891007) -- (0.358368, 0.933580);
\draw (0.358368, 0.933580) -- (-0.052336, 0.998630);
\draw (0.358368, 0.933580) -- (0.052336, 0.998630);
\draw (0.052336, 0.998630) -- (-0.052336, 0.998630);
\draw (0.358368, 0.933580) -- (0.258819, 0.965926);
\draw (0.258819, 0.965926) -- (0.052336, 0.998630);
\draw (0.258819, 0.965926) -- (0.156434, 0.987688);
\draw (0.156434, 0.987688) -- (0.052336, 0.998630);
\draw (-0.052336, 0.998630) -- (-0.156434, 0.987688);
\draw (-0.156434, 0.987688) -- (-0.258819, 0.965926);
\draw (-0.258819, 0.965926) -- (-0.358368, 0.933580);
\draw (-0.358368, 0.933580) -- (-0.998630, -0.052336);
\draw (-0.998630, -0.052336) -- (-0.987688, -0.156434);
\draw (-0.987688, -0.156434) -- (-0.965926, -0.258819);
\draw (-0.358368, 0.933580) -- (-0.891007, 0.453990);
\draw (-0.891007, 0.453990) -- (-0.998630, -0.052336);
\draw (-0.891007, 0.453990) -- (-0.933580, 0.358368);
\draw (-0.933580, 0.358368) -- (-0.998630, -0.052336);
\draw (-0.933580, 0.358368) -- (-0.998630, 0.052336);
\draw (-0.998630, 0.052336) -- (-0.998630, -0.052336);
\draw (-0.933580, 0.358368) -- (-0.965926, 0.258819);
\draw (-0.965926, 0.258819) -- (-0.998630, 0.052336);
\draw (-0.965926, 0.258819) -- (-0.987688, 0.156434);
\draw (-0.987688, 0.156434) -- (-0.998630, 0.052336);
\draw (-0.965926, -0.258819) -- (-0.933580, -0.358368);
\draw (-0.933580, -0.358368) -- (0.052336, -0.998630);
\draw (0.052336, -0.998630) -- (0.156434, -0.987688);
\draw (0.156434, -0.987688) -- (0.258819, -0.965926);
\draw (-0.933580, -0.358368) -- (-0.453990, -0.891007);
\draw (-0.453990, -0.891007) -- (0.052336, -0.998630);
\draw (-0.453990, -0.891007) -- (-0.358368, -0.933580);
\draw (-0.358368, -0.933580) -- (0.052336, -0.998630);
\draw (-0.358368, -0.933580) -- (-0.052336, -0.998630);
\draw (-0.052336, -0.998630) -- (0.052336, -0.998630);
\draw (-0.358368, -0.933580) -- (-0.258819, -0.965926);
\draw (-0.258819, -0.965926) -- (-0.052336, -0.998630);
\draw (-0.258819, -0.965926) -- (-0.156434, -0.987688);
\draw (-0.156434, -0.987688) -- (-0.052336, -0.998630);
\draw (0.258819, -0.965926) -- (0.358368, -0.933580);
\draw (0.358368, -0.933580) -- (0.998630, 0.052336);
\draw (0.998630, 0.052336) -- (0.987688, 0.156434);
\draw (0.987688, 0.156434) -- (0.965926, 0.258819);
\draw (0.358368, -0.933580) -- (0.891007, -0.453990);
\draw (0.891007, -0.453990) -- (0.998630, 0.052336);
\draw (0.891007, -0.453990) -- (0.933580, -0.358368);
\draw (0.933580, -0.358368) -- (0.998630, 0.052336);
\draw (0.933580, -0.358368) -- (0.998630, -0.052336);
\draw (0.998630, -0.052336) -- (0.998630, 0.052336);
\draw (0.933580, -0.358368) -- (0.965926, -0.258819);
\draw (0.965926, -0.258819) -- (0.998630, -0.052336);
\draw (0.965926, -0.258819) -- (0.987688, -0.156434);
\draw (0.987688, -0.156434) -- (0.998630, -0.052336);
\draw (0.933580, 0.358368) -- (0.891007, 0.453990);
\draw (0.891007, 0.453990) -- (0.838671, 0.544639);
\draw (0.838671, 0.544639) -- (0.707107, 0.707107);
\draw (0.707107, 0.707107) -- (0.453990, 0.891007);
\draw (0.838671, 0.544639) -- (0.777146, 0.629320);
\draw (0.777146, 0.629320) -- (0.707107, 0.707107);
\draw (0.707107, 0.707107) -- (0.544639, 0.838671);
\draw (0.544639, 0.838671) -- (0.453990, 0.891007);
\draw (0.707107, 0.707107) -- (0.629320, 0.777146);
\draw (0.629320, 0.777146) -- (0.544639, 0.838671);
\draw (-0.358368, 0.933580) -- (-0.453990, 0.891007);
\draw (-0.453990, 0.891007) -- (-0.544639, 0.838671);
\draw (-0.544639, 0.838671) -- (-0.707107, 0.707107);
\draw (-0.707107, 0.707107) -- (-0.891007, 0.453990);
\draw (-0.544639, 0.838671) -- (-0.629320, 0.777146);
\draw (-0.629320, 0.777146) -- (-0.707107, 0.707107);
\draw (-0.707107, 0.707107) -- (-0.838671, 0.544639);
\draw (-0.838671, 0.544639) -- (-0.891007, 0.453990);
\draw (-0.707107, 0.707107) -- (-0.777146, 0.629320);
\draw (-0.777146, 0.629320) -- (-0.838671, 0.544639);
\draw (-0.933580, -0.358368) -- (-0.891007, -0.453990);
\draw (-0.891007, -0.453990) -- (-0.838671, -0.544639);
\draw (-0.838671, -0.544639) -- (-0.707107, -0.707107);
\draw (-0.707107, -0.707107) -- (-0.453990, -0.891007);
\draw (-0.838671, -0.544639) -- (-0.777146, -0.629320);
\draw (-0.777146, -0.629320) -- (-0.707107, -0.707107);
\draw (-0.707107, -0.707107) -- (-0.544639, -0.838671);
\draw (-0.544639, -0.838671) -- (-0.453990, -0.891007);
\draw (-0.707107, -0.707107) -- (-0.629320, -0.777146);
\draw (-0.629320, -0.777146) -- (-0.544639, -0.838671);
\draw (0.358368, -0.933580) -- (0.453990, -0.891007);
\draw (0.453990, -0.891007) -- (0.544639, -0.838671);
\draw (0.544639, -0.838671) -- (0.707107, -0.707107);
\draw (0.707107, -0.707107) -- (0.891007, -0.453990);
\draw (0.544639, -0.838671) -- (0.629320, -0.777146);
\draw (0.629320, -0.777146) -- (0.707107, -0.707107);
\draw (0.707107, -0.707107) -- (0.838671, -0.544639);
\draw (0.838671, -0.544639) -- (0.891007, -0.453990);
\draw (0.707107, -0.707107) -- (0.777146, -0.629320);
\draw (0.777146, -0.629320) -- (0.838671, -0.544639);
\node at (-0.340292, 0.548358) {0};
\node at (-0.044047, -0.319005) {1};
\node at (0.562853, -0.317086) {2};
\node at (0.498948, 0.069182) {3};
\node at (-0.385880, 0.299995) {4};
\node at (0.655537, 0.270022) {5};
\node at (0.228704, 0.681362) {6};
\node at (0.253341, 0.941072) {7};
\node at (0.119456, 0.976946) {8};
\node at (0.223174, 0.966045) {9};
\node at (0.155863, 0.984081) {10};
\node at (-0.155863, 0.984081) {11};
\node at (-0.713886, 0.286383) {12};
\node at (-0.749335, 0.445078) {13};
\node at (-0.941072, 0.253341) {14};
\node at (-0.976946, 0.119456) {15};
\node at (-0.966045, 0.223174) {16};
\node at (-0.984081, 0.155863) {17};
\node at (-0.286383, -0.713886) {18};
\node at (-0.445078, -0.749335) {19};
\node at (-0.253341, -0.941072) {20};
\node at (-0.119456, -0.976946) {21};
\node at (-0.223174, -0.966045) {22};
\node at (-0.155863, -0.984081) {23};
\node at (0.713886, -0.286383) {24};
\node at (0.749335, -0.445078) {25};
\node at (0.941072, -0.253341) {26};
\node at (0.976946, -0.119456) {27};
\node at (0.966045, -0.223174) {28};
\node at (0.984081, -0.155863) {29};
\node at (0.764871, 0.591022) {30};
\node at (0.774308, 0.627022) {31};
\node at (0.568579, 0.812261) {32};
\node at (0.627022, 0.774308) {33};
\node at (-0.591022, 0.764871) {34};
\node at (-0.627022, 0.774308) {35};
\node at (-0.812261, 0.568579) {36};
\node at (-0.774308, 0.627022) {37};
\node at (-0.764871, -0.591022) {38};
\node at (-0.774308, -0.627022) {39};
\node at (-0.568579, -0.812261) {40};
\node at (-0.627022, -0.774308) {41};
\node at (0.591022, -0.764871) {42};
\node at (0.627022, -0.774308) {43};
\node at (0.812261, -0.568579) {44};
\node at (0.774308, -0.627022) {45};

    \end{scope}
  \end{tikzsubfigure}
\end{figure}
