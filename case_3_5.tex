\section{$5$-valent {\sc Eberhard}-like theorems with triangles}\label{sec:3:5}

In the previous two section, we could rely on \autoref{thm:eberhard:extended:3} and \autoref{thm:eberhard:extended:4} to do the main work for us and create an initial map we can begin to work with. Since the previous work on $5$-valent {\sc Eberhard} theorems is not as advanced as in the $3$-valent and $4$-valent cases and does not consider the additional $v$-vector, we need a new $5$-valent {\sc Eberhard} theorem which we can use as first construction step and which includes both a $p$-vector and a $v$-vector. But the situation is not as difficult as doing a whole construction of a polyhedral map by ourselves, as in fact, we can use \autoref{thm:eberhard:extended:4} nonetheless and just need to convert the $4$-valent vertices into $5$-valent ones.

\begin{proposition}
  Let $p = (p_3, p_4, \dots, p_n)$, $v = (v_3, v_4, \dots, v_m)$ be a pair of admissible sequences, then $p$ is $[2\times3, 4]$-$[5]$-realizable on a closed orientable $2$-manifold $S$ with {\sc Euler} characteristic $\chi$.
  \begin{proof}
    Let $p' \defeq p + v$ and $v' = [v'_4 \times 4]$ for some $v'_4$ we want to specify shortly. Since $p$ and $v$ are admissible, \eqref{eq:vp:4} holds for $p$ and $v$, and it also holds for $p'$ and $v'$:
    % Beginning with equation \eqref{eq:vp:5} we retrieve the following:
    % \begin{align*}
    %   &2 \sum_{k=3}^m \left(5 - k \right) v_k\,+\, 3 \sum_{k=3}^n \left( \frac{10}{3} - k \right) p_k = 10 \chi\\
    %   \implies &\frac{2}{3} \sum_{k=3}^m \left(5 - k \right) v_k\,+\, \sum_{k=3}^n \left( \frac{10}{3} - k \right) p_k = \frac{10}{3} \chi\\
    %   \implies & \frac{p_3}{3} = \frac{10}{3}\chi - \frac{2}{3} \sum_{k = 3}^m (5 - k) v_k + \sum_{k=4}^n \left(k - \frac{10}{3} \right) p_k  \geq \frac{10}{3} \chi - \frac{1}{3} \sum_{k = 3}^m (2 - k)v_k + \frac{2}{3} \sum_{k=4}^n p_k \\
    %   \implies & p_3 \geq \frac{10}{3}\chi - \frac{1}{3} \sum_{k = 3}^m (2 - k) v_k + \frac{2}{3} \sum_{k=3}^n p_k \geq \frac{2}{3}  \sum_{k=3}^n p_k - \frac{1}{3} \sum_{k = 3}^m (2 - k) v_k - \frac{2}{3}\chi.
    % \end{align*}
    % We want to show that the last term in the inequality chain is a whole number. For this we take \eqref{eq:vp:5} modulo $3$:
    % \begin{align*}
    %   0 &\cong{} 2 \sum_{k=3}^m \left(5 - k \right) v_k {}+{} \sum_{k=3}^n (10 - 3 k) p_k - 10 \chi &(\mod 3)\\
    %     &\cong{} 2 \sum_{k=3}^m \left(5 - k \right) v_k {}+{} 10 \sum_{k=3}^n p_k - 10 \chi &(\mod 3)\\
    %     &\cong{} \sum_{k=3}^n p_k {}-{} \sum_{k=3}^m \left(5 - k \right) v_k {}-{} \chi &(\mod 3)
    % \end{align*}
    % Therefore by setting $p'_3 := p_3 + \frac{2}{3}(\chi + \sum_{k = 3}^m (5 - k) v_k - \sum_{k=3}^n p_k) \geq 0$, $p'_k = p_k$, $(k\geq 4)$ and $v'_k = 0$, we get two sequences $p'$, $v'$, that satisfy:
    \begin{align*}
      &\sum_{k=3}^{\max(n, m)} (4 - k) p'_k + \sum_{k=3}^{\max(n, m)} (4 - k) v'_k \\
      ={}& \sum_{k=3}^n (4 - k) p_k + \sum_{k=3}^m (4 - k) v_k = 4 \chi
    \end{align*}
    as well as \eqref{eq:handshake}
    \begin{align*}
      \sum_{k = 3}^n k (p_k + v_k) = \sum_{k = 3}^n k p_k + \sum_{k = 3}^n k v_k = 4e = \sum_{k = 3} k v'_k
    \end{align*}
    if we set $v'_4 \defeq e$. Hence the sequences $p'$, $v'$ are $[4]$-$[4]$-realizable by \autoref{thm:eberhard:extended:4} as a polyhedral map on $S$. Let $M'$ be this realization with $e'$ edges and $v'$ vertices. We use $M'$ to create a realization with valence $5$ by inserting additional triangles and quadrangles. This is done by replacing each edge of the realization by four triangles and a single quadrangle while replacing each vertex by eight triangles and five squares as shown in \autoref{fig:case34:img1}. Since $S$ is orientable we can do this such that the resulting map is $5$-valent.

    \begin{tikzfigure}{\label{fig:case34:img1}}{Building a $5$-valent polyhedral map out of a $4$-valent one by adding triangles and quadrangles \todo{Finish this!}}
      \matrix (m) [ column sep=1cm] {
        \begin{scope}[scale=0.5]

          \filldraw[fill=gray!50!white] (-3.5, -0.5) -- (-3, 0) -- (-1, 0) -- (-1, -2) -- (-1.5, -2.5);
          \filldraw[fill=gray!50!white] (3.5, -0.5) -- (3, 0) -- (1, 0) -- (1, -2) -- (1.5, -2.5);
          \filldraw[fill=gray!50!white] (3.5, 0.5) -- (3, 0) -- (1, 0) -- (1, 2) -- (1.5, 2.5);
          \filldraw[fill=gray!50!white] (-3.5, 0.5) -- (-3, 0) -- (-1, 0) -- (-1, 2) -- (-1.5, 2.5);
          \filldraw[fill=gray!50!white] (-0.5, -2.5) -- (-1, -2) -- (-1, 0) -- (1, 0) -- (1, -2) -- (0.5, -2.5);
          \filldraw[fill=gray!50!white] (-0.5, 2.5) -- (-1, 2) -- (-1, 0) -- (1, 0) -- (1, 2) -- (0.5, 2.5);

          \draw (-3.5, 0) -- (-3, 0);
          \draw (3.5, 0) -- (3, 0);
          \draw (-1, 2.5) -- (-1, 2);
          \draw (-1, -2.5) -- (-1, -2);
          \draw (1, 2.5) -- (1, 2);
          \draw (1, -2.5) -- (1, -2);

          \draw[very thick] (-3, 0) -- (3, 0);
          \draw[very thick] (-1, -2) -- (-1, 2);
          \draw[very thick] (1, -2) -- (1, 2);

        \end{scope}
        &
        \begin{scope}[scale=0.5]
          \filldraw[fill=gray!50!white] (-9.5, -3.5) -- (-9, -3) -- (-7, -3) -- (-7, -5) -- (-7.5, -5.5);
          \filldraw[fill=gray!50!white] (9.5, -3.5) -- (9, -3) -- (7, -3) -- (7, -5) -- (7.5, -5.5);
          \filldraw[fill=gray!50!white] (9.5, 3.5) -- (9, 3) -- (7, 3) -- (7, 5) -- (7.5, 5.5);
          \filldraw[fill=gray!50!white] (-9.5, 3.5) -- (-9, 3) -- (-7, 3) -- (-7, 5) -- (-7.5, 5.5);
          \filldraw[fill=gray!50!white] (-0.5, -5.5) -- (-1, -5) -- (-1, -3) -- (1, -3) -- (1, -5) -- (0.5, -5.5);
          \filldraw[fill=gray!50!white] (-0.5, 5.5) -- (-1, 5) -- (-1, 3) -- (1, 3) -- (1, 5) -- (0.5, 5.5);

          %\filldraw[fill=gray!75!white] (-5, -1) -- (-3, 1) -- (-3, 3) -- (-1, 1) -- (1, 1) -- (-1, -1) -- (-1, -3) -- (-3, -1) -- cycle;

          \draw (-9, -3) -- (9, -3);
          \draw (-9, -1) -- (9, -1);
          \draw (-9, 1) -- (9, 1);
          \draw (-9, 3) -- (9, 3);

          \draw (-7, -5) -- (-7, 5);
          \draw (-5, -5) -- (-5, 5);
          \draw (-3, -5) -- (-3, 5);
          \draw (-1, -5) -- (-1, 5);
          \draw ( 1, -5) -- ( 1, 5);
          \draw ( 3, -5) -- ( 3, 5);
          \draw ( 5, -5) -- ( 5, 5);
          \draw ( 7, -5) -- ( 7, 5);

          \draw (-9, -3) -- (-7, -1);
          \draw (-9,  1) -- (-7,  3);
          \draw (-7, -3) -- (-5, -5);
          \draw (-7,  1) -- (-5, -1);
          \draw (-7,  5) -- (-5,  3);
          \draw (-5, -3) -- (-3, -1);
          \draw (-5,  1) -- (-3,  3);
          \draw (-3, -3) -- (-1, -5);
          \draw (-3,  1) -- (-1, -1);
          \draw (-3,  5) -- (-1,  3);
          \draw (-1, -3) -- ( 1, -1);
          \draw (-1,  1) -- ( 1,  3);
          \draw ( 1, -3) -- ( 3, -5);
          \draw ( 1,  1) -- ( 3, -1);
          \draw ( 1,  5) -- ( 3,  3);
          \draw ( 3, -3) -- ( 5, -1);
          \draw ( 3,  1) -- ( 5,  3);
          \draw ( 5, -3) -- ( 7, -5);
          \draw ( 5,  1) -- ( 7, -1);
          \draw ( 5,  5) -- ( 7,  3);
          \draw ( 7, -3) -- ( 9, -1);
          \draw ( 7,  1) -- ( 9,  3);

          \draw[very thick] (-9, -3) -- (-9,  3);
          \draw[very thick] ( 9, -3) -- ( 9,  3);
          \draw[very thick] (-7, -5) -- (-1, -5);
          \draw[very thick] ( 7, -5) -- ( 1, -5);
          \draw[very thick] (-7,  5) -- (-1,  5);
          \draw[very thick] ( 7,  5) -- ( 1,  5);

          \draw[very thick] ( 7,  3) -- ( 1,  3);
          \draw[very thick] ( 7, -3) -- ( 1, -3);
          \draw[very thick] ( 7,  3) -- ( 7, -3);
          \draw[very thick] ( 1,  3) -- ( 1, -3);
          \draw[very thick] (-7,  3) -- (-1,  3);
          \draw[very thick] (-7, -3) -- (-1, -3);
          \draw[very thick] (-7,  3) -- (-7, -3);
          \draw[very thick] (-1,  3) -- (-1, -3);

          % \draw[very thick] (-5, -1) -- (-5, 1); \draw[very thick] (5, -1) -- (5, 1);
          % \draw[very thick] (-3, -3) -- (-1, -3); \draw[very thick] (1, -3) -- (3, -3);
          % \draw[very thick] (-3, 3) -- (-1, 3); \draw[very thick] (1, 3) -- (3, 3);
          \draw (-7, -5.5) -- (-7, -5);
          \draw[very thick] (-7, -5) -- (-6.5, -5.5);
          \draw (-5, -5.5) -- (-5, -5);
          \draw (-3.5, -5.5) -- (-3, -5) -- (-3, -5.5);
          \draw[very thick] (-1, -5.5) -- (-1, -5);
          \draw ( 1, -5.5) -- ( 1, -5);
          \draw[very thick] (1, -5) -- (1.5, -5.5);
          \draw ( 3, -5.5) -- ( 3, -5);
          \draw ( 4.5, -5.5) -- (5, -5) -- (5, -5.5);
          \draw[very thick] ( 7, -5.5) -- ( 7, -5);
        \end{scope}
        \\
        };
    \end{tikzfigure}
    The resulting map $M''$ is $5$-valent and has a $p$-vector of $p' + [c_1 \times 3, c_2 \times 4]$ (for some $c_1, c_2 \in \nats$ we do not want to state explicitly), but lacks lacks the specified vertices of the original $v$-vector $v$. To fix this we take for $k$-valent entry in $v$ a different $k$-gon of $M''$ and replace the $k$-gon together with the ring of triangles and quadrangles as in \autoref{fig:case34:img2}.
    \begin{tikzfigure}{\label{fig:case34:img2}}{Replacing a $k$-gon to get a $k$-valent vertex}
      \matrix (m) [ column sep=1cm] {
        \begin{scope}[scale=0.9, xscale=-1]
          \fill[fill=gray!50!white](  0 :   1) -- ( 72 :   1) -- (144 :   1) -- (216 :   1) -- (288 :   1) -- (  0 :   1);
          \draw (  0.0 : 1.000) -- ( 72.0 : 1.000) -- (144.0 : 1.000) -- (216.0 : 1.000) -- (288 :   1) -- (  0 :   1);
          \draw (  0.0 : 3.000) -- ( 72.0 : 3.000) -- (144.0 : 3.000) -- (216.0 : 3.000) -- (288 :   3) -- (  0 :   3);
          \draw (  0.0 : 1.000) -- (337.6 : 2.497) -- (288.0 : 1.000) -- (310.4 : 2.497);
          \draw ( 72.0 : 1.000) -- ( 49.6 : 2.497) -- (  0.0 : 1.000) -- ( 22.4 : 2.497);
          \draw (144.0 : 1.000) -- (121.6 : 2.497) -- ( 72.0 : 1.000) -- ( 94.4 : 2.497);
          \draw (216.0 : 1.000) -- (193.6 : 2.497) -- (144.0 : 1.000) -- (166.4 : 2.497);
          \draw (288.0 : 1.000) -- (265.6 : 2.497) -- (216.0 : 1.000) -- (238.4 : 2.497);
        \end{scope}
        &
        \begin{scope}[scale=0.9, xscale=-1]
          \draw (  0.0 : 3.000) -- ( 72.0 : 3.000) -- (144.0 : 3.000) -- (216.0 : 3.000) -- (288 :   3) -- (  0 :   3);
          \draw (  0.0 : 0.000) -- (337.6 : 2.497) -- ( 22.4 : 2.497);
          \draw ( 72.0 : 0.000) -- ( 49.6 : 2.497) -- ( 94.4 : 2.497);
          \draw (144.0 : 0.000) -- (121.6 : 2.497) -- (166.4 : 2.497);
          \draw (216.0 : 0.000) -- (193.6 : 2.497) -- (238.4 : 2.497);
          \draw (288.0 : 0.000) -- (265.6 : 2.497) -- (310.4 : 2.497);
        \end{scope}
        \\
      };
    \end{tikzfigure}
    After this construction we have held a map with $p$-vector $p'' \defeq p' + [c'_3 \times 3, c'_4 \times 4] - v = p + [c'_3 \times 3, c'_4 \times 4]$ and with $v$-vector $v'' \defeq v + d \cdot [5]$ (for some $c'_3, c'_4, d \in \nats$) since we only inserted additional triangles, quadrangles and $5$-valent vertices while removing $k$-gons and inserting $k$-valent vertices as specified by $v$. By \autoref{lem:2:valued:eberhard}, this suffices to show the claim.
  \end{proof}
\end{proposition}

Our next corner stone is to revisit \autoref{thm:main:const} for the $5$-valent case:
\begin{proposition}\label{thm:main:const:5} Let $r = 5$, $p$ and $v$ be an admissible pair of sequences for which we can apply \autoref{thm:eberhard:extended:3} (or \autoref{thm:eberhard:extended:4}). Let $w = [r]$ and $q = [q_3 \times 3, q_l \times l]$, where $l \geq 4$, $q_s = 3l - 10$ and $q_l = 1$. Assume there exist
  \begin{itemize}
  \item an expansion $r$-patch $\mathcal{P}_N$ with outer arc $o$ consisting of triangles and $l$-gons,
  \item an $o$-$4$-gonal $r$-patch $\mathcal{P}_F$ consisting of triangles and $l$-gons,
  \item an expansion $r$-patch $\mathcal{P}_P$ with the polyhedral property consisting of triangles and $l$-gons.
  \end{itemize}
  Then $(p, v)$ is $q$-$[5]$-realizable.
  \begin{proof}
    Analogous to \autoref{thm:main:const}.
  \end{proof}
\end{proposition}

\begin{construction}\label{const:edge:replacement:3:5:1}
  \begin{cinput}
  \item A $3$-patch with $p$-vector $p$ and a specified edge with exactly one vertex incident to some $k_1$-gon and the other vertex  incident to some $k_2$-gon.
  \end{cinput}
  \begin{coutput}
  \item A new $3$-patch with $p$-vector $p - [k_1, k_2] + [18k \times 3] + [k_1 + 3k, k_2 + 3k]$ for all $k \in \nats$.
  \item If every two faces of the $3$-patch meet proper, then this is carried over to the new patch.
  \end{coutput}
  \begin{cdescription}
    Using the replacement of the single edge as seen in \autoref{fig:const:edge:replacement:3:5:1} results in a new map with $p$-vector $p - [k_1, k_2] + [18 \times 3] + [k_1 + 3, k_2 + 3]$. The thick line on the left is the specified line and the thick line on the right is a new line which we can use to repeat the construction. Every time we use this construction we add a new quadrangle while increasing the number of vertices of the left and the right polygon; doing this $k$ times has the desired result, since we only inserted $5$-valent vertices.
    \begin{tikzfigure}{\label{fig:const:edge:replacement:3:5:1}}{}
      \matrix (m) [column sep=1cm] {
        \begin{scope}
          \draw[very thick] (-1, 0) -- (1, 0);
          \draw (-1.2, 0.5) -- (-1, 0) -- (-1.2, -0.5);
          \draw (1.2, 0.5) -- (1, 0) -- (1.2, -0.5);
          \draw (-0.8, 0.5) -- (-1, 0) -- (-0.8, -0.5);
          \draw (0.8, 0.5) -- (1, 0) -- (0.8, -0.5);
          \node at (-1.5, 0) {$k_1$};
          \node at (1.5, 0) {$k_2$};
        \end{scope}
        &
        \begin{scope}
          \draw[very thick] (-1, 1.5) -- (1, 1.5);
          \draw (-1, -1.5) -- (1, -1.5);
          \draw (-1.2, 2) -- (-1, 1.5);
          \draw (-1.2, -2) -- (-1, -1.5);
          \draw (1.2, 2) -- (1, 1.5);
          \draw (1.2, -2) -- (1, -1.5);
          \draw (-0.8, 2) -- (-1, 1.5);
          \draw (-0.8, -2) -- (-1, -1.5);
          \draw (0.8, 2) -- (1, 1.5);
          \draw (0.8, -2) -- (1, -1.5);
          \draw (1, 1.5) -- (0.8, 0.5) -- (0.8, -0.5) -- (1, -1.5);
          \draw (-1, 1.5) -- (-0.8, 0.5) -- (-0.8, -0.5) -- (-1, -1.5);
          \draw (-1, -1.5) -- (0, -1) -- (1, -1.5);
          \draw (-1, 1.5) -- (0, 1) -- (1, 1.5);
          \draw (-0.8, -0.5) -- (0, -1) -- (0.8, -0.5);
          \draw (-0.8, 0.5) -- (0, 1) -- (0.8, 0.5);
          \draw (-0.8, -0.5) -- (0, -0.5) -- (0.8, -0.5);
          \draw (-0.8, 0.5) -- (0, 0.5) -- (0.8, 0.5);
          \draw (-0.8, -0.5) -- (-0.4, 0) -- (0, -0.5) -- (0.4, 0) -- (0.8, -0.5);
          \draw (-0.8, 0.5) -- (-0.4, 0) -- (0, 0.5) -- (0.4, 0) -- (0.8, 0.5);
          \draw (-0.4, 0) -- (0.4, 0);
          \draw (0, -1) -- (0, -0.5);
          \draw (0, 1) -- (0, 0.5);
          %\draw

          \node at (-1.8, 0) {$k_1 + 3$};
          \node at (1.8, 0) {$k_2 + 3$};
          %\node at (0, 0) {$4$};
        \end{scope}
        \\
      };
    \end{tikzfigure}
  \end{cdescription}
\end{construction}

\begin{theorem}
  Let $p = (p_3, p_4, \dots, p_n)$, $v = (v_3, v_4, \dots, v_m)$ be a pair of admissible sequences for a closed orientable $2$-manifold $S$ with {\sc Euler} characteristic $\chi$, i.e. \eqref{eq:vp:5} and \eqref{eq:handshake} hold for some $e \in \nats$. Then $(p, v)$ is $[(9k + 2) \times 3, 3k + 4]$-$[5]$-realizable for all $k \in \nats$.
  \begin{proof}
    Using \autoref{const:edge:replacement:3:5:1} on the edges drawn thick in both $5$-patches in \autoref{fig:expansion:patch:3:4:5} results in $5$-patches consisting only of triangles and $n$-gons with the specified amount of vertices. The left patch is an expansion patch with outer tuple $o$, while the right patch is $o$-$6$-gonal, these properties do not change when using \autoref{const:edge:replacement:5:1} and \autoref{const:edge:replacement:5:2} on the thick lines. In fact, the resulting patch on the left is even polyhedral, so we can apply \autoref{thm:main:const:5}.
    \begin{tikzfigure}{\label{fig:expansion:patch:3:4:5}}{\todo{better picture}}
      \matrix (m) [column sep=1cm] {
        \begin{scope}[scale=3]
          \coordinate (x0) at (-0.623490, -0.781831);
\coordinate (x1) at (0.194893, -0.296146);
\coordinate (x2) at (0.900969, 0.433884);
\coordinate (x3) at (0.222521, -0.974928);
\coordinate (x4) at (0.900969, -0.433884);
\coordinate (x5) at (0.222521, 0.974928);
\coordinate (x6) at (-0.623490, 0.781831);
\coordinate (x7) at (-1.000000, 0.000000);
\draw (-0.623490, -0.781831) -- (0.194893, -0.296146);
\draw (0.194893, -0.296146) -- (0.900969, 0.433884);
\draw (0.900969, 0.433884) -- (-0.623490, -0.781831);
\draw (-0.623490, -0.781831) -- (0.222521, -0.974928);
\draw (0.222521, -0.974928) -- (0.194893, -0.296146);
\draw (0.222521, -0.974928) -- (0.900969, 0.433884);
\draw (0.222521, -0.974928) -- (0.900969, -0.433884);
\draw (0.900969, -0.433884) -- (0.900969, 0.433884);
\draw (0.900969, 0.433884) -- (0.222521, 0.974928);
\draw (0.222521, 0.974928) -- (-0.623490, -0.781831);
\draw (0.222521, 0.974928) -- (-0.623490, 0.781831);
\draw (-0.623490, 0.781831) -- (-1.000000, 0.000000);
\draw (-1.000000, 0.000000) -- (-0.623490, -0.781831);

        \end{scope}
        &
        \begin{scope}[scale=3]
          \coordinate (x0) at (-0.987688, -0.156434);
\coordinate (x1) at (-0.407611, -0.078729);
\coordinate (x2) at (-0.156434, 0.987688);
\coordinate (x3) at (-0.891007, -0.453990);
\coordinate (x4) at (-0.707107, -0.707107);
\coordinate (x5) at (-0.453990, -0.891007);
\coordinate (x6) at (-0.156434, -0.987688);
\coordinate (x7) at (0.156434, 0.987688);
\coordinate (x8) at (0.156434, -0.987688);
\coordinate (x9) at (0.407605, 0.078709);
\coordinate (x10) at (0.987688, 0.156434);
\coordinate (x11) at (0.891007, 0.453990);
\coordinate (x12) at (0.707107, 0.707107);
\coordinate (x13) at (0.453990, 0.891007);
\coordinate (x14) at (-0.453990, 0.891007);
\coordinate (x15) at (-0.987688, 0.156434);
\coordinate (x16) at (-0.707107, 0.707107);
\coordinate (x17) at (-0.891007, 0.453990);
\coordinate (x18) at (0.453990, -0.891007);
\coordinate (x19) at (0.987688, -0.156434);
\coordinate (x20) at (0.707107, -0.707107);
\coordinate (x21) at (0.891007, -0.453990);
\draw (-0.987688, -0.156434) -- (-0.407611, -0.078729);
\draw (-0.407611, -0.078729) -- (-0.156434, 0.987688);
\draw (-0.156434, 0.987688) -- (-0.987688, -0.156434);
\draw (-0.987688, -0.156434) -- (-0.891007, -0.453990);
\draw (-0.891007, -0.453990) -- (-0.407611, -0.078729);
\draw (-0.891007, -0.453990) -- (-0.707107, -0.707107);
\draw (-0.707107, -0.707107) -- (-0.407611, -0.078729);
\draw (-0.707107, -0.707107) -- (-0.453990, -0.891007);
\draw (-0.453990, -0.891007) -- (-0.156434, -0.987688);
\draw (-0.156434, -0.987688) -- (-0.407611, -0.078729);
\draw[very thick] (-0.156434, -0.987688) -- (0.156434, 0.987688);
\draw[very thick] (0.156434, 0.987688) -- (-0.156434, 0.987688);
\draw[very thick] (-0.156434, -0.987688) -- (0.156434, -0.987688);
\draw (0.156434, -0.987688) -- (0.407605, 0.078709);
\draw (0.407605, 0.078709) -- (0.156434, 0.987688);
\draw (0.156434, -0.987688) -- (0.987688, 0.156434);
\draw (0.987688, 0.156434) -- (0.407605, 0.078709);
\draw (0.987688, 0.156434) -- (0.891007, 0.453990);
\draw (0.891007, 0.453990) -- (0.407605, 0.078709);
\draw (0.891007, 0.453990) -- (0.707107, 0.707107);
\draw (0.707107, 0.707107) -- (0.407605, 0.078709);
\draw (0.707107, 0.707107) -- (0.453990, 0.891007);
\draw (0.453990, 0.891007) -- (0.156434, 0.987688);
\draw (-0.156434, 0.987688) -- (-0.453990, 0.891007);
\draw (-0.453990, 0.891007) -- (-0.987688, 0.156434);
\draw (-0.987688, 0.156434) -- (-0.987688, -0.156434);
\draw (-0.453990, 0.891007) -- (-0.707107, 0.707107);
\draw (-0.707107, 0.707107) -- (-0.987688, 0.156434);
\draw (-0.707107, 0.707107) -- (-0.891007, 0.453990);
\draw (-0.891007, 0.453990) -- (-0.987688, 0.156434);
\draw (0.156434, -0.987688) -- (0.453990, -0.891007);
\draw (0.453990, -0.891007) -- (0.987688, -0.156434);
\draw (0.987688, -0.156434) -- (0.987688, 0.156434);
\draw (0.453990, -0.891007) -- (0.707107, -0.707107);
\draw (0.707107, -0.707107) -- (0.987688, -0.156434);
\draw (0.707107, -0.707107) -- (0.891007, -0.453990);
\draw (0.891007, -0.453990) -- (0.987688, -0.156434);

        \end{scope}
        \\
      };
    \end{tikzfigure}
  \end{proof}
\end{theorem}
