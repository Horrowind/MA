\section{Negative results}

\begin{definition} A planar graph is called multi-$r$-valent, $r \in \nats$, if the valence of every vertex is a multiple of $r$. Dually the graph is called multi-$k$-gonal, $k \in \nats$, if for every $k'$-gon $k$ divides $k'$. It is called almost multi-$k$-gonal / almost multi-$r$-valent, if there is exactly one $k'$-gonal face / exactly one $r'$-valent vertex with $k' \nmid k$.
\end{definition}

\begin{theorem} Let $r, k \in \set{2, 3, 4, 5, 6}$. No $2$-connected graph is both multi-$r$-valent and almost multi-$k$-gonal.
\end{theorem}

We can dualize this theorem and get:
\begin{corollary} No $2$-connected graph is both almost multi-$r$-valent and multi-$k$-gonal, $r, k \in \set{2, 3, 4, 5, 6}$.
\end{corollary}

\begin{corollary}
  Let $r \in \set{3, 4, 5}$, $s \in {3, 4, 5}$ if $r = 3$ or $s = 3$. Additionally let $c \in \nats$. Then the sequences $p$, $v$,
  \begin{alignat*}{2}
    p &= [l], & \qquad&s \nmid l \\
    v &= (v_3, \dots, v_n), & &\sum_{k = 3, r \nmid k}^n v_k = 0
  \end{alignat*}
  or dual
  \begin{alignat*}{2}
    p &= (p_3, \dots, p_m), & \qquad&\sum_{k = 3, s \nmid k}^m p_k = 0\\
    v &= [l], & &r \nmid l
  \end{alignat*}
  are not $[q_s \times s, q_{c \cdot s} \times (c \cdot s)]$-$[r]$-realizable as polyhedron (for all $q_s, q_{c \cdot s} \in \nats$).
\begin{proof}
  The $1$-skeleton of a $[q_s \times s, q_{c \cdot s} \times (c \cdot s)]$-$[r]$-realization of such a pair of $p$-vectors and $v$-vectors would directly contradict the previous two statements.
\end{proof}
\end{corollary}

\begin{proposition}
  Let $p$ and $v$ be sequences, $S$ a closed $2$-manifold and $l \in \nats, l \geq 13$, $q_3 = \frac{l - 6}{\gcd(3, l-6)}$, $q_l = \frac{3}{\gcd(3, l-6)}$. Then the set
  \begin{align*}
    C(l, p, v) \defeq \set{c \in \nats : (p + c \cdot [q_3 \times 3, q_l \times l], v + c \cdot [3]) \text{ is realizable on } S}
  \end{align*}
  is cofinal. Additionally, if $l \geq 11$ and $\sum_{k \geq 4} 2k \cdot v_k + \sum_{k \geq 4} \floor{\tfrac{k}{2}} p_k < 3p_3$, then $C(l, p, v)$ is empty.
  \begin{proof}
    Since any to faces of a polyhedral map on $S$ have to meet proper, there can not be two triangles which coincide on a single edge with two $3$-valence vertices (see \autoref{fig:adjacent:triangles}). %
    \begin{tikzfigure}{\label{fig:adjacent:triangles}}{Two triangles incident to a $3$-valent edge}
      \draw (-2, 0) -- (-1, 0) -- (0, 1) -- (1, 0) -- (2, 0);
      \draw (-1, 0) -- (0, -1) -- (1, 0);
      \draw (0, 1) -- (0, -1);
      \draw[loosely dotted] (-2.5, 0) -- (-2, 0);
      \draw[loosely dotted] (2.5, 0) -- (2, 0);
    \end{tikzfigure}%
    Therefore every edge adjacent to a triangle has to have either a vertex of valence greater than $3$ or it has to be also adjacent to a larger $k$-gon, $k > 3$. Using this we can estimate the number of adjacent triangle-edge pairs (of which there are $3p_3$ many). Around each vertex of valence $r > 3$ we can put $r$ triangles, which counts as $2r$ triangle-edge pairs. Around each $k$-gon we can fit no more than $\floor{\frac{k}{2}} + \lceil \frac{j}{2} \rceil$ triangles, where $j$ is the number of vertices with valence greater than $3$ adjacent to the $k$-gon \todo{insert figure}. We can only hope for the greatest number of triangle-edge pairs, if we do not waste the vertices with valence greater than $3$ to the larger $k$-gons. Thus we can have only $\floor{\frac{k}{2}}$ triangle-edge pairs around each $k$-gon. Let $p' \defeq p + c \cdot [q_3 \times 3, q_l \times l]$, $v' = v + c \cdot [3]$, we have
    \begin{align*}
      &\sum_{k \geq 4} 2k \cdot v'_k + \sum_{k \geq 4} \floor{\tfrac{k}{2}} p'_k \\
      ={}&\sum_{k \geq 4} 2k \cdot v_k + \sum_{k \geq 4} \floor{\tfrac{k}{2}} p_k + c \cdot q_l \cdot \floor{\tfrac{l}{2}}\\
      \geq{} & 3 p'_3 {}={} 3 p_3 + 3 c \cdot q_3 \\
      \implies & \sum_{k \geq 4} 2k \cdot v_k + \sum_{k \geq 4} \floor{\tfrac{k}{2}} p_k - 3p_3 \geq c \cdot (3 q_3 - q_l \cdot \floor{\tfrac{l}{2}})
    \end{align*}
    Since $3q_3 = (l - 6)q_l$, the coefficient $(3 q_3 - q_l \cdot \floor{\tfrac{l}{2}})$ is positive if $\floor{\frac{l}{2}} < l - 6$, i.e. if $l \geq 13$, and we can only hope for finally many solutions. If $\sum_{k \geq 4} 2k \cdot v_k + \sum_{k \geq 4} \floor{\tfrac{k}{2}} p_k < 3p_3$, the coefficient of $c$ has to be negative, which is only the case for $l < 11$. This finishes the proof.
  \end{proof}
\end{proposition}

